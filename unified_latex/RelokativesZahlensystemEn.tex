\documentclass[11pt,a4paper,openany]{book}

% Essential packages
\usepackage[utf8]{inputenc}
\usepackage[T1]{fontenc}
\usepackage[english]{babel}
\usepackage[a4paper,margin=2.5cm]{geometry}
\usepackage{lmodern}

% Math and physics packages
\usepackage{amsmath}
\usepackage{amssymb}
\usepackage{amsthm}
\usepackage{mathtools}
\usepackage{physics}
\usepackage{siunitx}

% Graphics and tables
\usepackage{graphicx}
\usepackage[table,xcdraw]{xcolor}
\usepackage{tikz}
\usepackage{pgfplots}
\usepackage{tcolorbox}
\usepackage{booktabs}
\usepackage{array}
\usepackage{longtable}
\usepackage{float}

% Document formatting
\usepackage{fancyhdr}
\usepackage{tocloft}
\usepackage{hyperref}
\usepackage{cleveref}
\usepackage{microtype}
\usepackage{enumitem}
\usepackage{newunicodechar}

% Additional packages
\usepackage{adjustbox}
\usepackage{algorithm}
\usepackage{algorithmic}
\usepackage{amsfonts}
\usepackage{amsmath,amsfonts,amssymb}
\usepackage{amsmath,amsfonts,amssymb,physics}
\usepackage{amsmath,amssymb}
\usepackage{amsmath,amssymb,amsfonts,amsthm}
\usepackage{amsmath,amssymb,amsthm}
\usepackage{amsmath,amssymb,physics,graphicx,xcolor,amsthm}
\usepackage{bm}
\usepackage{booktabs,array,longtable,multirow}
\usepackage{braket}
\usepackage{breakurl}
\usepackage{cancel}
\usepackage{caption}
\usepackage{cite}
\usepackage{color}
\usepackage{colortbl}
\usepackage{csquotes}
\usepackage{doi}
\usepackage{forest}
\usepackage{gensymb}
\usepackage{geometry,fancyhdr}
\usepackage{graphicx,tikz,pgfplots}
\usepackage{hyperref,url}
\usepackage{hyphenat}
\usepackage{listings}
\usepackage{listings,enumerate}
\usepackage{mdframed}
\usepackage{multicol}
\usepackage{multirow}
\usepackage{natbib}
\usepackage{pdflscape}
\usepackage{ragged2e}
\usepackage{setspace}
\usepackage{siunitx,xcolor,graphicx}
\usepackage{slashed}
\usepackage{tabularx}
\usepackage{textcomp}
\usepackage{textgreek}
\usepackage{tikz,pgfplots}
\usepackage{upgreek}
\usepackage{url}

% Custom commands and definitions
\definecolor{blue}
\definecolor{blue}{rgb}{0,0,1}
\definecolor{boxgray}
\definecolor{boxgray}{RGB}{240,240,240}
\definecolor{deepblue}
\definecolor{deepblue}{RGB}{0,0,127}
\definecolor{deepgreen}
\definecolor{deepgreen}{RGB}{0,127,0}
\definecolor{deepred}
\definecolor{deepred}{RGB}{191,0,0}
\definecolor{t0blue}
\definecolor{t0blue}{RGB}{0,102,204}
\definecolor{t0blue}{RGB}{33,150,243}
\definecolor{t0green}
\definecolor{t0green}{RGB}{0,153,0}
\definecolor{t0green}{RGB}{0,153,76}
\definecolor{t0green}{RGB}{76,175,80}
\definecolor{t0orange}
\definecolor{t0orange}{RGB}{255,152,0}
\definecolor{t0purple}
\definecolor{t0purple}{RGB}{102,0,204}
\definecolor{t0purple}{RGB}{156,39,176}
\definecolor{t0red}
\definecolor{t0red}{RGB}{204,0,0}
\definecolor{t0red}{RGB}{204,0,51}
\definecolor{t0red}{RGB}{244,67,54}
\definecolor{t0yellow}
\definecolor{t0yellow}{RGB}{255,204,0}
\geometry{a4paper, left=25mm, right=25mm, top=25mm, bottom=25mm}
\geometry{a4paper, margin=1in}
\geometry{a4paper, margin=2.5cm}
\geometry{a4paper, margin=2cm}
\geometry{left=2.5cm,right=2.5cm,top=2.5cm,bottom=2.5cm}
\geometry{left=2cm,right=2cm,top=2cm,bottom=2cm}
\geometry{margin=1in}
\geometry{margin=2.5cm}
\geometry{margin=2cm}
\hypersetup{
	colorlinks=true,
	linkcolor=blue,
	citecolor=blue,
	urlcolor=blue,
	pdftitle={Analysis and Implications of MNRAS Paper 544 for the T0-Theory}
\hypersetup{
	colorlinks=true,
	linkcolor=blue,
	citecolor=blue,
	urlcolor=blue,
	pdftitle={Beweis: Die Feinstrukturkonstante α = 1 in natürlichen Einheiten}
\hypersetup{
	colorlinks=true,
	linkcolor=blue,
	citecolor=blue,
	urlcolor=blue,
	pdftitle={Beweis: Die Koide-Formel enthält implizit $\xi$}
\hypersetup{
	colorlinks=true,
	linkcolor=blue,
	citecolor=blue,
	urlcolor=blue,
	pdftitle={Chinas Photonischer Quantenchip: 1000x-Speedup und T0-Integration}
\hypersetup{
	colorlinks=true,
	linkcolor=blue,
	citecolor=blue,
	urlcolor=blue,
	pdftitle={Complete Derivation of Higgs Mass and Wilson Coefficients}
\hypersetup{
	colorlinks=true,
	linkcolor=blue,
	citecolor=blue,
	urlcolor=blue,
	pdftitle={Complete Particle Spectrum: Standard Model vs T0 Theory}
\hypersetup{
	colorlinks=true,
	linkcolor=blue,
	citecolor=blue,
	urlcolor=blue,
	pdftitle={Conceptual Comparison of Unified Natural Units and Extended Standard Model}
\hypersetup{
	colorlinks=true,
	linkcolor=blue,
	citecolor=blue,
	urlcolor=blue,
	pdftitle={Connections between the Mizohata-Takeuchi Counterexample and the T0 Time-Mass Duality Theory}
\hypersetup{
	colorlinks=true,
	linkcolor=blue,
	citecolor=blue,
	urlcolor=blue,
	pdftitle={Das Relationale Zahlensystem: Primzahlen als fundamentale Verhältnisse}
\hypersetup{
	colorlinks=true,
	linkcolor=blue,
	citecolor=blue,
	urlcolor=blue,
	pdftitle={Das T0-Modell (Planck-Referenziert): Eine Neuformulierung der Physik}
\hypersetup{
	colorlinks=true,
	linkcolor=blue,
	citecolor=blue,
	urlcolor=blue,
	pdftitle={Das T0-Modell: Zeit-Energie-Dualität und geometrische Ruhemasse}
\hypersetup{
	colorlinks=true,
	linkcolor=blue,
	citecolor=blue,
	urlcolor=blue,
	pdftitle={Der Massenskalierungsexponent κ in der T0-Theorie}
\hypersetup{
	colorlinks=true,
	linkcolor=blue,
	citecolor=blue,
	urlcolor=blue,
	pdftitle={Der geometrische Formalismus der T0-Quantenmechanik und seine Anwendung auf Quantencomputer}
\hypersetup{
	colorlinks=true,
	linkcolor=blue,
	citecolor=blue,
	urlcolor=blue,
	pdftitle={Der xi Parameter und Teilchendifferenzierung in der T0-Theorie}
\hypersetup{
	colorlinks=true,
	linkcolor=blue,
	citecolor=blue,
	urlcolor=blue,
	pdftitle={Deterministic Quantum Mechanics via T0-Energy Field Formulation}
\hypersetup{
	colorlinks=true,
	linkcolor=blue,
	citecolor=blue,
	urlcolor=blue,
	pdftitle={Deterministische Quantenmechanik via T0-Energiefeld-Formulierung}
\hypersetup{
	colorlinks=true,
	linkcolor=blue,
	citecolor=blue,
	urlcolor=blue,
	pdftitle={Die Elektroneneinheitsladung in der T0-Theorie: Jenseits von Punkt-Singularitäten}
\hypersetup{
	colorlinks=true,
	linkcolor=blue,
	citecolor=blue,
	urlcolor=blue,
	pdftitle={Die Feinstrukturkonstante: Verschiedene Darstellungen und Beziehungen}
\hypersetup{
	colorlinks=true,
	linkcolor=blue,
	citecolor=blue,
	urlcolor=blue,
	pdftitle={Die Musikalische Spirale und die 137: Die mathematische Entdeckung der kosmischen Verstimmung}
\hypersetup{
	colorlinks=true,
	linkcolor=blue,
	citecolor=blue,
	urlcolor=blue,
	pdftitle={E=mc² = E=m: Die Konstanten-Illusion entlarvt}
\hypersetup{
	colorlinks=true,
	linkcolor=blue,
	citecolor=blue,
	urlcolor=blue,
	pdftitle={E=mc² = E=m: The Constants Illusion Exposed}
\hypersetup{
	colorlinks=true,
	linkcolor=blue,
	citecolor=blue,
	urlcolor=blue,
	pdftitle={Einfache Lagrange-Revolution: Von der Standardmodell-Komplexität zur T0-Eleganz}
\hypersetup{
	colorlinks=true,
	linkcolor=blue,
	citecolor=blue,
	urlcolor=blue,
	pdftitle={Einführung in die Umsetzung photonischer Bauteile auf Wafern für Nachrichtentechniker}
\hypersetup{
	colorlinks=true,
	linkcolor=blue,
	citecolor=blue,
	urlcolor=blue,
	pdftitle={Einführung in photonische Quantenchips für Nachrichtentechniker}
\hypersetup{
	colorlinks=true,
	linkcolor=blue,
	citecolor=blue,
	urlcolor=blue,
	pdftitle={Elimination der Masse als dimensionaler Platzhalter im T0-Modell}
\hypersetup{
	colorlinks=true,
	linkcolor=blue,
	citecolor=blue,
	urlcolor=blue,
	pdftitle={Elimination of Mass as Dimensional Placeholder in the T0 Model}
\hypersetup{
	colorlinks=true,
	linkcolor=blue,
	citecolor=blue,
	urlcolor=blue,
	pdftitle={Empirical Analysis of Deterministic Factorization Methods}
\hypersetup{
	colorlinks=true,
	linkcolor=blue,
	citecolor=blue,
	urlcolor=blue,
	pdftitle={Empirische Analyse deterministischer Faktorisierungsmethoden}
\hypersetup{
	colorlinks=true,
	linkcolor=blue,
	citecolor=blue,
	urlcolor=blue,
	pdftitle={Integration der Dirac-Gleichung im T0-Modell: Natürliche-Einheiten-Rahmenwerk}
\hypersetup{
	colorlinks=true,
	linkcolor=blue,
	citecolor=blue,
	urlcolor=blue,
	pdftitle={Integration of the Dirac Equation in the T0 Model: Natural Units Framework}
\hypersetup{
	colorlinks=true,
	linkcolor=blue,
	citecolor=blue,
	urlcolor=blue,
	pdftitle={Introduction to Photonic Quantum Chips for Communication Engineers}
\hypersetup{
	colorlinks=true,
	linkcolor=blue,
	citecolor=blue,
	urlcolor=blue,
	pdftitle={Introduction to the Implementation of Photonic Components on Wafers for Communication Engineers}
\hypersetup{
	colorlinks=true,
	linkcolor=blue,
	citecolor=blue,
	urlcolor=blue,
	pdftitle={Konzeptioneller Vergleich von Einheitlichen Natürlichen Einheiten und Erweitertem Standardmodell}
\hypersetup{
	colorlinks=true,
	linkcolor=blue,
	citecolor=blue,
	urlcolor=blue,
	pdftitle={Markov Chains in the Context of T0 Theory: Deterministic or Stochastic? A Treatise on Patterns, Preconditions, and Uncertainty}
\hypersetup{
	colorlinks=true,
	linkcolor=blue,
	citecolor=blue,
	urlcolor=blue,
	pdftitle={Markov-Ketten im Kontext der T0-Theorie: Deterministisch oder stochastisch? Ein Traktat zu Mustern, Voraussetzungen und Unsicherheit}
\hypersetup{
	colorlinks=true,
	linkcolor=blue,
	citecolor=blue,
	urlcolor=blue,
	pdftitle={Mathematical Analysis of T0-Shor Algorithm: Theoretical Framework and Computational Complexity}
\hypersetup{
	colorlinks=true,
	linkcolor=blue,
	citecolor=blue,
	urlcolor=blue,
	pdftitle={Mathematical Constructs of Alternative CMB Models: Unnikrishnan and Peratt in Harmony with the T0 Theory}
\hypersetup{
	colorlinks=true,
	linkcolor=blue,
	citecolor=blue,
	urlcolor=blue,
	pdftitle={Mathematische Analyse des T0-Shor Algorithmus: Theoretischer Rahmen und Berechnungskomplexität}
\hypersetup{
	colorlinks=true,
	linkcolor=blue,
	citecolor=blue,
	urlcolor=blue,
	pdftitle={Mathematische Konstrukte alternativer CMB-Modelle: Unnikrishnan und Peratt im Einklang mit der T0-Theorie}
\hypersetup{
	colorlinks=true,
	linkcolor=blue,
	citecolor=blue,
	urlcolor=blue,
	pdftitle={Natural Unit Systems: Universal Energy Conversion and Fundamental Length Scale Hierarchy}
\hypersetup{
	colorlinks=true,
	linkcolor=blue,
	citecolor=blue,
	urlcolor=blue,
	pdftitle={Natural Units in Theoretical Physics: A Treatise in the Context of T0 Theory}
\hypersetup{
	colorlinks=true,
	linkcolor=blue,
	citecolor=blue,
	urlcolor=blue,
	pdftitle={Natürliche Einheiten in der theoretischen Physik: Eine Abhandlung im Kontext der T0-Theorie}
\hypersetup{
	colorlinks=true,
	linkcolor=blue,
	citecolor=blue,
	urlcolor=blue,
	pdftitle={Natürliche Einheitensysteme: Universelle Energieumwandlung und fundamentale Längenskala-Hierarchie}
\hypersetup{
	colorlinks=true,
	linkcolor=blue,
	citecolor=blue,
	urlcolor=blue,
	pdftitle={Parameter System-Dependency in T0-Model: SI vs. Natural Units}
\hypersetup{
	colorlinks=true,
	linkcolor=blue,
	citecolor=blue,
	urlcolor=blue,
	pdftitle={Parameter-Systemabhängigkeit im T0-Modell: SI- vs. natürliche Einheiten}
\hypersetup{
	colorlinks=true,
	linkcolor=blue,
	citecolor=blue,
	urlcolor=blue,
	pdftitle={Proof: The Fine Structure Constant α = 1 in Natural Units}
\hypersetup{
	colorlinks=true,
	linkcolor=blue,
	citecolor=blue,
	urlcolor=blue,
	pdftitle={Proof: The Koide Formula Implicitly Contains $\xi$}
\hypersetup{
	colorlinks=true,
	linkcolor=blue,
	citecolor=blue,
	urlcolor=blue,
	pdftitle={Pure Energy T0 Theory: Ratio-Based Physics with SI Reference}
\hypersetup{
	colorlinks=true,
	linkcolor=blue,
	citecolor=blue,
	urlcolor=blue,
	pdftitle={Quantum Mechanics in the T0 Model: Field-Theoretic Foundations}
\hypersetup{
	colorlinks=true,
	linkcolor=blue,
	citecolor=blue,
	urlcolor=blue,
	pdftitle={Ratio-Based vs. Absolute: The Role of Fractal Correction in T0 Theory}
\hypersetup{
	colorlinks=true,
	linkcolor=blue,
	citecolor=blue,
	urlcolor=blue,
	pdftitle={Reine Energie T0-Theorie: Verhältnis-basierte Physik mit SI-Referenz}
\hypersetup{
	colorlinks=true,
	linkcolor=blue,
	citecolor=blue,
	urlcolor=blue,
	pdftitle={Simple Lagrangian Revolution: From Standard Model Complexity to T0 Elegance}
\hypersetup{
	colorlinks=true,
	linkcolor=blue,
	citecolor=blue,
	urlcolor=blue,
	pdftitle={Simplified Dirac Equation in T0 Theory: Field Node Approach}
\hypersetup{
	colorlinks=true,
	linkcolor=blue,
	citecolor=blue,
	urlcolor=blue,
	pdftitle={Simplified T0 Theory: Elegant Lagrangian Density for Time-Mass Duality}
\hypersetup{
	colorlinks=true,
	linkcolor=blue,
	citecolor=blue,
	urlcolor=blue,
	pdftitle={T0 Cosmology: Redshift as a Geometric Path Effect in a Static Universe}
\hypersetup{
	colorlinks=true,
	linkcolor=blue,
	citecolor=blue,
	urlcolor=blue,
	pdftitle={T0 Deterministic Quantum Computing: Complete Analysis of Important Algorithms}
\hypersetup{
	colorlinks=true,
	linkcolor=blue,
	citecolor=blue,
	urlcolor=blue,
	pdftitle={T0 Deterministisches Quantencomputing: Vollständige Analyse wichtiger Algorithmen}
\hypersetup{
	colorlinks=true,
	linkcolor=blue,
	citecolor=blue,
	urlcolor=blue,
	pdftitle={T0 Model: Complete Framework - From Time-Energy Duality to Universal Constants}
\hypersetup{
	colorlinks=true,
	linkcolor=blue,
	citecolor=blue,
	urlcolor=blue,
	pdftitle={T0 Model: Complete Parameter-Free Particle Mass Calculation}
\hypersetup{
	colorlinks=true,
	linkcolor=blue,
	citecolor=blue,
	urlcolor=blue,
	pdftitle={T0 Model: Unified Neutrino Formula Structure}
\hypersetup{
	colorlinks=true,
	linkcolor=blue,
	citecolor=blue,
	urlcolor=blue,
	pdftitle={T0 Model: Universal Energy Relations for Mol and Candela Units}
\hypersetup{
	colorlinks=true,
	linkcolor=blue,
	citecolor=blue,
	urlcolor=blue,
	pdftitle={T0 Modell: Vollständiges Framework - Von Zeit-Energie-Dualität zu universellen Konstanten}
\hypersetup{
	colorlinks=true,
	linkcolor=blue,
	citecolor=blue,
	urlcolor=blue,
	pdftitle={T0 Quantenfeldtheorie: QFT, QM und Quantencomputer}
\hypersetup{
	colorlinks=true,
	linkcolor=blue,
	citecolor=blue,
	urlcolor=blue,
	pdftitle={T0 Quantum Field Theory: QFT, QM and Quantum Computers}
\hypersetup{
	colorlinks=true,
	linkcolor=blue,
	citecolor=blue,
	urlcolor=blue,
	pdftitle={T0 Theory vs Bell's Theorem: How Deterministic Energy Fields Circumvent No-Go Theorems}
\hypersetup{
	colorlinks=true,
	linkcolor=blue,
	citecolor=blue,
	urlcolor=blue,
	pdftitle={T0 Theory: Final Extension to Hadrons - Physically Derived Corrections}
\hypersetup{
	colorlinks=true,
	linkcolor=blue,
	citecolor=blue,
	urlcolor=blue,
	pdftitle={T0 Theory: The Fine-Structure Constant}
\hypersetup{
	colorlinks=true,
	linkcolor=blue,
	citecolor=blue,
	urlcolor=blue,
	pdftitle={T0 Theory: The Gravitational Constant}
\hypersetup{
	colorlinks=true,
	linkcolor=blue,
	citecolor=blue,
	urlcolor=blue,
	pdftitle={T0-Kosmologie: Rotverschiebung als geometrischer Pfad-Effekt im statischen Universum}
\hypersetup{
	colorlinks=true,
	linkcolor=blue,
	citecolor=blue,
	urlcolor=blue,
	pdftitle={T0-Model: Complete Document Analysis and Structured Summary}
\hypersetup{
	colorlinks=true,
	linkcolor=blue,
	citecolor=blue,
	urlcolor=blue,
	pdftitle={T0-Model: Kinetic Energy of Electrons and Photons}
\hypersetup{
	colorlinks=true,
	linkcolor=blue,
	citecolor=blue,
	urlcolor=blue,
	pdftitle={T0-Model: The Hubble Parameter in Static Universe}
\hypersetup{
	colorlinks=true,
	linkcolor=blue,
	citecolor=blue,
	urlcolor=blue,
	pdftitle={T0-Modell-Verifikation: Skalen-Verhältnis-basierte Berechnungen}
\hypersetup{
	colorlinks=true,
	linkcolor=blue,
	citecolor=blue,
	urlcolor=blue,
	pdftitle={T0-Modell: Bewegungsenergie von Elektronen und Photonen}
\hypersetup{
	colorlinks=true,
	linkcolor=blue,
	citecolor=blue,
	urlcolor=blue,
	pdftitle={T0-Modell: Die Hubble-Konstante im statischen Universum}
\hypersetup{
	colorlinks=true,
	linkcolor=blue,
	citecolor=blue,
	urlcolor=blue,
	pdftitle={T0-Modell: Einheitliche Neutrino-Formel-Struktur}
\hypersetup{
	colorlinks=true,
	linkcolor=blue,
	citecolor=blue,
	urlcolor=blue,
	pdftitle={T0-Modell: Universelle Energiebeziehungen für Mol- und Candela-Einheiten}
\hypersetup{
	colorlinks=true,
	linkcolor=blue,
	citecolor=blue,
	urlcolor=blue,
	pdftitle={T0-Modell: Vollständige Dokumentenanalyse und strukturierte Zusammenfassung}
\hypersetup{
	colorlinks=true,
	linkcolor=blue,
	citecolor=blue,
	urlcolor=blue,
	pdftitle={T0-Modell: Vollständige parameterfreie Teilchenmassen-Berechnung}
\hypersetup{
	colorlinks=true,
	linkcolor=blue,
	citecolor=blue,
	urlcolor=blue,
	pdftitle={T0-QAT: $\xi$-Aware Quantization-Aware Training}
\hypersetup{
	colorlinks=true,
	linkcolor=blue,
	citecolor=blue,
	urlcolor=blue,
	pdftitle={T0-QFT ML Addendum: Machine Learning Derived Extensions}
\hypersetup{
	colorlinks=true,
	linkcolor=blue,
	citecolor=blue,
	urlcolor=blue,
	pdftitle={T0-QFT ML-Addendum: Maschinelle Lern-abgeleitete Erweiterungen}
\hypersetup{
	colorlinks=true,
	linkcolor=blue,
	citecolor=blue,
	urlcolor=blue,
	pdftitle={T0-Theorie vs Bells Theorem: Wie deterministische Energiefelder No-Go-Theoreme umgehen}
\hypersetup{
	colorlinks=true,
	linkcolor=blue,
	citecolor=blue,
	urlcolor=blue,
	pdftitle={T0-Theorie: Der Terrell-Penrose-Effekt und Massenvariation}
\hypersetup{
	colorlinks=true,
	linkcolor=blue,
	citecolor=blue,
	urlcolor=blue,
	pdftitle={T0-Theorie: Die Feinstrukturkonstante}
\hypersetup{
	colorlinks=true,
	linkcolor=blue,
	citecolor=blue,
	urlcolor=blue,
	pdftitle={T0-Theorie: Die Gravitationskonstante}
\hypersetup{
	colorlinks=true,
	linkcolor=blue,
	citecolor=blue,
	urlcolor=blue,
	pdftitle={T0-Theorie: Die T0-Zeit-Masse-Dualität}
\hypersetup{
	colorlinks=true,
	linkcolor=blue,
	citecolor=blue,
	urlcolor=blue,
	pdftitle={T0-Theorie: Die sieben Rätsel}
\hypersetup{
	colorlinks=true,
	linkcolor=blue,
	citecolor=blue,
	urlcolor=blue,
	pdftitle={T0-Theorie: Erweiterung auf Bell-Tests – ML-Simulationen (November 2025)}
\hypersetup{
	colorlinks=true,
	linkcolor=blue,
	citecolor=blue,
	urlcolor=blue,
	pdftitle={T0-Theorie: Finale Erweiterung auf Hadronen - Physikalisch abgeleitete Korrekturen}
\hypersetup{
	colorlinks=true,
	linkcolor=blue,
	citecolor=blue,
	urlcolor=blue,
	pdftitle={T0-Theorie: Finale Fraktale Massenformeln (November 2025)}
\hypersetup{
	colorlinks=true,
	linkcolor=blue,
	citecolor=blue,
	urlcolor=blue,
	pdftitle={T0-Theorie: Fraktaldimension aus Lepton-Massenverhältnis}
\hypersetup{
	colorlinks=true,
	linkcolor=blue,
	citecolor=blue,
	urlcolor=blue,
	pdftitle={T0-Theorie: Fundamentale Prinzipien}
\hypersetup{
	colorlinks=true,
	linkcolor=blue,
	citecolor=blue,
	urlcolor=blue,
	pdftitle={T0-Theorie: Herleitung der Gravitationskonstanten}
\hypersetup{
	colorlinks=true,
	linkcolor=blue,
	citecolor=blue,
	urlcolor=blue,
	pdftitle={T0-Theorie: Kosmische Beziehungen und universelle $\xi$-Konstante}
\hypersetup{
	colorlinks=true,
	linkcolor=blue,
	citecolor=blue,
	urlcolor=blue,
	pdftitle={T0-Theorie: Kosmologie}
\hypersetup{
	colorlinks=true,
	linkcolor=blue,
	citecolor=blue,
	urlcolor=blue,
	pdftitle={T0-Theorie: Netzwerkdarstellung und Dimensionsanalyse in der T0-Theorie}
\hypersetup{
	colorlinks=true,
	linkcolor=blue,
	citecolor=blue,
	urlcolor=blue,
	pdftitle={T0-Theorie: Teilchenmassen}
\hypersetup{
	colorlinks=true,
	linkcolor=blue,
	citecolor=blue,
	urlcolor=blue,
	pdftitle={T0-Theorie: Vollstaendiger Abschluss}
\hypersetup{
	colorlinks=true,
	linkcolor=blue,
	citecolor=blue,
	urlcolor=blue,
	pdftitle={T0-Theory: Complete Closure}
\hypersetup{
	colorlinks=true,
	linkcolor=blue,
	citecolor=blue,
	urlcolor=blue,
	pdftitle={T0-Theory: Complete Derivation of All Parameters Without Circularity}
\hypersetup{
	colorlinks=true,
	linkcolor=blue,
	citecolor=blue,
	urlcolor=blue,
	pdftitle={T0-Theory: Cosmic Relations and universal $\xi$-constant}
\hypersetup{
	colorlinks=true,
	linkcolor=blue,
	citecolor=blue,
	urlcolor=blue,
	pdftitle={T0-Theory: Cosmology}
\hypersetup{
	colorlinks=true,
	linkcolor=blue,
	citecolor=blue,
	urlcolor=blue,
	pdftitle={T0-Theory: Derivation of the Gravitational Constant}
\hypersetup{
	colorlinks=true,
	linkcolor=blue,
	citecolor=blue,
	urlcolor=blue,
	pdftitle={T0-Theory: Extension to Bell Tests – ML Simulations (November 2025)}
\hypersetup{
	colorlinks=true,
	linkcolor=blue,
	citecolor=blue,
	urlcolor=blue,
	pdftitle={T0-Theory: Final Fractal Mass Formulas (November 2025)}
\hypersetup{
	colorlinks=true,
	linkcolor=blue,
	citecolor=blue,
	urlcolor=blue,
	pdftitle={T0-Theory: Fractal Dimension from Lepton Mass Ratio}
\hypersetup{
	colorlinks=true,
	linkcolor=blue,
	citecolor=blue,
	urlcolor=blue,
	pdftitle={T0-Theory: Fundamental Principles}
\hypersetup{
	colorlinks=true,
	linkcolor=blue,
	citecolor=blue,
	urlcolor=blue,
	pdftitle={T0-Theory: Mass Variation as an Equivalent to Time Dilation}
\hypersetup{
	colorlinks=true,
	linkcolor=blue,
	citecolor=blue,
	urlcolor=blue,
	pdftitle={T0-Theory: Network Representation and Dimensional Analysis in the T0-Theory}
\hypersetup{
	colorlinks=true,
	linkcolor=blue,
	citecolor=blue,
	urlcolor=blue,
	pdftitle={T0-Theory: Neutrinos}
\hypersetup{
	colorlinks=true,
	linkcolor=blue,
	citecolor=blue,
	urlcolor=blue,
	pdftitle={T0-Theory: Particle Masses}
\hypersetup{
	colorlinks=true,
	linkcolor=blue,
	citecolor=blue,
	urlcolor=blue,
	pdftitle={T0-Theory: The Seven Riddles}
\hypersetup{
	colorlinks=true,
	linkcolor=blue,
	citecolor=blue,
	urlcolor=blue,
	pdftitle={T0-Theory: The T0-Time-Mass Duality}
\hypersetup{
	colorlinks=true,
	linkcolor=blue,
	citecolor=blue,
	urlcolor=blue,
	pdftitle={Temperature Units in Natural Units: T0-Theory}
\hypersetup{
	colorlinks=true,
	linkcolor=blue,
	citecolor=blue,
	urlcolor=blue,
	pdftitle={Temperatureinheiten in nat\"urlichen Einheiten: T0-Theorie}
\hypersetup{
	colorlinks=true,
	linkcolor=blue,
	citecolor=blue,
	urlcolor=blue,
	pdftitle={The Electron Unit Charge in T0 Theory: Beyond Point Singularities}
\hypersetup{
	colorlinks=true,
	linkcolor=blue,
	citecolor=blue,
	urlcolor=blue,
	pdftitle={The Fine Structure Constant: Various Representations and Relationships}
\hypersetup{
	colorlinks=true,
	linkcolor=blue,
	citecolor=blue,
	urlcolor=blue,
	pdftitle={The Geometric Formalism of T0 Quantum Mechanics and its Application to Quantum Computing}
\hypersetup{
	colorlinks=true,
	linkcolor=blue,
	citecolor=blue,
	urlcolor=blue,
	pdftitle={The Mass Scaling Exponent κ in T0 Theory}
\hypersetup{
	colorlinks=true,
	linkcolor=blue,
	citecolor=blue,
	urlcolor=blue,
	pdftitle={The Musical Spiral and 137: The Mathematical Discovery of Cosmic Detuning}
\hypersetup{
	colorlinks=true,
	linkcolor=blue,
	citecolor=blue,
	urlcolor=blue,
	pdftitle={The Relational Number System: Prime Numbers as Fundamental Ratios}
\hypersetup{
	colorlinks=true,
	linkcolor=blue,
	citecolor=blue,
	urlcolor=blue,
	pdftitle={The T0 Model (Planck-Referenced): A Reformulation of Physics}
\hypersetup{
	colorlinks=true,
	linkcolor=blue,
	citecolor=blue,
	urlcolor=blue,
	pdftitle={The T0 Model: Time-Energy Duality and Geometric Rest Mass}
\hypersetup{
	colorlinks=true,
	linkcolor=blue,
	citecolor=blue,
	urlcolor=blue,
	pdftitle={The T0-Model (Planck-Referenced): A Reformulation of Physics}
\hypersetup{
	colorlinks=true,
	linkcolor=blue,
	citecolor=blue,
	urlcolor=blue,
	pdftitle={Verbindungen zwischen dem Mizohata-Takeuchi-Gegenbeispiel und der T0-Zeit-Masse-Dualitätstheorie}
\hypersetup{
	colorlinks=true,
	linkcolor=blue,
	citecolor=blue,
	urlcolor=blue,
	pdftitle={Vereinfachte Dirac-Gleichung in der T0-Theorie: Feldknoten-Ansatz}
\hypersetup{
	colorlinks=true,
	linkcolor=blue,
	citecolor=blue,
	urlcolor=blue,
	pdftitle={Vereinfachte T0-Theorie: Elegante Lagrange-Dichte für Zeit-Masse-Dualität}
\hypersetup{
	colorlinks=true,
	linkcolor=blue,
	citecolor=blue,
	urlcolor=blue,
	pdftitle={Verhältnisbasiert vs. Absolut: Die Rolle der fraktalen Korrektur in der T0-Theorie}
\hypersetup{
	colorlinks=true,
	linkcolor=blue,
	citecolor=blue,
	urlcolor=blue,
	pdftitle={Vollständige Herleitung der Higgs-Masse und Wilson-Koeffizienten}
\hypersetup{
	colorlinks=true,
	linkcolor=blue,
	citecolor=blue,
	urlcolor=blue,
	pdftitle={Vollständiges Teilchenspektrum: Standard-Modell vs T0-Theorie}
\hypersetup{
	colorlinks=true,
	linkcolor=blue,
	citecolor=blue,
	urlcolor=blue,
	pdftitle={Warum Zahlenverhältnisse nicht direkt gekürzt werden dürfen}
\hypersetup{
	colorlinks=true,
	linkcolor=blue,
	citecolor=blue,
	urlcolor=blue,
	pdftitle={Why Numerical Ratios Must Not Be Directly Simplified}
\hypersetup{
	colorlinks=true,
	linkcolor=blue,
	citecolor=blue,
	urlcolor=blue,
}
\hypersetup{
	colorlinks=true,
	linkcolor=blue,
	citecolor=red,
	urlcolor=blue,
	bookmarks=true,
	bookmarksnumbered=true,
	pdfstartview=FitH,
	pdftitle={T0 Model - Field-Theoretic Derivation of the Beta Parameter}
\hypersetup{
	colorlinks=true,
	linkcolor=blue,
	citecolor=red,
	urlcolor=blue,
	bookmarks=true,
	bookmarksnumbered=true,
	pdfstartview=FitH,
	pdftitle={T0-Modell - Feldtheoretische Herleitung des Beta-Parameters}
\hypersetup{
	colorlinks=true,
	linkcolor=blue,
	filecolor=magenta,
	urlcolor=cyan,
}
\hypersetup{
	colorlinks=true,
	linkcolor=blue,
	urlcolor=blue,
	citecolor=blue,
	pdftitle={From Time Dilation to Mass Variation: Mathematical Core Formulations of Time-Mass Duality Theory - Updated Framework}
\hypersetup{
	colorlinks=true,
	linkcolor=blue,
	urlcolor=blue,
	citecolor=blue,
	pdftitle={T0 Model: Detailed Formula for Leptonic Anomalies}
\hypersetup{
	colorlinks=true,
	linkcolor=blue,
	urlcolor=blue,
	citecolor=blue,
	pdftitle={T0 Model: Detaillierte Formel für leptonische Anomalien}
\hypersetup{
	colorlinks=true,
	linkcolor=blue,
	urlcolor=blue,
	citecolor=blue,
	pdftitle={T0 Model: Energy-based Formulas with Quadratic Scaling}
\hypersetup{
	colorlinks=true,
	linkcolor=blue,
	urlcolor=blue,
	citecolor=blue,
	pdftitle={T0 Model: Granulation, Limits and Fundamental Asymmetry}
\hypersetup{
	colorlinks=true,
	linkcolor=blue,
	urlcolor=blue,
	citecolor=blue,
	pdftitle={T0-Modell: Energiebasierte Formeln mit quadratischer Skalierung}
\hypersetup{
	colorlinks=true,
	linkcolor=blue,
	urlcolor=blue,
	citecolor=blue,
	pdftitle={T0-Modell: Granulation, Limits und fundamentale Asymmetrie}
\hypersetup{
	colorlinks=true,
	linkcolor=blue,
	urlcolor=blue,
	citecolor=blue,
	pdftitle={Von Zeitdilatation zu Massenvariation: Mathematische Kernformulierungen der Zeit-Masse-Dualitätstheorie - Aktualisiertes Framework}
\hypersetup{
	colorlinks=true,
	linkcolor=t0blue,
	citecolor=t0blue,
	urlcolor=t0blue,
	pdftitle={T0 Model: Complete Theoretical Summary}
\hypersetup{
	colorlinks=true,
	linkcolor=t0blue,
	citecolor=t0blue,
	urlcolor=t0blue,
	pdftitle={T0 Theory: Resolution of Apparent Instantaneity}
\hypersetup{
	colorlinks=true,
	linkcolor=t0blue,
	citecolor=t0blue,
	urlcolor=t0blue,
	pdftitle={T0 vs Synergetics: Vereinfachung durch natürliche Einheiten}
\hypersetup{
	colorlinks=true,
	linkcolor=t0blue,
	citecolor=t0blue,
	urlcolor=t0blue,
	pdftitle={T0-Modell: Vollständige theoretische Zusammenfassung}
\hypersetup{
	colorlinks=true,
	linkcolor=t0blue,
	citecolor=t0blue,
	urlcolor=t0blue,
	pdftitle={T0-Theorie: Auflösung der scheinbaren Instantanität}
\hypersetup{
	colorlinks=true,
	linkcolor=t0blue,
	citecolor=t0blue,
	urlcolor=t0blue,
	pdftitle={T0-Theorie: Vollständige Dokumentenübersicht}
\hypersetup{
	colorlinks=true,
	linkcolor=t0blue,
	citecolor=t0blue,
	urlcolor=t0blue,
	pdftitle={T0-Theory: Complete Document Overview}
\hypersetup{
	colorlinks=true,
	linkcolor=t0blue,
	citecolor=t0blue,
	urlcolor=t0blue,
}
\hypersetup{
	colorlinks=true,
	linkcolor=t0blue,
	citecolor=t0green,
	urlcolor=t0blue,
	pdftitle={Das verborgene Geheimnis von 1/137}
\hypersetup{
	colorlinks=true,
	linkcolor=t0blue,
	citecolor=t0green,
	urlcolor=t0blue,
	pdftitle={The Hidden Secret of 1/137}
\hypersetup{
    colorlinks=true,
    linkcolor=blue,
    citecolor=blue,
    urlcolor=blue,
    pdftitle={Analyse und Implikationen des MNRAS-Papiers 544 für die T0-Theorie}
\hypersetup{
  colorlinks=true,
  linkcolor=blue,
  citecolor=blue,
  urlcolor=blue
}
\hypersetup{
  colorlinks=true,
  linkcolor=blue,
  citecolor=blue,
  urlcolor=blue,
  pdftitle={T0-Theorie: Ein-Uhr-Metrologie und Drei-Uhren-Experiment}
\hypersetup{
  colorlinks=true,
  linkcolor=blue,
  citecolor=blue,
  urlcolor=blue,
  pdftitle={T0-Theory: Single-Clock Metrology and Three-Clock Experiment}
\hypersetup{
colorlinks=true,
linkcolor=blue,
citecolor=blue,
urlcolor=blue,
pdftitle={Quantenmechanik im T0-Modell: Feldtheoretische Grundlagen}
\hypersetup{
colorlinks=true,
linkcolor=blue,
citecolor=blue,
urlcolor=blue,
pdftitle={T0-Theory: Neutrinos}
\newcommand{\Bzero}{B_0}
\newcommand{\CQCD}{C_{\text{QCD}
\newcommand{\Cconv}{C_{\text{conv}
\newcommand{\Cto}{C_{\text{T0}
\newcommand{\Czero}{C_0}
\newcommand{\DTmu}{D_{T,\mu}
\newcommand{\DcovT}[1]{\partial_\mu #1 + #1 \partial_\mu \Tfield}
\newcommand{\Dfrak}{D_f}
\newcommand{\Df}{D_f}
\newcommand{\DhiggsT}{\Tfield (\partial_\mu + ig A_\mu) \Phi + \Phi \partial_\mu \Tfield}
\newcommand{\EPlanck}{E_P}
\newcommand{\EPlanck}{E_{\text{Pl}
\newcommand{\EPratio}[1]{\frac{#1}
\newcommand{\EP}{E_P}
\newcommand{\EP}{E_{\text{P}
\newcommand{\EW}{E_W}
\newcommand{\EZ}{E_Z}
\newcommand{\Echar}{E_{\text{char}
\newcommand{\Ee}{E_e}
\newcommand{\Efield}{E(x,t)}
\newcommand{\Efield}{E_\text{field}
\newcommand{\Efield}{E_{\text{Feld}
\newcommand{\Efield}{E_{\text{Field}
\newcommand{\Efield}{E_{\text{field}
\newcommand{\Efield}{E}
\newcommand{\Egamma}{E_\gamma}
\newcommand{\Eh}{E_h}
\newcommand{\Emu}{E_\mu}
\newcommand{\Enorm}[1]{E_{\text{norm}
\newcommand{\En}{E_n}
\newcommand{\Ep}{E_p}
\newcommand{\Eratio}[2]{\frac{E_{#1}
\newcommand{\Etau}{E_\tau}
\newcommand{\Evis}{E_{\text{vis}
\newcommand{\Exi}{E_\xi}
\newcommand{\Ezero}{E_0}
\newcommand{\GeV}{\,\text{GeV}
\newcommand{\Gnat}{G_{\text{nat}
\newcommand{\Gsi}{G_{\text{SI}
\newcommand{\Hubble}{H_0}
\newcommand{\Kfrak}{K_{\text{frac}
\newcommand{\Kfrak}{K_{\text{frak}
\newcommand{\Kspec}{K_{\text{spec}
\newcommand{\LCDM}{\Lambda\text{CDM}
\newcommand{\LPlanck}{\ell_{\text{Pl}
\newcommand{\Lag}{\mathcal{L}
\newcommand{\Lambdat}{\Lambda_T}
\newcommand{\Leff}{L_{\text{eff}
\newcommand{\Lorentz}[2]{{\Lambda^\mu{}
\newcommand{\Lp}{L_{\text{P}
\newcommand{\Lxi}{L_\xi}
\newcommand{\Lzero}{L_0}
\newcommand{\MPl}{M_{\text{Pl}
\newcommand{\MSbar}{\overline{\text{MS}
\newcommand{\MeV}{\,\text{MeV}
\newcommand{\Mpl}{M_{\text{Pl}
\newcommand{\OmegaDM}{\Omega_{\text{DM}
\newcommand{\OmegaLambda}{\Omega_{\Lambda}
\newcommand{\Omegab}{\Omega_b}
\newcommand{\Phiphoton}{\Phi_{\text{photon}
\newcommand{\Ricci}{R_{\mu\nu}
\newcommand{\Riem}{R^\rho{}
\newcommand{\Rzero}{R_\infty}
\newcommand{\Scal}{R}
\newcommand{\SynchPower}{P_{\text{synch}
\newcommand{\TPlanck}{t_{\text{Pl}
\newcommand{\Tfieldt}{T(\vec{x}
\newcommand{\Tfieldt}{T(x,t)}
\newcommand{\Tfield}{T(x)}
\newcommand{\Tfield}{T(x,t)}
\newcommand{\Tfield}{T_{\text{field}
\newcommand{\Tfield}{T}
\newcommand{\Tfield}{\mathcal{T}
\newcommand{\Tzerot}{T_0(\Tfield)}
\newcommand{\Tzero}{T_0}
\newcommand{\Weyl}{C^\rho{}
\newcommand{\ZPinch}{J \times B = \nabla p}
\newcommand{\aleph}{\aleph}
\newcommand{\alphaEMSI}{\alpha_{\text{EM,SI}
\newcommand{\alphaEMnat}{\alpha_{\text{EM,nat}
\newcommand{\alphaEM}{\alpha_{\text{EM}
\newcommand{\alphaEM}{\ensuremath{\alpha_{\text{EM}
\newcommand{\alphaQCD}{\alpha_s}
\newcommand{\alphaQED}{\alpha_{\text{QED}
\newcommand{\alphaSI}{\alpha_{\text{SI}
\newcommand{\alphaT}{\alpha_{\text{T}
\newcommand{\alphaWSI}{\alpha_{\text{W,SI}
\newcommand{\alphaWnat}{\alpha_{\text{W,nat}
\newcommand{\alphaW}{\alpha_{\text{W}
\newcommand{\alphaem}{\alpha_{EM}
\newcommand{\alphaem}{\alpha}
\newcommand{\alphafine}{\alpha}
\newcommand{\alphagem}{\alpha}
\newcommand{\alphanat}{\alpha_{\text{nat}
\newcommand{\alphapar}{\alpha}
\newcommand{\betaTSI}{\beta_{\text{T,SI}
\newcommand{\betaTnat}{\beta_{\text{T,nat}
\newcommand{\betaT}{\beta_T}
\newcommand{\betaT}{\beta_{T}
\newcommand{\betaT}{\beta_{\text{T}
\newcommand{\betaT}{\ensuremath{\beta_T}
\newcommand{\betapar}{\beta}
\newcommand{\calL}{\mathcal{L}
\newcommand{\checked}{\checkmark}
\newcommand{\checkmarkx}{\checkmark}
\newcommand{\dTdt}{\frac{d\Tfieldt}
\newcommand{\deltaE}{\delta E}
\newcommand{\deltafield}{\ensuremath{\delta m}
\newcommand{\deltam}{\delta m}
\newcommand{\deq}{\displaystyle}
\newcommand{\docref}[1]{\texttt{#1}
\newcommand{\eV}{\,\text{eV}
\newcommand{\epsilonT}{\varepsilon_T}
\newcommand{\epsilonzero}{\varepsilon_0}
\newcommand{\etavis}{\eta_{\text{visual}
\newcommand{\e}{\mathrm{e}
\newcommand{\gW}{g_W}
\newcommand{\gammaf}{\gamma_{\text{Lorentz}
\newcommand{\gammamu}{\gamma^\mu}
\newcommand{\gs}{g_s}
\newcommand{\inftytext}{$\infty$}
\newcommand{\interval}[2]{#1:#2}
\newcommand{\kfrac}{K_{\text{frak}
\newcommand{\lP}{\ell_{\text{P}
\newcommand{\lP}{l_P}
\newcommand{\lambdah}{\ensuremath{\lambda_h}
\newcommand{\lambdah}{\lambda_h}
\newcommand{\lambdazero}{\lambda_0}
\newcommand{\mP}{m_{\text{P}
\newcommand{\mfield}{m(x,t)}
\newcommand{\mfield}{m}
\newcommand{\mh}{m_h}
\newcommand{\micrometer}{\ensuremath{\mu}
\newcommand{\mikrometer}{\ensuremath{\mu}
\newcommand{\myRightarrow}{\ensuremath{\Rightarrow}
\newcommand{\myapprox}{\ensuremath{\approx}
\newcommand{\myomega}{\ensuremath{\omega}
\newcommand{\myphi}{\ensuremath{\phi}
\newcommand{\mypi}{\ensuremath{\pi}
\newcommand{\mypropto}{\ensuremath{\propto}
\newcommand{\myrightarrow}{\ensuremath{\rightarrow}
\newcommand{\mysim}{\ensuremath{\sim}
\newcommand{\mysqrt}{\ensuremath{\sqrt}
\newcommand{\mytimes}{\ensuremath{\times}
\newcommand{\natunits}{\hbar = c = G = k_B = 1}
\newcommand{\natunits}{\text{(nat. Einh.)}
\newcommand{\natunits}{\text{(nat. units)}
\newcommand{\nulep}{\nu}
\newcommand{\nuzero}{\nu_0}
\newcommand{\partialop}{\ensuremath{\partial}
\newcommand{\pdTdt}{\frac{\partial\Tfieldt}
\newcommand{\pdTdx}{\nabla\Tfieldt}
\newcommand{\phiT}{\phi}
\newcommand{\pichar}{\pi}
\newcommand{\primrel}[1]{\mathbf{#1}
\newcommand{\rhoCMB}{\rho_{\text{CMB}
\newcommand{\rhoCasimir}{\rho_{\text{Casimir}
\newcommand{\rhoE}{\rho_E}
\newcommand{\rhofield}{\ensuremath{\rho}
\newcommand{\rzero}{r_0}
\newcommand{\slashk}{\cancel{k}
\newcommand{\slashp}{\cancel{p}
\newcommand{\slashq}{\cancel{q}
\newcommand{\tP}{t_P}
\newcommand{\tP}{t_{\text{P}
\newcommand{\tablescale}{0.9}
\newcommand{\tzero}{t_0}
\newcommand{\vect}[1]{\boldsymbol{#1}
\newcommand{\vecx}{\vec{x}
\newcommand{\vh}{v}
\newcommand{\vr}{\vec{r}
\newcommand{\warningx}{\color{red}
\newcommand{\warningx}{\textbf{!}
\newcommand{\warningx}{{\color{red}
\newcommand{\xiT}{\xi}
\newcommand{\xiconst}{\xi = \frac{4}
\newcommand{\xicoupling}{f(E/\Exi)}
\newcommand{\xigeom}{\xi_{\text{geom}
\newcommand{\xigeom}{\xi}
\newcommand{\xikonst}{\xi = \frac{4}
\newcommand{\xiparticle}{\xi_{\text{particle}
\newcommand{\xipar}{\ensuremath{\xi}
\newcommand{\xipar}{\xi_0}
\newcommand{\xipar}{\xi}
\newcommand{\xirat}{\xi_{\text{ratio}
\newtheorem{axiom}{Axiom}
\newtheorem{category}{Category-Theoretic Basis}
\newtheorem{category}{Kategorientheoretische Basis}
\newtheorem{corollary}[theorem]{Corollary}
\newtheorem{corollary}[theorem]{Korollar}
\newtheorem{corollary}{Corollary}
\newtheorem{corollary}{Korollar}
\newtheorem{definition}[theorem]{Definition}
\newtheorem{definition}{Definition}
\newtheorem{discovery}{Discovery}
\newtheorem{discovery}{Neue Entdeckung}
\newtheorem{discovery}{New Discovery}
\newtheorem{discovery}{Revolutionary Discovery}
\newtheorem{entdeckung}{Entdeckung}
\newtheorem{entdeckung}{Revolutionäre Entdeckung}
\newtheorem{erkenntnis}{Erkenntnis}
\newtheorem{erkenntnis}{Schlüsselerkenntnis}
\newtheorem{example}[theorem]{Beispiel}
\newtheorem{example}[theorem]{Example}
\newtheorem{example}{Beispiel}
\newtheorem{example}{Example}
\newtheorem{insight}{Central Insight}
\newtheorem{insight}{Insight}
\newtheorem{insight}{Key Insight}
\newtheorem{insight}{Wichtige Einsicht}
\newtheorem{insight}{Zentrale Einsicht}
\newtheorem{lemma}[theorem]{Lemma}
\newtheorem{lemma}{Lemma}
\newtheorem{principle}{Fundamental Principle}
\newtheorem{principle}{Fundamentales Prinzip}
\newtheorem{principle}{Grundlegendes Prinzip}
\newtheorem{principle}{Principle}
\newtheorem{principle}{Prinzip}
\newtheorem{prinzip}{Grundprinzip}
\newtheorem{proof_step}{Beweisschritt}
\newtheorem{proof_step}{Proof Step}
\newtheorem{proposition}[theorem]{Proposition}
\newtheorem{proposition}{Proposition}
\newtheorem{remark}[theorem]{Bemerkung}
\newtheorem{remark}[theorem]{Remark}
\newtheorem{theorem}{Theorem}
\newtheorem{warning}[theorem]{Warning}
\newtheorem{warning}[theorem]{Warnung}
\newunicodechar{±}{\ensuremath{\pm}
\newunicodechar{×}{\ensuremath{\times}
\newunicodechar{÷}{\ensuremath{\div}
\newunicodechar{ħ}{\ensuremath{\hbar}
\newunicodechar{Α}{\ensuremath{A}
\newunicodechar{Β}{\ensuremath{B}
\newunicodechar{Γ}{\ensuremath{\Gamma}
\newunicodechar{Δ}{\ensuremath{\Delta}
\newunicodechar{Ε}{\ensuremath{E}
\newunicodechar{Ζ}{\ensuremath{Z}
\newunicodechar{Η}{\ensuremath{H}
\newunicodechar{Θ}{\ensuremath{\Theta}
\newunicodechar{Ι}{\ensuremath{I}
\newunicodechar{Κ}{\ensuremath{K}
\newunicodechar{Λ}{\ensuremath{\Lambda}
\newunicodechar{Μ}{\ensuremath{M}
\newunicodechar{Ν}{\ensuremath{N}
\newunicodechar{Ξ}{\ensuremath{\Xi}
\newunicodechar{Ο}{\ensuremath{O}
\newunicodechar{Π}{\ensuremath{\Pi}
\newunicodechar{Ρ}{\ensuremath{P}
\newunicodechar{Σ}{\ensuremath{\Sigma}
\newunicodechar{Τ}{\ensuremath{T}
\newunicodechar{Υ}{\ensuremath{\Upsilon}
\newunicodechar{Φ}{\ensuremath{\Phi}
\newunicodechar{Χ}{\ensuremath{X}
\newunicodechar{Ψ}{\ensuremath{\Psi}
\newunicodechar{Ω}{\ensuremath{\Omega}
\newunicodechar{α}{\ensuremath{\alpha}
\newunicodechar{β}{\ensuremath{\beta}
\newunicodechar{γ}{\ensuremath{\gamma}
\newunicodechar{δ}{\ensuremath{\delta}
\newunicodechar{ε}{\ensuremath{\varepsilon}
\newunicodechar{ζ}{\ensuremath{\zeta}
\newunicodechar{η}{\ensuremath{\eta}
\newunicodechar{θ}{\ensuremath{\theta}
\newunicodechar{ι}{\ensuremath{\iota}
\newunicodechar{κ}{\ensuremath{\kappa}
\newunicodechar{λ}{\ensuremath{\lambda}
\newunicodechar{μ}{\ensuremath{\mu}
\newunicodechar{ν}{\ensuremath{\nu}
\newunicodechar{ξ}{\ensuremath{\xi}
\newunicodechar{ο}{\ensuremath{o}
\newunicodechar{π}{\ensuremath{\pi}
\newunicodechar{ρ}{\ensuremath{\rho}
\newunicodechar{σ}{\ensuremath{\sigma}
\newunicodechar{τ}{\ensuremath{\tau}
\newunicodechar{υ}{\ensuremath{\upsilon}
\newunicodechar{φ}{\ensuremath{\phi}
\newunicodechar{φ}{\ensuremath{\varphi}
\newunicodechar{χ}{\ensuremath{\chi}
\newunicodechar{ψ}{\ensuremath{\psi}
\newunicodechar{ω}{\ensuremath{\omega}
\newunicodechar{←}{\ensuremath{\leftarrow}
\newunicodechar{→}{\ensuremath{\rightarrow}
\newunicodechar{↔}{\ensuremath{\leftrightarrow}
\newunicodechar{⇐}{\ensuremath{\Leftarrow}
\newunicodechar{⇒}{\ensuremath{\Rightarrow}
\newunicodechar{⇔}{\ensuremath{\Leftrightarrow}
\newunicodechar{∂}{\ensuremath{\partial}
\newunicodechar{∅}{\ensuremath{\emptyset}
\newunicodechar{∇}{\ensuremath{\nabla}
\newunicodechar{∈}{\ensuremath{\in}
\newunicodechar{∉}{\ensuremath{\notin}
\newunicodechar{∏}{\ensuremath{\prod}
\newunicodechar{∑}{\ensuremath{\sum}
\newunicodechar{√}{\ensuremath{\sqrt}
\newunicodechar{∝}{\ensuremath{\propto}
\newunicodechar{∞}{\ensuremath{\infty}
\newunicodechar{∩}{\ensuremath{\cap}
\newunicodechar{∪}{\ensuremath{\cup}
\newunicodechar{∫}{\ensuremath{\int}
\newunicodechar{≈}{\ensuremath{\approx}
\newunicodechar{≠}{\ensuremath{\neq}
\newunicodechar{≤}{\ensuremath{\leq}
\newunicodechar{≥}{\ensuremath{\geq}
\newunicodechar{★}{\ensuremath{\star}
\newunicodechar{✓}{\checkmark}
\pgfplotsset{compat=1.17}
\pgfplotsset{compat=1.18}
\renewcommand{\cftchapfont}{\large\bfseries\color{blue}
\renewcommand{\cftchappagefont}{\large\bfseries\color{blue}
\renewcommand{\cftsecfont}{\bfseries}
\renewcommand{\cftsecfont}{\color{blue}
\renewcommand{\cftsecfont}{\large\bfseries\color{blue}
\renewcommand{\cftsecpagefont}{\bfseries}
\renewcommand{\cftsecpagefont}{\color{blue}
\renewcommand{\cftsecpagefont}{\large\bfseries\color{blue}
\renewcommand{\cftsubsecfont}{\color{blue!80!black}
\renewcommand{\cftsubsecfont}{\color{blue}
\renewcommand{\cftsubsecpagefont}{\color{blue!80!black}
\renewcommand{\cftsubsecpagefont}{\color{blue}
\renewcommand{\cftsubsubsecfont}{\color{blue!60!black}
\renewcommand{\cftsubsubsecfont}{\color{blue}
\renewcommand{\cftsubsubsecpagefont}{\color{blue!60!black}
\renewcommand{\cftsubsubsecpagefont}{\color{blue}
\renewcommand{\cfttoctitlefont}{\huge\bfseries\color{blue}
\renewcommand{\cfttoctitlefont}{\huge\bfseries}
\renewcommand{\familydefault}{\sfdefault}
\renewcommand{\footrulewidth}{0.4pt}
\renewcommand{\headrulewidth}{0.4pt}
\sisetup{locale = DE, group-separator = {.}
\sisetup{locale = DE}
\usetikzlibrary{arrows.meta,positioning,shapes.geometric}
\usetikzlibrary{decorations.pathmorphing, patterns, shapes.arrows}
\usetikzlibrary{intersections}
\usetikzlibrary{positioning, arrows.meta}
\usetikzlibrary{positioning, arrows}
\usetikzlibrary{positioning, shapes.geometric, arrows.meta}
\usetikzlibrary{positioning,shapes,arrows}

% Common settings
\setlength{\headheight}{15pt}
\pgfplotsset{compat=1.18}
\usetikzlibrary{positioning,shapes,arrows,arrows.meta}

% Hyperref setup
\hypersetup{
    colorlinks=true,
    linkcolor=blue,
    citecolor=blue,
    urlcolor=blue
}


\title{RelokativesZahlensystemEn}
\author{Johann Pascher}
\date{\today}

\begin{document}

\maketitle
\tableofcontents

\begin{abstract}
		Prime numbers correspond to ratios in an alternative number system that is fundamentally more basic than our familiar set-based system. This document develops a relational number system in which prime numbers are defined as elementary, indivisible ratios or proportional transformations. By shifting the reference point from absolute quantities to pure relations, a system emerges that establishes multiplication as the primary operation and reflects the logarithmic structure of many natural laws.
	\end{abstract}
	
	\tableofcontents
	\newpage
	
	# List of Symbols and Notation
	
	{\small
		\begin{table}[htbp]
			\centering
			\begin{adjustbox}{width=0.98\textwidth}
				\begin{tabular}{lll}
					\toprule
					\textbf{Symbol} & \textbf{Meaning} & \textbf{Notes} \\
					\midrule
					\multicolumn{3}{c}{\textbf{Relational Basic Operations}} \\
					$\primrel{1}$ & Identity relation & $1:1$, starting point of all transformations \\
					$\primrel{2}$ & Doubling relation & $2:1$, elementary scaling \\
					$\primrel{3}$ & Fifth relation & $3:2$, musical fifth \\
					$\primrel{5}$ & Third relation & $5:4$, musical major third \\
					$\primrel{p}$ & Prime number relation & Elementary, indivisible proportion \\
					\midrule
					\multicolumn{3}{c}{\textbf{Interval Representation}} \\
					$I$ & Musical interval & As frequency ratio \\
					$\vect{v}$ & Exponent vector & $(a_1, a_2, a_3, \ldots)$ for $2^{a_1} \cdot 3^{a_2} \cdot 5^{a_3} \cdots$ \\
					$p_i$ & i-th prime number & $p_1=2, p_2=3, p_3=5, p_4=7, \ldots$ \\
					$a_i$ & Exponent of i-th prime & Integer, can be negative \\
					$n\text{-limit}$ & Prime number limitation & System with primes up to $n$ \\
					\midrule
					\multicolumn{3}{c}{\textbf{Operations}} \\
					$\circ$ & Composition of relations & Corresponds to multiplication \\
					$\oplus$ & Addition of exponent vectors & Logarithmic addition \\
					$\log$ & Logarithmic transformation & Multiplication $\to$ addition \\
					$\exp$ & Exponential function & Addition $\to$ multiplication \\
					\midrule
					\multicolumn{3}{c}{\textbf{Transformations}} \\
					$\text{FFT}$ & Fast Fourier Transform & Practical application \\
					$\text{QFT}$ & Quantum Fourier Transform & Quantum algorithm \\
					$\text{Shor}$ & Shor's Algorithm & Prime factorization \\
					\bottomrule
				\end{tabular}
			\end{adjustbox}
			\caption{Symbols and notation of the relational number system}
			\label{tab:symbole}
		\end{table}
	}
	
	\newpage
	
	# Introduction: Shifting the Reference Point
	
	The idea of shifting the reference point to construct a number system based on ratios while reinterpreting the role of prime numbers is the key to a more fundamental understanding of mathematics. \textbf{Prime numbers correspond to ratios in an alternative number system that is fundamentally more basic} than our familiar set-based system.
	
	## What does shifting the reference point mean?
	
	Previously, we have thought of the reference point (the denominator in a fraction like $P/X$) often as 1, representing a fixed, absolute unit. However, when we shift the reference point, we no longer think of absolute numerical values, but of \textbf{relational steps or transformations}.
	
	Imagine we define numbers not as three apples, but as the \textbf{relationship or operation} that transforms one quantity into another.
	
	# Music as a Model: Intervals as Operations
	
	In music, an interval (e.g., a fifth, $3/2$) is not just a static ratio, but an \textbf{operation} that transforms one tone into another. When you shift a tone up by a fifth, you multiply its frequency by $3/2$.
	
	## Musical Intervals as a Ratio System
	
	In just intonation, intervals are represented as ratios of whole numbers:
	
	\begin{table}[htbp]
		\centering
		\begin{adjustbox}{width=0.85\textwidth}
			\begin{tabular}{lccc}
				\toprule
				\textbf{Interval} & \textbf{Ratio} & \textbf{Prime Factor} & \textbf{Vector} \\
				\midrule
				Octave & $2:1$ & $2^1$ & $(1, 0, 0)$ \\
				Fifth & $3:2$ & $2^{-1} \cdot 3^1$ & $(-1, 1, 0)$ \\
				Fourth & $4:3$ & $2^2 \cdot 3^{-1}$ & $(2, -1, 0)$ \\
				Major third & $5:4$ & $2^{-2} \cdot 5^1$ & $(-2, 0, 1)$ \\
				Minor third & $6:5$ & $2^1 \cdot 3^1 \cdot 5^{-1}$ & $(1, 1, -1)$ \\
				\bottomrule
			\end{tabular}
		\end{adjustbox}
		\caption{Musical intervals in relational representation}
		\label{tab:intervalle}
	\end{table}
	
	These ratios can be written as \textbf{products of prime numbers with integer exponents}:
	
	
```math-equation

		\text{Interval} = 2^a \cdot 3^b \cdot 5^c \cdot 7^d \cdot \ldots
	
```

	
	Depending on how many prime numbers one allows (2, 3, 5 – or also 7, 11, 13 \ldots), one speaks of a \textbf{5-limit}, \textbf{7-limit} or \textbf{13-limit} system.
	
	\begin{example}[A major third]
		The major third ($5/4$) can be expressed as $2^{-2} \cdot 5^1$:
		
```math-align

			\frac{5}{4} &= 2^{-2} \cdot 5^1 \\
			\text{Exponent vector:} \quad &(-2, 0, 1) \text{ for } (2, 3, 5)
		
```

		
		Here this means:
		
			- $2^{-2}$: The prime number 2 appears twice in the denominator
			- $5^{+1}$: The prime number 5 appears once in the numerator
		
	\end{example}
	
	## Vector Representation of Intervals
	
	A useful representation is:
	
	\begin{definition}[Interval Vector]
		
```math-equation

			I = (a_1, a_2, a_3, \ldots) \text{ with } I = \prod_{i} p_i^{a_i}
		
```

		
		Where:
		
			- $p_i$: the $i$-th prime number $(2, 3, 5, 7, \ldots)$
			- $a_i$: integer exponent (can be negative)
		
	\end{definition}
	
	This allows a clear \textbf{algebraic structure} for intervals, including addition, inversion, etc. over the exponent vectors.
	
	## Application: Interval Multiplication = Exponent Addition
	
	\begin{example}[Major chord construction]
		A C major chord in the 5-limit system:
		
```math-align

			\text{C-E-G} &= \primrel{1} \circ \text{Major third} \circ \text{Fifth} \\
			&= (0,0,0) \oplus (-2,0,1) \oplus (-1,1,0) \\
			&= (-3,1,1) \\
			&= \frac{2^{-3} \cdot 3^1 \cdot 5^1}{1} = \frac{15}{8}
		
```

		This shows how complex harmonic structures emerge as compositions of elementary prime relations.
	\end{example}
	
	# Historical Precedents
	
	The relational number system stands in a long tradition of mathematical-philosophical approaches:
	
	
		- \textbf{Pythagorean harmony doctrine}: The Pythagoreans already recognized that \textit{Everything is number} -- understood as ratio, not as quantity
		- \textbf{Euler's Tonnetz} (1739): Prime number-based representation of musical intervals in a two-dimensional lattice
		- \textbf{Grassmann's Ausdehnungslehre} (1844): Multiplication as fundamental operation that creates new geometric objects
		- \textbf{Dedekind cuts} (1872): Numbers as relations between rational sets
	
	
	# Category-Theoretic Foundation
	
	\begin{category}
		The relational system can be interpreted as a free monoidal category, where:
		
			- \textbf{Objects} = ratio vectors $\vect{v} = (a_1, a_2, a_3, \ldots)$
			- \textbf{Morphisms} = proportional transformations between relations
			- \textbf{Tensor product} $\otimes$ = composition $\circ$ of relations
			- \textbf{Unit object} = identity relation $\primrel{1}$
		
		
		This structure makes explicit that the relational system has a natural category-theoretic interpretation.
	\end{category}
	
	# Prime Numbers as Elementary Relations
	
	If we transfer this musical approach to numbers, we can interpret prime numbers not as independent numbers, but as \textbf{fundamental, irreducible proportional steps or transformations}:
	
	## The Elementary Ratios
	
	\begin{definition}[Prime Number Relations]
		
```math-align

			\primrel{1}: \quad &\text{Identity relation } (1:1) \\
			&\text{The state of equality, starting point of all transformations} \\[0.5em]
			\primrel{2}: \quad &\text{Doubling relation } (2:1) \\
			&\text{The elementary gesture of doubling} \\[0.5em]
			\primrel{3}: \quad &\text{Fifth relation } (3:2) \\
			&\text{Fundamental proportional transformation} \\[0.5em]
			\primrel{5}: \quad &\text{Third relation } (5:4) \\
			&\text{Further elementary proportional transformation}
		
```

	\end{definition}
	
	## Numbers as Compositions of Ratios
	
	In a relational system, numbers would not be static quantities, but \textbf{compositions of ratios}:
	
	
		- \textbf{Starting point}: Base unit $(1:1)$
		- \textbf{Numbers as paths}: Each number is a path of operations
		
			- The number 2: Path of the $2:1$ operation
			- The number 3: Path of the $3:1$ operation  
			- The number 6: Path $2:1$ followed by $3:1$
			- The number 12: $2 \times 2 \times 3$ (three operations)
		
	
	
	# Axiomatic Foundations
	
	\begin{axiom}[Relational Arithmetic]
		For all relations $\primrel{a}, \primrel{b}, \primrel{c}$ in a relational number system:
		
			- \textbf{Associativity}: $(\primrel{a} \circ \primrel{b}) \circ \primrel{c} = \primrel{a} \circ (\primrel{b} \circ \primrel{c})$
			- \textbf{Neutral element}: $\exists \primrel{1} \forall \primrel{a}: \primrel{a} \circ \primrel{1} = \primrel{a}$
			- \textbf{Invertibility}: $\forall \primrel{a} \exists \primrel{a}^{-1}: \primrel{a} \circ \primrel{a}^{-1} = \primrel{1}$
			- \textbf{Commutativity}: $\primrel{a} \circ \primrel{b} = \primrel{b} \circ \primrel{a}$
		
	\end{axiom}
	
	These axioms establish the relational system as an abelian group under the composition operation $\circ$.
	
	# The Fundamental Difference: Addition vs. Multiplication
	
	## Addition: The Parts Continue to Exist
	
	When we add, we essentially bring things together that exist side by side or sequentially. The original components remain preserved in some way:
	
	
		- \textbf{Sets}: $2 + 3 = 5$ apples (original parts recognizable as subsets)
		- \textbf{Wave superposition}: Frequencies $f_1$ and $f_2$ are still detectable in the spectrum
		- \textbf{Forces}: Vector addition - both original forces are present
	
	
	## Multiplication: Something New Emerges
	
	With multiplication, something fundamentally different happens. This involves scaling, transformation, or the creation of a new quality:
	
	
		- \textbf{Area calculation}: $2m \times 3m = 6m^2$ (new dimension)
		- \textbf{Proportional change}: Doubling $\circ$ tripling = sixfolding
		- \textbf{Musical intervals}: Fifth $\times$ octave = new harmonic position
	
	
	# The Power of the Logarithm: Multiplication Becomes Addition
	
	The fact that taking logarithms turns multiplications into additions is fundamental:
	
	
```math-equation

		\log(A \times B) = \log(A) + \log(B)
	
```

	
	## What does logarithmization teach us?
	
	
		- \textbf{Scale transformation}: From proportional to linear scale
		- \textbf{Nature of perception}: Many sensory perceptions are logarithmic
		
			- \textbf{Hearing}: Frequency ratios as equal steps
			- \textbf{Light}: Logarithmic brightness perception
			- \textbf{Sound}: Decibel scale
		
		- \textbf{Physical systems}: Exponential growth becomes linear
		- \textbf{Unification}: Addition and multiplication are connected by transformation
	
	
	## Logarithmic Perception
	
	The nature of perception follows the Weber-Fechner law, which reflects the logarithmic structure of relational systems:
	
	\begin{figure}[htbp]
		\centering
		\begin{tikzpicture}[scale=0.8]
			\draw[->] (0,0) -- (6,0) node[right] {Stimulus intensity $I$};
			\draw[->] (0,0) -- (0,4) node[above] {Perception $W$};
			\draw[domain=0.1:5.5, smooth, blue, thick] plot (\x, {1.5*ln(\x + 0.5)});
			\node[blue] at (4,2.5) {$W = k \log(I/I_0)$};
			\node at (3,0.8) {\footnotesize Weber-Fechner law};
			\draw[dashed, gray] (1,0) -- (1,1.04);
			\draw[dashed, gray] (2,0) -- (2,1.66);
			\draw[dashed, gray] (4,0) -- (4,2.28);
			\node[below] at (1,0) {\footnotesize $I_1$};
			\node[below] at (2,0) {\footnotesize $2I_1$};
			\node[below] at (4,0) {\footnotesize $4I_1$};
		\end{tikzpicture}
		\caption{Logarithmic perception corresponds to the structure of relational systems}
		\label{fig:logarithmische_wahrnehmung}
	\end{figure}
	
	# Physical Analogies and Applications
	
	## Renormalization Group Flow
	
	A remarkable parallel exists between relational composition and renormalization group flow in quantum field theory:
	
	
```math-equation

		\beta(g) = \mu\frac{dg}{d\mu} = \sum_{k=1}^n \primrel{p_k} \circ \log\left(\frac{E}{E_0}\right)
	
```

	
	Here the energy scaling corresponds to the composition of prime relations.
	
	## Quantum Entanglement and Relations
	
	\begin{table}[htbp]
		\centering
		\begin{adjustbox}{width=0.85\textwidth}
			\begin{tabular}{ll}
				\toprule
				\textbf{Relational System} & \textbf{Quantum Mechanics} \\
				\midrule
				Prime relation $\primrel{p}$ & Basis state $|p\rangle$ \\
				Composition $\circ$ & Tensor product $\otimes$ \\
				Vector addition $\oplus$ & Superposition principle \\
				Logarithmic structure & Phase relationships \\
				\bottomrule
			\end{tabular}
		\end{adjustbox}
		\caption{Structural analogies between relational and quantum systems}
		\label{tab:quantenanalogien}
	\end{table}
	
	# Additive and Multiplicative Modulation in Nature
	
	## Electromagnetism and Physics
	
	\begin{table}[htbp]
		\centering
		\begin{adjustbox}{width=0.9\textwidth}
			\begin{tabular}{lll}
				\toprule
				\textbf{Modulation} & \textbf{Description} & \textbf{Examples} \\
				\midrule
				Multiplicative (AM) & Proportional amplitude change & Amplitude modulation, scaling \\
				Additive (FM) & Superposition of frequencies & Frequency modulation, interference \\
				\bottomrule
			\end{tabular}
		\end{adjustbox}
		\caption{Modulation in physics and technology}
		\label{tab:modulation}
	\end{table}
	
	## Music and Acoustics
	
	
		- \textbf{Timbre}: Additive superposition of harmonic overtones with multiplicative frequency ratios
		- \textbf{Harmony}: Consonance through simple multiplicative ratios ($3:2$, $5:4$)
		- \textbf{Melody}: Multiplicative frequency steps in additive time sequence
	
	
	# The Elimination of Absolute Quantities
	
	A central feature of this system is that the concrete assignment to a quantity is not necessary in the fundamental definitions. \textbf{The assignment to a specific quantity can be omitted and only becomes important when these relational numbers are applied to real things.}
	
	\begin{definition}[Relational vs. Absolute Numbers]
		
			- \textbf{Fundamental level}: Numbers are abstract relationships
			- \textbf{Application level}: Measurement in concrete units (meters, kilograms, hertz)
			- \textbf{Natural units}: $E = m$ (energy-mass identity as pure relation)
		
	\end{definition}
	
	# FFT, QFT and Shor's Algorithm: Practical Applications
	
	These algorithms already use the relational principle:
	
	## Fast Fourier Transform (FFT)
	
	The FFT reduces complexity from $O(N^2)$ to $O(N \log N)$ through:
	
		- Decomposition of the DFT matrix into sparsely populated factors
		- Rader's algorithm for prime-sized transforms uses multiplicative groups
		- Works with frequency ratios instead of absolute values
	
	
	## Quantum Fourier Transform (QFT)
	
	
		- Quantum version of the classical DFT
		- Core component of Shor's algorithm
		- Works with exponential functions for period finding
	
	
	## Algorithmic Details: Shor's Algorithm
	
	\begin{algorithm}[htbp]
		\caption{Shor's Algorithm for Prime Factorization}
		\label{alg:shor}
		\begin{algorithmic}[1]
			\STATE \textbf{Input:} Odd composite number $N$
			\STATE \textbf{Output:} Non-trivial factor of $N$
			\STATE 
			\STATE Choose random $a$ with $1 < a < N$ and $\gcd(a,N) = 1$
			\STATE Use quantum computer for period finding:
			\STATE \quad Find period $r$ of function $f(x) = a^x \bmod N$
			\STATE \quad Use QFT for efficient computation
			\IF{$r$ is odd OR $a^{r/2} \equiv -1 \pmod{N}$}
			\STATE Go to step 4 (choose new $a$)
			\ENDIF
			\STATE Compute $d_1 = \gcd(a^{r/2} - 1, N)$
			\STATE Compute $d_2 = \gcd(a^{r/2} + 1, N)$
			\IF{$1 < d_1 < N$}
			\RETURN $d_1$
			\ELSIF{$1 < d_2 < N$}
			\RETURN $d_2$
			\ELSE
			\STATE Go to step 4
			\ENDIF
		\end{algorithmic}
	\end{algorithm}
	
	The key lies in period finding through QFT, which recognizes relational patterns in modular arithmetic.
	
	\begin{table}[htbp]
		\centering
		\begin{adjustbox}{width=0.85\textwidth}
			\begin{tabular}{llll}
				\toprule
				\textbf{Algorithm} & \textbf{Property} & \textbf{Complexity} & \textbf{Application} \\
				\midrule
				FFT & Ratios & $O(N \log N)$ & Signal processing \\
				QFT & Superposition & Polynomial & Quantum algorithms \\
				Shor & Period patterns & Polynomial & Cryptography \\
				\bottomrule
			\end{tabular}
		\end{adjustbox}
		\caption{Relational algorithms in practice}
		\label{tab:algorithmen}
	\end{table}
	
	# Mathematical Framework
	
	## Formal Definition of the Relational System
	
	\begin{theorem}[Relational Number System]
		A relational number system $\mathcal{R}$ is defined by:
		
			- A set of prime number relations $\{\primrel{p_1}, \primrel{p_2}, \ldots\}$
			- A composition operation $\circ$ (corresponds to multiplication)
			- A vector representation $\vect{v} = (a_1, a_2, \ldots)$ with $\prod_i p_i^{a_i}$
			- A logarithmic addition operation $\oplus$ on vectors
		
	\end{theorem}
	
	## Properties of the System
	
	
		- \textbf{Closure}: $\primrel{a} \circ \primrel{b} \in \mathcal{R}$
		- \textbf{Associativity}: $(\primrel{a} \circ \primrel{b}) \circ \primrel{c} = \primrel{a} \circ (\primrel{b} \circ \primrel{c})$
		- \textbf{Identity}: $\primrel{1}$ is neutral element
		- \textbf{Inverses}: Each relation $\primrel{a}$ has inverse $\primrel{a}^{-1}$
	
	
	# Advantages and Challenges
	
	## Advantages of the Relational System
	
	
		- \textbf{Fundamental nature}: Captures the essence of relationships
		- \textbf{Logarithmic harmony}: Compatible with natural laws
		- \textbf{Multiplicative primary operation}: Natural connection
		- \textbf{Practical application}: Already implemented in FFT/QFT/Shor
	
	
	## Challenges
	
	
		- \textbf{Addition}: Complex definition in purely relational spaces
		- \textbf{Intuition}: Unfamiliar for set-based thinking
		- \textbf{Practical implementation}: Requires new mathematical tools
	
	
	# Epistemological Implications
	
	The relational number system has profound philosophical consequences:
	
	
		- \textbf{Operationalism}: Numbers are defined by their transformative effects, not by static properties
		- \textbf{Process ontology}: Being is understood as a dynamic network of transformations
		- \textbf{Neo-Pythagoreanism}: Mathematical relations as fundamental substrate of reality
		- \textbf{Structuralism}: The structure of relationships is primary over \textit{objects}
	
	
	# Open Research Questions
	
	The relational number system opens various research directions:
	
	
		- \textbf{Canonical addition}: How can addition be naturally defined in the relational system without transitioning to logarithmic space?
		- \textbf{Topological structure}: Is there a natural topology on the space of prime relations?
		- \textbf{Non-commutative generalizations}: Can the system capture quantum groups and non-commutative structures?
		- \textbf{Algorithmic complexity}: Which computational problems become easier or harder in the relational system?
		- \textbf{Cognitive modeling}: How is relational thinking reflected in neural structures?
	
	
	# Conclusion
	
	The relational number system represents a paradigm shift: from "How much?" to "How does it relate?". 
	
	\textbf{Core insights}:
	
		- Prime numbers are elementary, indivisible ratios
		- Multiplication is the natural, primary operation
		- The system is intrinsically logarithmically structured
		- Practical applications already exist in computer science
		- Energy can serve as a universal relational dimension
	
	
	This framework offers both theoretical insights and practical tools for a deeper understanding of the mathematical structure of reality.
	
	# Appendix A: Practical Application - T0-Framework Factorization Tool
	
	This appendix shows a real implementation of the relational number system in a factorization tool that practically implements the theoretical concepts.
	
	## Adaptive Relational Parameter Scaling
	
	The T0-Framework implements adaptive ξ-parameters that follow the relational principle:
	
	\begin{algorithm}[htbp]
		\caption{Adaptive $\xi$-Parameters in the Relational System}
		\label{alg:adaptive_xi}
		\begin{algorithmic}[1]
			\STATE \textbf{function} adaptive\_xi\_for\_hardware(problem\_bits):
			\IF{problem\_bits $\leq$ 64}
			\STATE base\_xi = $1 \times 10^{-5}$ \COMMENT{Standard relations}
			\ELSIF{problem\_bits $\leq$ 256}
			\STATE base\_xi = $1 \times 10^{-6}$ \COMMENT{Reduced coupling}
			\ELSIF{problem\_bits $\leq$ 1024}
			\STATE base\_xi = $1 \times 10^{-7}$ \COMMENT{Minimal coupling}
			\ELSE
			\STATE base\_xi = $1 \times 10^{-8}$ \COMMENT{Extreme stability}
			\ENDIF
			\RETURN base\_xi $\times$ hardware\_factor
		\end{algorithmic}
	\end{algorithm}
	
	This scaling demonstrates the \textbf{relational principle}: The parameter $\xi$ is not set absolutely, but \textbf{relative to the problem size}.
	
	## Energy Field Relations instead of Absolute Values
	
	The T0-Framework defines physical constants relationally:
	
	
```math-align

		c^2 &= 1 + \xi \quad \text{(relational coupling)} \\
		\text{correction} &= 1 + \xi \quad \text{(adaptive correction factor)} \\
		E_{\text{corr}} &= \xi \cdot \frac{E_1 \cdot E_2}{r^2} \quad \text{(energy field ratio)}
	
```

	
	The wave velocity is defined \textbf{not as an absolute constant}, but as a \textbf{relation to $\xi$}.
	
	## Quantum Gates as Relational Transformations
	
	The implementation shows how quantum operations function as \textbf{compositions of ratios}:
	
	\begin{example}[T0-Hadamard Gate]
		
```math-align

			\text{correction} &= 1 + \xi \\
			E_{\text{out},0} &= \frac{E_0 + E_1}{\sqrt{2}} \cdot \text{correction} \\
			E_{\text{out},1} &= \frac{E_0 - E_1}{\sqrt{2}} \cdot \text{correction}
		
```

		
		The Hadamard gate uses \textbf{relational corrections} instead of fixed transformations.
	\end{example}
	
	\begin{example}[T0-CNOT Gate]
		\begin{algorithmic}[1]
			\IF{$|$control\_field$|$ > threshold}
			\STATE target\_out = $-$target\_field $\times$ correction
			\ELSE
			\STATE target\_out = target\_field $\times$ correction
			\ENDIF
		\end{algorithmic}
		
		The CNOT operation is based on \textbf{ratios and thresholds}, not on discrete states.
	\end{example}
	
	## Period Finding through Resonance Relations
	
	The heart of prime factorization uses \textbf{relational resonances}:
	
	
```math-align

		\omega &= \frac{2\pi}{r} \quad \text{(period frequency)} \\
		E_{\text{corr}} &= \xi \cdot \frac{E_1 \cdot E_2}{r^2} \quad \text{(energy field correlation)} \\
		\text{resonance}_{\text{base}} &= \exp\left(-\frac{(\omega - \pi)^2}{4|\xi|}\right) \\
		\text{resonance}_{\text{total}} &= \text{resonance}_{\text{base}} \cdot (1 + E_{\text{corr}})^{2.5}
	
```

	
	This implementation shows how \textbf{Shor's period finding} is replaced by \textbf{relational energy field correlations}.
	
	## Bell State Verification as Relational Consistency
	
	The tool implements Bell states with relational corrections:
	
	\begin{algorithm}[htbp]
		\caption{T0-Bell State Generation}
		\label{alg:bell_t0}
		\begin{algorithmic}[1]
			\STATE Start: $|00\rangle$
			\STATE correction = $1 + \xi$
			\STATE inv\_sqrt2 = $1/\sqrt{2}$
			\STATE 
			\COMMENT{Hadamard on first qubit}
			\STATE $E_{00} = 1.0 \times$ inv\_sqrt2 $\times$ correction
			\STATE $E_{10} = 1.0 \times$ inv\_sqrt2 $\times$ correction
			\STATE 
			\COMMENT{CNOT: $|10\rangle \to |11\rangle$}
			\STATE $E_{11} = E_{10} \times$ correction
			\STATE $E_{10} = 0$
			\STATE 
			\COMMENT{Final result: $(|00\rangle + |11\rangle)/\sqrt{2}$ with ξ-correction}
			\RETURN $\{P(00), P(01), P(10), P(11)\}$
		\end{algorithmic}
	\end{algorithm}
	
	## Empirical Validation of Relational Theory
	
	The tool conducts \textbf{ablation studies} that confirm the relational principle:
	
	\begin{table}[htbp]
		\centering
		\begin{adjustbox}{width=0.9\textwidth}
			\begin{tabular}{lccc}
				\toprule
				\textbf{$\xi$-Parameter} & \textbf{Success Rate} & \textbf{Average Time} & \textbf{Stability} \\
				\midrule
				$\xi = 1 \times 10^{-5}$ (relational) & 100\% & 1.2s & Stable up to 64-bit \\
				$\xi = 1.33 \times 10^{-4}$ (absolute) & 95\% & 1.8s & Unstable at >32-bit \\
				$\xi = 1 \times 10^{-4}$ (absolute) & 90\% & 2.1s & Overflow problems \\
				$\xi = 5 \times 10^{-5}$ (absolute) & 98\% & 1.4s & Good but not optimal \\
				\bottomrule
			\end{tabular}
		\end{adjustbox}
		\caption{Empirical validation: Relational vs. absolute $\xi$-parameters}
		\label{tab:xi_validation}
	\end{table}
	
	The results show: \textbf{Relational parameters} (that adapt to problem size) are \textbf{significantly more effective} than absolute constants.
	
	## Implementation Code Examples
	
	### Relational Parameter Adaptation
	\begin{verbatim}
		def adaptive_xi_for_hardware(self, hardware_type: str = "standard") -> float:
		# Adaptive xi-scaling based on problem size
		if self.rsa_bits <= 64:
		base_xi = 1e-5  # Optimal for standard problems
		elif self.rsa_bits <= 256:
		base_xi = 1e-6  # Reduced coupling for medium sizes
		elif self.rsa_bits <= 1024:
		base_xi = 1e-7  # Minimal coupling for large problems
		else:
		base_xi = 1e-8  # Extremely reduced for stability
		
		hardware_factor = {"standard": 1.0, "gpu": 1.2, "quantum": 0.5}
		return base_xi * hardware_factor.get(hardware_type, 1.0)
	\end{verbatim}
	
	### Energy Field Relations
	\begin{verbatim}
		def solve_energy_field(self, x: np.ndarray, t: np.ndarray) -> np.ndarray:
		# T0-Framework: c² = 1 + xi (relational coupling)
		c_squared = 1.0 + abs(self.xi)  # NOT just xi!
		
		for i in range(2, len(t)):
		for j in range(1, len(x)-1):
		spatial_laplacian = (E[j+1,i-1] - 2\textit{E[j,i-1] + E[j-1,i-1]) / (dx}*2)
		# Wave equation with relational velocity
		E[j,i] = 2\textit{E[j,i-1] - E[j,i-2] + c_squared } (dt\textit{*2) } spatial_laplacian
	\end{verbatim}
	
	### Relational Quantum Gates
	\begin{verbatim}
		def hadamard_t0(self, E_field_0: float, E_field_1: float) -> Tuple[float, float]:
		xi = self.adaptive_xi_for_hardware()
		correction = 1 + xi  # Relational correction, not absolute
		inv_sqrt2 = 1 / math.sqrt(2)
		
		# Hadamard with relational xi-correction
		E_out_0 = (E_field_0 + E_field_1) \textit{ inv_sqrt2 } correction
		E_out_1 = (E_field_0 - E_field_1) \textit{ inv_sqrt2 } correction
		return (E_out_0, E_out_1)
	\end{verbatim}
	
	### Period Finding through Ratio Resonance
	\begin{verbatim}
		def quantum_period_finding(self, a: int) -> Optional[int]:
		for r in range(1, max_period):
		if self.mod_pow(a, r, self.rsa_N) == 1:
		omega = 2 * math.pi / r
		
		# Relational energy field correlation instead of absolute calculation
		E_corr = self.xi \textit{ (E1 } E2) / (r**2)
		base_resonance = math.exp(-((omega - math.pi)\textit{*2) / (4 } abs(self.xi)))
		
		# Resonance amplified by ratio correlations
		total_resonance = base_resonance \textit{ (1 + E_corr)}*2.5
	\end{verbatim}
	
	## Insights for the Relational Number System
	
	The T0-Framework implementation demonstrates several core principles of the relational number system:
	
	
		- \textbf{Adaptive parameters}: No universal constants, but context-sensitive relations
		- \textbf{Ratio-based operations}: All calculations use correction factors like $(1 + \xi)$
		- \textbf{Logarithmic scaling}: Parameters change exponentially with problem size
		- \textbf{Composition of relations}: Complex operations as concatenation of simple ratios
		- \textbf{Empirical validation}: Relational approaches measurably outperform absolute constants
	
	
	This implementation shows that the \textbf{relational number system is not only theoretically elegant}, but also \textbf{practically superior} for complex calculations like prime factorization.
	
	# Outlook
	
	## Future Research Directions
	
	
		- Development of a complete addition theory for relational numbers
		- Application to quantum field theory and string theory
		- Computer algebra systems for relational arithmetic
		- Pedagogical approaches for relational mathematics education
	
	
	## Potential Applications
	
	
		- New algorithms for prime factorization
		- Improved quantum computing protocols
		- Innovative approaches in music theory and acoustics
		- Fundamentally new perspectives in theoretical physics

\end{document}
