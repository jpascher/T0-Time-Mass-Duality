\documentclass[11pt,a4paper,openany]{book}

% Essential packages
\usepackage[utf8]{inputenc}
\usepackage[T1]{fontenc}
\usepackage[ngerman]{babel}
\usepackage[a4paper,margin=2.5cm]{geometry}
\usepackage{lmodern}

% Math and physics packages
\usepackage{amsmath}
\usepackage{amssymb}
\usepackage{amsthm}
\usepackage{mathtools}
\usepackage{physics}
\usepackage{siunitx}

% Graphics and tables
\usepackage{graphicx}
\usepackage[table,xcdraw]{xcolor}
\usepackage{tikz}
\usepackage{pgfplots}
\usepackage{tcolorbox}
\usepackage{booktabs}
\usepackage{array}
\usepackage{longtable}
\usepackage{float}

% Document formatting
\usepackage{fancyhdr}
\usepackage{tocloft}
\usepackage{hyperref}
\usepackage{cleveref}
\usepackage{microtype}
\usepackage{enumitem}
\usepackage{newunicodechar}

% Additional packages (cleaned up - removed duplicates)
\usepackage{adjustbox}
\usepackage{algorithm}
\usepackage{algorithmic}
\usepackage{amsfonts}
\usepackage{bm}
\usepackage{braket}
\usepackage{breakurl}
\usepackage{cancel}
\usepackage{caption}
\usepackage{cite}
\usepackage{csquotes}
\usepackage{doi}
\usepackage{forest}
\usepackage{gensymb}
\usepackage{hyphenat}
\usepackage{listings}
\usepackage{mdframed}
\usepackage{multicol}
\usepackage{multirow}
\usepackage{natbib}
\usepackage{pdflscape}
\usepackage{ragged2e}
\usepackage{setspace}
\usepackage{slashed}
\usepackage{tabularx}
\usepackage{textcomp}
\usepackage{textgreek}
\usepackage{upgreek}
\usepackage{url}

% Color definitions (FIXED: removed extra \definecolor commands)
\definecolor{blue}{rgb}{0,0,1}
\definecolor{boxgray}{RGB}{240,240,240}
\definecolor{deepblue}{RGB}{0,0,127}
\definecolor{deepgreen}{RGB}{0,127,0}
\definecolor{deepred}{RGB}{191,0,0}
\definecolor{t0blue}{RGB}{0,102,204}
\definecolor{t0green}{RGB}{0,153,0}
\definecolor{t0orange}{RGB}{255,152,0}
\definecolor{t0purple}{RGB}{102,0,204}
\definecolor{t0red}{RGB}{204,0,0}
\definecolor{t0yellow}{RGB}{255,204,0}

% TikZ libraries
\usetikzlibrary{arrows,shapes,positioning,calc,patterns,decorations.pathmorphing,decorations.markings}

% PGFPlots setup
\pgfplotsset{compat=1.18}

% Hyperref setup
\hypersetup{
    colorlinks=true,
    linkcolor=blue,
    filecolor=magenta,
    urlcolor=cyan,
    citecolor=green,
    pdftitle={T0 Theory Document},
    pdfauthor={Johann Pascher},
    pdfsubject={T0 Theory},
    pdfkeywords={T0, physics, theory}
}

% Header and footer
\pagestyle{fancy}
\fancyhf{}
\fancyhead[LE,RO]{\thepage}
\fancyhead[RE]{\leftmark}
\fancyhead[LO]{\rightmark}
\fancyfoot[C]{T0 Theory - Johann Pascher}

% Theorem environments
\theoremstyle{definition}
\newtheorem{definition}{Definition}[section]
\newtheorem{theorem}{Theorem}[section]
\newtheorem{lemma}[theorem]{Lemma}
\newtheorem{proposition}[theorem]{Proposition}
\newtheorem{corollary}[theorem]{Corollary}
\theoremstyle{remark}
\newtheorem{remark}{Remark}[section]
\newtheorem{example}{Example}[section]

% Custom commands (common across T0 documents)
\newcommand{\T}[1]{\text{#1}}
\newcommand{\mat}[1]{\mathbf{#1}}
\newcommand{\E}{\mathrm{e}}
\newcommand{\I}{\mathrm{i}}
\newcommand{\diff}{\mathrm{d}}
\newcommand{\Real}{\mathrm{Re}}
\newcommand{\Imag}{\mathrm{Im}}


\begin{document}

\maketitle
\tableofcontents

\begin{abstract}
		Dieses Dokument stellt die fundamentalen Prinzipien der T0-Theorie vor, einer geometrischen Reformulierung der Physik basierend auf einem einzigen universellen Parameter $\xipar = \frac{4}{3} \times 10^{-4}$. Die Theorie zeigt, wie alle fundamentalen Konstanten und Teilchenmassen aus der dreidimensionalen Raumgeometrie ableitbar sind. Dabei werden verschiedene Interpretationsansätze - harmonisch, geometrisch und feldtheoretisch - gleichberechtigt dargestellt. Die fraktale Struktur der Quantenraumzeit wird durch den Korrekturfaktor $\Kfrak = 0.986$ systematisch berücksichtigt.
	\end{abstract}
	
	\tableofcontents
	\newpage
	
	# Einführung in die T0-Theorie
		## Zeit-Masse-Dualitaet
	
	
	In natuerlichen Einheiten ($\hbar = c = 1$) gilt die fundamentale Beziehung:
	
```math-equation

		T \cdot m = 1
		\label{eq:time_mass_duality}
	
```

	Zeit und Masse sind dual zueinander verknuepft: Schwere Teilchen haben kurze charakteristische Zeitskalen, leichte Teilchen lange.
	## Die zentrale Hypothese
	
	Die T0-Theorie basiert auf der revolutionären Hypothese, dass alle physikalischen Phänomene aus der geometrischen Struktur des dreidimensionalen Raums ableitbar sind. Im Zentrum steht ein einziger universeller Parameter:
	
	\begin{foundation}
		\textbf{Der fundamentale geometrische Parameter:}
		
```math-equation

			\boxed{\xipar = \frac{4}{3} \times 10^{-4} = 1.333333\dots \times 10^{-4}}
			\label{eq:xi_fundamental}
		
```

		Dieser Parameter ist dimensionslos und enthält die gesamte Information über die physikalische Struktur des Universums.
	\end{foundation}
	
	## Paradigmenwechsel gegenüber dem Standardmodell
	
	\begin{table}[htbp]
		\centering
		\begin{tabular}{lcc}
			\toprule
			\textbf{Aspekt} & \textbf{Standardmodell} & \textbf{T0-Theorie} \\
			\midrule
			Freie Parameter & $> 20$ & $1$ \\
			Theoretische Basis & Empirische Anpassung & Geometrische Ableitung \\
			Teilchenmassen & Willkürlich & Aus Quantenzahlen berechenbar \\
			Konstanten & Experimentell bestimmt & Geometrisch abgeleitet \\
			Vereinigung & Separate Theorien & Einheitlicher Rahmen \\
			\bottomrule
		\end{tabular}
		\caption{Vergleich zwischen Standardmodell und T0-Theorie}
	\end{table}
	
	# Der geometrische Parameter $\xipar$
	
	## Mathematische Struktur
	
	Der Parameter $\xipar$ setzt sich aus zwei fundamentalen Komponenten zusammen:
	
	
```math-equation

		\xipar = \underbrace{\frac{4}{3}}_{\text{Harmonisch-geometrisch}} \times \underbrace{10^{-4}}_{\text{Skalenhierarchie}}
		\label{eq:xi_components}
	
```

	
	## Die harmonisch-geometrische Komponente: 4/3
	
	\begin{alternative}
		\textbf{Harmonische Interpretation:}
		
		Der Faktor $\frac{4}{3}$ entspricht dem \textbf{perfekten Quart}, einem der fundamentalen harmonischen Intervalle:
		
			- \textbf{Oktave:} 2:1 (immer universell)
			- \textbf{Quinte:} 3:2 (immer universell)  
			- \textbf{Quarte:} 4:3 (immer universell!)
		
		
		Diese Verhältnisse sind \textbf{geometrisch/mathematisch}, nicht materialabhängig. Der Raum selbst hat eine harmonische Struktur, und 4/3 (die Quarte) ist seine fundamentale Signatur.
	\end{alternative}
	
	\begin{alternative}
		\textbf{Geometrische Interpretation:}
		
		Der Faktor $\frac{4}{3}$ ergibt sich aus der tetraedrischen Packungsstruktur des dreidimensionalen Raums:
		
			- \textbf{Tetraeder-Volumen:} $V = \frac{\sqrt{2}}{12}a^3$
			- \textbf{Kugel-Volumen:} $V = \frac{4\pi}{3}r^3$ 
			- \textbf{Packungsdichte:} $\eta = \frac{\pi}{3\sqrt{2}} \approx 0.74$
			- \textbf{Geometrisches Verhältnis:} $\frac{4}{3}$ aus der optimalen Raumaufteilung
		
	\end{alternative}
	
	## Die Skalenhierarchie: $10^{-4$}
	
	\begin{foundation}
		\textbf{Quantenfeldtheoretische Herleitung von $10^{-4}$:}
		
		Der Faktor $10^{-4}$ entsteht durch die Kombination von:
		
		\textbf{1. Loop-Suppression (Quantenfeldtheorie):}
		
```math-equation

			\frac{1}{16\pi^3} = 2.01 \times 10^{-3}
		
```

		
		\textbf{2. T0-Higgs-Parameter:}
		
```math-equation

			(\lambda_h^{(T0)})^2 \frac{(v^{(T0)})^2}{(m_h^{(T0)})^2} = 0.0647
		
```

		
		\textbf{3. Vollständige Berechnung:}
		
```math-equation

			2.01 \times 10^{-3} \times 0.0647 = 1.30 \times 10^{-4}
		
```

		
		Also: \textbf{QFT Loop-Suppression} ($\sim 10^{-3}$) $\times$ \textbf{T0 Higgs-Sektor} ($\sim 10^{-1}$) = $10^{-4}$
	\end{foundation}
	
	# Fraktale Raumzeitstruktur
	
	## Quantenraumzeit-Effekte
	
	Die T0-Theorie erkennt an, dass die Raumzeit auf Planck-Skalen aufgrund von Quantenfluktuationen eine fraktale Struktur aufweist:
	
	\begin{keyresult}
		\textbf{Fraktale Raumzeit-Parameter:}
		
```math-align

			\Dfrak &= 2.94 \quad \text{(effektive fraktale Dimension)} \\
			\Kfrak &= 1 - \frac{\Dfrak - 2}{68} = 1 - \frac{0.94}{68} = 0.986
		
```

		
		\textbf{Physikalische Interpretation:}
		
			- $\Dfrak < 3$: Raumzeit ist auf kleinsten Skalen ''porös''
			- $\Kfrak = 0.986 < 1$: Reduzierte effektive Interaktionsstärke
			- Die Konstante 68 ergibt sich aus der tetraedralen Symmetrie des 3D-Raums
			- Quantenfluktuationen und Vakuumstruktur-Effekte
		
	\end{keyresult}
	
	## Ursprung der Konstante 68
	
	\begin{alternative}
		\textbf{Tetraeder-Geometrie:}
		
		Alle Tetraeder-Kombinationen ergeben 72:
		
```math-align

			6 \times 12 &= 72 \quad \text{(Kanten $\times$ Rotationen)} \\
			4 \times 18 &= 72 \quad \text{(Flächen $\times$ 18)} \\
			24 \times 3 &= 72 \quad \text{(Symmetrien $\times$ Dimensionen)}
		
```

		
		Der Wert 68 = 72 - 4 berücksichtigt die 4 Eckpunkte des Tetraeders als Ausnahmen.
	\end{alternative}

Diese Dualitaet ist nicht nur eine mathematische Beziehung, sondern spiegelt eine fundamentale Eigenschaft der Raumzeit wider. Sie erklaert, warum schwere Teilchen staerker an die temporale Struktur der Raumzeit koppeln.
	
	# Charakteristische Energieskalen
	
	## Die T0-Energiehierarchie
	
	Aus dem Parameter $\xipar$ ergeben sich natürliche Energieskalen:
	
	
```math-align

		(E_0)_{\xipar} &= \frac{1}{\xipar} = 7500 \quad \text{(in natürlichen Einheiten)} \\
		(E_0)_{\text{EM}} &= 7.398\,\mathrm{MeV} \quad \text{(charakteristische EM-Energie)} \\
		(E_0)_{\text{char}} &= 28.4 \quad \text{(charakteristische T0-Energie)}
	
```

	
	## Die charakteristische elektromagnetische Energie
	
	\begin{keyresult}
		\textbf{Gravitativ-geometrische Herleitung von $E_0$:}
		
		Die charakteristische Energie folgt aus der Kopplungsbeziehung:
		
```math-equation

			E_0^2 = \frac{4\sqrt{2} \cdot m_\mu}{\xipar^4}
		
```

		
		Dies ergibt $E_0 = 7.398$ MeV als fundamentale elektromagnetische Energieskala.
	\end{keyresult}
	
	\begin{alternative}
		\textbf{Geometrisches Mittel der Leptonmassen:}
		
		Alternativ kann $E_0$ als geometrisches Mittel definiert werden:
		
```math-equation

			E_0 = \sqrt{m_e \cdot m_\mu} = 7.35\,\mathrm{MeV}
		
```

		
		Die Differenz zu 7.398 MeV (< 1\%) ist durch Quantenkorrekturen erklärbar.
	\end{alternative}
	
	# Dimensionsanalytische Grundlagen
	
	## Natürliche Einheiten
	
	Die T0-Theorie arbeitet in natürlichen Einheiten, wobei:
	
	
```math-align

		\hbar = c = 1 \quad \text{(Konvention)}
	
```

	
	In diesem System haben alle Größen Energie-Dimension oder sind dimensionslos:
	
	
```math-align

		[M] &= [E] \quad \text{(aus $E = mc^2$ mit $c = 1$)} \\
		[L] &= [E^{-1}] \quad \text{(aus $\lambda = \hbar/p$ mit $\hbar = 1$)} \\
		[T] &= [E^{-1}] \quad \text{(aus $\omega = E/\hbar$ mit $\hbar = 1$)}
	
```

	
	## Umrechnungsfaktoren
	
	\begin{warning}
		\textbf{Kritische Bedeutung von Umrechnungsfaktoren:}
		
		Für experimentellen Vergleich sind Umrechnungsfaktoren von natürlichen zu SI-Einheiten essentiell:
		
			- Diese sind \textbf{nicht} willkürlich, sondern folgen aus fundamentalen Konstanten
			- Sie kodieren die Verbindung zwischen geometrischer Theorie und messbaren Größen
			- Beispiel: $C_{\text{conv}} = 7.783 \times 10^{-3}$ für die Gravitationskonstante $G$ in $\si{m^3 kg^{-1} s^{-2}}$
		
	\end{warning}
	
	# Die universelle T0-Formelstruktur
	
	## Grundmuster der T0-Beziehungen
	
	Alle T0-Formeln folgen dem universellen Muster:
	
	
```math-equation

		\boxed{\text{Physikalische Größe} = f(\xipar, \text{Quantenzahlen}) \times \text{Umrechnungsfaktor}}
		\label{eq:universal_pattern}
	
```

	
	wobei:
	
		- $f(\xipar, \text{Quantenzahlen})$ die geometrische Beziehung kodiert
		- Quantenzahlen $(n,l,j)$ die spezifische Konfiguration bestimmen
		- Umrechnungsfaktoren die Verbindung zu SI-Einheiten herstellen
	
	
	## Beispiele der universellen Struktur
	
	
```math-align

		\text{Gravitationskonstante:} \quad G &= \frac{\xipar^2}{4m_e} \times C_{\text{conv}} \times \Kfrak \\
		\text{Teilchenmassen:} \quad m_i &= \frac{\Kfrak}{\xipar \cdot f(n_i,l_i,j_i)} \times C_{\text{conv}} \\
		\text{Feinstrukturkonstante:} \quad \alpha &= \xipar \times \left(\frac{E_0}{1\,\mathrm{MeV}}\right)^2
	
```

	
	# Verschiedene Interpretationsebenen
	
	## Hierarchie der Verständnisebenen
	
	\begin{foundation}
		\textbf{Die T0-Theorie kann auf verschiedenen Ebenen verstanden werden:}
		
		\textbf{1. Phänomenologische Ebene:}
		
			- Empirische Beobachtung: Eine Konstante erklärt alles
			- Praktische Anwendung: Vorhersage neuer Werte
		
		
		\textbf{2. Geometrische Ebene:}
		
			- Raumstruktur bestimmt physikalische Eigenschaften
			- Tetraedrische Packung als Grundprinzip
		
		
		\textbf{3. Harmonische Ebene:}
		
			- Raumzeit als harmonisches System
			- Teilchen als ''Töne'' in kosmischer Harmonie
		
		
		\textbf{4. Quantenfeldtheoretische Ebene:}
		
			- Loop-Suppressionen und Higgs-Mechanismus
			- Fraktale Korrekturen als Quanteneffekte
		
	\end{foundation}
	
	## Komplementäre Sichtweisen
	
	\begin{alternative}
		\textbf{Reduktionistische vs. holistische Sichtweise:}
		
		\textbf{Reduktionistisch:}
		
			- $\xipar$ als empirischer Parameter, der ''zufällig'' funktioniert
			- Geometrische Interpretationen als nachträglich hinzugefügt
		
		
		\textbf{Holistisch:}
		
			- Raum-Zeit-Materie als untrennbare Einheit
			- $\xipar$ als Ausdruck einer tieferen kosmischen Ordnung
		
	\end{alternative}
	

	# Grundlegende Berechnungsmethoden

\section{Direkte geometrische Methode}

Die einfachste Anwendung der T0-Theorie verwendet direkte geometrische Beziehungen:

```math-equation

	\text{Physikalische Groesse} = \text{Geometrischer Faktor} \times \xi^n \times \text{Normierung}
	\label{eq:direct_method}

```

wobei der Exponent $n$ aus der Dimensionsanalyse folgt und der geometrische Faktor rationale Zahlen wie $\frac{4}{3}$, $\frac{16}{5}$, etc. enthaelt.

\section{Erweiterte Yukawa-Methode}

Fuer Teilchenmassen wird zusaetzlich der Higgs-Mechanismus beruecksichtigt:

```math-equation

	m_i = y_i \cdot v
	\label{eq:yukawa_method}

```

wobei die Yukawa-Kopplungen $y_i$ geometrisch aus der T0-Struktur berechnet werden:

```math-equation

	y_i = r_i \times \xi^{p_i}
	\label{eq:yukawa_coupling}

```

Die Parameter $r_i$ und $p_i$ sind exakte rationale Zahlen, die aus der Quantenzahlen-Zuordnung der T0-Geometrie folgen.
	# Philosophische Implikationen
	
	## Das Problem der Natürlichkeit
	
	\begin{foundation}
		\textbf{Warum ist das Universum mathematisch beschreibbar?}
		
		Die T0-Theorie bietet eine mögliche Antwort: Das Universum ist mathematisch beschreibbar, weil es \textbf{selbst} mathematisch strukturiert ist. Der Parameter $\xipar$ ist nicht nur eine Beschreibung der Natur - er \textbf{ist} die Natur.
		
		
			- \textbf{Platonische Sichtweise:} Mathematische Strukturen sind fundamental
			- \textbf{Pythagoräische Sichtweise:} ''Alles ist Zahl und Harmonie''
			- \textbf{Moderne Interpretation:} Geometrie als Grundlage der Physik
		
	\end{foundation}
	
	## Das anthropische Prinzip
	
	\begin{alternative}
		\textbf{Schwaches vs. starkes anthropisches Prinzip:}
		
		\textbf{Schwach (beobachtungsbedingt):}
		
			- Wir beobachten $\xipar = \frac{4}{3} \times 10^{-4}$, weil nur in einem solchen Universum Beobachter existieren können
			- Multiversum mit verschiedenen $\xipar$-Werten
		
		
		\textbf{Stark (prinzipiell):}
		
			- $\xipar$ hat diesen Wert, \textbf{weil} er aus der Logik der Raumzeit folgt
			- Nur dieser Wert ist mathematisch konsistent
		
	\end{alternative}

	

		
	# Experimentelle Bestaetigung

\section{Erfolgreiche Vorhersagen}

Die T0-Theorie hat bereits mehrere experimentelle Tests bestanden:

\section{Testbare Vorhersagen}

\begin{keyresult}[Konkrete T0-Vorhersagen]
	Die Theorie macht spezifische, falsifizierbare Vorhersagen:
	
		- Neutrino-Masse: $m_\nu = 4{,}54$ meV (geometrische Vorhersage)
		- Tau-Anomalie: $\Delta a_\tau = 7{,}1 \times 10^{-9}$ (noch nicht messbar)
		- Modifizierte Gravitation bei charakteristischen T0-Laengenskalen
		- Alternative kosmologische Parameter ohne dunkle Energie
	
\end{keyresult}
	# Zusammenfassung und Ausblick
	
	## Die zentralen Erkenntnisse
	
	\begin{foundation}
		\textbf{Fundamentale T0-Prinzipien:}
		
		
			- \textbf{Geometrische Einheit:} Ein Parameter $\xipar = \frac{4}{3} \times 10^{-4}$ bestimmt alle Physik
			- \textbf{Fraktale Struktur:} Quantenraumzeit mit $D_f = 2.94$ und $K_{\text{frak}} = 0.986$
			- \textbf{Harmonische Ordnung:} 4/3 als fundamentales harmonisches Verhältnis
			- \textbf{Hierarchische Skalen:} Von Planck- bis kosmologischen Dimensionen
			- \textbf{Experimentelle Testbarkeit:} Konkrete, falsifizierbare Vorhersagen
		
	\end{foundation}
	
		
	## Die nächsten Schritte
	
	Dieses erste Dokument der T0-Serie hat die fundamentalen Prinzipien etabliert. Die folgenden Dokumente werden diese Grundlagen in spezifischen Anwendungen vertiefen:
	
	# Struktur der T0-Dokumentenserie

Dieses Grundlagendokument bildet den Ausgangspunkt einer systematischen Darstellung der T0-Theorie. Die folgenden Dokumente vertiefen spezielle Aspekte:

	- \textbf{T0\_Feinstruktur\_De.tex}: Mathematische Herleitung der Feinstrukturkonstante
	- \textbf{T0\_Gravitationskonstante\_De.tex}: Detaillierte Berechnung der Gravitation
	- \textbf{T0\_Teilchenmassen\_De.tex}: Systematische Massenberechnung aller Fermionen
	- \textbf{T0\_Neutrinos\_De.tex}: Spezialbehandlung der Neutrino-Physik
	- \textbf{T0\_Anomale\_Magnetische\_Momente\_De.tex}: Loesung der Myon g-2 Anomalie
	- \textbf{T0\_Kosmologie\_De.tex}: Kosmologische Anwendungen der T0-Theorie

Jedes Dokument baut auf den hier etablierten Grundprinzipien auf und zeigt deren Anwendung in einem spezifischen Bereich der Physik.
\chapter{Struktur der T0-Dokumentenserie}

Dieses Grundlagendokument bildet den Ausgangspunkt einer systematischen Darstellung der T0-Theorie. Die folgenden Dokumente vertiefen spezielle Aspekte:

	- \textbf{T0\_Feinstruktur\_De.tex}: Mathematische Herleitung der Feinstrukturkonstante
	- \textbf{T0\_Gravitationskonstante\_De.tex}: Detaillierte Berechnung der Gravitation
	- \textbf{T0\_Teilchenmassen\_De.tex}: Systematische Massenberechnung aller Fermionen
	- \textbf{T0\_Neutrinos\_De.tex}: Spezialbehandlung der Neutrino-Physik
	- \textbf{T0\_Anomale\_Magnetische\_Momente\_De.tex}: Lösung der Myon g-2 Anomalie
	- \textbf{T0\_Kosmologie\_De.tex}: Kosmologische Anwendungen der T0-Theorie
	- \textbf{T0\_QM-QFT-RT\_De.tex}: Vollständige Quantenfeldtheorie im T0-Framework mit Quantenmechanik und Quantencomputer-Anwendungen

	# Literaturverweise
	
	## Grundlegende T0-Dokumente
	
	
		- Pascher, J. (2025). \textit{T0-Theorie: Herleitung der Gravitationskonstanten}. Technische Dokumentation.
		- Pascher, J. (2025). \textit{T0-Modell: Parameterfreie Partikelmasseberechnung mit fraktalen Korrekturen}. Wissenschaftliche Abhandlung.
		- Pascher, J. (2025). \textit{T0-Modell: Einheitliche Neutrino-Formel-Struktur}. Spezielle Analyse.
	
	
	## Verwandte Arbeiten
	
	
		- Einstein, A. (1915). \textit{Die Feldgleichungen der Gravitation}. Sitzungsberichte der K\''oniglich Preussischen Akademie der Wissenschaften.
		- Planck, M. (1900). \textit{Zur Theorie des Gesetzes der Energieverteilung im Normalspektrum}. Verhandlungen der Deutschen Physikalischen Gesellschaft.
		- Wheeler, J.A. (1989). \textit{Information, physics, quantum: The search for links}. Proceedings of the 3rd International Symposium on Foundations of Quantum Mechanics.
	
	
	\begin{center}
		\hrule
		\vspace{0.5cm}
		\textit{Dieses Dokument ist Teil der neuen T0-Serie}\\
		\textit{und ersetzt die \''alteren, inkonsistenten Darstellungen}\\
		\vspace{0.3cm}
		\textbf{T0-Theorie: Zeit-Masse-Dualit\''at Framework}\\
		\textit{Johann Pascher, HTL Leonding, {\''O}sterreich}\\
	\end{center}

\end{document}
