\documentclass[11pt,a4paper,openany]{book}

% Essential packages
\usepackage[utf8]{inputenc}
\usepackage[T1]{fontenc}
\usepackage[english]{babel}
\usepackage[a4paper,margin=2.5cm]{geometry}
\usepackage{lmodern}

% Math and physics packages
\usepackage{amsmath}
\usepackage{amssymb}
\usepackage{amsthm}
\usepackage{mathtools}
\usepackage{physics}
\usepackage{siunitx}

% Graphics and tables
\usepackage{graphicx}
\usepackage[table,xcdraw]{xcolor}
\usepackage{tikz}
\usepackage{pgfplots}
\usepackage{tcolorbox}
\usepackage{booktabs}
\usepackage{array}
\usepackage{longtable}
\usepackage{float}

% Document formatting
\usepackage{fancyhdr}
\usepackage{tocloft}
\usepackage{hyperref}
\usepackage{cleveref}
\usepackage{microtype}
\usepackage{enumitem}
\usepackage{newunicodechar}

% Additional packages
\usepackage{adjustbox}
\usepackage{algorithm}
\usepackage{algorithmic}
\usepackage{amsfonts}
\usepackage{amsmath,amsfonts,amssymb}
\usepackage{amsmath,amsfonts,amssymb,physics}
\usepackage{amsmath,amssymb}
\usepackage{amsmath,amssymb,amsfonts,amsthm}
\usepackage{amsmath,amssymb,amsthm}
\usepackage{amsmath,amssymb,physics,graphicx,xcolor,amsthm}
\usepackage{bm}
\usepackage{booktabs,array,longtable,multirow}
\usepackage{braket}
\usepackage{breakurl}
\usepackage{cancel}
\usepackage{caption}
\usepackage{cite}
\usepackage{color}
\usepackage{colortbl}
\usepackage{csquotes}
\usepackage{doi}
\usepackage{forest}
\usepackage{gensymb}
\usepackage{geometry,fancyhdr}
\usepackage{graphicx,tikz,pgfplots}
\usepackage{hyperref,url}
\usepackage{hyphenat}
\usepackage{listings}
\usepackage{listings,enumerate}
\usepackage{mdframed}
\usepackage{multicol}
\usepackage{multirow}
\usepackage{natbib}
\usepackage{pdflscape}
\usepackage{ragged2e}
\usepackage{setspace}
\usepackage{siunitx,xcolor,graphicx}
\usepackage{slashed}
\usepackage{tabularx}
\usepackage{textcomp}
\usepackage{textgreek}
\usepackage{tikz,pgfplots}
\usepackage{upgreek}
\usepackage{url}

% Custom commands and definitions
\definecolor{blue}
\definecolor{blue}{rgb}{0,0,1}
\definecolor{boxgray}
\definecolor{boxgray}{RGB}{240,240,240}
\definecolor{deepblue}
\definecolor{deepblue}{RGB}{0,0,127}
\definecolor{deepgreen}
\definecolor{deepgreen}{RGB}{0,127,0}
\definecolor{deepred}
\definecolor{deepred}{RGB}{191,0,0}
\definecolor{t0blue}
\definecolor{t0blue}{RGB}{0,102,204}
\definecolor{t0blue}{RGB}{33,150,243}
\definecolor{t0green}
\definecolor{t0green}{RGB}{0,153,0}
\definecolor{t0green}{RGB}{0,153,76}
\definecolor{t0green}{RGB}{76,175,80}
\definecolor{t0orange}
\definecolor{t0orange}{RGB}{255,152,0}
\definecolor{t0purple}
\definecolor{t0purple}{RGB}{102,0,204}
\definecolor{t0purple}{RGB}{156,39,176}
\definecolor{t0red}
\definecolor{t0red}{RGB}{204,0,0}
\definecolor{t0red}{RGB}{204,0,51}
\definecolor{t0red}{RGB}{244,67,54}
\definecolor{t0yellow}
\definecolor{t0yellow}{RGB}{255,204,0}
\geometry{a4paper, left=25mm, right=25mm, top=25mm, bottom=25mm}
\geometry{a4paper, margin=1in}
\geometry{a4paper, margin=2.5cm}
\geometry{a4paper, margin=2cm}
\geometry{left=2.5cm,right=2.5cm,top=2.5cm,bottom=2.5cm}
\geometry{left=2cm,right=2cm,top=2cm,bottom=2cm}
\geometry{margin=1in}
\geometry{margin=2.5cm}
\geometry{margin=2cm}
\hypersetup{
	colorlinks=true,
	linkcolor=blue,
	citecolor=blue,
	urlcolor=blue,
	pdftitle={Analysis and Implications of MNRAS Paper 544 for the T0-Theory}
\hypersetup{
	colorlinks=true,
	linkcolor=blue,
	citecolor=blue,
	urlcolor=blue,
	pdftitle={Beweis: Die Feinstrukturkonstante α = 1 in natürlichen Einheiten}
\hypersetup{
	colorlinks=true,
	linkcolor=blue,
	citecolor=blue,
	urlcolor=blue,
	pdftitle={Beweis: Die Koide-Formel enthält implizit $\xi$}
\hypersetup{
	colorlinks=true,
	linkcolor=blue,
	citecolor=blue,
	urlcolor=blue,
	pdftitle={Chinas Photonischer Quantenchip: 1000x-Speedup und T0-Integration}
\hypersetup{
	colorlinks=true,
	linkcolor=blue,
	citecolor=blue,
	urlcolor=blue,
	pdftitle={Complete Derivation of Higgs Mass and Wilson Coefficients}
\hypersetup{
	colorlinks=true,
	linkcolor=blue,
	citecolor=blue,
	urlcolor=blue,
	pdftitle={Complete Particle Spectrum: Standard Model vs T0 Theory}
\hypersetup{
	colorlinks=true,
	linkcolor=blue,
	citecolor=blue,
	urlcolor=blue,
	pdftitle={Conceptual Comparison of Unified Natural Units and Extended Standard Model}
\hypersetup{
	colorlinks=true,
	linkcolor=blue,
	citecolor=blue,
	urlcolor=blue,
	pdftitle={Connections between the Mizohata-Takeuchi Counterexample and the T0 Time-Mass Duality Theory}
\hypersetup{
	colorlinks=true,
	linkcolor=blue,
	citecolor=blue,
	urlcolor=blue,
	pdftitle={Das Relationale Zahlensystem: Primzahlen als fundamentale Verhältnisse}
\hypersetup{
	colorlinks=true,
	linkcolor=blue,
	citecolor=blue,
	urlcolor=blue,
	pdftitle={Das T0-Modell (Planck-Referenziert): Eine Neuformulierung der Physik}
\hypersetup{
	colorlinks=true,
	linkcolor=blue,
	citecolor=blue,
	urlcolor=blue,
	pdftitle={Das T0-Modell: Zeit-Energie-Dualität und geometrische Ruhemasse}
\hypersetup{
	colorlinks=true,
	linkcolor=blue,
	citecolor=blue,
	urlcolor=blue,
	pdftitle={Der Massenskalierungsexponent κ in der T0-Theorie}
\hypersetup{
	colorlinks=true,
	linkcolor=blue,
	citecolor=blue,
	urlcolor=blue,
	pdftitle={Der geometrische Formalismus der T0-Quantenmechanik und seine Anwendung auf Quantencomputer}
\hypersetup{
	colorlinks=true,
	linkcolor=blue,
	citecolor=blue,
	urlcolor=blue,
	pdftitle={Der xi Parameter und Teilchendifferenzierung in der T0-Theorie}
\hypersetup{
	colorlinks=true,
	linkcolor=blue,
	citecolor=blue,
	urlcolor=blue,
	pdftitle={Deterministic Quantum Mechanics via T0-Energy Field Formulation}
\hypersetup{
	colorlinks=true,
	linkcolor=blue,
	citecolor=blue,
	urlcolor=blue,
	pdftitle={Deterministische Quantenmechanik via T0-Energiefeld-Formulierung}
\hypersetup{
	colorlinks=true,
	linkcolor=blue,
	citecolor=blue,
	urlcolor=blue,
	pdftitle={Die Elektroneneinheitsladung in der T0-Theorie: Jenseits von Punkt-Singularitäten}
\hypersetup{
	colorlinks=true,
	linkcolor=blue,
	citecolor=blue,
	urlcolor=blue,
	pdftitle={Die Feinstrukturkonstante: Verschiedene Darstellungen und Beziehungen}
\hypersetup{
	colorlinks=true,
	linkcolor=blue,
	citecolor=blue,
	urlcolor=blue,
	pdftitle={Die Musikalische Spirale und die 137: Die mathematische Entdeckung der kosmischen Verstimmung}
\hypersetup{
	colorlinks=true,
	linkcolor=blue,
	citecolor=blue,
	urlcolor=blue,
	pdftitle={E=mc² = E=m: Die Konstanten-Illusion entlarvt}
\hypersetup{
	colorlinks=true,
	linkcolor=blue,
	citecolor=blue,
	urlcolor=blue,
	pdftitle={E=mc² = E=m: The Constants Illusion Exposed}
\hypersetup{
	colorlinks=true,
	linkcolor=blue,
	citecolor=blue,
	urlcolor=blue,
	pdftitle={Einfache Lagrange-Revolution: Von der Standardmodell-Komplexität zur T0-Eleganz}
\hypersetup{
	colorlinks=true,
	linkcolor=blue,
	citecolor=blue,
	urlcolor=blue,
	pdftitle={Einführung in die Umsetzung photonischer Bauteile auf Wafern für Nachrichtentechniker}
\hypersetup{
	colorlinks=true,
	linkcolor=blue,
	citecolor=blue,
	urlcolor=blue,
	pdftitle={Einführung in photonische Quantenchips für Nachrichtentechniker}
\hypersetup{
	colorlinks=true,
	linkcolor=blue,
	citecolor=blue,
	urlcolor=blue,
	pdftitle={Elimination der Masse als dimensionaler Platzhalter im T0-Modell}
\hypersetup{
	colorlinks=true,
	linkcolor=blue,
	citecolor=blue,
	urlcolor=blue,
	pdftitle={Elimination of Mass as Dimensional Placeholder in the T0 Model}
\hypersetup{
	colorlinks=true,
	linkcolor=blue,
	citecolor=blue,
	urlcolor=blue,
	pdftitle={Empirical Analysis of Deterministic Factorization Methods}
\hypersetup{
	colorlinks=true,
	linkcolor=blue,
	citecolor=blue,
	urlcolor=blue,
	pdftitle={Empirische Analyse deterministischer Faktorisierungsmethoden}
\hypersetup{
	colorlinks=true,
	linkcolor=blue,
	citecolor=blue,
	urlcolor=blue,
	pdftitle={Integration der Dirac-Gleichung im T0-Modell: Natürliche-Einheiten-Rahmenwerk}
\hypersetup{
	colorlinks=true,
	linkcolor=blue,
	citecolor=blue,
	urlcolor=blue,
	pdftitle={Integration of the Dirac Equation in the T0 Model: Natural Units Framework}
\hypersetup{
	colorlinks=true,
	linkcolor=blue,
	citecolor=blue,
	urlcolor=blue,
	pdftitle={Introduction to Photonic Quantum Chips for Communication Engineers}
\hypersetup{
	colorlinks=true,
	linkcolor=blue,
	citecolor=blue,
	urlcolor=blue,
	pdftitle={Introduction to the Implementation of Photonic Components on Wafers for Communication Engineers}
\hypersetup{
	colorlinks=true,
	linkcolor=blue,
	citecolor=blue,
	urlcolor=blue,
	pdftitle={Konzeptioneller Vergleich von Einheitlichen Natürlichen Einheiten und Erweitertem Standardmodell}
\hypersetup{
	colorlinks=true,
	linkcolor=blue,
	citecolor=blue,
	urlcolor=blue,
	pdftitle={Markov Chains in the Context of T0 Theory: Deterministic or Stochastic? A Treatise on Patterns, Preconditions, and Uncertainty}
\hypersetup{
	colorlinks=true,
	linkcolor=blue,
	citecolor=blue,
	urlcolor=blue,
	pdftitle={Markov-Ketten im Kontext der T0-Theorie: Deterministisch oder stochastisch? Ein Traktat zu Mustern, Voraussetzungen und Unsicherheit}
\hypersetup{
	colorlinks=true,
	linkcolor=blue,
	citecolor=blue,
	urlcolor=blue,
	pdftitle={Mathematical Analysis of T0-Shor Algorithm: Theoretical Framework and Computational Complexity}
\hypersetup{
	colorlinks=true,
	linkcolor=blue,
	citecolor=blue,
	urlcolor=blue,
	pdftitle={Mathematical Constructs of Alternative CMB Models: Unnikrishnan and Peratt in Harmony with the T0 Theory}
\hypersetup{
	colorlinks=true,
	linkcolor=blue,
	citecolor=blue,
	urlcolor=blue,
	pdftitle={Mathematische Analyse des T0-Shor Algorithmus: Theoretischer Rahmen und Berechnungskomplexität}
\hypersetup{
	colorlinks=true,
	linkcolor=blue,
	citecolor=blue,
	urlcolor=blue,
	pdftitle={Mathematische Konstrukte alternativer CMB-Modelle: Unnikrishnan und Peratt im Einklang mit der T0-Theorie}
\hypersetup{
	colorlinks=true,
	linkcolor=blue,
	citecolor=blue,
	urlcolor=blue,
	pdftitle={Natural Unit Systems: Universal Energy Conversion and Fundamental Length Scale Hierarchy}
\hypersetup{
	colorlinks=true,
	linkcolor=blue,
	citecolor=blue,
	urlcolor=blue,
	pdftitle={Natural Units in Theoretical Physics: A Treatise in the Context of T0 Theory}
\hypersetup{
	colorlinks=true,
	linkcolor=blue,
	citecolor=blue,
	urlcolor=blue,
	pdftitle={Natürliche Einheiten in der theoretischen Physik: Eine Abhandlung im Kontext der T0-Theorie}
\hypersetup{
	colorlinks=true,
	linkcolor=blue,
	citecolor=blue,
	urlcolor=blue,
	pdftitle={Natürliche Einheitensysteme: Universelle Energieumwandlung und fundamentale Längenskala-Hierarchie}
\hypersetup{
	colorlinks=true,
	linkcolor=blue,
	citecolor=blue,
	urlcolor=blue,
	pdftitle={Parameter System-Dependency in T0-Model: SI vs. Natural Units}
\hypersetup{
	colorlinks=true,
	linkcolor=blue,
	citecolor=blue,
	urlcolor=blue,
	pdftitle={Parameter-Systemabhängigkeit im T0-Modell: SI- vs. natürliche Einheiten}
\hypersetup{
	colorlinks=true,
	linkcolor=blue,
	citecolor=blue,
	urlcolor=blue,
	pdftitle={Proof: The Fine Structure Constant α = 1 in Natural Units}
\hypersetup{
	colorlinks=true,
	linkcolor=blue,
	citecolor=blue,
	urlcolor=blue,
	pdftitle={Proof: The Koide Formula Implicitly Contains $\xi$}
\hypersetup{
	colorlinks=true,
	linkcolor=blue,
	citecolor=blue,
	urlcolor=blue,
	pdftitle={Pure Energy T0 Theory: Ratio-Based Physics with SI Reference}
\hypersetup{
	colorlinks=true,
	linkcolor=blue,
	citecolor=blue,
	urlcolor=blue,
	pdftitle={Quantum Mechanics in the T0 Model: Field-Theoretic Foundations}
\hypersetup{
	colorlinks=true,
	linkcolor=blue,
	citecolor=blue,
	urlcolor=blue,
	pdftitle={Ratio-Based vs. Absolute: The Role of Fractal Correction in T0 Theory}
\hypersetup{
	colorlinks=true,
	linkcolor=blue,
	citecolor=blue,
	urlcolor=blue,
	pdftitle={Reine Energie T0-Theorie: Verhältnis-basierte Physik mit SI-Referenz}
\hypersetup{
	colorlinks=true,
	linkcolor=blue,
	citecolor=blue,
	urlcolor=blue,
	pdftitle={Simple Lagrangian Revolution: From Standard Model Complexity to T0 Elegance}
\hypersetup{
	colorlinks=true,
	linkcolor=blue,
	citecolor=blue,
	urlcolor=blue,
	pdftitle={Simplified Dirac Equation in T0 Theory: Field Node Approach}
\hypersetup{
	colorlinks=true,
	linkcolor=blue,
	citecolor=blue,
	urlcolor=blue,
	pdftitle={Simplified T0 Theory: Elegant Lagrangian Density for Time-Mass Duality}
\hypersetup{
	colorlinks=true,
	linkcolor=blue,
	citecolor=blue,
	urlcolor=blue,
	pdftitle={T0 Cosmology: Redshift as a Geometric Path Effect in a Static Universe}
\hypersetup{
	colorlinks=true,
	linkcolor=blue,
	citecolor=blue,
	urlcolor=blue,
	pdftitle={T0 Deterministic Quantum Computing: Complete Analysis of Important Algorithms}
\hypersetup{
	colorlinks=true,
	linkcolor=blue,
	citecolor=blue,
	urlcolor=blue,
	pdftitle={T0 Deterministisches Quantencomputing: Vollständige Analyse wichtiger Algorithmen}
\hypersetup{
	colorlinks=true,
	linkcolor=blue,
	citecolor=blue,
	urlcolor=blue,
	pdftitle={T0 Model: Complete Framework - From Time-Energy Duality to Universal Constants}
\hypersetup{
	colorlinks=true,
	linkcolor=blue,
	citecolor=blue,
	urlcolor=blue,
	pdftitle={T0 Model: Complete Parameter-Free Particle Mass Calculation}
\hypersetup{
	colorlinks=true,
	linkcolor=blue,
	citecolor=blue,
	urlcolor=blue,
	pdftitle={T0 Model: Unified Neutrino Formula Structure}
\hypersetup{
	colorlinks=true,
	linkcolor=blue,
	citecolor=blue,
	urlcolor=blue,
	pdftitle={T0 Model: Universal Energy Relations for Mol and Candela Units}
\hypersetup{
	colorlinks=true,
	linkcolor=blue,
	citecolor=blue,
	urlcolor=blue,
	pdftitle={T0 Modell: Vollständiges Framework - Von Zeit-Energie-Dualität zu universellen Konstanten}
\hypersetup{
	colorlinks=true,
	linkcolor=blue,
	citecolor=blue,
	urlcolor=blue,
	pdftitle={T0 Quantenfeldtheorie: QFT, QM und Quantencomputer}
\hypersetup{
	colorlinks=true,
	linkcolor=blue,
	citecolor=blue,
	urlcolor=blue,
	pdftitle={T0 Quantum Field Theory: QFT, QM and Quantum Computers}
\hypersetup{
	colorlinks=true,
	linkcolor=blue,
	citecolor=blue,
	urlcolor=blue,
	pdftitle={T0 Theory vs Bell's Theorem: How Deterministic Energy Fields Circumvent No-Go Theorems}
\hypersetup{
	colorlinks=true,
	linkcolor=blue,
	citecolor=blue,
	urlcolor=blue,
	pdftitle={T0 Theory: Final Extension to Hadrons - Physically Derived Corrections}
\hypersetup{
	colorlinks=true,
	linkcolor=blue,
	citecolor=blue,
	urlcolor=blue,
	pdftitle={T0 Theory: The Fine-Structure Constant}
\hypersetup{
	colorlinks=true,
	linkcolor=blue,
	citecolor=blue,
	urlcolor=blue,
	pdftitle={T0 Theory: The Gravitational Constant}
\hypersetup{
	colorlinks=true,
	linkcolor=blue,
	citecolor=blue,
	urlcolor=blue,
	pdftitle={T0-Kosmologie: Rotverschiebung als geometrischer Pfad-Effekt im statischen Universum}
\hypersetup{
	colorlinks=true,
	linkcolor=blue,
	citecolor=blue,
	urlcolor=blue,
	pdftitle={T0-Model: Complete Document Analysis and Structured Summary}
\hypersetup{
	colorlinks=true,
	linkcolor=blue,
	citecolor=blue,
	urlcolor=blue,
	pdftitle={T0-Model: Kinetic Energy of Electrons and Photons}
\hypersetup{
	colorlinks=true,
	linkcolor=blue,
	citecolor=blue,
	urlcolor=blue,
	pdftitle={T0-Model: The Hubble Parameter in Static Universe}
\hypersetup{
	colorlinks=true,
	linkcolor=blue,
	citecolor=blue,
	urlcolor=blue,
	pdftitle={T0-Modell-Verifikation: Skalen-Verhältnis-basierte Berechnungen}
\hypersetup{
	colorlinks=true,
	linkcolor=blue,
	citecolor=blue,
	urlcolor=blue,
	pdftitle={T0-Modell: Bewegungsenergie von Elektronen und Photonen}
\hypersetup{
	colorlinks=true,
	linkcolor=blue,
	citecolor=blue,
	urlcolor=blue,
	pdftitle={T0-Modell: Die Hubble-Konstante im statischen Universum}
\hypersetup{
	colorlinks=true,
	linkcolor=blue,
	citecolor=blue,
	urlcolor=blue,
	pdftitle={T0-Modell: Einheitliche Neutrino-Formel-Struktur}
\hypersetup{
	colorlinks=true,
	linkcolor=blue,
	citecolor=blue,
	urlcolor=blue,
	pdftitle={T0-Modell: Universelle Energiebeziehungen für Mol- und Candela-Einheiten}
\hypersetup{
	colorlinks=true,
	linkcolor=blue,
	citecolor=blue,
	urlcolor=blue,
	pdftitle={T0-Modell: Vollständige Dokumentenanalyse und strukturierte Zusammenfassung}
\hypersetup{
	colorlinks=true,
	linkcolor=blue,
	citecolor=blue,
	urlcolor=blue,
	pdftitle={T0-Modell: Vollständige parameterfreie Teilchenmassen-Berechnung}
\hypersetup{
	colorlinks=true,
	linkcolor=blue,
	citecolor=blue,
	urlcolor=blue,
	pdftitle={T0-QAT: $\xi$-Aware Quantization-Aware Training}
\hypersetup{
	colorlinks=true,
	linkcolor=blue,
	citecolor=blue,
	urlcolor=blue,
	pdftitle={T0-QFT ML Addendum: Machine Learning Derived Extensions}
\hypersetup{
	colorlinks=true,
	linkcolor=blue,
	citecolor=blue,
	urlcolor=blue,
	pdftitle={T0-QFT ML-Addendum: Maschinelle Lern-abgeleitete Erweiterungen}
\hypersetup{
	colorlinks=true,
	linkcolor=blue,
	citecolor=blue,
	urlcolor=blue,
	pdftitle={T0-Theorie vs Bells Theorem: Wie deterministische Energiefelder No-Go-Theoreme umgehen}
\hypersetup{
	colorlinks=true,
	linkcolor=blue,
	citecolor=blue,
	urlcolor=blue,
	pdftitle={T0-Theorie: Der Terrell-Penrose-Effekt und Massenvariation}
\hypersetup{
	colorlinks=true,
	linkcolor=blue,
	citecolor=blue,
	urlcolor=blue,
	pdftitle={T0-Theorie: Die Feinstrukturkonstante}
\hypersetup{
	colorlinks=true,
	linkcolor=blue,
	citecolor=blue,
	urlcolor=blue,
	pdftitle={T0-Theorie: Die Gravitationskonstante}
\hypersetup{
	colorlinks=true,
	linkcolor=blue,
	citecolor=blue,
	urlcolor=blue,
	pdftitle={T0-Theorie: Die T0-Zeit-Masse-Dualität}
\hypersetup{
	colorlinks=true,
	linkcolor=blue,
	citecolor=blue,
	urlcolor=blue,
	pdftitle={T0-Theorie: Die sieben Rätsel}
\hypersetup{
	colorlinks=true,
	linkcolor=blue,
	citecolor=blue,
	urlcolor=blue,
	pdftitle={T0-Theorie: Erweiterung auf Bell-Tests – ML-Simulationen (November 2025)}
\hypersetup{
	colorlinks=true,
	linkcolor=blue,
	citecolor=blue,
	urlcolor=blue,
	pdftitle={T0-Theorie: Finale Erweiterung auf Hadronen - Physikalisch abgeleitete Korrekturen}
\hypersetup{
	colorlinks=true,
	linkcolor=blue,
	citecolor=blue,
	urlcolor=blue,
	pdftitle={T0-Theorie: Finale Fraktale Massenformeln (November 2025)}
\hypersetup{
	colorlinks=true,
	linkcolor=blue,
	citecolor=blue,
	urlcolor=blue,
	pdftitle={T0-Theorie: Fraktaldimension aus Lepton-Massenverhältnis}
\hypersetup{
	colorlinks=true,
	linkcolor=blue,
	citecolor=blue,
	urlcolor=blue,
	pdftitle={T0-Theorie: Fundamentale Prinzipien}
\hypersetup{
	colorlinks=true,
	linkcolor=blue,
	citecolor=blue,
	urlcolor=blue,
	pdftitle={T0-Theorie: Herleitung der Gravitationskonstanten}
\hypersetup{
	colorlinks=true,
	linkcolor=blue,
	citecolor=blue,
	urlcolor=blue,
	pdftitle={T0-Theorie: Kosmische Beziehungen und universelle $\xi$-Konstante}
\hypersetup{
	colorlinks=true,
	linkcolor=blue,
	citecolor=blue,
	urlcolor=blue,
	pdftitle={T0-Theorie: Kosmologie}
\hypersetup{
	colorlinks=true,
	linkcolor=blue,
	citecolor=blue,
	urlcolor=blue,
	pdftitle={T0-Theorie: Netzwerkdarstellung und Dimensionsanalyse in der T0-Theorie}
\hypersetup{
	colorlinks=true,
	linkcolor=blue,
	citecolor=blue,
	urlcolor=blue,
	pdftitle={T0-Theorie: Teilchenmassen}
\hypersetup{
	colorlinks=true,
	linkcolor=blue,
	citecolor=blue,
	urlcolor=blue,
	pdftitle={T0-Theorie: Vollstaendiger Abschluss}
\hypersetup{
	colorlinks=true,
	linkcolor=blue,
	citecolor=blue,
	urlcolor=blue,
	pdftitle={T0-Theory: Complete Closure}
\hypersetup{
	colorlinks=true,
	linkcolor=blue,
	citecolor=blue,
	urlcolor=blue,
	pdftitle={T0-Theory: Complete Derivation of All Parameters Without Circularity}
\hypersetup{
	colorlinks=true,
	linkcolor=blue,
	citecolor=blue,
	urlcolor=blue,
	pdftitle={T0-Theory: Cosmic Relations and universal $\xi$-constant}
\hypersetup{
	colorlinks=true,
	linkcolor=blue,
	citecolor=blue,
	urlcolor=blue,
	pdftitle={T0-Theory: Cosmology}
\hypersetup{
	colorlinks=true,
	linkcolor=blue,
	citecolor=blue,
	urlcolor=blue,
	pdftitle={T0-Theory: Derivation of the Gravitational Constant}
\hypersetup{
	colorlinks=true,
	linkcolor=blue,
	citecolor=blue,
	urlcolor=blue,
	pdftitle={T0-Theory: Extension to Bell Tests – ML Simulations (November 2025)}
\hypersetup{
	colorlinks=true,
	linkcolor=blue,
	citecolor=blue,
	urlcolor=blue,
	pdftitle={T0-Theory: Final Fractal Mass Formulas (November 2025)}
\hypersetup{
	colorlinks=true,
	linkcolor=blue,
	citecolor=blue,
	urlcolor=blue,
	pdftitle={T0-Theory: Fractal Dimension from Lepton Mass Ratio}
\hypersetup{
	colorlinks=true,
	linkcolor=blue,
	citecolor=blue,
	urlcolor=blue,
	pdftitle={T0-Theory: Fundamental Principles}
\hypersetup{
	colorlinks=true,
	linkcolor=blue,
	citecolor=blue,
	urlcolor=blue,
	pdftitle={T0-Theory: Mass Variation as an Equivalent to Time Dilation}
\hypersetup{
	colorlinks=true,
	linkcolor=blue,
	citecolor=blue,
	urlcolor=blue,
	pdftitle={T0-Theory: Network Representation and Dimensional Analysis in the T0-Theory}
\hypersetup{
	colorlinks=true,
	linkcolor=blue,
	citecolor=blue,
	urlcolor=blue,
	pdftitle={T0-Theory: Neutrinos}
\hypersetup{
	colorlinks=true,
	linkcolor=blue,
	citecolor=blue,
	urlcolor=blue,
	pdftitle={T0-Theory: Particle Masses}
\hypersetup{
	colorlinks=true,
	linkcolor=blue,
	citecolor=blue,
	urlcolor=blue,
	pdftitle={T0-Theory: The Seven Riddles}
\hypersetup{
	colorlinks=true,
	linkcolor=blue,
	citecolor=blue,
	urlcolor=blue,
	pdftitle={T0-Theory: The T0-Time-Mass Duality}
\hypersetup{
	colorlinks=true,
	linkcolor=blue,
	citecolor=blue,
	urlcolor=blue,
	pdftitle={Temperature Units in Natural Units: T0-Theory}
\hypersetup{
	colorlinks=true,
	linkcolor=blue,
	citecolor=blue,
	urlcolor=blue,
	pdftitle={Temperatureinheiten in nat\"urlichen Einheiten: T0-Theorie}
\hypersetup{
	colorlinks=true,
	linkcolor=blue,
	citecolor=blue,
	urlcolor=blue,
	pdftitle={The Electron Unit Charge in T0 Theory: Beyond Point Singularities}
\hypersetup{
	colorlinks=true,
	linkcolor=blue,
	citecolor=blue,
	urlcolor=blue,
	pdftitle={The Fine Structure Constant: Various Representations and Relationships}
\hypersetup{
	colorlinks=true,
	linkcolor=blue,
	citecolor=blue,
	urlcolor=blue,
	pdftitle={The Geometric Formalism of T0 Quantum Mechanics and its Application to Quantum Computing}
\hypersetup{
	colorlinks=true,
	linkcolor=blue,
	citecolor=blue,
	urlcolor=blue,
	pdftitle={The Mass Scaling Exponent κ in T0 Theory}
\hypersetup{
	colorlinks=true,
	linkcolor=blue,
	citecolor=blue,
	urlcolor=blue,
	pdftitle={The Musical Spiral and 137: The Mathematical Discovery of Cosmic Detuning}
\hypersetup{
	colorlinks=true,
	linkcolor=blue,
	citecolor=blue,
	urlcolor=blue,
	pdftitle={The Relational Number System: Prime Numbers as Fundamental Ratios}
\hypersetup{
	colorlinks=true,
	linkcolor=blue,
	citecolor=blue,
	urlcolor=blue,
	pdftitle={The T0 Model (Planck-Referenced): A Reformulation of Physics}
\hypersetup{
	colorlinks=true,
	linkcolor=blue,
	citecolor=blue,
	urlcolor=blue,
	pdftitle={The T0 Model: Time-Energy Duality and Geometric Rest Mass}
\hypersetup{
	colorlinks=true,
	linkcolor=blue,
	citecolor=blue,
	urlcolor=blue,
	pdftitle={The T0-Model (Planck-Referenced): A Reformulation of Physics}
\hypersetup{
	colorlinks=true,
	linkcolor=blue,
	citecolor=blue,
	urlcolor=blue,
	pdftitle={Verbindungen zwischen dem Mizohata-Takeuchi-Gegenbeispiel und der T0-Zeit-Masse-Dualitätstheorie}
\hypersetup{
	colorlinks=true,
	linkcolor=blue,
	citecolor=blue,
	urlcolor=blue,
	pdftitle={Vereinfachte Dirac-Gleichung in der T0-Theorie: Feldknoten-Ansatz}
\hypersetup{
	colorlinks=true,
	linkcolor=blue,
	citecolor=blue,
	urlcolor=blue,
	pdftitle={Vereinfachte T0-Theorie: Elegante Lagrange-Dichte für Zeit-Masse-Dualität}
\hypersetup{
	colorlinks=true,
	linkcolor=blue,
	citecolor=blue,
	urlcolor=blue,
	pdftitle={Verhältnisbasiert vs. Absolut: Die Rolle der fraktalen Korrektur in der T0-Theorie}
\hypersetup{
	colorlinks=true,
	linkcolor=blue,
	citecolor=blue,
	urlcolor=blue,
	pdftitle={Vollständige Herleitung der Higgs-Masse und Wilson-Koeffizienten}
\hypersetup{
	colorlinks=true,
	linkcolor=blue,
	citecolor=blue,
	urlcolor=blue,
	pdftitle={Vollständiges Teilchenspektrum: Standard-Modell vs T0-Theorie}
\hypersetup{
	colorlinks=true,
	linkcolor=blue,
	citecolor=blue,
	urlcolor=blue,
	pdftitle={Warum Zahlenverhältnisse nicht direkt gekürzt werden dürfen}
\hypersetup{
	colorlinks=true,
	linkcolor=blue,
	citecolor=blue,
	urlcolor=blue,
	pdftitle={Why Numerical Ratios Must Not Be Directly Simplified}
\hypersetup{
	colorlinks=true,
	linkcolor=blue,
	citecolor=blue,
	urlcolor=blue,
}
\hypersetup{
	colorlinks=true,
	linkcolor=blue,
	citecolor=red,
	urlcolor=blue,
	bookmarks=true,
	bookmarksnumbered=true,
	pdfstartview=FitH,
	pdftitle={T0 Model - Field-Theoretic Derivation of the Beta Parameter}
\hypersetup{
	colorlinks=true,
	linkcolor=blue,
	citecolor=red,
	urlcolor=blue,
	bookmarks=true,
	bookmarksnumbered=true,
	pdfstartview=FitH,
	pdftitle={T0-Modell - Feldtheoretische Herleitung des Beta-Parameters}
\hypersetup{
	colorlinks=true,
	linkcolor=blue,
	filecolor=magenta,
	urlcolor=cyan,
}
\hypersetup{
	colorlinks=true,
	linkcolor=blue,
	urlcolor=blue,
	citecolor=blue,
	pdftitle={From Time Dilation to Mass Variation: Mathematical Core Formulations of Time-Mass Duality Theory - Updated Framework}
\hypersetup{
	colorlinks=true,
	linkcolor=blue,
	urlcolor=blue,
	citecolor=blue,
	pdftitle={T0 Model: Detailed Formula for Leptonic Anomalies}
\hypersetup{
	colorlinks=true,
	linkcolor=blue,
	urlcolor=blue,
	citecolor=blue,
	pdftitle={T0 Model: Detaillierte Formel für leptonische Anomalien}
\hypersetup{
	colorlinks=true,
	linkcolor=blue,
	urlcolor=blue,
	citecolor=blue,
	pdftitle={T0 Model: Energy-based Formulas with Quadratic Scaling}
\hypersetup{
	colorlinks=true,
	linkcolor=blue,
	urlcolor=blue,
	citecolor=blue,
	pdftitle={T0 Model: Granulation, Limits and Fundamental Asymmetry}
\hypersetup{
	colorlinks=true,
	linkcolor=blue,
	urlcolor=blue,
	citecolor=blue,
	pdftitle={T0-Modell: Energiebasierte Formeln mit quadratischer Skalierung}
\hypersetup{
	colorlinks=true,
	linkcolor=blue,
	urlcolor=blue,
	citecolor=blue,
	pdftitle={T0-Modell: Granulation, Limits und fundamentale Asymmetrie}
\hypersetup{
	colorlinks=true,
	linkcolor=blue,
	urlcolor=blue,
	citecolor=blue,
	pdftitle={Von Zeitdilatation zu Massenvariation: Mathematische Kernformulierungen der Zeit-Masse-Dualitätstheorie - Aktualisiertes Framework}
\hypersetup{
	colorlinks=true,
	linkcolor=t0blue,
	citecolor=t0blue,
	urlcolor=t0blue,
	pdftitle={T0 Model: Complete Theoretical Summary}
\hypersetup{
	colorlinks=true,
	linkcolor=t0blue,
	citecolor=t0blue,
	urlcolor=t0blue,
	pdftitle={T0 Theory: Resolution of Apparent Instantaneity}
\hypersetup{
	colorlinks=true,
	linkcolor=t0blue,
	citecolor=t0blue,
	urlcolor=t0blue,
	pdftitle={T0 vs Synergetics: Vereinfachung durch natürliche Einheiten}
\hypersetup{
	colorlinks=true,
	linkcolor=t0blue,
	citecolor=t0blue,
	urlcolor=t0blue,
	pdftitle={T0-Modell: Vollständige theoretische Zusammenfassung}
\hypersetup{
	colorlinks=true,
	linkcolor=t0blue,
	citecolor=t0blue,
	urlcolor=t0blue,
	pdftitle={T0-Theorie: Auflösung der scheinbaren Instantanität}
\hypersetup{
	colorlinks=true,
	linkcolor=t0blue,
	citecolor=t0blue,
	urlcolor=t0blue,
	pdftitle={T0-Theorie: Vollständige Dokumentenübersicht}
\hypersetup{
	colorlinks=true,
	linkcolor=t0blue,
	citecolor=t0blue,
	urlcolor=t0blue,
	pdftitle={T0-Theory: Complete Document Overview}
\hypersetup{
	colorlinks=true,
	linkcolor=t0blue,
	citecolor=t0blue,
	urlcolor=t0blue,
}
\hypersetup{
	colorlinks=true,
	linkcolor=t0blue,
	citecolor=t0green,
	urlcolor=t0blue,
	pdftitle={Das verborgene Geheimnis von 1/137}
\hypersetup{
	colorlinks=true,
	linkcolor=t0blue,
	citecolor=t0green,
	urlcolor=t0blue,
	pdftitle={The Hidden Secret of 1/137}
\hypersetup{
    colorlinks=true,
    linkcolor=blue,
    citecolor=blue,
    urlcolor=blue,
    pdftitle={Analyse und Implikationen des MNRAS-Papiers 544 für die T0-Theorie}
\hypersetup{
  colorlinks=true,
  linkcolor=blue,
  citecolor=blue,
  urlcolor=blue
}
\hypersetup{
  colorlinks=true,
  linkcolor=blue,
  citecolor=blue,
  urlcolor=blue,
  pdftitle={T0-Theorie: Ein-Uhr-Metrologie und Drei-Uhren-Experiment}
\hypersetup{
  colorlinks=true,
  linkcolor=blue,
  citecolor=blue,
  urlcolor=blue,
  pdftitle={T0-Theory: Single-Clock Metrology and Three-Clock Experiment}
\hypersetup{
colorlinks=true,
linkcolor=blue,
citecolor=blue,
urlcolor=blue,
pdftitle={Quantenmechanik im T0-Modell: Feldtheoretische Grundlagen}
\hypersetup{
colorlinks=true,
linkcolor=blue,
citecolor=blue,
urlcolor=blue,
pdftitle={T0-Theory: Neutrinos}
\newcommand{\Bzero}{B_0}
\newcommand{\CQCD}{C_{\text{QCD}
\newcommand{\Cconv}{C_{\text{conv}
\newcommand{\Cto}{C_{\text{T0}
\newcommand{\Czero}{C_0}
\newcommand{\DTmu}{D_{T,\mu}
\newcommand{\DcovT}[1]{\partial_\mu #1 + #1 \partial_\mu \Tfield}
\newcommand{\Dfrak}{D_f}
\newcommand{\Df}{D_f}
\newcommand{\DhiggsT}{\Tfield (\partial_\mu + ig A_\mu) \Phi + \Phi \partial_\mu \Tfield}
\newcommand{\EPlanck}{E_P}
\newcommand{\EPlanck}{E_{\text{Pl}
\newcommand{\EPratio}[1]{\frac{#1}
\newcommand{\EP}{E_P}
\newcommand{\EP}{E_{\text{P}
\newcommand{\EW}{E_W}
\newcommand{\EZ}{E_Z}
\newcommand{\Echar}{E_{\text{char}
\newcommand{\Ee}{E_e}
\newcommand{\Efield}{E(x,t)}
\newcommand{\Efield}{E_\text{field}
\newcommand{\Efield}{E_{\text{Feld}
\newcommand{\Efield}{E_{\text{Field}
\newcommand{\Efield}{E_{\text{field}
\newcommand{\Efield}{E}
\newcommand{\Egamma}{E_\gamma}
\newcommand{\Eh}{E_h}
\newcommand{\Emu}{E_\mu}
\newcommand{\Enorm}[1]{E_{\text{norm}
\newcommand{\En}{E_n}
\newcommand{\Ep}{E_p}
\newcommand{\Eratio}[2]{\frac{E_{#1}
\newcommand{\Etau}{E_\tau}
\newcommand{\Evis}{E_{\text{vis}
\newcommand{\Exi}{E_\xi}
\newcommand{\Ezero}{E_0}
\newcommand{\GeV}{\,\text{GeV}
\newcommand{\Gnat}{G_{\text{nat}
\newcommand{\Gsi}{G_{\text{SI}
\newcommand{\Hubble}{H_0}
\newcommand{\Kfrak}{K_{\text{frac}
\newcommand{\Kfrak}{K_{\text{frak}
\newcommand{\Kspec}{K_{\text{spec}
\newcommand{\LCDM}{\Lambda\text{CDM}
\newcommand{\LPlanck}{\ell_{\text{Pl}
\newcommand{\Lag}{\mathcal{L}
\newcommand{\Lambdat}{\Lambda_T}
\newcommand{\Leff}{L_{\text{eff}
\newcommand{\Lorentz}[2]{{\Lambda^\mu{}
\newcommand{\Lp}{L_{\text{P}
\newcommand{\Lxi}{L_\xi}
\newcommand{\Lzero}{L_0}
\newcommand{\MPl}{M_{\text{Pl}
\newcommand{\MSbar}{\overline{\text{MS}
\newcommand{\MeV}{\,\text{MeV}
\newcommand{\Mpl}{M_{\text{Pl}
\newcommand{\OmegaDM}{\Omega_{\text{DM}
\newcommand{\OmegaLambda}{\Omega_{\Lambda}
\newcommand{\Omegab}{\Omega_b}
\newcommand{\Phiphoton}{\Phi_{\text{photon}
\newcommand{\Ricci}{R_{\mu\nu}
\newcommand{\Riem}{R^\rho{}
\newcommand{\Rzero}{R_\infty}
\newcommand{\Scal}{R}
\newcommand{\SynchPower}{P_{\text{synch}
\newcommand{\TPlanck}{t_{\text{Pl}
\newcommand{\Tfieldt}{T(\vec{x}
\newcommand{\Tfieldt}{T(x,t)}
\newcommand{\Tfield}{T(x)}
\newcommand{\Tfield}{T(x,t)}
\newcommand{\Tfield}{T_{\text{field}
\newcommand{\Tfield}{T}
\newcommand{\Tfield}{\mathcal{T}
\newcommand{\Tzerot}{T_0(\Tfield)}
\newcommand{\Tzero}{T_0}
\newcommand{\Weyl}{C^\rho{}
\newcommand{\ZPinch}{J \times B = \nabla p}
\newcommand{\aleph}{\aleph}
\newcommand{\alphaEMSI}{\alpha_{\text{EM,SI}
\newcommand{\alphaEMnat}{\alpha_{\text{EM,nat}
\newcommand{\alphaEM}{\alpha_{\text{EM}
\newcommand{\alphaEM}{\ensuremath{\alpha_{\text{EM}
\newcommand{\alphaQCD}{\alpha_s}
\newcommand{\alphaQED}{\alpha_{\text{QED}
\newcommand{\alphaSI}{\alpha_{\text{SI}
\newcommand{\alphaT}{\alpha_{\text{T}
\newcommand{\alphaWSI}{\alpha_{\text{W,SI}
\newcommand{\alphaWnat}{\alpha_{\text{W,nat}
\newcommand{\alphaW}{\alpha_{\text{W}
\newcommand{\alphaem}{\alpha_{EM}
\newcommand{\alphaem}{\alpha}
\newcommand{\alphafine}{\alpha}
\newcommand{\alphagem}{\alpha}
\newcommand{\alphanat}{\alpha_{\text{nat}
\newcommand{\alphapar}{\alpha}
\newcommand{\betaTSI}{\beta_{\text{T,SI}
\newcommand{\betaTnat}{\beta_{\text{T,nat}
\newcommand{\betaT}{\beta_T}
\newcommand{\betaT}{\beta_{T}
\newcommand{\betaT}{\beta_{\text{T}
\newcommand{\betaT}{\ensuremath{\beta_T}
\newcommand{\betapar}{\beta}
\newcommand{\calL}{\mathcal{L}
\newcommand{\checked}{\checkmark}
\newcommand{\checkmarkx}{\checkmark}
\newcommand{\dTdt}{\frac{d\Tfieldt}
\newcommand{\deltaE}{\delta E}
\newcommand{\deltafield}{\ensuremath{\delta m}
\newcommand{\deltam}{\delta m}
\newcommand{\deq}{\displaystyle}
\newcommand{\docref}[1]{\texttt{#1}
\newcommand{\eV}{\,\text{eV}
\newcommand{\epsilonT}{\varepsilon_T}
\newcommand{\epsilonzero}{\varepsilon_0}
\newcommand{\etavis}{\eta_{\text{visual}
\newcommand{\e}{\mathrm{e}
\newcommand{\gW}{g_W}
\newcommand{\gammaf}{\gamma_{\text{Lorentz}
\newcommand{\gammamu}{\gamma^\mu}
\newcommand{\gs}{g_s}
\newcommand{\inftytext}{$\infty$}
\newcommand{\interval}[2]{#1:#2}
\newcommand{\kfrac}{K_{\text{frak}
\newcommand{\lP}{\ell_{\text{P}
\newcommand{\lP}{l_P}
\newcommand{\lambdah}{\ensuremath{\lambda_h}
\newcommand{\lambdah}{\lambda_h}
\newcommand{\lambdazero}{\lambda_0}
\newcommand{\mP}{m_{\text{P}
\newcommand{\mfield}{m(x,t)}
\newcommand{\mfield}{m}
\newcommand{\mh}{m_h}
\newcommand{\micrometer}{\ensuremath{\mu}
\newcommand{\mikrometer}{\ensuremath{\mu}
\newcommand{\myRightarrow}{\ensuremath{\Rightarrow}
\newcommand{\myapprox}{\ensuremath{\approx}
\newcommand{\myomega}{\ensuremath{\omega}
\newcommand{\myphi}{\ensuremath{\phi}
\newcommand{\mypi}{\ensuremath{\pi}
\newcommand{\mypropto}{\ensuremath{\propto}
\newcommand{\myrightarrow}{\ensuremath{\rightarrow}
\newcommand{\mysim}{\ensuremath{\sim}
\newcommand{\mysqrt}{\ensuremath{\sqrt}
\newcommand{\mytimes}{\ensuremath{\times}
\newcommand{\natunits}{\hbar = c = G = k_B = 1}
\newcommand{\natunits}{\text{(nat. Einh.)}
\newcommand{\natunits}{\text{(nat. units)}
\newcommand{\nulep}{\nu}
\newcommand{\nuzero}{\nu_0}
\newcommand{\partialop}{\ensuremath{\partial}
\newcommand{\pdTdt}{\frac{\partial\Tfieldt}
\newcommand{\pdTdx}{\nabla\Tfieldt}
\newcommand{\phiT}{\phi}
\newcommand{\pichar}{\pi}
\newcommand{\primrel}[1]{\mathbf{#1}
\newcommand{\rhoCMB}{\rho_{\text{CMB}
\newcommand{\rhoCasimir}{\rho_{\text{Casimir}
\newcommand{\rhoE}{\rho_E}
\newcommand{\rhofield}{\ensuremath{\rho}
\newcommand{\rzero}{r_0}
\newcommand{\slashk}{\cancel{k}
\newcommand{\slashp}{\cancel{p}
\newcommand{\slashq}{\cancel{q}
\newcommand{\tP}{t_P}
\newcommand{\tP}{t_{\text{P}
\newcommand{\tablescale}{0.9}
\newcommand{\tzero}{t_0}
\newcommand{\vect}[1]{\boldsymbol{#1}
\newcommand{\vecx}{\vec{x}
\newcommand{\vh}{v}
\newcommand{\vr}{\vec{r}
\newcommand{\warningx}{\color{red}
\newcommand{\warningx}{\textbf{!}
\newcommand{\warningx}{{\color{red}
\newcommand{\xiT}{\xi}
\newcommand{\xiconst}{\xi = \frac{4}
\newcommand{\xicoupling}{f(E/\Exi)}
\newcommand{\xigeom}{\xi_{\text{geom}
\newcommand{\xigeom}{\xi}
\newcommand{\xikonst}{\xi = \frac{4}
\newcommand{\xiparticle}{\xi_{\text{particle}
\newcommand{\xipar}{\ensuremath{\xi}
\newcommand{\xipar}{\xi_0}
\newcommand{\xipar}{\xi}
\newcommand{\xirat}{\xi_{\text{ratio}
\newtheorem{axiom}{Axiom}
\newtheorem{category}{Category-Theoretic Basis}
\newtheorem{category}{Kategorientheoretische Basis}
\newtheorem{corollary}[theorem]{Corollary}
\newtheorem{corollary}[theorem]{Korollar}
\newtheorem{corollary}{Corollary}
\newtheorem{corollary}{Korollar}
\newtheorem{definition}[theorem]{Definition}
\newtheorem{definition}{Definition}
\newtheorem{discovery}{Discovery}
\newtheorem{discovery}{Neue Entdeckung}
\newtheorem{discovery}{New Discovery}
\newtheorem{discovery}{Revolutionary Discovery}
\newtheorem{entdeckung}{Entdeckung}
\newtheorem{entdeckung}{Revolutionäre Entdeckung}
\newtheorem{erkenntnis}{Erkenntnis}
\newtheorem{erkenntnis}{Schlüsselerkenntnis}
\newtheorem{example}[theorem]{Beispiel}
\newtheorem{example}[theorem]{Example}
\newtheorem{example}{Beispiel}
\newtheorem{example}{Example}
\newtheorem{insight}{Central Insight}
\newtheorem{insight}{Insight}
\newtheorem{insight}{Key Insight}
\newtheorem{insight}{Wichtige Einsicht}
\newtheorem{insight}{Zentrale Einsicht}
\newtheorem{lemma}[theorem]{Lemma}
\newtheorem{lemma}{Lemma}
\newtheorem{principle}{Fundamental Principle}
\newtheorem{principle}{Fundamentales Prinzip}
\newtheorem{principle}{Grundlegendes Prinzip}
\newtheorem{principle}{Principle}
\newtheorem{principle}{Prinzip}
\newtheorem{prinzip}{Grundprinzip}
\newtheorem{proof_step}{Beweisschritt}
\newtheorem{proof_step}{Proof Step}
\newtheorem{proposition}[theorem]{Proposition}
\newtheorem{proposition}{Proposition}
\newtheorem{remark}[theorem]{Bemerkung}
\newtheorem{remark}[theorem]{Remark}
\newtheorem{theorem}{Theorem}
\newtheorem{warning}[theorem]{Warning}
\newtheorem{warning}[theorem]{Warnung}
\newunicodechar{±}{\ensuremath{\pm}
\newunicodechar{×}{\ensuremath{\times}
\newunicodechar{÷}{\ensuremath{\div}
\newunicodechar{ħ}{\ensuremath{\hbar}
\newunicodechar{Α}{\ensuremath{A}
\newunicodechar{Β}{\ensuremath{B}
\newunicodechar{Γ}{\ensuremath{\Gamma}
\newunicodechar{Δ}{\ensuremath{\Delta}
\newunicodechar{Ε}{\ensuremath{E}
\newunicodechar{Ζ}{\ensuremath{Z}
\newunicodechar{Η}{\ensuremath{H}
\newunicodechar{Θ}{\ensuremath{\Theta}
\newunicodechar{Ι}{\ensuremath{I}
\newunicodechar{Κ}{\ensuremath{K}
\newunicodechar{Λ}{\ensuremath{\Lambda}
\newunicodechar{Μ}{\ensuremath{M}
\newunicodechar{Ν}{\ensuremath{N}
\newunicodechar{Ξ}{\ensuremath{\Xi}
\newunicodechar{Ο}{\ensuremath{O}
\newunicodechar{Π}{\ensuremath{\Pi}
\newunicodechar{Ρ}{\ensuremath{P}
\newunicodechar{Σ}{\ensuremath{\Sigma}
\newunicodechar{Τ}{\ensuremath{T}
\newunicodechar{Υ}{\ensuremath{\Upsilon}
\newunicodechar{Φ}{\ensuremath{\Phi}
\newunicodechar{Χ}{\ensuremath{X}
\newunicodechar{Ψ}{\ensuremath{\Psi}
\newunicodechar{Ω}{\ensuremath{\Omega}
\newunicodechar{α}{\ensuremath{\alpha}
\newunicodechar{β}{\ensuremath{\beta}
\newunicodechar{γ}{\ensuremath{\gamma}
\newunicodechar{δ}{\ensuremath{\delta}
\newunicodechar{ε}{\ensuremath{\varepsilon}
\newunicodechar{ζ}{\ensuremath{\zeta}
\newunicodechar{η}{\ensuremath{\eta}
\newunicodechar{θ}{\ensuremath{\theta}
\newunicodechar{ι}{\ensuremath{\iota}
\newunicodechar{κ}{\ensuremath{\kappa}
\newunicodechar{λ}{\ensuremath{\lambda}
\newunicodechar{μ}{\ensuremath{\mu}
\newunicodechar{ν}{\ensuremath{\nu}
\newunicodechar{ξ}{\ensuremath{\xi}
\newunicodechar{ο}{\ensuremath{o}
\newunicodechar{π}{\ensuremath{\pi}
\newunicodechar{ρ}{\ensuremath{\rho}
\newunicodechar{σ}{\ensuremath{\sigma}
\newunicodechar{τ}{\ensuremath{\tau}
\newunicodechar{υ}{\ensuremath{\upsilon}
\newunicodechar{φ}{\ensuremath{\phi}
\newunicodechar{φ}{\ensuremath{\varphi}
\newunicodechar{χ}{\ensuremath{\chi}
\newunicodechar{ψ}{\ensuremath{\psi}
\newunicodechar{ω}{\ensuremath{\omega}
\newunicodechar{←}{\ensuremath{\leftarrow}
\newunicodechar{→}{\ensuremath{\rightarrow}
\newunicodechar{↔}{\ensuremath{\leftrightarrow}
\newunicodechar{⇐}{\ensuremath{\Leftarrow}
\newunicodechar{⇒}{\ensuremath{\Rightarrow}
\newunicodechar{⇔}{\ensuremath{\Leftrightarrow}
\newunicodechar{∂}{\ensuremath{\partial}
\newunicodechar{∅}{\ensuremath{\emptyset}
\newunicodechar{∇}{\ensuremath{\nabla}
\newunicodechar{∈}{\ensuremath{\in}
\newunicodechar{∉}{\ensuremath{\notin}
\newunicodechar{∏}{\ensuremath{\prod}
\newunicodechar{∑}{\ensuremath{\sum}
\newunicodechar{√}{\ensuremath{\sqrt}
\newunicodechar{∝}{\ensuremath{\propto}
\newunicodechar{∞}{\ensuremath{\infty}
\newunicodechar{∩}{\ensuremath{\cap}
\newunicodechar{∪}{\ensuremath{\cup}
\newunicodechar{∫}{\ensuremath{\int}
\newunicodechar{≈}{\ensuremath{\approx}
\newunicodechar{≠}{\ensuremath{\neq}
\newunicodechar{≤}{\ensuremath{\leq}
\newunicodechar{≥}{\ensuremath{\geq}
\newunicodechar{★}{\ensuremath{\star}
\newunicodechar{✓}{\checkmark}
\pgfplotsset{compat=1.17}
\pgfplotsset{compat=1.18}
\renewcommand{\cftchapfont}{\large\bfseries\color{blue}
\renewcommand{\cftchappagefont}{\large\bfseries\color{blue}
\renewcommand{\cftsecfont}{\bfseries}
\renewcommand{\cftsecfont}{\color{blue}
\renewcommand{\cftsecfont}{\large\bfseries\color{blue}
\renewcommand{\cftsecpagefont}{\bfseries}
\renewcommand{\cftsecpagefont}{\color{blue}
\renewcommand{\cftsecpagefont}{\large\bfseries\color{blue}
\renewcommand{\cftsubsecfont}{\color{blue!80!black}
\renewcommand{\cftsubsecfont}{\color{blue}
\renewcommand{\cftsubsecpagefont}{\color{blue!80!black}
\renewcommand{\cftsubsecpagefont}{\color{blue}
\renewcommand{\cftsubsubsecfont}{\color{blue!60!black}
\renewcommand{\cftsubsubsecfont}{\color{blue}
\renewcommand{\cftsubsubsecpagefont}{\color{blue!60!black}
\renewcommand{\cftsubsubsecpagefont}{\color{blue}
\renewcommand{\cfttoctitlefont}{\huge\bfseries\color{blue}
\renewcommand{\cfttoctitlefont}{\huge\bfseries}
\renewcommand{\familydefault}{\sfdefault}
\renewcommand{\footrulewidth}{0.4pt}
\renewcommand{\headrulewidth}{0.4pt}
\sisetup{locale = DE, group-separator = {.}
\sisetup{locale = DE}
\usetikzlibrary{arrows.meta,positioning,shapes.geometric}
\usetikzlibrary{decorations.pathmorphing, patterns, shapes.arrows}
\usetikzlibrary{intersections}
\usetikzlibrary{positioning, arrows.meta}
\usetikzlibrary{positioning, arrows}
\usetikzlibrary{positioning, shapes.geometric, arrows.meta}
\usetikzlibrary{positioning,shapes,arrows}

% Common settings
\setlength{\headheight}{15pt}
\pgfplotsset{compat=1.18}
\usetikzlibrary{positioning,shapes,arrows,arrows.meta}

% Hyperref setup
\hypersetup{
    colorlinks=true,
    linkcolor=blue,
    citecolor=blue,
    urlcolor=blue
}


\title{T0 Vollstaendige Berchnungen De}
\author{Johann Pascher}
\date{\today}

\begin{document}

\maketitle
\tableofcontents

\begin{abstract}
		Die T0-Theorie stellt einen neuen Ansatz zur Vereinigung von Teilchenphysik und Kosmologie dar, indem alle fundamentalen Massen und physikalischen Konstanten aus nur drei geometrischen Parametern abgeleitet werden: der Konstante $\xi = \frac{4}{3} \times 10^{-4}$, der Planck-Länge $\ell_P = 1.616e-35$ m und der charakteristischen Energie $E_0 = 7.398$ MeV wobei Energie auch abgeleitet werden kann. Diese Version demonstriert die bemerkenswerte Präzision des T0-Frameworks mit über 99\% Genauigkeit bei fundamentalen Konstanten.
	\end{abstract}
	
	\tableofcontents
	\newpage
	
	# Einführung
	
	Die T0-Theorie basiert auf der fundamentalen Hypothese einer geometrischen Konstante $\xi$, die alle physikalischen Phänomene auf makroskopischen und mikroskopischen Skalen vereint. Im Gegensatz zu Standardansätzen, die auf empirischen Anpassungen basieren, leitet T0 alle Parameter aus exakten mathematischen Beziehungen ab.
	
	## Fundamentale Parameter
	
	Das gesamte T0-System basiert ausschließlich auf drei Eingabewerten:
	
	
```math-align

		\xi &= \frac{4}{3} \times 10^{-4} \approx 1.33333333e-04 \quad \text{(geometrische Konstante)} \\
		\ell_P &= 1.616e-35 \text{ m} \quad \text{(Planck-Länge)} \\
		E_0 &= 7.398 \text{ MeV} \quad \text{(charakteristische Energie)} \\
		v &= 246.0 \text{ GeV} \quad \text{(Higgs-VEV)}
	
```

	
	# T0-Fundamentalformel für die Gravitationskonstante
	
	## Mathematische Herleitung
	
	Die zentrale Erkenntnis der T0-Theorie ist die Beziehung:
	
```math-equation

		\xi = 2\sqrt{G \cdot m_{\text{char}}}
	
```

	
	wobei $m_{\text{char}} = \xi/2$ die charakteristische Masse ist. Auflösung nach $G$ ergibt:
	
	
```math-equation

		\boxed{G = \frac{\xi^2}{4m_{\text{char}}} = \frac{\xi^2}{4 \cdot (\xi/2)} = \frac{\xi}{2}}
	
```

	
	## Dimensionsanalyse
	
	In natürlichen Einheiten ($\hbar = c = 1$) ergibt die T0-Grundformel zunächst:
	
```math-equation

		[G_{\text{T0}}] = \frac{[\xi^2]}{[m]} = \frac{[1]}{[E]} = [E^{-1}]
	
```

	
	Da die physikalische Gravitationskonstante jedoch die Dimension $[E^{-2}]$ benötigt, ist ein Umrechnungsfaktor erforderlich:
	
	
```math-equation

		G_{\text{nat}} = G_{\text{T0}} \times 3{,}521 \times 10^{-2} \quad [E^{-2}]
	
```

	
	## Herkunft des Faktors 1 ($3{,521 \times 10^{-2}$)}
	
	Der Faktor $3{,}521 \times 10^{-2}$ entstammt der charakteristischen T0-Energieskala $E_{\text{char}} \approx 28.4$ in natürlichen Einheiten. Dieser Faktor korrigiert die Dimension von $[E^{-1}]$ nach $[E^{-2}]$ und repräsentiert die Kopplung der T0-Geometrie an die Raumzeit-Krümmung, wie sie durch die $\xi$-Feldstruktur definiert ist.
	

	
	
\section{Verifikation des charakteristischen T0-Faktors}

\textbf{Der Faktor $3{,}521 \times 10^{-2}$ ist exakt $\frac{1}{28{,}4}$!}
\subsection{Kernerkenntnisse der Nachrechnung}

	- \textbf{Faktor-Identifikation:}
	
		- $3{,}521 \times 10^{-2} = \frac{1}{28{,}4}$ (perfekte Übereinstimmung)
		- Dies entspricht einer charakteristischen T0-Energieskala von $\mathbf{E_{\text{char}} \approx 28{,}4}$ in natürlichen Einheiten
	
	
	- \textbf{Dimensionsstruktur:}
	
		- $\mathbf{E_{\text{char}} = 28{,}4}$ hat Dimension $[E]$
		- $\mathbf{\text{Faktor} = \frac{1}{28{,}4} \approx 0{,}03521}$ hat Dimension $[E^{-1}] = [L]$
		- Dies ist eine \textbf{charakteristische Länge} im T0-System
	
	
	- \textbf{Dimensionskorrektur $[E^{-1}] \rightarrow [E^{-2}]$:}
	
		- $\mathbf{\text{Faktor} \times \xi = 4{,}695 \times 10^{-6}}$ ergibt Dimension $[E^{-2}]$
		- Dies ist die Kopplung an die Raumzeit-Krümmung
		- $\mathbf{264\times}$ stärker als die reine Gravitationskopplung $\alpha_G = \xi^2 = 1{,}778 \times 10^{-8}$
	
	
	- \textbf{Skalenhierarchie bestätigt:}
	
```math-align

		E_0 &\approx 7{,}398 \text{ MeV} \quad \text{(elektromagnetische Skala)} \\
		E_{\text{char}} &\approx 28{,}4 \quad \text{(T0-Zwischen-Energieskala)} \\
		E_{T0} &= \frac{1}{\xi} = 7500 \quad \text{(fundamentale T0-Skala)}
	
```

	
	- \textbf{Physikalische Bedeutung:}
	\\Der Faktor repräsentiert die \textbf{$\xi$-Feldstruktur-Kopplung}, die die T0-Geometrie an die Raumzeit-Krümmung bindet -- genau wie wir beschrieben haben!

\textbf{Formel für die charakteristische T0-Energieskala:}

```math-equation

	\boxed{E_{\text{char}} = \frac{1}{3{,}521 \times 10^{-2}} = 28{,}4 \quad \text{(natürliche Einheiten)}}

```

Die Dimensionskorrektur erfolgt durch die $\xi$-Feldstruktur:

```math-equation

	\underbrace{3{,}521 \times 10^{-2}}_{[E^{-1}]} \times \underbrace{\xi}_{[1]} = \underbrace{4{,}695 \times 10^{-6}}_{[E^{-2}]}

```

Diese Kopplung bindet die T0-Geometrie an die Raumzeit-Krümmung.

\subsection{Charakteristische T0-Einheiten: $r_0 = E_0 = m_0$}

In charakteristischen T0-Einheiten des natürlichen Einheitensystems gilt die fundamentale Beziehung:

```math-equation

	r_0 = E_0 = m_0 \quad \text{(in charakteristischen Einheiten)}

```

\textbf{Korrekte Interpretation in natürlichen Einheiten:}

```math-align

	r_0 &= 0{,}035211 \quad [E^{-1}] = [L] \quad \text{(charakteristische Länge)} \\
	E_0 &= 28{,}4 \quad [E] \quad \text{(charakteristische Energie)} \\
	m_0 &= 28{,}4 \quad [E] = [M] \quad \text{(charakteristische Masse)} \\
	t_0 &= 0{,}035211 \quad [E^{-1}] = [T] \quad \text{(charakteristische Zeit)}

```

\textbf{Fundamentale Konjugation:}

```math-equation

	r_0 \times E_0 = 0{,}035211 \times 28{,}4 = 1{,}000 \quad \text{(dimensionslos)}

```

Die charakteristischen Skalen sind \textbf{konjugierte Größen} der T0-Geometrie. Die T0-Formel $r_0 = 2GE$ wird mit der charakteristischen Gravitationskonstante:

```math-equation

	G_{\text{char}} = \frac{r_0}{2 \times E_0} = \frac{\xi^2}{2 \times E_{\text{char}}}

```

\section{SI-Umrechnung}

Der Übergang zu SI-Einheiten erfolgt durch den Umrechnungsfaktor:

```math-equation

	\boxed{G_{\text{SI}} = G_{\text{nat}} \times 2{,}843 \times 10^{-5} \quad \si{\meter^3 \kilogram^{-1} \second^{-2}}}

```

\section{Herkunft des Faktors 2 ($2{,843 \times 10^{-5}$)}}

Der Faktor $2{,}843 \times 10^{-5}$ ergibt sich aus der fundamentalen T0-Feldkopplung:

```math-equation

	\boxed{2{,}843 \times 10^{-5} = 2 \times (E_{\text{char}} \times \xi)^2}

```

Diese Formel hat klare physikalische Bedeutung:

	- \textbf{Faktor 2:} Fundamentale Dualität der T0-Theorie
	- \textbf{$E_{\text{char}} \times \xi$:} Kopplung der charakteristischen Energieskala an die $\xi$-Geometrie
	- \textbf{Quadrierung:} Charakteristisch für Feldtheorien (analog zu $E^2$-Termen)

\textbf{Numerische Verifikation:}

```math-align

	2 \times (E_{\text{char}} \times \xi)^2 &= 2 \times (28{,}4 \times 1{,}333 \times 10^{-4})^2 \\
	&= 2 \times (3{,}787 \times 10^{-3})^2 \\
	&= 2{,}868 \times 10^{-5}

```

\textbf{Abweichung vom verwendeten Wert:} $< 1\%$ (praktisch perfekte Übereinstimmung)

\section{Schritt-für-Schritt Berechnung}

```math-align

	\text{Schritt 1: } m_{\text{char}} &= \frac{\xi}{2} = \frac{1.333333 \times 10^{-4}}{2} = 6{,}666667 \times 10^{-5} \\
	\text{Schritt 2: } G_{\text{T0}} &= \frac{\xi^2}{4m_{\text{char}}} = \frac{\xi}{2} = 6{,}666667 \times 10^{-5} \text{ [dimensionslos]} \\
	\text{Schritt 3: } G_{\text{nat}} &= G_{\text{T0}} \times 3{,}521 \times 10^{-2} = 2{,}347333 \times 10^{-6} \text{ [E}^{-2}\text{]} \\
	\text{Schritt 4: } G_{\text{SI}} &= G_{\text{nat}} \times 2{,}843 \times 10^{-5} = 6{,}673469 \times 10^{-11} \si{\meter^3 \kilogram^{-1} \second^{-2}}

```

\textbf{Experimenteller Vergleich:}

```math-align

	G_{\text{exp}} &= 6{,}674300 \times 10^{-11} \si{\meter^3 \kilogram^{-1} \second^{-2}} \\
	\text{Relativer Fehler} &= 0{,}0125\%

```

	
	# Teilchenmassen-Berechnungen
	
	## Yukawa-Methode der T0-Theorie
	
	Alle Fermionmassen werden durch die universelle T0-Yukawa-Formel bestimmt:
	
	
```math-equation

		\boxed{m = r \times \xi^p \times v}
	
```

	
	wobei $r$ und $p$ exakte rationale Zahlen sind, die aus der T0-Geometrie folgen.
	
	## Detaillierte Massenberechnungen
	
	\begin{longtable}{>{\raggedright}p{4cm}ccccccc}
		\caption{T0-Yukawa-Massenberechnungen für alle Standardmodell-Fermionen} \\
		\toprule
		\textbf{Teilchen} & \textbf{$r$} & \textbf{$p$} & \textbf{$\xi^p$} & \textbf{T0-Masse [MeV]} & \textbf{Exp. [MeV]} & \textbf{Fehler [\%]} \\
		\midrule
		\endfirsthead
		\multicolumn{7}{c}{\textit{Fortsetzung von vorheriger Seite}} \\
		\toprule
		\textbf{Teilchen} & \textbf{$r$} & \textbf{$p$} & \textbf{$\xi^p$} & \textbf{T0-Masse [MeV]} & \textbf{Exp. [MeV]} & \textbf{Fehler [\%]} \\
		\midrule
		\endhead
		\midrule
		\multicolumn{7}{r}{\textit{Fortsetzung auf nächster Seite}} \\
		\endfoot
		\bottomrule
		\endlastfoot
		Elektron & $\frac{4}{3}$ & $\frac{3}{2}$ & 1.540e-06 & 0.5 & 0.5 & 1.18 \\
		Myon & $\frac{16}{5}$ & $1$ & 1.333e-04 & 105.0 & 105.7 & 0.66 \\
		Tau & $\frac{8}{3}$ & $\frac{2}{3}$ & 2.610e-03 & 1712.1 & 1776.9 & 3.64 \\
		Up & $6$ & $\frac{3}{2}$ & 1.540e-06 & 2.3 & 2.3 & 0.11 \\
		Down & $\frac{25}{2}$ & $\frac{3}{2}$ & 1.540e-06 & 4.7 & 4.7 & 0.30 \\
		Strange & $\frac{26}{9}$ & $1$ & 1.333e-04 & 94.8 & 93.4 & 1.45 \\
		Charm & $2$ & $\frac{2}{3}$ & 2.610e-03 & 1284.1 & 1270.0 & 1.11 \\
		Bottom & $\frac{3}{2}$ & $\frac{1}{2}$ & 1.155e-02 & 4260.8 & 4180.0 & 1.93 \\
		Top & $\frac{1}{28}$ & $\frac{-1}{3}$ & 1.957e+01 & 171974.5 & 172760.0 & 0.45 \\
	\end{longtable}
	
	## Beispielberechnung: Elektron
	
	Die Elektronmasse dient als paradigmatisches Beispiel der T0-Yukawa-Methode:
	
	
```math-align

		r_e &= \frac{4}{3}, \quad p_e = \frac{3}{2} \\
		m_e &= \frac{4}{3} \times \left(\frac{4}{3} \times 10^{-4}\right)^{3/2} \times 246 \text{ GeV} \\
		&= \frac{4}{3} \times 1.539601e-06 \times 246 \text{ GeV} \\
		&= 0.505 \text{ MeV}
	
```

	
	\textbf{Experimenteller Wert:} $m_{e,\text{exp}} = 0.511$ MeV
	
	\textbf{Relative Abweichung:} 1.176\%
	
	# Magnetische Momente und g-2 Anomalien
	
	## Standardmodell + T0-Korrekturen
	
	Die T0-Theorie sagt spezifische Korrekturen zu den magnetischen Momenten der Leptonen vorher. Die anomalen magnetischen Momente werden durch die Kombination von Standardmodell-Beiträgen und T0-Korrekturen beschrieben:
	
	
```math-equation

		a_{\text{gesamt}} = a_{\text{SM}} + a_{\text{T0}}
	
```

	
	\begin{table}[h]
		\centering
		\begin{tabular}{>{\raggedright}p{4cm}ccccc}
			\toprule
			\textbf{Lepton} & \textbf{T0-Masse [MeV]} & \textbf{$a_{\text{SM}}$} & \textbf{$a_{\text{T0}}$} & \textbf{$a_{\text{exp}}$} & \textbf{$\sigma$-Abw.} \\
			\midrule
			Elektron & 504.989 & 1.160e-03 & 5.810e-14 & 1.160e-03 & +0.9 \\
			Myon & 104960.000 & 1.166e-03 & 2.510e-09 & 1.166e-03 & +1.3 \\
			Tau & 1712102.115 & 1.177e-03 & 6.679e-07 & --- & --- \\
			\bottomrule
		\end{tabular}
		\caption{Magnetische Moment-Anomalien: SM + T0-Vorhersagen vs. Experiment}
	\end{table}
	
	# Vollständige Liste physikalischer Konstanten
	
	Die T0-Theorie berechnet über 40 fundamentale physikalische Konstanten in einer hierarchischen 8-Level-Struktur. Diese Sektion dokumentiert alle berechneten Werte mit ihren Einheiten und Abweichungen von experimentellen Referenzwerten.
	
	## Kategorienbasierte Konstantenübersicht
	
	\begin{table}[h]
		\centering
		\begin{tabular}{>{\raggedright}p{4cm}ccccc}
			\toprule
			\textbf{Kategorie} & \textbf{Anzahl} & \textbf{Ø-Fehler [\%]} & \textbf{Min [\%]} & \textbf{Max [\%]} & \textbf{Präzision} \\
			\midrule
			Fundamental & 1 & 0.0005 & 0.0005 & 0.0005 & Exzellent \\
			Gravitation & 1 & 0.0125 & 0.0125 & 0.0125 & Exzellent \\
			Planck & 6 & 0.0131 & 0.0062 & 0.0220 & Exzellent \\
			Elektromagnetisch & 4 & 0.0001 & 0.0000 & 0.0002 & Exzellent \\
			Atomphysik & 7 & 0.0005 & 0.0000 & 0.0009 & Exzellent \\
			Metrologie & 5 & 0.0002 & 0.0000 & 0.0005 & Exzellent \\
			Thermodynamik & 3 & 0.0008 & 0.0000 & 0.0023 & Exzellent \\
			Kosmologie & 4 & 11.6528 & 0.0601 & 45.6741 & Akzeptabel \\
			\bottomrule
		\end{tabular}
		\caption{Kategorienbasierte Fehlerstatistik der T0-Konstantenberechnungen}
	\end{table}
	
	## Detaillierte Konstantenliste
	
	\begin{longtable}{>{\raggedright}p{5.cm}p{1.5cm}p{2cm}p{2.5cm}p{2cm}p{2.5cm}}
		\caption{Vollständige Liste aller berechneten physikalischen Konstanten} \\
		\toprule
		\textbf{Konstante} & \textbf{Symbol} & \textbf{T0-Wert} & \textbf{Referenzwert} & \textbf{Fehler [\%]} & \textbf{Einheit} \\
		\midrule
		\endfirsthead
		\multicolumn{6}{c}{\textit{Fortsetzung von vorheriger Seite}} \\
		\toprule
		\textbf{Konstante} & \textbf{Symbol} & \textbf{T0-Wert} & \textbf{Referenzwert} & \textbf{Fehler [\%]} & \textbf{Einheit} \\
		\midrule
		\endhead
		\midrule
		\multicolumn{6}{r}{\textit{Fortsetzung auf nächster Seite}} \\
		\endfoot
		\bottomrule
		\endlastfoot
		Feinstrukturkonstante & $\alpha$ & 7.297e-03 & 7.297e-03 & 0.0005 & \text{dimensionslos} \\
		Gravitationskonstante & $G$ & 6.673e-11 & 6.674e-11 & 0.0125 & $\si{\meter^3 \kilogram^{-1} \second^{-2}}$ \\
		Planck-Masse & $m_P$ & 2.177e-08 & 2.176e-08 & 0.0062 & $\si{\kilogram}$ \\
		Planck-Zeit & $t_P$ & 5.390e-44 & 5.391e-44 & 0.0158 & $\si{\second}$ \\
		Planck-Temperatur & $T_P$ & 1.417e+32 & 1.417e+32 & 0.0062 & $\si{\kelvin}$ \\
		Lichtgeschwindigkeit & $c$ & 2.998e+08 & 2.998e+08 & 0.0000 & $\si{\meter \per \second}$ \\
		Reduzierte Planck-Konstante & $\hbar$ & 1.055e-34 & 1.055e-34 & 0.0000 & $\si{\joule \second}$ \\
		Planck-Energie & $E_P$ & 1.956e+09 & 1.956e+09 & 0.0062 & $\si{\joule}$ \\
		Planck-Kraft & $F_P$ & 1.211e+44 & 1.210e+44 & 0.0220 & $\si{\newton}$ \\
		Planck-Leistung & $P_P$ & 3.629e+52 & 3.628e+52 & 0.0220 & $\si{\watt}$ \\
		Magnetische Feldkonstante & $\mu_0$ & 1.257e-06 & 1.257e-06 & 0.0000 & $\si{\henry \per \meter}$ \\
		Elektrische Feldkonstante & $\epsilon_0$ & 8.854e-12 & 8.854e-12 & 0.0000 & $\si{\farad \per \meter}$ \\
		Elementarladung & $e$ & 1.602e-19 & 1.602e-19 & 0.0002 & $\si{\coulomb}$ \\
		Wellenwiderstand Vakuum & $Z_0$ & 3.767e+02 & 3.767e+02 & 0.0000 & $\si{\ohm}$ \\
		Coulomb-Konstante & $k_e$ & 8.988e+09 & 8.988e+09 & 0.0000 & $\si{\newton \meter^2 \per \coulomb^2}$ \\
		Stefan-Boltzmann-Konstante & $\sigma_{SB}$ & 5.670e-08 & 5.670e-08 & 0.0000 & $\si{\watt \per \meter^2 \kelvin^4}$ \\
		Wien-Konstante & $b$ & 2.898e-03 & 2.898e-03 & 0.0023 & $\si{\meter \kelvin}$ \\
		Planck-Konstante & $h$ & 6.626e-34 & 6.626e-34 & 0.0000 & $\si{\joule \second}$ \\
		Bohr-Radius & $a_0$ & 5.292e-11 & 5.292e-11 & 0.0005 & $\si{\meter}$ \\
		Rydberg-Konstante & $R_\infty$ & 1.097e+07 & 1.097e+07 & 0.0009 & $\si{\meter^{-1}}$ \\
		Bohr-Magneton & $\mu_B$ & 9.274e-24 & 9.274e-24 & 0.0002 & $\si{\joule \per \tesla}$ \\
		Kern-Magneton & $\mu_N$ & 5.051e-27 & 5.051e-27 & 0.0002 & $\si{\joule \per \tesla}$ \\
		Hartree-Energie & $E_h$ & 4.360e-18 & 4.360e-18 & 0.0009 & $\si{\joule}$ \\
		Compton-Wellenlänge & $\lambda_C$ & 2.426e-12 & 2.426e-12 & 0.0000 & $\si{\meter}$ \\
		Elektronenradius & $r_e$ & 2.818e-15 & 2.818e-15 & 0.0005 & $\si{\meter}$ \\
		Faraday-Konstante & $F$ & 9.649e+04 & 9.649e+04 & 0.0002 & $\si{\coulomb \per \mole}$ \\
		von-Klitzing-Konstante & $R_K$ & 2.581e+04 & 2.581e+04 & 0.0005 & $\si{\ohm}$ \\
		Josephson-Konstante & $K_J$ & 4.836e+14 & 4.836e+14 & 0.0002 & $\si{\hertz \per \volt}$ \\
		Magnetischer Flussquant & $\Phi_0$ & 2.068e-15 & 2.068e-15 & 0.0002 & $\si{\weber}$ \\
		Gaskonstante & $R$ & 8.314e+00 & 8.314e+00 & 0.0000 & $\si{\joule \per \mole \kelvin}$ \\
		Loschmidt-Konstante & $n_0$ & 2.687e+22 & 2.687e+25 & 99.9000 & $\si{\meter^{-3}}$ \\
		Hubble-Konstante & $H_0$ & 2.196e-18 & 2.196e-18 & 0.0000 & $\si{\second^{-1}}$ \\
		Kosmologische Konstante & $\Lambda$ & 1.610e-52 & 1.105e-52 & 45.6741 & $\si{\meter^{-2}}$ \\
		Alter Universum & $t_{\text{Universum}}$ & 4.554e+17 & 4.551e+17 & 0.0601 & $\si{\second}$ \\
		Kritische Dichte & $\rho_{\text{krit}}$ & 8.626e-27 & 8.558e-27 & 0.7911 & $\si{\kilogram \per \meter^3}$ \\
		Hubble-Länge & $l_{\text{Hubble}}$ & 1.365e+26 & 1.364e+26 & 0.0862 & $\si{\meter}$ \\
		Boltzmann-Konstante & $k_B$ & 1.381e-23 & 1.381e-23 & 0.0000 & $\si{\joule \per \kelvin}$ \\
		Avogadro-Konstante & $N_A$ & 6.022e+23 & 6.022e+23 & 0.0000 & $\si{\mole^{-1}}$ \\
	\end{longtable}
	
	# Mathematische Eleganz und Theoretische Bedeutung
	
	## Exakte Bruchverhältnisse
	
	Ein bemerkenswertes Merkmal der T0-Theorie ist die ausschließliche Verwendung \textbf{exakter mathematischer Konstanten}:
	
	
		- \textbf{Grundkonstante:} $\xi = \frac{4}{3} \times 10^{-4}$ (exakter Bruch)
		- \textbf{Teilchen-r-Parameter:} $\frac{4}{3}$, $\frac{16}{5}$, $\frac{8}{3}$, $\frac{25}{2}$, $\frac{26}{9}$, $\frac{3}{2}$, $\frac{1}{28}$
		- \textbf{Teilchen-p-Parameter:} $\frac{3}{2}$, $1$, $\frac{2}{3}$, $\frac{1}{2}$, $-\frac{1}{3}$
		- \textbf{Gravitationsfaktoren:} $\frac{\xi}{2}$, $3{,}521 \times 10^{-2}$, $2{,}843 \times 10^{-5}$
	
	
	\textcolor{t0green}{\textbf{Keine willkürlichen Dezimalanpassungen!}} Alle Beziehungen folgen aus der fundamentalen geometrischen Struktur.
	
	## Dimensionsbasierte Hierarchie
	
	Die T0-Konstantenberechnung folgt einer natürlichen 8-Level-Hierarchie:
	
	
		- \textbf{Level 1:} Primäre $\xi$-Ableitungen ($\alpha$, $m_{\text{char}}$)
		- \textbf{Level 2:} Gravitationskonstante ($G$, $G_{\text{nat}}$)
		- \textbf{Level 3:} Planck-System ($m_P$, $t_P$, $T_P$, etc.)
		- \textbf{Level 4:} Elektromagnetische Konstanten ($e$, $\epsilon_0$, $\mu_0$)
		- \textbf{Level 5:} Thermodynamische Konstanten ($\sigma_{SB}$, Wien-Konstante)
		- \textbf{Level 6:} Atom- und Quantenkonstanten ($a_0$, $R_\infty$, $\mu_B$)
		- \textbf{Level 7:} Metrologische Konstanten ($R_K$, $K_J$, Faraday-Konstante)
		- \textbf{Level 8:} Kosmologische Konstanten ($H_0$, $\Lambda$, kritische Dichte)
	
	
	## Fundamentale Bedeutung der Umrechnungsfaktoren
	
	Die Umrechnungsfaktoren in der T0-Gravitationsberechnung haben tiefe theoretische Bedeutung:
	
	
```math-align

		\text{Faktor 1: } &3{,}521 \times 10^{-2} \quad \text{[E}^{-1} \rightarrow \text{E}^{-2}\text{]} \\
		\text{Faktor 2: } &2{,}843 \times 10^{-5} \quad \text{[E}^{-2} \rightarrow \si{\meter^3 \kilogram^{-1} \second^{-2}}\text{]}
	
```

	
	\textbf{Interpretation:} Diese Faktoren entstehen nicht durch willkürliche Anpassung, sondern repräsentieren die fundamentale geometrische Struktur des $\xi$-Feldes und seine Kopplung an die Raumzeit-Krümmung.
	
	## Experimentelle Testbarkeit
	
	Die T0-Theorie macht spezifische, testbare Vorhersagen:
	
	
		- \textbf{Casimir-CMB-Verhältnis:} Bei $d \approx 100\,\si{\micro\meter}$ sollte $|\rho_{\text{Casimir}}|/\rho_{\text{CMB}} \approx 308$
		- \textbf{Präzisions-g-2-Messungen:} T0-Korrekturen für Elektron und Tau
		- \textbf{Fünfte Kraft:} Modifikationen der Newtonschen Gravitation bei $\xi$-charakteristischen Skalen
		- \textbf{Kosmologische Parameter:} Alternative zu $\Lambda$-CDM mit $\xi$-basierten Vorhersagen
	
	
	# Methodische Aspekte und Implementierung
	
	## Numerische Präzision
	
	Die T0-Berechnungen verwenden durchgängig:
	
	
		- \textbf{Exakte Bruchrechnungen:} Python \texttt{fractions.Fraction} für $r$- und $p$-Parameter
		- \textbf{CODATA 2018 Konstanten:} Alle Referenzwerte aus offiziellen Quellen
		- \textbf{Dimensionsvalidierung:} Automatische Überprüfung aller Einheiten
		- \textbf{Fehlerfilterung:} Intelligente Behandlung von Ausreißern und T0-spezifischen Konstanten
	
	
	## Kategorienbasierte Analyse
	
	Die 40+ berechneten Konstanten werden in physikalisch sinnvolle Kategorien eingeteilt:
	
	\begin{center}
		\begin{tabular}{ll}
			\textbf{Fundamental} & $\alpha$, $m_{\text{char}}$ (direkt aus $\xi$) \\
			\textbf{Gravitation} & $G$, $G_{\text{nat}}$, Umrechnungsfaktoren \\
			\textbf{Planck} & $m_P$, $t_P$, $T_P$, $E_P$, $F_P$, $P_P$ \\
			\textbf{Elektromagnetisch} & $e$, $\epsilon_0$, $\mu_0$, $Z_0$, $k_e$ \\
			\textbf{Atomphysik} & $a_0$, $R_\infty$, $\mu_B$, $\mu_N$, $E_h$, $\lambda_C$, $r_e$ \\
			\textbf{Metrologie} & $R_K$, $K_J$, $\Phi_0$, $F$, $R_{\text{gas}}$ \\
			\textbf{Thermodynamik} & $\sigma_{SB}$, Wien-Konstante, $h$ \\
			\textbf{Kosmologie} & $H_0$, $\Lambda$, $t_{\text{Universum}}$, $\rho_{\text{krit}}$ \\
		\end{tabular}
	\end{center}
	
	# Statistische Zusammenfassung
	
	## Gesamtperformance
	
	\begin{table}[h]
		\centering
		\begin{tabular}{>{\raggedright}p{4cm}cc}
			\toprule
			\textbf{Kategorie} & \textbf{Anzahl} & \textbf{Durchschn. Fehler [\%]} \\
			\midrule
			Fundamental & 1 & 0.0005 \\
			Gravitation & 1 & 0.0125 \\
			Planck & 6 & 0.0131 \\
			Elektromagnetisch & 4 & 0.0001 \\
			Atomphysik & 7 & 0.0005 \\
			Metrologie & 5 & 0.0002 \\
			Thermodynamik & 3 & 0.0008 \\
			Kosmologie & 4 & 11.6528 \\
			\midrule
			\textbf{Gesamt} & 45 & 1.4600 \\
			\bottomrule
		\end{tabular}
		\caption{Statistische Performance der T0-Konstantenvorhersagen}
	\end{table}
	
	## Beste und schlechteste Vorhersagen
	
	\textbf{Beste Massenvorhersage:} Up (0.108\% Fehler)
	
	\textbf{Schlechteste Massenvorhersage:} Tau (3.645\% Fehler)
	
	\textbf{Beste Konstantenvorhersage:} C (0.0000\% Fehler)
	
	\textbf{Schlechteste Konstantenvorhersage:} N0 (99.9000\% Fehler)
	
	# Vergleich mit Standardans\"{atzen}
	
	## Vorteile der T0-Theorie
	
	
		- \textbf{Parameterreduktion:} 3 Eingaben statt $>20$ im Standardmodell
		- \textbf{Mathematische Eleganz:} Exakte Br\"{u}che statt empirischer Anpassungen
		- \textbf{Vereinheitlichung:} Teilchenphysik + Kosmologie + Quantengravitation
		- \textbf{Vorhersagekraft:} Neue Ph\"{a}nomene (Casimir-CMB, modifizierte g-2)
		- \textbf{Experimentelle Testbarkeit:} Spezifische, falsifizierbare Vorhersagen
	
	
	## Theoretische Herausforderungen
	
	
		- \textbf{Umrechnungsfaktoren:} Theoretische Ableitung der numerischen Faktoren
		- \textbf{Quantisierung:} Integration in eine vollst\"{a}ndige Quantenfeldtheorie
		- \textbf{Renormierung:} Behandlung von Divergenzen und Skaleninvarianzen
		- \textbf{Symmetrien:} Verbindung zu bekannten Eichsymmetrien
		- \textbf{Dunkle Materie/Energie:} Explizite T0-Behandlung kosmologischer R\"{a}tsel
	
	
	# Technische Details der Implementierung
	
	## Python-Code-Struktur
	
	Das T0-Berechnungsprogramm T0\_calc\_De.py ist als objektorientierte Python-Klasse implementiert:
	
	\begin{lstlisting}[language=Python, basicstyle=\small\ttfamily]
		class T0VereinigterRechner:
		def __init__(self):
		self.xi = Fraction(4, 3) * 1e-4  # Exakter Bruch
		self.v = 246.0  # Higgs VEV [GeV]
		self.l_P = 1.616e-35  # Planck-L\"ange [m]
		self.E0 = 7.398  # Charakteristische Energie [MeV]
		
		def berechne_yukawa_masse_exakt(self, teilchen_name):
		# Exakte Bruchrechnungen f\"ur r und p
		# T0-Formel: m = r \times \xi^p \times v
		
		def berechne_level_2(self):
		# Gravitationskonstante mit Faktoren
		# G = \xi^2/(4m) \times 3.521e-2 \times 2.843e-5
	\end{lstlisting}
	
	## Qualitätssicherung
	
	
		- \textbf{Dimensionsvalidierung:} Automatische Überprüfung aller physikalischen Einheiten
		- \textbf{Referenzwertverifikation:} Vergleich mit CODATA 2018 und Planck 2018
		- \textbf{Numerische Stabilität:} Verwendung von \texttt{fractions.Fraction} für exakte Arithmetik
		- \textbf{Fehlerbehandlung:} Intelligente Behandlung von T0-spezifischen vs. experimentellen Konstanten
	
	
	# Fazit und wissenschaftliche Einordnung
	
	## Revolutionäre Aspekte
	
	Die T0-Theorie Version 3.2 stellt einen paradigmatischen Wandel in der theoretischen Physik dar:
	
	
		- \textbf{Alle 9 Standardmodell-Fermionmassen} aus einer einzigen Formel
		- \textbf{Über 40 physikalische Konstanten} aus 3 geometrischen Parametern
		- \textbf{Magnetische Momente} mit SM + T0-Korrekturen
		- \textbf{Kosmologische Verbindungen} über Casimir-CMB-Beziehungen
		- \textbf{Geometrische Fundamentierung:} Alle Physik aus einer einzigen Konstante $\xi$
		- \textbf{Mathematische Perfektion:} Ausschließlich exakte Beziehungen, keine freien Parameter
		- \textbf{Experimentelle Validierung:} >99\% Übereinstimmung bei kritischen Tests
		- \textbf{Prädiktive Macht:} Neue Phänomene und testbare Vorhersagen
		- \textbf{Konzeptuelle Eleganz:} Vereinigung aller fundamentalen Kräfte und Skalen
	
	
	## Wissenschaftlicher Impact
	
	Die T0-Theorie adressiert fundamentale offene Fragen der modernen Physik:
	
	
		- \textbf{Hierarchieproblem:} Warum sind Teilchenmassen so unterschiedlich?
		- \textbf{Konstanten-Problem:} Warum haben Naturkonstanten ihre spezifischen Werte?
		- \textbf{Quantengravitation:} Wie vereinigt man Quantenmechanik und Gravitation?
		- \textbf{Kosmologische Konstante:} Was ist die Natur der dunklen Energie?
		- \textbf{Feinabstimmung:} Warum ist das Universum für Leben "optimiert"?
	
	
	\textcolor{t0green}{\textbf{Die T0-Antwort:}} Alle diese scheinbar unabhängigen Probleme sind Manifestationen der einzigen geometrischen Konstante $\xi = \frac{4}{3} \times 10^{-4}$.
	
		# Anhang: Vollständige Datenreferenzen
	
	## Experimentelle Referenzwerte
	
	Alle in diesem Bericht verwendeten experimentellen Werte stammen aus den folgenden authorisierten Quellen:
	
	
		- \textbf{CODATA 2018:} Committee on Data for Science and Technology, "2018 CODATA Recommended Values"
		- \textbf{PDG 2020:} Particle Data Group, "Review of Particle Physics", Prog. Theor. Exp. Phys. 2020
		- \textbf{Planck 2018:} Planck Collaboration, "Planck 2018 results VI. Cosmological parameters"
		- \textbf{NIST:} National Institute of Standards and Technology, Physics Laboratory
	
	
	## Software und Berechnungsdetails
	
	
		- \textbf{Python Version:} 3.8+
		- \textbf{Abhängigkeiten:} math, fractions, datetime, json
		- \textbf{Präzision:} Floating-point: IEEE 754 double precision
		- \textbf{Bruchrechnungen:} Python fractions.Fraction für exakte Arithmetik
		- \textbf{Code-Repository:} \url{https://github.com/jpascher/T0-Time-Mass-Duality}
	
	
	\vfill
	
	\begin{center}
		\hrule
		\vspace{0.5cm}
		\textit{Dieser Bericht wurde automatisch generiert durch den T0-Vereinigten Rechner v3.2}\\
		\textit{am \today\space durch das T0-LaTeX-Generierungsmodul}\\
		\vspace{0.3cm}
		\textbf{T0-Theorie: Zeit-Masse-Dualitäts-Framework}\\
		\textit{Johann Pascher, HTL Leonding, Österreich}\\
		\textit{Verfügbar unter: \url{https://github.com/jpascher/T0-Time-Mass-Duality}}
	\end{center}

\end{document}
