\documentclass[a4paper,12pt]{article}
\usepackage[utf8]{inputenc}
\usepackage[ngerman]{babel} % Für deutsche Sprache und Silbentrennung
\usepackage{amsmath, amssymb}

\begin{document}
	
	\title{Folgen der energieabhängigen Zeitdefinition für die Nichtlokalität bei Photonen}
	\author{Johann Pascher}
	\date{24. März 2025}
	\maketitle
	
	\tableofcontents % Inhaltsverzeichnis
	\newpage % Optional: Beginnt den Hauptteil auf einer neuen Seite
	
	\section{Folgen der energieabhängigen Zeitdefinition für die Nichtlokalität bei Photonen}
	Dieser Anhang untersucht die Konsequenzen der Erweiterung der Quantenmechanik durch eine energieabhängige Zeitdefinition \( T = \frac{1}{E} \) für masselose Teilchen wie Photonen, insbesondere im Hinblick auf die Nichtlokalität. Diese Erweiterung wurde entwickelt, um die ursprünglichen Modelle mit absoluter Zeit (\( T_0 \)) und intrinsischer Zeit (\( T = \frac{\hbar}{m c^2} \)) zu verbessern, die bei Photonen mit \( m = 0 \) versagen. Hierbei werden Planck-Einheiten (\( \hbar = c_0 = G = 1 \)) verwendet, um die Darstellung zu vereinfachen.
	
	\subsection{Ursprüngliches Modell und seine Grenzen}
	Im Standardmodell der Quantenmechanik wird die Nichtlokalität bei verschränkten Photonen als „instantane“ Korrelation beschrieben, wobei die Lichtkegelstruktur durch die Lichtgeschwindigkeit (\( c_0 = 1 \)) die Kausalität sichert. Die Zeit ist relativistisch variabel, und Photonen haben keine Ruhemasse (\( m = 0 \)). Das \( T_0 \)-Modell mit absoluter Zeit postuliert eine konstante Zeit, aber die Massevariabilität (\( m = \gamma m_0 \)) bleibt für Photonen undefiniert, da \( m_0 = 0 \). Ebenso führt die intrinsische Zeit \( T = \frac{\hbar}{m c^2} \) bei \( m = 0 \) zu \( T \to \infty \), was die Zeitentwicklung zum Stillstand bringt und nicht mit der Dynamik von Photonen übereinstimmt.
	
	\subsection{Erweiterung der Quantenmechanik}
	Um diese Grenzen zu überwinden, wird die Zeitdefinition für masselose Teilchen auf \( T = \frac{1}{E} \) erweitert, wobei \( E = p \) für Photonen gilt (da \( E = p c_0 \) und \( c_0 = 1 \) in Planck-Einheiten). Für massive Teilchen bleibt \( T = \frac{1}{m} \), und eine hybride Definition \( T = \frac{1}{\max(m, E)} \) vereinheitlicht die Behandlung. Dies führt zu einer modifizierten Schrödinger-Gleichung:
	\[
	i \frac{\partial \psi}{\partial (t/T)} = H \psi,
	\]
	mit \( H = p \) für Photonen und \( H = -\frac{1}{2m} \nabla^2 + V \) für massive Teilchen. Für ein Photon mit Energie \( E = 2\pi \nu \) (wobei \( \nu \) die Frequenz ist) ergibt sich \( T = \frac{1}{2\pi \nu} \), was der Periodendauer entspricht und eine endliche Zeitentwicklung ermöglicht.
	
	\subsection{Folgen für die Nichtlokalität bei Photonen}
	Die Einführung von \( T = \frac{1}{E} \) hat folgende Auswirkungen auf die Nichtlokalität:
	
	\begin{enumerate}
		\item \textbf{Endliche Zeitentwicklung}: Photonen erhalten eine energieabhängige Zeitskala \( T = \frac{1}{E} \), statt eines unendlichen \( T \). Dies ermöglicht eine dynamische Evolution, die von ihrer Frequenz abhängt.
		
		\item \textbf{Verzögerungen in der Verschränkung}: Bei verschränkten Photonen mit Energien \( E_1 \) und \( E_2 \) ergeben sich Zeitskalen \( T_1 = \frac{1}{E_1} \) und \( T_2 = \frac{1}{E_2} \). Unterschiedliche Energien (z. B. \( E_1 \neq E_2 \)) führen zu unterschiedlichen Entwicklungsraten, was eine Verzögerung \( |T_1 - T_2| = \left| \frac{1}{E_1} - \frac{1}{E_2} \right| \) in der Korrelation impliziert. Dies widerspricht der instantanen Korrelation des Standardmodells.
		
		\item \textbf{Photon mit massivem Teilchen}: In einem hybriden System (z. B. Photon und Elektron) hängt die Korrelation von \( T_\text{photon} = \frac{1}{E} \) und \( T_e = \frac{1}{m_e} \) ab. Da \( E \) typischerweise kleiner als \( m_e \) ist (z. B. \( E = 1 \, \text{eV} \) vs. \( m_e \approx 5.11 \times 10^5 \, \text{eV} \)), ist \( T_\text{photon} \gg T_e \), was eine signifikante Verzögerung im Photon-Zustand bedeutet.
		
		\item \textbf{Energieabhängige Nichtlokalität}: Die Korrelation wird energieabhängig, wodurch die Nichtlokalität als emergente Eigenschaft der Energie-Zeit-Beziehung erscheint, ähnlich der Masse-Zeit-Beziehung bei massiven Teilchen.
		
		\item \textbf{Kausalität}: Die Lichtkegelstruktur bleibt durch \( c_0 = 1 \) erhalten, aber die Zeitentwicklung innerhalb des Kegels wird für Photonen durch \( T = \frac{1}{E} \) skaliert, was die Dynamik der Zustandsänderung verändert.
	\end{enumerate}
	
	\subsection{Experimentelle Überprüfung}
	Die Folgen könnten durch Bell-Tests mit Photonen unterschiedlicher Frequenzen getestet werden, um Verzögerungen proportional zu \( \frac{1}{E} \) nachzuweisen. Ebenso könnten Messungen in hybriden Systemen (Photon-Elektron) die unterschiedlichen Zeitskalen \( T = \frac{1}{E} \) vs. \( T = \frac{1}{m} \) bestätigen.
	
	\subsection{Schlussfolgerung}
	Die Erweiterung der Quantenmechanik mit \( T = \frac{1}{E} \) für Photonen integriert masselose Teilchen konsistent in das Modell und verschiebt die Nichtlokalität von einer instantanen zu einer energieabhängigen Dynamik. Dies erfordert eine Neubetrachtung der Korrelationen in verschränkten Systemen und bietet neue experimentelle Ansätze zur Validierung.
	
\end{document}