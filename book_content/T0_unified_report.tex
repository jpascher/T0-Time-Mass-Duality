
\maketitle

\tableofcontents
\newpage

\section{Introduction}

The T0-Theory presents a revolutionary approach where all physical constants and particle masses are derived from only three fundamental geometric parameters. This work presents the complete results of the unified T0 calculator.

\section{Fundamental Input Parameters}

The entire T0-Theory is based on only three input values:

\begin{align}
\xi &= \frac{4}{3} \times 10^{-4} \approx 1.33333333e-04 \text{ (geometric constant)} \\
\ell_\text{P} &= 1.616000e-35 \text{ m (Planck length)} \\
E_0 &= 7.398 \text{ MeV (characteristic energy)} \\
v &= 246.0 \text{ GeV (Higgs VEV, derived from } \xi \text{)}
\end{align}

\subsection{Geometric Derivation of $\xi$}

The geometric constant $\xi$ arises from the fundamental field equation:
\begin{equation}
\nabla^2 m(x,t) = 4\pi G \rho(x,t) \cdot m(x,t)
\end{equation}

For a spherically symmetric point mass, this leads to the characteristic length:
\begin{equation}
r_0 = 2Gm \quad \text{and} \quad \xi = \frac{r_0}{\ell_\text{P}}
\end{equation}

\section{Particle Mass Calculations}

The T0-Theory calculates all particle masses using the Yukawa method:
\begin{equation}
m = r \times \xi^p \times v
\end{equation}

where $r$ and $p$ are particle-specific parameters from the geometric structure.

\begin{longtable}{lccccc}
\caption{T0 Mass Predictions with Exact Fraction Parameters} \\
\toprule
Particle & $r$ & $p$ & T0 Mass [\si{\mega\electronvolt}] & Exp. Mass [\si{\mega\electronvolt}] & Error [\%] \\
\midrule
\endfirsthead
\multicolumn{6}{c}{\tablename\ \thetable\ -- Continuation from previous page} \\
\toprule
Particle & $r$ & $p$ & T0 Mass [\si{\mega\electronvolt}] & Exp. Mass [\si{\mega\electronvolt}] & Error [\%] \\
\midrule
\endhead
\bottomrule
\multicolumn{6}{r}{Continuation on next page} \\
\endfoot
\bottomrule
\endlastfoot
Electron & $\frac{4}{3}$ & $\frac{3}{2}$ & 0.5 & 0.5 & 1.18 \\
Muon & $\frac{16}{5}$ & $1$ & 105.0 & 105.7 & 0.66 \\
Tau & $\frac{8}{3}$ & $\frac{2}{3}$ & 1712.1 & 1776.9 & 3.64 \\
Up & $6$ & $\frac{3}{2}$ & 2.3 & 2.3 & 0.11 \\
Down & $\frac{25}{2}$ & $\frac{3}{2}$ & 4.7 & 4.7 & 0.30 \\
Strange & $\frac{26}{9}$ & $1$ & 94.8 & 93.4 & 1.45 \\
Charm & $2$ & $\frac{2}{3}$ & 1284.1 & 1270.0 & 1.11 \\
Bottom & $\frac{3}{2}$ & $\frac{1}{2}$ & 4260.8 & 4180.0 & 1.93 \\
Top & $\frac{1}{28}$ & $\frac{-1}{3}$ & 171974.5 & 172760.0 & 0.45 \\
\end{longtable}

\subsection{Statistical Analysis of Mass Results}

The T0-Theory achieves remarkable accuracy in predicting particle masses:

\begin{itemize}
\item Number of calculated particles: 9
\item Average error: 1.20\%
\item Best prediction: up (0.11\% error)
\item All masses calculated from only 3 parameters
\end{itemize}

\section{Physical Constants}

The T0-Theory systematically derives all fundamental physical constants in an 8-level hierarchy:

\subsection{Level 1: Primary Derivations}
\begin{align}
\alpha &= \xi \left(\frac{E_0}{1 \text{ MeV}}\right)^2 = 7.297387e-03 \\
m_{\text{char}} &= \frac{\xi}{2} = 6.666667e-05
\end{align}

\subsection{Level 2: Gravitational Constant}

The gravitational constant is directly derived from $\xi$:
\begin{align}
G_{\text{nat}} &= \frac{\xi^2}{4 m_{\text{char}}} = \frac{\xi}{2} = 6.666667e-05 \text{ (dimensionless)} \\
G &= G_{\text{nat}} \times \frac{\ell_\text{P}^2 c^3}{\hbar} = 6.672194e-11 \text{ \si{\cubic\meter\per\kilogram\per\second\squared}}
\end{align}

\subsection{Overview of All Calculated Constants}

\begin{longtable}{p{1.5cm}p{3cm}S[table-format=1.6e2]S[table-format=1.6e2]S[table-format=2.4]}
\caption{T0 Constant Calculations by Hierarchy Level} \\
\toprule
{Level} & {Constant} & {T0 Value} & {Reference Value} & {Error [\%]} \\
\midrule
\endfirsthead
\multicolumn{5}{c}{\tablename\ \thetable\ -- Continuation from previous page} \\
\toprule
{Level} & {Constant} & {T0 Value} & {Reference Value} & {Error [\%]} \\
\midrule
\endhead
\bottomrule
\multicolumn{5}{r}{Continuation on next page} \\
\endfoot
\bottomrule
\endlastfoot
1 & $\alpha$ & 7.297387e-03 & 7.297353e-03 & 0.0005 \\
1 & $m_{\text{char}}$ & 6.666667e-05 & {T0-derived} & {-} \\
2 & $G$ & 6.672194e-11 & 6.674300e-11 & 0.0316 \\
2 & $G_{\text{nat}}$ & 6.666667e-05 & {T0-derived} & {-} \\
2 & $G_{\text{conversion factor}}$ & 6.672194e-11 & {T0-derived} & {-} \\
3 & $c$ & 2.997925e+08 & 2.997925e+08 & 0.0000 \\
3 & $\hbar$ & 1.054572e-34 & 1.054572e-34 & 0.0000 \\
3 & $m_{\text{P}}$ & 2.176778e-08 & 2.176434e-08 & 0.0158 \\
3 & $t_{\text{P}}$ & 5.390396e-44 & 5.391247e-44 & 0.0158 \\
3 & $T_{\text{P}}$ & 1.417008e+32 & 1.416784e+32 & 0.0158 \\
3 & $E_{\text{P}}$ & 1.956390e+09 & 1.956082e+09 & 0.0158 \\
3 & $F_{\text{P}}$ & 1.210638e+44 & 1.210256e+44 & 0.0315 \\
3 & $P_{\text{P}}$ & 3.629400e+52 & 3.628255e+52 & 0.0316 \\
4 & $\mu_0$ & 1.256637e-06 & 1.256637e-06 & 0.0000 \\
4 & $\epsilon_0$ & 8.854188e-12 & 8.854188e-12 & 0.0000 \\
4 & $e$ & 1.602180e-19 & 1.602177e-19 & 0.0002 \\
4 & $Z_0$ & 3.767303e+02 & 3.767303e+02 & 0.0000 \\
4 & $k_{\text{e}}$ & 8.987552e+09 & 8.987552e+09 & 0.0000 \\
5 & $\sigma_{\text{SB}}$ & 5.670374e-08 & 5.670374e-08 & 0.0000 \\
5 & $b_{\text{Wien}}$ & 2.897839e-03 & 2.897772e-03 & 0.0023 \\
5 & $h$ & 6.626070e-34 & 6.626070e-34 & 0.0000 \\
6 & $a_0$ & 5.291747e-11 & 5.291772e-11 & 0.0005 \\
6 & $R_{\infty}$ & 1.097384e+07 & 1.097373e+07 & 0.0009 \\
6 & $\mu_{\text{B}}$ & 9.274032e-24 & 9.274010e-24 & 0.0002 \\
6 & $\mu_{\text{N}}$ & 5.050796e-27 & 5.050784e-27 & 0.0002 \\
6 & $E_{\text{h}}$ & 4.359786e-18 & 4.359745e-18 & 0.0009 \\
6 & $\lambda_{\text{C}}$ & 2.426310e-12 & 2.426310e-12 & 0.0000 \\
6 & $r_{\text{e}}$ & 2.817954e-15 & 2.817940e-15 & 0.0005 \\
7 & $F$ & 9.648556e+04 & 9.648533e+04 & 0.0002 \\
7 & $R_{\text{K}}$ & 2.581268e+04 & 2.581281e+04 & 0.0005 \\
7 & $K_{\text{J}}$ & 4.835990e+14 & 4.835978e+14 & 0.0002 \\
7 & $\Phi_0$ & 2.067829e-15 & 2.067834e-15 & 0.0002 \\
7 & $R_{\text{gas}}$ & 8.314463e+00 & 8.314463e+00 & 0.0000 \\
8 & $H_0$ & 2.196000e-18 & {T0-derived} & {-} \\
8 & $\Lambda$ & 1.609698e-52 & {T0-derived} & {-} \\
8 & $t_{\text{universe}}$ & 4.553734e+17 & {T0-derived} & {-} \\
8 & $\rho_{\text{crit}}$ & 8.627350e-27 & {T0-derived} & {-} \\
8 & $l_{\text{Hubble}}$ & 1.365175e+26 & {T0-derived} & {-} \\
\end{longtable}

\section{Summary}

\subsection{Key Results}

The T0-Theory achieves a remarkable unification of physics:

\begin{enumerate}
\item \textbf{Complete Mass Calculation}: All 9 particle masses from geometric principles
\item \textbf{Constant Hierarchy}: 39 physical constants derived in 8 levels
\item \textbf{High Precision}: Average mass error only 1.2 \%
\item \textbf{Minimal Input}: Only 3 fundamental parameters required
\item \textbf{Open Source}: All documents and source code are available at \url{https://github.com/jpascher/T0-Time-Mass-Duality} under the MIT License.
\end{enumerate}

\section{Conclusion}

The T0 Unified Calculator demonstrates that geometric principles can lead to astonishingly accurate predictions in particle physics. The numerical accuracy warrants scientific attention.

\vfill
\begin{center}
\textit{Generated on \today\ with the T0 Unified Calculator v3.0}\\
\textit{Johann Pascher, HTL Leonding, Austria}
\end{center}

