\documentclass[a4paper,12pt]{article}
\usepackage{amsmath,amssymb}
\usepackage{hyperref}
\usepackage{graphicx}
\usepackage{float}

\title{Vollständige deutsche Übersetzung eines wissenschaftlichen Textes: Minkowskische und galileische Raumzeit}
\author{}
\date{}

\begin{document}

\maketitle

\section*{Einleitung}
Wir betrachten als „Apparat“ jedes Gerät, das die „Einheiten“ festlegt, die zur Beschreibung von Observablen erforderlich sind. Beispiele hierfür sind:
\begin{itemize}
    \item das \textit{Internationale Prototyp des Meters} (IPM) und
    \item das \textit{Internationale Prototyp der Sekunde} (IPS), der mit $9,192,631,770$ Perioden der Strahlung assoziiert ist, die dem Übergang zwischen zwei Hyperfeinzuständen des Grundzustands des Cäsium-133-Atoms entspricht.
\end{itemize}
Diese Einheiten werden durch echte Messinstrumente definiert: Lineale und Uhren. Ihre Existenz ist keine Option, sondern eine Notwendigkeit, um das Verständnis der zugrunde liegenden Raumzeit zu ermöglichen. Das Fehlen einer solchen Beobachtung führte in der Vergangenheit zu erheblichen Missverständnissen.

\section*{Strategie}
Unsere Strategie zur Beantwortung der oben gestellten Frage besteht darin, vom Konzept der Raumzeit (sei es relativistisch oder nicht) auszugehen, auf dem alle anderen Theorien basieren. Um eine Raumzeit zu konstruieren, sind einige Apparate erforderlich. Minkowski- und andere relativistische Raumzeiten verlangen die Existenz echter Uhren, während die galileische Raumzeit auch Lineale benötigt. Die durch diese Apparate definierten Einheiten sind ausreichend, um alle Observablen der entsprechenden physikalischen Gesetze auszudrücken. Dies führt zu der Schlussfolgerung, dass die Anzahl fundamentaler Konstanten (im Sinne von DOV) zwei in der galileischen Raumzeit und eine in relativistischen Raumzeiten beträgt.

\section*{Minkowskische Raumzeit und Gleichungen}
Die Beziehung zwischen Ereignissen setzt die Minkowskische Linienmetrik voraus:
\[
\mathrm{d}s^2 = -\mathrm{d}t^2 + \mathrm{d}x^2 + \mathrm{d}y^2 + \mathrm{d}z^2,
\]
wobei die Zeitabstände $\tau_{RP}$ und $\tau_{PS}$ wie folgt definiert sind:
\[
\Delta \tau_{RP} = \sqrt{\frac{1 + v}{1 - v}} (t_Q - x_Q), \quad \Delta \tau_{PS} = \sqrt{\frac{1 - v}{1 + v}} (t_Q + x_Q).
\]
Das Produkt dieser Zeiten ergibt:
\[
\Delta \tau_{RP} \cdot \Delta \tau_{PS} = -(t_Q - t_P)^2 + (x_Q - x_P)^2,
\]
wobei $t_P = x_P = 0$ angenommen wird. Dies entspricht Gleichung (4).

\section*{Die Länge eines ruhenden Stabes}
Die Länge $D$ eines ruhenden Stabes wird mit drei identischen Uhren gemessen: Eine Uhr $C_1$ misst die hinführende Zeit $\tau_1$, $C_2$ die rückführende Zeit $\tau_2$, und $C_3$ die Gesamtzeit $\tau$. Die Formel lautet:
\[
D = \frac{\sqrt{(\tau^2 - \tau_1^2 - \tau_2^2)^2 - 4\tau_1^2\tau_2^2}}{2\tau}.
\]
Dies basiert auf relativistischer Zeitmessung und zeigt, dass $D$ unabhängig von den Geschwindigkeiten $v_1$ und $v_2$ ist.

\section*{Fazit}
Es wurde gezeigt, dass alle Observablen in relativistischen Raumzeiten mit einer einzigen fundamentalen Einheit (Zeit) beschrieben werden können. In galileischen Raumzeiten hingegen sind zwei Einheiten erforderlich: Raum und Zeit. Diese Reduktion ermöglicht eine minimalistische Beschreibung der physikalischen Gesetze.

\end{document}
