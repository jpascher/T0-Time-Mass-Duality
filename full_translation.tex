\documentclass[a4paper,12pt]{article}
\usepackage{amsmath,amssymb}
\usepackage{hyperref}
\usepackage{graphicx}
\usepackage{float}

\title{Vollständige deutsche Übersetzung eines wissenschaftlichen Textes: Minkowskische und galileische Raumzeit}
\author{}
\date{}

\begin{document}

\maketitle

\section*{Einleitung}
Die zentrale Frage in diesem Text betrifft die fundamentalen Einheiten und Konstanten der Physik im Rahmen galileischer und minkowskischer Raumzeiten. Ziel ist es, den physikalischen Bezug zwischen Zeit, Raum und Masse in verschiedenen metrologischen Systemen zu analysieren.

\section*{Zwei Einheiten in galileischen Raumzeiten}
Im vorherigen Abschnitt haben wir festgestellt, dass das MKS-System (Meter-Kilogramm-Sekunde) ausreicht, um alle physikalischen Observablen auszudrücken. Nun zeigen wir, dass in galileischen Raumzeiten alle Observablen ausschließlich in den Einheiten Zeit und Raum ausgedrückt werden können.

Vor 2019 wurde das Kilogramm durch einen Zylinder aus Platin-Iridium definiert, der im Internationalen Büro für Maß und Gewicht (BIPM) aufbewahrt wurde. Seit 2019 wird das Kilogramm jedoch durch eine Festlegung des Planckschen Wirkungsquantums definiert:
\[
    h = 6.62607015 \times 10^{-34} \ \mathrm{J \cdot s},
\]
während der Wert der Newtonschen Gravitationskonstanten $G$ gemessen wird. Dies ist zwar aus metrologischer Sicht sinnvoll, es wäre jedoch konzeptionell besser, das Kilogramm durch eine Festlegung von $G$ zu definieren.

Das Plancksche Wirkungsquantum $h$ gibt die Skala des Spins von Elementarteilchen an, während $G$ allein keine natürliche Skala bietet. Die physikalische Größe, die für die Gravitation zwischen Körpern verantwortlich ist, ist $GM$, mit den Einheiten:
\[
    \mathrm{m^3 \cdot s^{-2}},
\]
die durch Uhren und Lineale gemessen werden kann.

\subsection*{Reformulierung aller Observablen}
Hätte man $G$ nicht eingeführt, könnten alle physikalischen Größen im sogenannten MS-System ausgedrückt werden. Dies ist eine Reduktion des üblichen MKS-Systems, bei der Masse durch Zeit und Raum ersetzt wird. Die Umrechnung lautet:
\[
    O^{MS}_i = G^{\gamma_i} \cdot O_i = \ell_i \cdot \mathrm{m}^{\alpha_i + 3\gamma_i} \cdot \mathrm{s}^{\beta_i - 2\gamma_i}.
\]

\section*{Eine Einheit in minkowskischen Raumzeiten}
In minkowskischen Raumzeiten können Raum und Zeit miteinander verbunden werden. Dadurch lässt sich zeigen, dass alle Observablen in Einheiten der Zeit ausgedrückt werden können.

\subsection*{Protokoll zur Längenmessung}
Ein elegantes Protokoll von Unruh beschreibt, wie die Länge eines ruhenden Stabes allein durch Zeitmessungen bestimmt werden kann. Mit drei identischen Uhren $C_1$, $C_2$ und $C_3$ wird folgende Methode angewandt:
\begin{enumerate}
    \item $C_1$ wird zurückgesetzt und frei entlang des Stabes geschickt. Die gemessene Zeit sei $\tau_1$.
    \item $C_2$ startet nach Ankunft von $C_1$ am anderen Ende und kehrt zurück. Die gemessene Zeit sei $\tau_2$.
    \item $C_3$ misst die Gesamtzeit $\tau$ zwischen Abgang von $C_1$ und Rückkehr von $C_2$.
\end{enumerate}
Die Länge $D$ des Stabes ergibt sich aus:
\[
    D = \frac{\sqrt{(\tau^2 - \tau_1^2 - \tau_2^2)^2 - 4\tau_1^2\tau_2^2}}{2\tau}.
\]

\subsection*{Interpretation und Umrechnung}
Die Geschwindigkeit des Lichts $c \approx 299,792,458 \ \mathrm{m/s}$ ist als Umrechnungsfaktor zwischen Zeit- und Längeneinheiten zu betrachten. In geometrisierten Einheiten, wie sie in der Hochenergiephysik verwendet werden, gilt:
\[
    G = c = \hbar = k_B = 1.
\]

\section*{Schlussfolgerung}
Wir haben gezeigt, dass alle Observablen in minkowskischen Raumzeiten in einer einzigen fundamentalen Einheit ausgedrückt werden können: der Zeit. In galileischen Raumzeiten sind zwei Einheiten erforderlich: Zeit und Raum. Diese Erkenntnisse ermöglichen eine minimale, aber vollständige Beschreibung der Naturgesetze.

\end{document}