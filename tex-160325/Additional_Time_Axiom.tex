
\section{The Nature of Time as a Physical Parameter}

In many physical theories, time appears merely as a mathematical parameter that describes the sequence of events. It is not an independent physical substance like energy or mass. To avoid misunderstandings in the interpretation of physical equations, the role of time should be explicitly formulated.

\subsection{Axiomatic Definition of Time}

To ensure a correct interpretation of physical theories, we propose the following axiom regarding time:

\begin{itemize}
    \item Time is a mathematical parameter that describes the order of events but does not constitute a physical substance itself.
    \item Time has a natural direction, which manifests macroscopically through the irreversibility of processes.
    \item The mathematical reversibility of time in many physical equations does not imply that processes are reversible in reality.
    \item Within a system, its own time is always experienced as constant, regardless of an external observer's perspective.
    \item Any physical interpretation of time must account for its relational existence rather than treating it as absolute.
\end{itemize}

\subsection{Practical Consequences}

Although time is often treated symmetrically in physical equations, a macroscopically observable direction of time exists. The cause-effect relationship, the thermodynamic arrow of time, and the expansion of the universe indicate that time has a distinct direction in real systems. The introduction of this axiom ensures that time is not misinterpreted in physical theories or equated with substantive quantities like energy or matter.

