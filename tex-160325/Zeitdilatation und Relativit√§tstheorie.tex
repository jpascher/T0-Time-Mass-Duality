\documentclass[a4paper,12pt]{article}
\usepackage[utf8]{inputenc}
\usepackage{amsmath, amssymb}

\title{Zeitdilatation und Relativitätstheorie}
\author{Johann Pascher}
\date{März 19, 2025}

\begin{document}
	
	\maketitle
	
	\section{Einführung in die Zeitdilatation}
	Die Diskussion begann mit der Frage nach zusätzlichen Aspekten der Zeitdilatation in der Relativitätstheorie (RT), die über die Standardformeln hinausgehen.
	
	\subsection{Spezielle Relativitätstheorie (SRT)}
	Die Zeitdilatation durch Geschwindigkeit wird beschrieben durch:
	\[
	t = \frac{t_0}{\sqrt{1 - \frac{v^2}{c^2}}}
	\]
	wobei \(t_0\) die Eigenzeit, \(v\) die Geschwindigkeit und \(c\) die Lichtgeschwindigkeit ist.
	
	\subsection{Allgemeine Relativitätstheorie (ART)}
	Die gravitative Zeitdilatation lautet:
	\[
	t = \frac{t_0}{\sqrt{1 - \frac{2GM}{rc^2}}}
	\]
	mit \(G\) als Gravitationskonstante, \(M\) als Masse und \(r\) als Abstand.
	
	\section{Weitere Aspekte der Zeitdilatation}
	Zusätzliche Effekte umfassen:
	\begin{itemize}
		\item \textbf{Rahmenverschleppung}: Durch Rotation, proportional zu \(\frac{J}{c^2 r^3}\).
		\item \textbf{Kosmologische Zeitdilatation}: Durch die Expansion, \(t = t_0 (1 + z)\), mit \(z\) als Rotverschiebung.
	\end{itemize}
	
	\section{Lichtgeschwindigkeit ohne RT}
	Die Lichtgeschwindigkeit \(c\) lässt sich aus den Maxwell-Gleichungen ableiten:
	\[
	c = \frac{1}{\sqrt{\mu_0 \varepsilon_0}}
	\]
	Dies ist ein klassischer Ansatz, unabhängig von der RT.
	
	\section{RT ohne Verformung}
	Ohne Gravitation (ART) reduziert sich die Raumzeit zur flachen Minkowski-Raumzeit der SRT. Ohne Bewegung entfällt die RT zugunsten klassischer Physik.
	
\end{document}