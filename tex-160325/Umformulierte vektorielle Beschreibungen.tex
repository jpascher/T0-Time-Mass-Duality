\documentclass{article}
\usepackage[utf8]{inputenc}
\usepackage{amsmath}
\usepackage{amssymb}
\usepackage{hyperref}
\usepackage{geometry}
\geometry{a4paper, margin=2cm}

\title{Umformulierte vektorielle Beschreibungen zur Integration der Gravitation als vierte Kraft unter Berücksichtigung des Higgs-Feldes}
\author{Johann Pascher}
\date{15.3.2025}

\begin{document}
	
	\maketitle
	
	\section{Einleitung}
	
	Das Higgs-Feld ist ein **skalares Quantenfeld**, das im Standardmodell der Teilchenphysik eine zentrale Rolle spielt. Es ist verantwortlich für die Erzeugung der Massen der Elementarteilchen durch den Higgs-Mechanismus. In diesem Dokument werden die vektoriellen Beschreibungen der vier fundamentalen Kräfte (starke Kernkraft, elektromagnetische Kraft, schwache Kernkraft und Gravitation) so umformuliert, dass die Rolle des Higgs-Feldes als skalares Quantenfeld korrekt berücksichtigt wird. Dies ermöglicht eine konsistente Integration der Gravitation als vierte Kraft.
	
	\section{Das Higgs-Feld als skalares Quantenfeld}
	
	Das Higgs-Feld ist ein komplexes skalares Quantenfeld mit vier reellen Komponenten, das im Standardmodell durch den Higgs-Mechanismus die Massen der Eichbosonen (W- und Z-Bosonen) und der Fermionen (Quarks und Leptonen) erzeugt. Die Lagrange-Dichte des Higgs-Feldes lautet:
	
	\begin{equation}
		\mathcal{L}_\text{Higgs} = (D_\mu \phi)^\dagger (D^\mu \phi) - V(\phi),
	\end{equation}
	
	wobei $\phi$ das Higgs-Feld ist, $D_\mu$ die kovariante Ableitung darstellt und $V(\phi)$ das Higgs-Potential beschreibt:
	
	\begin{equation}
		V(\phi) = \mu^2 \phi^\dagger \phi + \lambda (\phi^\dagger \phi)^2.
	\end{equation}
	
	Das Higgs-Feld hat einen nicht-verschwindenden Vakuumerwartungswert (VEV), der die spontane Symmetriebrechung des elektroschwachen Sektors bewirkt und den Teilchen ihre Masse verleiht.
	
	\section{Die starke Kernkraft}
	
	\subsection{Klassische Beschreibung}
	Die starke Kernkraft hält Quarks zusammen, um Protonen und Neutronen zu bilden. Die Massen der Quarks werden oft als gegebene Parameter behandelt.
	
	\subsection{Gravitationskompatible Beschreibung}
	In einer gravitationskompatiblen Beschreibung wird die Masse der Quarks durch das Higgs-Feld und die Raumzeitkrümmung beeinflusst. Die starke Kernkraft kann dann als Wechselwirkung beschrieben werden, die sowohl von der Farbladung als auch von der durch die Gravitation verursachten Raumzeitkrümmung abhängt.
	
	\begin{equation}
		\mathcal{L}_\text{stark} = -\frac{1}{4} F_{\mu\nu}^a F^{a\mu\nu} + \bar{\psi}(i \gamma^\mu D_\mu - m_\psi(\phi, g_{\mu\nu}))\psi,
	\end{equation}
	
	wobei $m_\psi(\phi, g_{\mu\nu})$ die Masse des Quarks ist, die vom Higgs-Feld $\phi$ und der Metrik $g_{\mu\nu}$ abhängt.
	
	\section{Die elektromagnetische Kraft}
	
	\subsection{Klassische Beschreibung}
	Die elektromagnetische Kraft beschreibt die Wechselwirkung zwischen geladenen Teilchen, wobei die Masse der Teilchen als gegebener Parameter behandelt wird.
	
	\subsection{Gravitationskompatible Beschreibung}
	In einer gravitationskompatiblen Beschreibung wird die Masse der geladenen Teilchen durch das Higgs-Feld und die Raumzeitkrümmung beeinflusst. Die elektromagnetische Kraft kann dann als Wechselwirkung beschrieben werden, die sowohl von der elektrischen Ladung als auch von der Raumzeitkrümmung abhängt.
	
	\begin{equation}
		\mathcal{L}_\text{em} = -\frac{1}{4} F_{\mu\nu} F^{\mu\nu} + \bar{\psi}(i \gamma^\mu D_\mu - m_\psi(\phi, g_{\mu\nu}))\psi,
	\end{equation}
	
	wobei $m_\psi(\phi, g_{\mu\nu})$ die Masse des geladenen Teilchens ist, die vom Higgs-Feld $\phi$ und der Metrik $g_{\mu\nu}$ abhängt.
	
	\section{Die schwache Kernkraft}
	
	\subsection{Klassische Beschreibung}
	Die schwache Kernkraft ist verantwortlich für bestimmte Arten von radioaktivem Zerfall. Die Massen der W- und Z-Bosonen sowie der beteiligten Teilchen werden oft als gegebene Parameter behandelt.
	
	\subsection{Gravitationskompatible Beschreibung}
	In einer gravitationskompatiblen Beschreibung werden die Massen der W- und Z-Bosonen sowie der beteiligten Teilchen durch das Higgs-Feld und die Raumzeitkrümmung beeinflusst. Die schwache Kernkraft kann dann als Wechselwirkung beschrieben werden, die sowohl von den schwachen Ladungen als auch von der Raumzeitkrümmung abhängt.
	
	\begin{equation}
		\mathcal{L}_\text{schwach} = -\frac{1}{4} W_{\mu\nu}^a W^{a\mu\nu} + \bar{\psi}(i \gamma^\mu D_\mu - m_\psi(\phi, g_{\mu\nu}))\psi,
	\end{equation}
	
	wobei $m_\psi(\phi, g_{\mu\nu})$ die Masse der beteiligten Teilchen ist, die vom Higgs-Feld $\phi$ und der Metrik $g_{\mu\nu}$ abhängt.
	
	\section{Die Gravitation}
	
	\subsection{Klassische Beschreibung}
	Die Gravitation wird in der Allgemeinen Relativitätstheorie durch die Krümmung der Raumzeit beschrieben, die durch den Energie-Impuls-Tensor verursacht wird.
	
	\subsection{Gravitationskompatible Beschreibung}
	In einer gravitationskompatiblen Beschreibung wird die Masse als Quelle der Raumzeitkrümmung betrachtet, die durch den Energie-Impuls-Tensor beschrieben wird. Die Gravitation kann dann als Wechselwirkung beschrieben werden, die sowohl von der Masse als auch von der Raumzeitkrümmung abhängt.
	
	\begin{equation}
		\mathcal{L}_\text{grav} = -\frac{1}{16\pi G} \sqrt{-g}R + \mathcal{L}_\text{Materie}(g_{\mu\nu}, \phi),
	\end{equation}
	
	wobei $\mathcal{L}_\text{Materie}(g_{\mu\nu}, \phi)$ die Lagrange-Dichte der Materie ist, die von der Metrik $g_{\mu\nu}$ und dem Higgs-Feld $\phi$ abhängt.
	
	\section{Zusammenfassung}
	
	Durch die Umformulierung der vektoriellen Beschreibungen der vier fundamentalen Kräfte wird die Rolle des Higgs-Feldes als skalares Quantenfeld korrekt berücksichtigt. Die Masse der Teilchen wird als dynamische Größe betrachtet, die vom Higgs-Feld und der Raumzeitkrümmung beeinflusst wird. Dies ermöglicht eine konsistente Integration der Gravitation als vierte Kraft und führt zu einer einheitlichen Beschreibung aller vier fundamentalen Kräfte.
	
\end{document}