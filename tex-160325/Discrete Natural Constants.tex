\documentclass{article}
\usepackage{amsmath, amssymb}

\title{Discrete Natural Constants and Stable Physical States}
\author{Johann Pascher}
\date{25.2.2025}

\begin{document}
	
	\maketitle
	\tableofcontents
	\section{Introduction: The Reducibility of All Constants to Energy}
	
	In natural unit systems, particularly those in which the speed of light \( c \) and the reduced Planck constant \( \hbar \) are set to 1, all physical units can be reduced to energy units. Special relativity provides a central foundation: through the relation \( E = mc^2 \), it becomes evident that mass can be directly expressed in energy units. Similarly, length and time can be linked to energy via \( c \):
	
	\begin{itemize}
		\item Length \( L \): \( L \sim \frac{1}{E} \)
		\item Time \( T \): \( T \sim \frac{1}{E} \)
		\item Mass \( m \): \( m \sim E \)
	\end{itemize}
	
	Since the coupling constants of fundamental interactions also exhibit energy dependencies, it becomes apparent that all fundamental physical quantities can ultimately be traced back to a fundamental energy scale.
	
	\section{The Idea of Discrete Natural Constants}
	
	The previous consideration of natural constants is based on the assumption that they can vary continuously or are indeed constant. However, if all fundamental constants can be reduced to a fundamental energy quantity, the question arises as to whether this energy exists in continuous or discrete form. Based on quantum mechanical principles, the hypothesis emerges that even seemingly constant natural laws follow discrete rules.
	
	\section{Discrete States of the Constants}
	
	Analogous to quantum mechanics, where electrons can only occupy certain energy levels, it is conceivable that fundamental constants can also take on only discrete values. The observed stability of physical processes would then be the result of particularly stable discrete configuration spaces of the constants. In an expanding universe, transitions between these states could occur in jumps rather than through continuous variation.
	
	\section{Consequences for Physics}
	
	\begin{enumerate}
		\item \textbf{Energy quantities as the primary physical entity}: All other quantities are derived from it and can only take discrete values.
		\item \textbf{Cosmological variability of constants}: If the universe expands, natural constants might not change continuously but rather in jumps.
		\item \textbf{Explanation for observed stabilities}: The stability of physical processes could result from the fact that certain discrete states are energetically favored.
		\item \textbf{Possible observations}: If natural constants have discrete states, abrupt changes might be observable in astrophysical measurements.
	\end{enumerate}
	
	\section{Implications for the Standard Model of Physics}
	
	This perspective could have profound implications for the Standard Model of physics, particularly in the following areas:
	
	\begin{enumerate}
		\item \textbf{Running coupling constants and energy scales}: In the Standard Model, interactions between particles are described by coupling constants that change with energy scale ("running"). If these coupling constants instead take on discrete states, renormalization group theory would need to be revised to account for jumps rather than continuous changes.
		\item \textbf{Electroweak unification and Grand Unified Theories (GUTs)}: Theories extending the Standard Model beyond electroweak unification assume that coupling constants converge at high energies. If these constants do not run continuously but take on only specific discrete values, unification scenarios may need to be reinterpreted.
		\item \textbf{Particle masses and the Higgs mechanism}: In the Standard Model, particles acquire their mass through the Higgs mechanism. If natural constants exist in discrete states, it could mean that the generated masses can also take only specific values. This could provide alternative explanations for observed mass spectra or enable new predictions.
		\item \textbf{Cosmological variability and the constancy of natural laws}: The Standard Model assumes constant natural laws. However, if discrete jumps in constants occur due to energy density differences in different regions of the universe, this could lead to locally varying physical laws. This could be tested through high-precision measurements in various cosmic environments.
		\item \textbf{Dark matter and dark energy}: If certain discrete energy values are preferred, this could provide new explanations for the existence of dark matter and dark energy. It is conceivable that these phenomena are expressions of transitions between stable energy states rather than unknown particles or fields.
	\end{enumerate}
	
	\section{Conclusion}
	
	If all fundamental constants can be traced back to energy and exist in discrete states, it follows that the apparent continuity of physical laws might be an illusion. Instead, the universe could be structured through stable, preferred configurations of natural constants that change only abruptly. This hypothesis presents a new perspective on the structure of natural laws and could be tested in future experiments.
	\appendix
	\section{Mathematische Modelle zur Beschreibung diskreter Naturkonstanten}
	
	Ein m\"oglicher Ansatz zur Modellierung der Naturkonstanten als dynamische, diskrete Felder ist die Definition von Tensorfeldern \( \alpha_i \), die durch partielle Differentialgleichungen beschrieben werden:
	
	\begin{equation}
		\square \alpha_i + f(\alpha_j, T_{\mu\nu}, R_{\mu\nu}) = 0,
	\end{equation}
	
	wobei \( T_{\mu\nu} \) der Energie-Impuls-Tensor und \( R_{\mu\nu} \) der Ricci-Tensor der Raumzeit ist. Weiterhin kann ein Potential \( V(\alpha) \) eingef\"uhrt werden:
	
	\begin{equation}
		\square \alpha_i = - \frac{dV}{d\alpha_i}.
	\end{equation}
	
	Diese Gleichungen beschreiben stabile diskrete Zust\"ande der Konstanten durch lokale Minima von \( V(\alpha) \).
	
\end{document}
