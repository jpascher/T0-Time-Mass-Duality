\documentclass[12pt,a4paper]{article}
\usepackage[utf8]{inputenc}
\usepackage{amsmath}
\usepackage{amsfonts}
\usepackage{amssymb}
\usepackage{graphicx}
\usepackage{hyperref}
\usepackage{physics}
\usepackage{siunitx}
\usepackage{booktabs}

\title{Derivation of Maxwell's Equations from a Fundamental Vector Field $\psi$ with Enhanced Consistency and Historical Context}
\author{Johann Pascher}
\date{24.2.2025}

\begin{document}
	\maketitle
	
	\section{Historical Introduction}
	Maxwell's equations, formulated by James Clerk Maxwell in the 1860s, unified the previously separate theories of electricity and magnetism. This unification not only led to a deeper understanding of electromagnetic phenomena but also predicted the existence of electromagnetic waves, which were later experimentally verified by Heinrich Hertz.
	
	Our modern derivation of these equations is based on symmetry principles and field theory --- concepts that were developed after Maxwell's original work. This approach demonstrates the fundamental role of symmetries in physics and shows how mathematical elegance leads to physical insight.
	
	\section{Fundamental Principles}
	\subsection{Basic Field Equation and Generalization}
	The fundamental field equation in its original form is:
	\begin{equation}
		\left(\frac{\partial^2}{\partial t^2} - c^2\nabla^2 + V'(\psi_\nu \psi^\nu)\right)\psi_\mu = J_\mu
	\end{equation}
	
	This equation represents a special case of a more general structure. In its complete form, the theory can also include higher-order non-linear self-interactions:
	\begin{equation}
		\partial_\alpha F^{\alpha\mu} + \frac{\partial V}{\partial(F^2)} \partial_\alpha (F^{\alpha\mu} F^{\rho\sigma}F_{\rho\sigma}) = J^\mu
	\end{equation}
	where $F_{\mu\nu} = \partial_\mu \psi_\nu - \partial_\nu \psi_\mu$ is the field strength tensor. The original form is sufficient for deriving classical Maxwell's equations, while the extended form can describe additional phenomena such as non-linear quantum effects.
	
	\subsection{Revised Linearization}
	Direct identification of the potential:
	\begin{equation}
		\psi_\mu \equiv A_\mu = (\phi/c, \vec{A})
	\end{equation}
	with the relationships:
	\begin{align*}
		\vec{E} &= -\nabla\phi - \partial_t\vec{A} \\
		\vec{B} &= \nabla\times\vec{A}
	\end{align*}
	
	\section{Mathematical Derivation}
	\subsection{Consistent Lagrangian Density}
	\begin{equation}
		\mathcal{L} = -\frac{1}{4\mu_0}F^{\mu\nu}F_{\mu\nu} + V(F^2) - J^\mu A_\mu
	\end{equation}
	The Euler-Lagrange equations directly yield:
	\begin{equation}
		\partial_\mu\left(\frac{\partial\mathcal{L}}{\partial(\partial_\mu A_\nu)}\right) - \frac{\partial\mathcal{L}}{\partial A_\nu} = 0
	\end{equation}
	
	\section{Experimental Verification}
	\subsection{Relevant Experiments}
	\begin{itemize}
		\item Hertzian Dipole (1887): Generation of electromagnetic waves
		\item Hall Effect (1879): Charge carrier motion
		\item Quantum Hall Effect (1980): Topological order
	\end{itemize}
	
	\section{Critical Analysis}
	\subsection{Limitations}
	\begin{table}[h]
		\centering
		\begin{tabular}{@{}lll@{}}
			\toprule
			Aspect & Classical & Quantum Field Theory \\
			\midrule
			Validity Range & $\lambda \gg \lambda_C$ & All scales \\
			Self-energy & Divergent & Renormalizable \\
			Charge Quantization & No & Yes \\
			\bottomrule
		\end{tabular}
		\caption{Comparison of Theories}
	\end{table}
	
	\section{Applications and Implications}
	\subsection{Classical Applications}
	\begin{itemize}
		\item Electromagnetic wave propagation
		\item Antenna theory and design
		\item Optical phenomena
	\end{itemize}
	
	\subsection{Quantum Extensions}
	\begin{itemize}
		\item Quantum electrodynamics (QED)
		\item Cavity quantum electrodynamics
		\item Quantum optics
	\end{itemize}
	
	\section{Modern Developments}
	\subsection{Theoretical Extensions}
	\begin{itemize}
		\item Gauge field theories
		\item Electroweak unification
		\item String theory approaches
	\end{itemize}
	
	\subsection{Experimental Frontiers}
	\begin{itemize}
		\item Precision QED tests
		\item Quantum Hall systems
		\item Topological phases of matter
	\end{itemize}
	
	\section{References}
	\subsection{Essential Literature}
	\begin{itemize}
		\item Jackson: Classical Electrodynamics (Wiley)
		\item Weinberg: The Quantum Theory of Fields Vol. I (Cambridge)
		\item Feynman: Lectures on Physics Vol. II (Addison-Wesley)
	\end{itemize}
	
	\subsection{Advanced Topics}
	\begin{itemize}
		\item Peskin \& Schroeder: An Introduction to Quantum Field Theory
		\item Weinberg: The Quantum Theory of Fields Vol. II
		\item Zee: Quantum Field Theory in a Nutshell
	\end{itemize}
	
\end{document}