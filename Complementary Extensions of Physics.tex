\documentclass[a4paper,12pt]{article}
\usepackage[utf8]{inputenc}
\usepackage[english]{babel} % For English language and hyphenation
\usepackage{amsmath, amssymb}
\usepackage{graphicx}
\usepackage{hyperref} % For references to external documents

\begin{document}
	
	\title{Complementary Extensions of Physics: Absolute Time and Intrinsic Time}
	\author{Johann Pascher} % Your name as author
	\date{March 24, 2025}
	\maketitle
	
	\tableofcontents % Table of contents added
	\newpage % Optional: Starts the main content on a new page
	
	\section{Introduction}
	This work investigates two novel and logically coherent approaches in theoretical physics: the complementary standard model of relativity with absolute time and a modified Schrödinger equation with a mass-dependent intrinsic time. Both concepts not only offer alternative perspectives on the nature of time, energy, and quantum mechanics but are also internally consistent and build upon established physical principles. The complementary standard model introduces an absolute time linked to a dynamic energy-mass relationship, while the modified Schrödinger equation postulates an intrinsic time derived from the mass of a system. These dual approaches extend the wave-particle duality in a manner that is both mathematically consistent and physically plausible, inviting deeper reflection on the foundations of modern physics.
	
	\section{Wave-Particle Duality and Its Extension}
	Classical quantum mechanics views light and matter as both waves and particles, depending on the type of experiment. This work extends this duality by assuming that wave and particle properties arise not only from the measurement process but are determined by a fundamental interaction with an intrinsic time structure. This intrinsic time emerges from the mass of the observed object and directly influences the system's evolution.
	
	\section{Complementary Standard Model of Relativity}
	
	\subsection{Introduction}
	This model is based on the assumption of an absolute time \( T_0 \) and variable energy \( E \) as well as mass \( m \). It provides an alternative perspective to the special theory of relativity (STR) by reinterpreting the role of time.
	
	\subsection{Basic Assumptions}
	1. Absolute time: \( T_0 \) is constant. \\
	2. Constant speed of light: \( c_0 \approx 3 \times 10^8 \, \text{m/s} \). \\
	3. Variable energy: \( E \) is not fixed but dynamic. \\
	4. Mass as a function of energy: \( m = f(E) \).
	
	\subsection{Mathematical Formulation}
	The central energy relation is:
	\[
	E = \frac{\hbar}{T_0}
	\]
	Using the known relation \( E = m c_0^2 \), it follows:
	\[
	m = \frac{E}{c_0^2} = \frac{\hbar}{T_0 c_0^2}
	\]
	Thus, the mass \( m \) varies with \( E \), while \( T_0 \) remains fixed.
	
	\subsection{Implications for the Standard Model}
	- The classical assumption of a fixed rest mass must be extended. \\
	- The model could offer alternative explanations for quantum correlations. \\
	- The interpretation of time in quantum field theory might be modified. \\
	This theory presents a complementary perspective to established physics and provides new approaches to unifying quantum mechanics and relativity.
	
	\section{Modified Schrödinger Equation with Intrinsic Time}
	In my work \textit{Time as an Emergent Property in Quantum Mechanics} (March 23, 2025), the Schrödinger equation is extended to account for a mass-dependent time. The key modification lies in replacing the time \( t \) in the Schrödinger equation with an intrinsic time \( T \), which depends on the mass \( m \) of the quantum mechanical system. The intrinsic time \( T \) is defined as:
	\[
	T = \frac{\hbar}{m c^2}
	\]
	This leads to a modified Schrödinger equation where the system's time evolution depends on its mass. The modified formula is:
	\[
	i\hbar \frac{\partial}{\partial (t/T)} \Psi = \hat{H} \Psi
	\]
	Here, the time \( t \) is scaled by the intrinsic time \( T \), meaning that the time evolution proceeds at different rates for different masses. For a system with a larger mass \( m \), the intrinsic time \( T \) is shorter, leading to faster time evolution, while for a system with a smaller mass \( m \), the time evolution is slower.
	
	\section{Mathematical Comparison of Wave-Particle Duality and Time-Mass Duality}
	
	\subsection{Wave-Particle Duality}
	
	\subsubsection{Particle Description}
	The particle description of a quantum mechanical system focuses on localized mass/energy with a defined position:
	\begin{itemize}
		\item Particle of mass \( m \) with position \( \vec{x} \)
		\item Momentum \( \vec{p} = m\vec{v} \)
		\item Energy \( E = \frac{1}{2}mv^2 \) (non-relativistic) or \( E = \gamma mc^2 \) (relativistic)
	\end{itemize}
	
	\subsubsection{Wave Description}
	The wave description focuses on the spatially extended wavefunction:
	\begin{itemize}
		\item Wavefunction \( \Psi(\vec{x},t) \)
		\item De Broglie wavelength \( \lambda = \frac{h}{p} \)
		\item Wave vector \( \vec{k} = \frac{\vec{p}}{\hbar} \)
		\item Angular frequency \( \omega = \frac{E}{\hbar} \)
	\end{itemize}
	
	\subsubsection{Mathematical Connection}
	The two descriptions are linked through the Fourier transform:
	\[
	\Psi(\vec{x}) = \frac{1}{(2\pi\hbar)^{3/2}} \int \phi(\vec{p}) e^{i\vec{p}\cdot\vec{x}/\hbar} d^3p
	\]
	\[
	\phi(\vec{p}) = \frac{1}{(2\pi\hbar)^{3/2}} \int \Psi(\vec{x}) e^{-i\vec{p}\cdot\vec{x}/\hbar} d^3x
	\]
	Here, \( \phi(\vec{p}) \) is the wavefunction in momentum space.
	
	\subsection{Time-Mass Duality}
	
	\subsubsection{Time Dilation Description (Standard Model)}
	\begin{itemize}
		\item Variable time \( t \) with time dilation: \( t' = \gamma t \)
		\item Constant rest mass \( m_0 \)
		\item Relativistic energy: \( E = \gamma m_0c^2 \)
		\item Time dilation factor: \( \gamma = \frac{1}{\sqrt{1-v^2/c^2}} \)
	\end{itemize}
	
	\subsubsection{Mass Variation Description (This Model)}
	\begin{itemize}
		\item Absolute, constant time \( T_0 \)
		\item Variable mass \( m = \gamma m_0 \)
		\item Energy: \( E = mc^2 = \frac{\hbar}{T} \)
		\item Intrinsic time: \( T = \frac{\hbar}{mc^2} \)
	\end{itemize}
	
	\subsubsection{Mathematical Connection}
	The connection between the two descriptions can be expressed through the following transformations:
	\begin{enumerate}
		\item Time coordinate transformation:
		\[
		\frac{dt}{dt_0} = \frac{m_0}{m} = \frac{1}{\gamma}
		\]
		\item Equivalent formulation of time evolution:
		\begin{itemize}
			\item Standard model: \( i\hbar\frac{\partial}{\partial t}\Psi = \hat{H}\Psi \)
			\item This model: \( i\hbar\frac{\partial}{\partial (t/T)}\Psi = \hat{H}\Psi \)
		\end{itemize}
		\item Transformation between descriptions:
		\begin{itemize}
			\item If \( t' = \gamma t \) (time dilation) in the standard model
			\item Then \( m' = \gamma m_0 \) (mass variation) in this model
			\item With \( T' = \frac{\hbar}{m'c^2} = \frac{T_0}{\gamma} \)
		\end{itemize}
	\end{enumerate}
	
	\subsection{Parallels Between the Dualities}
	\begin{enumerate}
		\item \textbf{Complementarity}: \\
		- Wave-particle: Position (\( \vec{x} \)) and momentum (\( \vec{p} \)) are complementary observables \\
		- Time-mass: Time (\( t \) or \( T \)) and energy/mass (\( E \) or \( m \)) are complementary quantities
		\item \textbf{Uncertainty Relations}: \\
		- Wave-particle: \( \Delta x \Delta p \geq \frac{\hbar}{2} \) \\
		- Time-mass: \( \Delta t \Delta E \geq \frac{\hbar}{2} \) or \( \Delta T \Delta m \geq \frac{\hbar}{2c^2} \)
		\item \textbf{Transformations}: \\
		- Wave-particle: Fourier transformation between position and momentum space \\
		- Time-mass: Lorentz transformation (standard model) or mass variation transformation (this model)
	\end{enumerate}
	
	\subsection{Mathematical Structure of the Duality}
	In both cases, we can understand the duality as a transformation between complementary representations of the same physical system:
	\begin{itemize}
		\item \textbf{Wave-Particle:} \\
		\( \mathcal{F}: \Psi(\vec{x}) \rightarrow \phi(\vec{p}) \) \\
		Where \( \mathcal{F} \) is the Fourier transform operator.
		\item \textbf{Time-Mass (in this model):} \\
		\( \mathcal{L}: (T_0, m_0) \rightarrow (T, m) \) \\
		Where \( \mathcal{L} \) represents a modified Lorentz transformation that induces mass variation instead of time dilation, with: \\
		\( m = \gamma m_0 \) \\
		\( T = \frac{T_0}{\gamma} \)
	\end{itemize}
	The invariance in both dualities is evident in:
	\begin{itemize}
		\item Wave-particle: \( |\Psi|^2 dx = |\phi|^2 dp \) (probability conservation)
		\item Time-mass: \( m_0c^2T_0 = mc^2T = \hbar \) (energy-time product)
	\end{itemize}
	
	\section{Conclusion}
	This work presents two innovative approaches to extending physical theories: the complementary standard model of relativity with absolute time and the modified Schrödinger equation with mass-dependent intrinsic time. Both models offer new perspectives on the nature of time, energy, and quantum mechanics, demonstrating their internal coherence and complementarity through a mathematical comparison with wave-particle duality. However, a central objection to the concept of absolute time is that we measure time directly, and time dilation appears as an observable reality, as seen in GPS corrections, muon decay, or photon-based light travel time measurements. This objection requires thorough examination, as it challenges the foundation of the proposed models.
	
	In the standard model of special relativity, time dilation (\( t' = \gamma t \)) is interpreted as a real change in time, confirmed by precise measurements. For instance, GPS satellites exhibit a time shift of approximately \( 38 \, \mu\text{s} \) per day, and muons have a longer lifetime when in motion. Photon measurements, such as light travel time (\( t = \frac{d}{c_0} \)), also seem to support time as a variable quantity, as the measured time differences align with the Lorentz transformation. However, the \( T_0 \) model with absolute time and variable mass (\( m = \gamma m_0 \)) proposed here offers an alternative interpretation, which I explore in detail in my work \textit{A Model with Absolute Time and Variable Energy: A Comprehensive Investigation of the Foundations} (March 24, 2025), particularly in the section "Fundamentals of Time Measurement." There, I demonstrate that time measurements are indirectly derived from frequencies (\( f = \frac{E}{h} \)), which are linked to energy (\( E = m c_0^2 \)) and thus mass. In the standard model, this is interpreted as time dilation, while in the \( T_0 \) model, the same measurement reflects a change in mass with constant time.
	
	This dualistic approach extends to photon measurements as well. Light travel time could be interpreted as an energy loss (\( E \)-variation) rather than a time variation, since photons travel at \( c_0 \) in both models, and their frequency (\( \nu = \frac{E}{h} \)) depends on energy. For example, a redshift in the \( T_0 \) model could be explained as an \( E \)- or \( m \)-loss rather than time dilation. Crucially, all measurements—whether with material particles (muons, GPS) or photons (light travel time, Doppler effect)—implicitly allow both interpretations. The mathematical equivalence between \( t' = \gamma t \) (standard model) and \( m' = \gamma m_0 \) (this model) shows that the observable outcomes are identical, regardless of whether we consider time or mass as variable. This means that photon measurements do not resolve the distinction but rather reinforce the duality: we cannot definitively determine whether time dilation or mass variation is the "more real" description, as both models equally account for the data.
	
	The core issue lies in the fact that our measurement methods—whether through clocks, frequency spectra, or light travel time—always presuppose an operational definition of time linked to energy and mass (\( E = h f = m c_0^2 \)). This linkage makes distinguishing between the models experimentally challenging, as every measurement can be interpreted dualistically. The objection that time dilation is real because we measure time is thus not refuted but placed in an interpretive context: the reality of time dilation depends on the chosen perspective, not the measurements themselves. In my further work \textit{A Model with Absolute Time and Variable Energy} (March 24, 2025), particularly in the sections "Fundamentals of Time Measurement" and "Experimental Verification," I discuss this interpretive duality in depth and suggest approaches for future experiments that might reveal differences between the models. Likewise, in \textit{Time as an Emergent Property in Quantum Mechanics} (March 23, 2025), I further explore the role of intrinsic time as an emergent property. Until such distinctions become experimentally feasible, the models presented here remain complementary perspectives that enhance the flexibility and depth of physical description.
	
\end{document}