% ==============================================================================
% T0 Theory: Shared German Preamble – Optimiert für Buch (ca. 500 Seiten)
% Version: 1.2
% Author: Johann Pascher
% Date: 2026
% ==============================================================================
%
% Diese Präambel ist speziell für dicke Bücher (ca. 500 Seiten) optimiert.
% Papierformat: 8.25" × 11"
% Steganpassungen:
%   - Vergrößerter innerer Rand und bindingoffset für perfekte Bindung
%   - Etwas kleinere äußere Ränder für mehr Text pro Seite
%   - Vergrößerter bottom-Rand für angenehme Lesbarkeit
%   - Openright für neue Kapitel (bei book-Klasse)
%
% Empfohlene Dokumentklasse:
%   \documentclass[11pt,openright,twoside]{book}
%
% ==============================================================================

% =============================================================================
% SECTION 1: Encoding and Language
% =============================================================================
\usepackage[utf8]{inputenc}
\usepackage[T1]{fontenc}
\usepackage[ngerman]{babel}  % oder [german] für altere Rechtschreibung
\usepackage{microtype} % Verbessert die Mikrotypographie erheblich
\usepackage{ragged2e}  % Für besseren Blocksatz
\usepackage{booktabs}  % Für professionelle Tabellen
\usepackage{hyphenat}
\babelprovide[import, main]{ngerman}
% Für bessere Worttrennung in Tabellen
\newcommand{\tabcell}[1]{\begin{tabular}{@{}c@{}}#1\end{tabular}}
\hyphenation{
	Al-go-rit-men
	Perio-den-fin-dung
	Pollard-Rho
	Shor-Al-go-rit-mus
	Frame-work
	Kom-ple-xi-täts-klas-se
	Theo-re-tisch
	Prak-tisch
	Er-war-tet
	Det-er-mi-nis-tisch
	Pro-ba-bi-lis-tisch
	Quan-ten-rech-ner
	Quan-ten-pro-zes-sor
	Quan-ten-Fourier-trans-for-ma-ti-on
	QFT
	Super-po-si-ti-on
	Deko-hä-renz
	Mes-sung
	Mes-ser-geb-nis
	Mo-du-lar
	Ra-ti-o-nal
	Arith-me-tik
	Ge-mein-sa-mer-Teiler
	GGT
	Kong-ru-enz
	Ex-po-nen-ti-a-ti-on
	Se-quenz
	Zy-klus
	De-tek-ti-on
	Re-pro-du-zier-bar
	Re-pro-du-zie-rung
	Har-d-ware
	Soft-ware
	Im-ple-men-tie-rung
	Ef-fi-zi-enz
	Ge-schwin-dig-keit
	Spei-cher-be-darf
	Re-chen-zeit
	Para-me-ter
	Ein-stel-lung
	Tu-ning
	Schwel-le
	Krite-ri-um
	Erfolgs-rate
	Wahr-schein-lich-keit
	RAM-Limi-tie-rung
	Spei-cher-ska-lie-rung
	Elek-tro-nen-zu-stän-de
	Ex-po-nen-ti-el-le
	Ko-die-rung
	Mess-kol-laps
	Kom-ple-xi-tät
	Ein-ga-be
	Aus-ga-be
	Abbil-dung
	Aus-le-sen
	Eng-pass
	Fun-da-men-tal
	Sys-tem
	Quan-ten-com-pu-ter
	Hy-brid
	Sha-lling
	Limi-tie-rung
	Kom-ple-xi-tät
	Pro-blem
	Sel-bes
	Di-rekt
	Dis-kre-panz
	Un-si-cher-heit
	The-o-rie
	Ex-pe-ri-ment
	Ana-lyse
	Syn-the-se
	Hy-po-the-se
	Mo-dell
	Phy-sik
	Kos-mo-lo-gie
	As-tro-no-mie
	Teil-chen
	Wel-le
	Feld
	Kraft
	Ener-gie
	Ma-ter-ie
	Raum
	Zeit
	Ge-schwin-dig-keit
	Be-schleu-ni-gung
	Ro-tver-schie-bung
	Dop-pler-effekt
	Ex-pan-si-on
	Uni-ver-sum
	Ur-knall
	In-fla-ti-on
	Gra-vi-ta-ti-on
	Schwarz-schild
	Ein-stein
	Lo-renz
	Ma-xwell
	Quan-ten
	Quan-ten-me-cha-nik
	Quanten-feld-theo-rie
	Fein-struk-tur-kon-stan-te
	Plancksche
	Un-schär-fe
	Su-per-po-si-ti-on
	Ver-schrän-kung
	Ko-hä-renz
	De-ko-hä-renz
	Stan-dard-mo-dell
	Elek-tro-mag-ne-tis-mus
	Schwa-che-Wech-sel-wir-kung
	Star-ke-Wech-sel-wir-kung
	Quan-ten-chro-mo-dy-na-mik
	Had-ro-nen
	Lep-to-nen
	Bo-so-nen
	Fer-mi-o-nen
	Quark
	Glu-on
	Pho-ton
	W-Bo-son
	Z-Bo-son
	Hi-ggs
	Zeit-Mas-se-Du-a-li-tät
	Fraktale-Raum-zeit
	Frak-tal
	Dimen-si-on
	Skalar-feld
	La-grange-dich-te
	Mas-sen-hi-er-ar-chie
	Ein-schleifen-bei-trag
	Ein-bet-tung
	A-no-ma-lie
	My-on
	Tau
	Elek-tron
	String-theo-rie
	M-Theo-rie
	Su-per-string-theo-rie
	Su-per-gra-vi-ta-ti-on
	Kalabi-Yau-Ma-nig-fal-tig-keit
	Ex-tra-di-men-si-o-nen
	Kom-pak-ti-fi-zie-rung
	Bran-en
	Du-a-li-tät
	T-Du-a-li-tät
	S-Du-a-li-tät
	Quan-ten-schaum
	Loop-Quan-ten-gra-vi-ta-ti-on
	Ma-the-ma-tisch
	Ma-the-ma-tik
	Ge-o-me-trie
	To-po-lo-gie
	Al-ge-bra
	Ana-ly-sis
	Dif-fe-ren-ti-al
	In-te-gral
	Gle-i-chung
	Funk-ti-on
	Pa-ra-me-ter
	Kon-stan-te
	Va-ri-able
	Äqui-va-lenz
	Pro-por-ti-o-nal
	On-to-lo-gisch
	Epi-ste-mo-lo-gisch
	Phä-no-me-no-lo-gisch
	Kon-zenp-ti-o-nell
	Theo-re-tisch
	Em-pi-risch
	Prag-ma-tisch
	Fun-da-men-tal
	Re-duk-ti-o-nis-tisch
	Ho-lis-tisch
	Be-rech-nung
	Be-rech-nungs-me-tho-de
	Si-mu-la-ti-on
	Nu-me-risch
	Ana-ly-tisch
	Ap-pro-xi-ma-ti-on
	Prä-zi-si-on
	Ak-ku-ra-tes
	Ge-nau-ig-keit
	Feh-ler
	Ab-weich-ung
	Kor-rek-tur
	Ka-li-bra-ti-on
	LHC
	CERN
	Fermilab
	DESY
	SLAC
	Belle-II
	ATLAS
	CMS
	LHCb
	ALICE
	GeV
	TeV
	eV
	keV
	MeV
	Planck-Ein-hei-ten
	Na-tür-li-che-Ein-hei-ten
	SI-Ein-hei-ten
	cgs-Sys-tem
	QED
	QCD
	HVP
	BSM
	SM
	ESM
	SUSY
	MSSM
	NMSSM
	GUT
	TOE
	BBN
	CMB
	LSS
}
\usepackage{lmodern}

% =============================================================================
% SECTION 2: Page Geometry – Buch-Optimierung für ~500 Seiten (8.25" × 11")
% =============================================================================
\usepackage[paperwidth=8.25 in, paperheight=11in,
top=1.0in,
bottom=1.2in,           % vergrößert für bessere Lesbarkeit
inner=1.0in,            % größerer Innenrand (Bindung)
outer=0.75in,           % kleinerer Außenrand → mehr Text pro Seite
bindingoffset=0.75in,   % zusätzlicher Puffer für die Bindung (Steg)
twoside]{geometry}
\setlength{\headheight}{15pt}

% =============================================================================
% SECTION 3: Mathematics and Physics
% =============================================================================
\usepackage{amsmath,amssymb,amsfonts,amsthm}
\usepackage{mathtools}
\usepackage{physics}
\usepackage{siunitx}
\sisetup{
	locale=DE,
	group-separator={.},
	output-decimal-marker={,},
	per-mode=symbol
}

% =============================================================================
% SECTION 4: Graphics and Tables
% =============================================================================
\usepackage{graphicx}
\usepackage[table,xcdraw]{xcolor}
\usepackage{tikz}
\usetikzlibrary{arrows.meta,positioning,shapes.geometric,decorations.pathmorphing,patterns,shapes.arrows,intersections}
\usepackage{pgfplots}
\pgfplotsset{compat=1.18}
\usepackage[most]{tcolorbox}
\tcbuselibrary{breakable}
\usepackage{array}
\usepackage{longtable}
\usepackage{float}
\usepackage{adjustbox}
\usepackage{rotating}
\usepackage{tabularx}
\usepackage{makecell}
\usepackage{multirow}
%\usepackage{alphalph}
% =============================================================================
% SECTION 5: Document Formatting
% =============================================================================
\usepackage{fancyhdr}
\renewcommand{\headrulewidth}{0.4pt}
\renewcommand{\footrulewidth}{0.4pt}
\usepackage{tocloft}
\usepackage{hyperref}
\hypersetup{
	colorlinks=true,
	linkcolor=black,
	citecolor=black,
	urlcolor=black,
	breaklinks=true,
	bookmarksnumbered=true,
	unicode=true
}
\usepackage{bookmark}
\usepackage{cleveref}

\setcounter{tocdepth}{3}

\usepackage{enumitem}
\usepackage{setspace}
\usepackage{multicol}

% =============================================================================
% SECTION 6: Code and Algorithms
% =============================================================================
\usepackage{algorithm}
\usepackage{algorithmic}
\usepackage{listings}
\lstset{
	basicstyle=\ttfamily\footnotesize,
	breaklines=true,
	breakatwhitespace=true,
	columns=flexible,
	keepspaces=true,
	showstringspaces=false,
	frame=single,
	xleftmargin=0pt,
	xrightmargin=0pt
}
\usepackage{mdframed}

% =============================================================================
% SECTION 7: Additional Packages
% =============================================================================
\usepackage{pdflscape}
\usepackage{braket}
\usepackage{cancel}
\usepackage{caption}
\usepackage{csquotes}
\usepackage{gensymb}
\usepackage{textcomp}
\usepackage{textgreek}
\usepackage{upgreek}
\usepackage{url}
\usepackage{slashed}
\usepackage{bm}
\usepackage{newunicodechar}

% =============================================================================
% SECTION 8: Citation Commands (Compatibility)
% =============================================================================
\providecommand{\citep}[1]{\cite{#1}}
\providecommand{\citet}[1]{\cite{#1}}

% =============================================================================
% SECTION 9: Colors
% =============================================================================
\definecolor{gold}{RGB}{255,215,0}
\definecolor{blue}{rgb}{0,0,1}
\definecolor{boxgray}{RGB}{240,240,240}
\definecolor{deepblue}{RGB}{0,0,127}
\definecolor{deepgreen}{RGB}{0,127,0}
\definecolor{deepred}{RGB}{191,0,0}
\definecolor{t0blue}{RGB}{33,150,243}
\definecolor{t0green}{RGB}{76,175,80}
\definecolor{t0orange}{RGB}{255,152,0}
\definecolor{t0purple}{RGB}{156,39,176}
\definecolor{t0red}{RGB}{244,67,54}
\definecolor{t0yellow}{RGB}{255,204,0}

% =============================================================================
% SECTION 10: Column Types
% =============================================================================
\newcolumntype{L}[1]{>{\raggedright\arraybackslash}p{#1}}
\newcolumntype{C}[1]{>{\centering\arraybackslash}p{#1}}



% Custom commands for specific documents
\newcommand{\Weyl}{\mathrm{Weyl}}
\newcommand{\ZPinch}{\nabla p = \mathbf{J} \times \mathbf{B}}
\newcommand{\SynchPower}{P_{\text{synch}}}

% =============================================================================
% SECTION 12: Hyperref Settings
% =============================================================================
\hypersetup{
	colorlinks=true,
	linkcolor=blue,
	citecolor=blue,
	urlcolor=blue,
	breaklinks=true,
	bookmarksnumbered=true,
	pdfstartview=FitH
}

% =============================================================================
% SECTION 13: Theorem Environments (Deutsch)
% =============================================================================
\theoremstyle{plain}
\newtheorem{theorem}{Satz}[section]
\newtheorem{lemma}[theorem]{Lemma}
\newtheorem{proposition}[theorem]{Proposition}
\newtheorem{corollary}[theorem]{Korollar}

\theoremstyle{definition}
\newtheorem{definition}[theorem]{Definition}
\newtheorem{example}[theorem]{Beispiel}
\newtheorem{insight}[theorem]{Erkenntnis}
\newtheorem{discovery}[theorem]{Entdeckung}
\newtheorem{erkenntnis}[theorem]{Erkenntnis}

\theoremstyle{remark}
\newtheorem{remark}[theorem]{Bemerkung}
\newtheorem{axiom}{Axiom}
\newtheorem{principle}{Prinzip}
\newtheorem{bemerkung}[theorem]{Bemerkung}
\newtheorem{warnung}[theorem]{Warnung}

% =============================================================================
% SECTION 14: T0-Specific Commands
% =============================================================================

% --- Core T0 Fields ---
\newcommand{\Tfield}{T(x,t)}
\providecommand{\Tfieldt}{T(\vec{x},t)}
\newcommand{\Efield}{E(x,t)}
\newcommand{\mfield}{m(x,t)}
\providecommand{\vecx}{\vec{x}}

% --- Lagrangian ---
\newcommand{\Lag}{\mathcal{L}}
\newcommand{\calL}{\mathcal{L}}

% --- Greek Letters and Constants ---
\newcommand{\alphaem}{\alpha}
\newcommand{\betaT}{\beta_T}
\newcommand{\xiT}{\xi}
\newcommand{\xipar}{\xi}

% --- Energy and Planck Units ---
\newcommand{\Ezero}{E_0}
\newcommand{\EPlanck}{E_{\text{Pl}}}
\newcommand{\Mpl}{M_{\text{Pl}}}
\newcommand{\mP}{m_{\text{P}}}
\newcommand{\lP}{\ell_{\text{P}}}
\newcommand{\tP}{t_{\text{P}}}
\newcommand{\LPlanck}{\ell_{\text{Pl}}}
\newcommand{\TPlanck}{t_{\text{Pl}}}

% --- Coupling Constants ---
\newcommand{\Gnat}{G_{\text{nat}}}
\newcommand{\alphaEM}{\alpha_{\text{EM}}}
\newcommand{\alphaSI}{\alpha_{\text{SI}}}
\newcommand{\Hubble}{H_0}
\newcommand{\LCDM}{\Lambda\text{CDM}}
\newcommand{\natunits}{(nat. units)}

% --- T0 Model Parameters ---
\newcommand{\xigeom}{\xi_{\mathrm{geom}}}
\newcommand{\rzero}{r_{0}}
\newcommand{\xirat}{\xi_{\mathrm{rat}}}
\newcommand{\tzero}{t_{0}}
\newcommand{\Lambdat}{\Lambda_{\mathrm{t}}}
\newcommand{\EP}{E_{\mathrm{P}}}
\newcommand{\Emu}{E_{\mu}}
\newcommand{\Ee}{E_{e}}
\newcommand{\Etau}{E_{\tau}}
\newcommand{\alphafine}{\alpha_{\mathrm{fine}}}
\newcommand{\alphal}{\alpha_{\ell}}
\newcommand{\Lzero}{\ell_{0}}
\newcommand{\Lp}{\ell_{\mathrm{P}}}

% --- Additional T0 Commands ---
\newcommand{\Kfrak}{K_{\text{frak}}}
\newcommand{\Dfrak}{D_{\text{frak}}}
\newcommand{\betapar}{\ensuremath{\beta_T}}
\newcommand{\alphapar}{\alpha}
\newcommand{\deltafield}{\delta \phi}
\newcommand{\deltam}{\delta m}
\newcommand{\deltaE}{\delta E}
\newcommand{\Exi}{E_{\xi}}
\newcommand{\Lxi}{\ell_{\xi}}
\newcommand{\rhoCMB}{\rho_{\text{CMB}}}
\newcommand{\rhoCasimir}{\rho_{\text{Casimir}}}
\newcommand{\Leff}{L_{\text{eff}}}
\newcommand{\CQCD}{C_{\mathrm{QCD}}}
\newcommand{\Kspec}{K_{\mathrm{spec}}}
\newcommand{\Tzero}{\ensuremath{T_0}}
\newcommand{\Eabs}{E_{\text{abs}}}
\newcommand{\taupar}{\tau}

% --- Provided Commands (may be redefined elsewhere) ---
\providecommand{\xiconst}{\xi_{\text{const}}}
\providecommand{\DhiggsT}{D_{\text{Higgs-T}}}
\providecommand{\rhoE}{\rho_{E}}
\providecommand{\Echar}{E_{\text{char}}}
\providecommand{\kfrac}{k_{\text{frac}}}
\providecommand{\alphaEMSI}{\alpha_{\text{EM,SI}}}
\providecommand{\alphaEMnat}{\alpha_{\text{EM,nat}}}
\providecommand{\betaTSI}{\beta_{T,\text{SI}}}
\providecommand{\betaTnat}{\beta_{T,\text{nat}}}
\providecommand{\Gsi}{G_{\text{SI}}}
\providecommand{\xiparSI}{\xi_{\text{SI}}}
\providecommand{\xiparnat}{\xi_{\text{nat}}}
\providecommand{\meff}{m_{\text{eff}}}
\providecommand{\Tzerot}{T_{0}(t)}
\providecommand{\mzerot}{m_{0}(t)}
\providecommand{\Ezeroabs}{E_{0,\text{abs}}}
\providecommand{\Epar}{E_{\text{par}}}
\providecommand{\Lnat}{\ell_{\text{nat}}}
\providecommand{\Tnat}{T_{\text{nat}}}
\providecommand{\xifrak}{\xi_{\text{frac}}}
\providecommand{\Tfrak}{T_{\text{frac}}}
\providecommand{\mfrak}{m_{\text{frac}}}
\providecommand{\Dfrac}{D_{\text{frac}}}
\providecommand{\EphotSI}{E_{\gamma,\text{SI}}}
\providecommand{\EphotNat}{E_{\gamma,\text{nat}}}
\providecommand{\Eabsint}{E_{\text{abs,int}}}
\providecommand{\mphoton}{m_{\gamma}}
\providecommand{\Evis}{E_{\text{vis}}}
\providecommand{\Cto}{C_{T0}}
\providecommand{\mytimes}{\times}
\providecommand{\lambdah}{\lambda_h}
\providecommand{\checkmarkx}{\checkmark}
\providecommand{\Enorm}{E_{\text{norm}}}
\providecommand{\Tobs}{T_{\text{obs}}}
\providecommand{\mobs}{m_{\text{obs}}}
\providecommand{\Eobs}{E_{\text{obs}}}
\providecommand{\Lobs}{\ell_{\text{obs}}}
\providecommand{\xobs}{\xi_{\text{obs}}}
\providecommand{\calE}{\mathcal{E}}
\providecommand{\calT}{\mathcal{T}}
\providecommand{\calM}{\mathcal{M}}
\providecommand{\alphag}{\alpha_g}
\providecommand{\Tmax}{T_{\text{max}}}
\providecommand{\mmin}{m_{\text{min}}}
\providecommand{\Lmax}{\ell_{\text{max}}}
\providecommand{\Emin}{E_{\text{min}}}
\providecommand{\Geff}{G_{\text{eff}}}
\providecommand{\rhoeff}{\rho_{\text{eff}}}
\providecommand{\xieff}{\xi_{\text{eff}}}
\providecommand{\Teff}{T_{\text{eff}}}
\providecommand{\hPlanck}{h}
\providecommand{\kB}{k_B}
\providecommand{\muB}{\mu_B}
\providecommand{\lambdaC}{\lambda_C}
\providecommand{\omegaP}{\omega_P}
\providecommand{\rhoP}{\rho_P}
\providecommand{\Tref}{T_{\text{ref}}}
\providecommand{\Eref}{E_{\text{ref}}}
\providecommand{\mref}{m_{\text{ref}}}
\providecommand{\Lref}{\ell_{\text{ref}}}
\providecommand{\xikonst}{\xi_0}
\providecommand{\Phiphoton}{\Phi_{\gamma}}
\providecommand{\etavis}{\eta_{\text{vis}}}
\providecommand{\pichar}{\pi}
\providecommand{\primrel}{\mathcal{P}_{\text{rel}}}
\providecommand{\warningx}{\textcolor{orange}{\textbf{!}}}
\providecommand{\phiT}{\phi_T}
\providecommand{\Lorentz}{\Lambda}
\providecommand{\Cconv}{C_{\text{conv}}}
\providecommand{\Df}{\Delta f}
\providecommand{\lambdazero}{\lambda_0}
\providecommand{\myapprox}{\approx}
\providecommand{\checked}{\checkmark}
\providecommand{\alphaWSI}{\alpha_W^{\text{SI}}}
\providecommand{\alphaWnat}{\alpha_W^{\text{nat}}}
\providecommand{\vect}[1]{\vec{#1}}
\providecommand{\Rzero}{R_0}
\providecommand{\Riem}{\mathcal{R}}
\providecommand{\nuzero}{\nu_0}
\providecommand{\mypi}{\pi}

% =============================================================================
% SECTION 15: tcolorbox Styles and Environments
% =============================================================================

\tcbset{
	keyresult/.style={
		colback=blue!5!white,
		colframe=blue!75!black,
		title=Schlüsselresultat,
		fonttitle=\bfseries
	},
	foundation/.style={
		colback=green!5!white,
		colframe=green!75!black,
		title=Grundlage,
		fonttitle=\bfseries
	},
	alternative/.style={
		colback=orange!5!white,
		colframe=orange!75!black,
		title=Alternative,
		fonttitle=\bfseries
	},
	warningbox/.style={
		colback=red!5!white,
		colframe=red!75!black,
		title=Warnung,
		fonttitle=\bfseries
	}
}

\newtcolorbox{keyresultbox}[1][]{colback=blue!5!white,colframe=blue!75!black,fonttitle=\bfseries,title={#1},breakable}
\newtcolorbox{keyresult}[1][Schlüsselresultat]{colback=blue!5!white,colframe=blue!75!black,fonttitle=\bfseries,title={#1},breakable}
\newtcolorbox{foundationbox}[1][]{colback=green!5!white,colframe=green!75!black,fonttitle=\bfseries,title={#1},breakable}
\newtcolorbox{foundation}[1][Grundlage]{colback=green!5!white,colframe=green!75!black,fonttitle=\bfseries,title={#1},breakable}
\newtcolorbox{alternativebox}[1][]{colback=orange!5!white,colframe=orange!75!black,fonttitle=\bfseries,title={#1},breakable}
\newtcolorbox{warningboxenv}[1][]{colback=red!5!white,colframe=red!75!black,fonttitle=\bfseries,title={#1},breakable}

\newtcolorbox{fundamental}[1][]{
	colback=boxgray,
	colframe=t0blue,
	fonttitle=\bfseries,
	title=#1,
	sharp corners,
	boxrule=2pt
}

\newtcolorbox{newperspective}[1][]{
	colback=red!5!white,
	colframe=t0red,
	fonttitle=\bfseries,
	title=#1,
	sharp corners,
	boxrule=2pt
}

\newtcolorbox{formula}[1][]{
	colback=blue!5!white,
	colframe=blue!75!black,
	fonttitle=\bfseries,
	title=#1
}

\newtcolorbox{result}[1][]{
	colback=green!5!white,
	colframe=green!75!black,
	fonttitle=\bfseries,
	title=#1
}

\newtcolorbox{derivation}[1][]{
	colback=green!5!white,
	colframe=green!75!black,
	title=#1,
	fonttitle=\bfseries,
	breakable
}

\newtcolorbox{summary}[1][]{
	colback=gray!10!white,
	colframe=gray!75!black,
	title=#1,
	fonttitle=\bfseries,
	breakable
}

\newtcolorbox{comparison}[1][]{
	colback=purple!5!white,
	colframe=purple!75!black,
	title=#1,
	fonttitle=\bfseries,
	breakable
}

\newtcolorbox{relation}[1][]{
	colback=cyan!5!white,
	colframe=cyan!75!black,
	title=#1,
	fonttitle=\bfseries,
	breakable
}

\newtcolorbox{principleBox}[1][]{
	colback=yellow!5!white,
	colframe=yellow!75!black,
	title=#1,
	fonttitle=\bfseries,
	breakable
}

\newtcolorbox{insightBox}[1][]{colback=blue!5,colframe=t0blue,title={#1},fonttitle=\bfseries,breakable}
\newtcolorbox{discoveryBox}[1][]{colback=green!5,colframe=t0green,title={#1},fonttitle=\bfseries,breakable}
\newtcolorbox{revelation}[1][]{colback=red!5,colframe=t0red,title={#1},fonttitle=\bfseries,breakable}
\newtcolorbox{keypoint}[1][]{colback=blue!5,colframe=t0blue,title={#1},fonttitle=\bfseries,breakable}
\newtcolorbox{evidence}[1][]{colback=green!5,colframe=t0green,title={#1},fonttitle=\bfseries,breakable}
\newtcolorbox{conclusionBox}[1][]{colback=gray!5,colframe=gray,title={#1},fonttitle=\bfseries,breakable}
\newtcolorbox{significance}[1][]{colback=yellow!5,colframe=orange,title={#1},fonttitle=\bfseries,breakable}
\newtcolorbox{philosophical}[1][]{colback=purple!5,colframe=purple,title={#1},fonttitle=\bfseries,breakable}
\newtcolorbox{implicationBox}[1][]{colback=cyan!5,colframe=cyan,title={#1},fonttitle=\bfseries,breakable}
\newtcolorbox{perspectiveBox}[1][]{colback=blue!5,colframe=t0blue,title={#1},fonttitle=\bfseries,breakable}
\newtcolorbox{revolutionary}[1][]{colback=red!5,colframe=t0red,title={#1},fonttitle=\bfseries,breakable}

\newtcolorbox{technical}[1][]{colback=gray!5,colframe=gray!75!black,title={#1},fonttitle=\bfseries,breakable}
\newtcolorbox{technicalBox}[1][]{colback=gray!5,colframe=gray!75!black,title={#1},fonttitle=\bfseries,breakable}
\newtcolorbox{notationBox}[1][]{colback=yellow!5,colframe=yellow!75!black,title={#1},fonttitle=\bfseries,breakable}
\newtcolorbox{verification}[1][]{colback=orange!5!white,colframe=orange!75!black,fonttitle=\bfseries,title={#1}}
\newtcolorbox{explanationBox}[1][]{colback=purple!5!white,colframe=purple!75!black,fonttitle=\bfseries,title={#1}}
\newtcolorbox{interpretationBox}[1][]{colback=cyan!5!white,colframe=cyan!75!black,fonttitle=\bfseries,title={#1}}
\newtcolorbox{explanation}[1][]{colback=purple!5!white,colframe=purple!75!black,fonttitle=\bfseries,title={#1},breakable}
\newtcolorbox{interpretation}[1][]{colback=cyan!5!white,colframe=cyan!75!black,fonttitle=\bfseries,title={#1},breakable}
\newtcolorbox{proof_step}[1][]{colback=gray!5!white,colframe=gray!75!black,fonttitle=\bfseries,title={#1},breakable}
\newtcolorbox{experimental}[1][]{colback=teal!5!white,colframe=teal!75!black,fonttitle=\bfseries,title={#1},breakable}

\newtcolorbox{important}[1][]{colback=red!5!white,colframe=red!75!black,title={#1},fonttitle=\bfseries,breakable}
\newtcolorbox{warning}[1][]{colback=orange!5!white,colframe=orange!75!black,title={#1},fonttitle=\bfseries,breakable}
\newtcolorbox{caution}[1][]{colback=yellow!5!white,colframe=yellow!75!black,title={#1},fonttitle=\bfseries,breakable}
\newtcolorbox{highlight}[1][]{colback=yellow!10!white,colframe=yellow!75!black,title={#1},fonttitle=\bfseries,breakable}
\newtcolorbox{critical}[1][]{colback=red!10!white,colframe=red!75!black,title={#1},fonttitle=\bfseries,breakable}

\newtcolorbox{analysis}[1][]{colback=blue!5!white,colframe=blue!75!black,title={#1},fonttitle=\bfseries,breakable}
\newtcolorbox{application}[1][]{colback=green!5!white,colframe=green!75!black,title={#1},fonttitle=\bfseries,breakable}
\newtcolorbox{experiment}[1][]{colback=cyan!5!white,colframe=cyan!75!black,title={#1},fonttitle=\bfseries,breakable}
\newtcolorbox{historical}[1][]{colback=brown!5!white,colframe=brown!75!black,title={#1},fonttitle=\bfseries,breakable}
\newtcolorbox{numerical}[1][]{colback=gray!5!white,colframe=gray!75!black,title={#1},fonttitle=\bfseries,breakable}
\newtcolorbox{overview}[1][]{colback=blue!5!white,colframe=blue!75!black,title={#1},fonttitle=\bfseries,breakable}
\newtcolorbox{speculation}[1][]{colback=purple!5!white,colframe=purple!75!black,title={#1},fonttitle=\bfseries,breakable}
\newtcolorbox{question}[1][]{colback=orange!5!white,colframe=orange!75!black,title={#1},fonttitle=\bfseries,breakable}
\newtcolorbox{method}[1][]{colback=teal!5!white,colframe=teal!75!black,title={#1},fonttitle=\bfseries,breakable}
\newtcolorbox{correct}[1][]{colback=green!10!white,colframe=green!75!black,title={#1},fonttitle=\bfseries,breakable}
\newtcolorbox{units}[1][]{colback=gray!5!white,colframe=gray!75!black,title={#1},fonttitle=\bfseries,breakable}
\newtcolorbox{achievement}[1][]{colback=gold!5!white,colframe=orange!75!black,title={#1},fonttitle=\bfseries,breakable}
\newtcolorbox{equivalence}[1][]{colback=cyan!5!white,colframe=cyan!75!black,title={#1},fonttitle=\bfseries,breakable}
\newtcolorbox{dimensional}[1][]{colback=purple!5!white,colframe=purple!75!black,title={#1},fonttitle=\bfseries,breakable}

\newtcolorbox{photon}[1][]{colback=yellow!5!white,colframe=yellow!75!black,title={#1},fonttitle=\bfseries,breakable}
\newtcolorbox{neutrino}[1][]{colback=blue!5!white,colframe=blue!75!black,title={#1},fonttitle=\bfseries,breakable}
\newtcolorbox{revolution}[1][]{colback=red!5!white,colframe=red!75!black,title={#1},fonttitle=\bfseries,breakable}
\newtcolorbox{t0box}[1][]{colback=blue!5!white,colframe=t0blue,title={#1},fonttitle=\bfseries,breakable}
\newtcolorbox{documentbox}[1][]{colback=gray!5!white,colframe=gray!75!black,title={#1},fonttitle=\bfseries,breakable}
\newtcolorbox{sibox}[1][]{colback=green!5!white,colframe=green!75!black,title={#1},fonttitle=\bfseries,breakable}
\newtcolorbox{smbox}[1][]{colback=blue!5!white,colframe=blue!75!black,title={#1},fonttitle=\bfseries,breakable}
\newtcolorbox{pvbox}[1][]{colback=purple!5!white,colframe=purple!75!black,title={#1},fonttitle=\bfseries,breakable}
\newtcolorbox{koidebox}[1][]{colback=orange!5!white,colframe=orange!75!black,title={#1},fonttitle=\bfseries,breakable}

\newtcolorbox{formel}[1][]{colback=blue!5!white,colframe=blue!75!black,title={#1},fonttitle=\bfseries,breakable}
\newtcolorbox{schluessel}[1][]{colback=blue!5!white,colframe=blue!75!black,title={#1},fonttitle=\bfseries,breakable}
\newtcolorbox{wichtig}[1][]{colback=red!5!white,colframe=red!75!black,title={#1},fonttitle=\bfseries,breakable}
\newtcolorbox{vorsicht}[1][]{colback=orange!5!white,colframe=orange!75!black,title={#1},fonttitle=\bfseries,breakable}
\newtcolorbox{revolutionaer}[1][]{colback=red!5!white,colframe=red!75!black,title={#1},fonttitle=\bfseries,breakable}
\newtcolorbox{numerisch}[1][]{colback=gray!5!white,colframe=gray!75!black,title={#1},fonttitle=\bfseries,breakable}
\newtcolorbox{experimentell}[1][]{colback=cyan!5!white,colframe=cyan!75!black,title={#1},fonttitle=\bfseries,breakable}
\newtcolorbox{anwendung}[1][]{colback=green!5!white,colframe=green!75!black,title={#1},fonttitle=\bfseries,breakable}
\newtcolorbox{alternative}[1][]{colback=orange!5!white,colframe=orange!75!black,title={#1},fonttitle=\bfseries,breakable}
\newtcolorbox{beziehung}[1][]{colback=cyan!5!white,colframe=cyan!75!black,title={#1},fonttitle=\bfseries,breakable}
\newtcolorbox{folgerung}[1][]{colback=green!5!white,colframe=green!75!black,title={#1},fonttitle=\bfseries,breakable}
\newtcolorbox{abhandlung}[1][]{colback=gray!5!white,colframe=gray!75!black,title={#1},fonttitle=\bfseries,breakable}
\newtcolorbox{prinzipBox}[1][]{colback=blue!5!white,colframe=blue!75!black,title={#1},fonttitle=\bfseries,breakable}
\newtcolorbox{prinzip}[1][]{colback=blue!5!white,colframe=blue!75!black,title={#1},fonttitle=\bfseries,breakable}
\newtcolorbox{beweis}[1][]{colback=gray!5!white,colframe=gray!75!black,title={#1},fonttitle=\bfseries,breakable}
\newtcolorbox{key}[2][]{colback=blue!5!white,colframe=blue!75!black,title={#2},fonttitle=\bfseries,breakable}
\newtcolorbox{category}[1][]{colback=purple!5!white,colframe=purple!75!black,title={#1},fonttitle=\bfseries,breakable}

% =============================================================================
% SECTION 16: Additional Simple Environments
% =============================================================================
\newenvironment{treatise}{\begin{quote}}{\end{quote}}
\newenvironment{gemeinsam}{\begin{quote}}{\end{quote}}
\newenvironment{vergleich}{\begin{quote}}{\end{quote}}
\newenvironment{vorteil}{\begin{quote}}{\end{quote}}
\newenvironment{quantum}{\begin{quote}}{\end{quote}}

% =============================================================================
% SECTION 17: Layout Settings (Kindle-compatible)
% =============================================================================
\raggedbottom

\usepackage{environ}
\let\oldtabular\tabular
\let\endoldtabular\endtabular

\newenvironment{scaledtable}[1][0.85]{%
	\begingroup\footnotesize\setlength{\LTleft}{0pt}\setlength{\LTright}{0pt}%
}{%
	\endgroup%
}

\newcommand{\widetable}[1]{\resizebox{\textwidth}{!}{#1}}

% =============================================================================
% SECTION 18: Table of Contents Formatting
% =============================================================================
\renewcommand{\cftsecfont}{\color{blue}}
\renewcommand{\cftsubsecfont}{\color{blue}}
\renewcommand{\cftsecpagefont}{\color{blue}}
\renewcommand{\cftsubsecpagefont}{\color{blue}}
\renewcommand{\cfttoctitlefont}{\huge\bfseries\color{blue}}

% =============================================================================
% SECTION 19: Default Header and Footer – Buch-Style
% =============================================================================
\pagestyle{fancy}
\fancyhf{}
\fancyfoot[C]{\thepage}
% Wichtig: Auch der plain-Stil (Kapitelanfangsseiten) soll genau dasselbe haben
\fancypagestyle{plain}{%
	\fancyfoot[C]{\thepage}
}
% =============================================================================
% SECTION 20: Zusätzliche Buch-spezifische Einstellungen
% =============================================================================
\usepackage{titlesec}
\titleformat{\chapter}[display]
{\normalfont\huge\bfseries}{\chaptertitlename\ \thechapter}{20pt}{\Huge}
\titlespacing*{\chapter}{0pt}{50pt}{40pt}

\setlength{\parindent}{1.5em}
\setlength{\parskip}{0pt}

% ==============================================================================
% Ende der gemeinsamen Präambel – Buchversion
% ==============================================================================
% ================================================
% Schlüsselresultat-Box – NUR HIER definieren!
% ================================================
% ================================================
% Schlüsselresultat-Box – NUR HIER definieren!
% ================================================
