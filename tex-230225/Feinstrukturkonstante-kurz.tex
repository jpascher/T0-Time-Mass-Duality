\documentclass{article}
\usepackage[utf8]{inputenc}
\usepackage{amsmath}
\usepackage{amssymb}
\usepackage{geometry}
\usepackage{hyperref}
\usepackage{siunitx}

\title{Die Feinstrukturkonstante: Verschiedene Darstellungen und Zusammenhänge}
\author{Johann Pascher}
\date{21-03-2025}

\begin{document}
	
	\maketitle
	
	\section{Einführung zur Feinstrukturkonstante}
	
	Die Feinstrukturkonstante ($\alpha$) ist eine dimensionslose physikalische Konstante, die eine fundamentale Rolle in der Quantenelektrodynamik spielt. Sie beschreibt die Stärke der elektromagnetischen Wechselwirkung zwischen Elementarteilchen. In ihrer bekanntesten Form lautet die Formel:
	
	$$\alpha = \frac{e^2}{4\pi\varepsilon_0\hbar c} \approx \frac{1}{137.035999}$$
	\tableofcontents
	\section{Erklärung der verwendeten Formelzeichen und Einheiten}
	
	- $\alpha$: Feinstrukturkonstante (dimensionslos)
	- $e$: Elementarladung (Einheit: Coulomb, C)
	- $\varepsilon_0$: Elektrische Feldkonstante (Einheit: Farad pro Meter, F/m)
	- $\mu_0$: Magnetische Feldkonstante (Einheit: Newton pro Ampere-Quadrat, N/A²)
	- $\hbar$: Reduziertes Planck'sches Wirkungsquantum (Einheit: Joule-Sekunde, Js)
	- $c$: Lichtgeschwindigkeit (Einheit: Meter pro Sekunde, m/s)
	(Einheit: \si{\meter\per\second})	- $h$: Planck'sches Wirkungsquantum (Einheit: Joule-Sekunde, Js)
	- $m_e$: Elektronenmasse (Einheit: Kilogramm, kg)
	- $\lambda_C$: Compton-Wellenlänge (Einheit: Meter, m)
	- $Q$: Elektrische Ladung (Einheit: Coulomb, C)
	- $C$: Kapazität (Einheit: Farad, F)
	- $V$: Spannung (Einheit: Volt, V)
	- $E$: Energie (Einheit: Joule, J)
	
	\section{Alternative Formulierungen der Feinstrukturkonstante}
	
	\subsection{Darstellung mit Permeabilität}
	Ausgehend von der Standardform können wir die elektrische Feldkonstante $\varepsilon_0$ durch die magnetische Feldkonstante $\mu_0$ ersetzen, indem wir die Beziehung $c^2 = \frac{1}{\varepsilon_0\mu_0}$ nutzen:
	
	$$\varepsilon_0 = \frac{1}{\mu_0c^2}$$
	$$\alpha = \frac{e^2}{4\pi\left(\frac{1}{\mu_0c^2}\right)\hbar c}$$
	$$= \frac{e^2\mu_0c^2}{4\pi\hbar c}$$
	$$= \frac{e^2\mu_0c}{4\pi\hbar}$$
	
	Mit der Beziehung $\hbar = \frac{h}{2\pi}$ erhalten wir eine alternative Form:
	
	$$\alpha = \frac{\mu_0e^2}{2h}$$
	
	\subsection{Formulierung mit Elektronenmasse und Compton-Wellenlänge}
	Das Plancksche Wirkungsquantum $h$ lässt sich durch andere physikalische Größen ausdrücken:
	
	$$h = \frac{m_e c \lambda_C}{2\pi}$$
	
	wobei $\lambda_C$ die Compton-Wellenlänge des Elektrons ist:
	
	$$\lambda_C = \frac{h}{m_e c}$$
	
	Setzt man dies in die Feinstrukturkonstante ein:
	
	$$\alpha = \frac{\mu_0e^2}{2h}$$
	$$= \frac{\mu_0e^2}{2\frac{m_e c \lambda_C}{2\pi}}$$
	$$= \frac{\mu_0e^2 \cdot 2\pi}{2m_e c \lambda_C}$$
	$$= \frac{\mu_0e^2\pi}{m_e c \lambda_C}$$
	
	\subsection{Ausdruck mit klassischem Elektronenradius}
	Der klassische Elektronenradius ist definiert als:
	
	$$r_e = \frac{e^2}{4\pi\varepsilon_0 m_e c^2}$$
	
	Mit $\varepsilon_0 = \frac{1}{\mu_0c^2}$ ergibt sich:
	
	$$r_e = \frac{e^2\mu_0}{4\pi m_e c^2}$$
	
	Die Feinstrukturkonstante kann als Verhältnis des klassischen Elektronenradius zur Compton-Wellenlänge geschrieben werden:
	
	$$\alpha = \frac{r_e}{\lambda_C}$$
	
	Dies führt zu einer weiteren Form:
	
	$$\alpha = \frac{e^2\mu_0}{4\pi m_e c^2} \cdot \frac{2\pi m_e c}{h}$$
	$$= \frac{e^2\mu_0}{2hc}$$
	
	\subsection{Formulierung mit $\mu_0$ und $\varepsilon_0$ als fundamentale Konstanten}
	Unter Verwendung der Beziehung $c = \frac{1}{\sqrt{\mu_0\varepsilon_0}}$ kann die Feinstrukturkonstante als:
	
	$$\alpha = \frac{e^2}{4\pi\varepsilon_0\hbar c} \cdot \sqrt{\mu_0\varepsilon_0}$$
	$$= \frac{e^2}{4\pi\varepsilon_0\hbar} \cdot \sqrt{\mu_0\varepsilon_0}$$
	
	ausgedrückt werden.
	
	Diese verschiedenen Darstellungen ermöglichen unterschiedliche physikalische Interpretationen und zeigen die Verbindungen zwischen fundamentalen Naturkonstanten.
	
	
	\section{Herleitung des Planckschen Wirkungsquantums durch fundamentale elektromagnetische Konstanten}
	
	\subsection{Dimensionale Austauschbarkeit: Eine neue Perspektive}
	
	Bevor wir mit der Herleitung beginnen, ist es wichtig, die grundlegende Annahme dieser Arbeit zu erläutern: In einer fundamentalen Theorie könnten die Dimensionen von Quantengrößen (wie dem Planckschen Wirkungsquantum) und elektromagnetischen Größen (wie Permittivität und Permeabilität) austauschbar oder auf eine gemeinsame Basis zurückführbar sein.
	
	Diese Annahme basiert auf der Beobachtung, dass die Natur auf ihrer fundamentalsten Ebene eine vereinheitlichte Beschreibung erfordern könnte, in der die scheinbar unterschiedlichen Dimensionen von Quantenmechanik und Elektrodynamik als verschiedene Manifestationen derselben grundlegenden Struktur erscheinen. In einer solchen Theorie wären die etablierten Dimensionsbarrieren nicht absolut, sondern Artefakte unserer makroskopischen Perspektive.
	
	Diese Austauschbarkeit der Dimensionen ermöglicht es uns, eine direkte Verbindung zwischen h und den elektromagnetischen Konstanten herzustellen, die über die konventionelle Physik hinausgeht und möglicherweise auf eine tiefere Struktur des Universums hindeutet.
	
	\subsection{Beziehung zwischen $h$, $\mu_0$ und $\varepsilon_0$}
	
	Zunächst betrachten wir die grundlegende Beziehung zwischen der Lichtgeschwindigkeit $c$, der Permeabilität $\mu_0$ und der Permittivität $\varepsilon_0$:
	(Einheit: \si{\meter\per\second})	
	$$c = \frac{1}{\sqrt{\mu_0\varepsilon_0}}$$
	
	Wir nutzen außerdem die fundamentale Relation zwischen dem Planckschen Wirkungsquantum $h$ und der Compton-Wellenlänge $\lambda_C$ des Elektrons:
	
	$$h = \frac{m_e c \lambda_C}{2\pi}$$
	
	Die Compton-Wellenlänge ist definiert als:
	
	$$\lambda_C = \frac{h}{m_e c}$$
	
	Durch Einsetzen der Lichtgeschwindigkeit $c = \frac{1}{\sqrt{\mu_0\varepsilon_0}}$ erhalten wir:
	(Einheit: \si{\meter\per\second})	
	$$h = \frac{m_e}{2\pi} \cdot \frac{\lambda_C}{\sqrt{\mu_0\varepsilon_0}}$$
	
	Nun ersetzen wir $\lambda_C$ durch seine Definition:
	
	$$h = \frac{m_e}{2\pi} \cdot \frac{h}{m_e c \sqrt{\mu_0\varepsilon_0}}$$
	
	Dies führt zu:
	
	$$h^2 = \frac{1}{\mu_0\varepsilon_0} \cdot \frac{m_e^2 \lambda_C^2}{4\pi^2}$$
	
	Mit $\lambda_C = \frac{h}{m_e c}$ folgt:
	
	$$h^2 = \frac{1}{\mu_0\varepsilon_0} \cdot \frac{m_e^2}{4\pi^2} \cdot \frac{h^2}{m_e^2c^2}$$
	
	Nach dem Kürzen von $m_e^2$ und Einsetzen von $c^2 = \frac{1}{\mu_0\varepsilon_0}$ erhalten wir schließlich:
	
	$$h = \frac{1}{2\pi\sqrt{\mu_0\varepsilon_0}}$$
	
	Diese Gleichung zeigt, dass unter der Annahme der dimensionalen Austauschbarkeit das Plancksche Wirkungsquantum $h$ tatsächlich durch die elektromagnetischen Vakuumkonstanten $\mu_0$ und $\varepsilon_0$ ausgedrückt werden kann. Diese Beziehung deutet auf eine tiefere Verbindung zwischen Quantenmechanik und Elektrodynamik hin.
	
	Für das reduzierte Plancksche Wirkungsquantum $\hbar = \frac{h}{2\pi}$ ergibt sich entsprechend:
	
	$$\hbar = \frac{h}{2\pi} = \frac{1}{2\pi} \cdot \frac{1}{2\pi\sqrt{\mu_0\varepsilon_0}} = \frac{1}{4\pi^2\sqrt{\mu_0\varepsilon_0}}$$
	
	\section{Neudefinition der Feinstrukturkonstante}
	
	\subsection{ Was bedeutet die Elementarladung $e$?}
	
	Die Elementarladung $e$ steht für die elektrische Ladung eines Elektrons oder Protons und beträgt etwa $e \approx 1,602 \times 10^{-19}$ C (Coulomb).
	
	\subsection{Die Feinstrukturkonstante durch elektromagnetische Vakuumkonstanten}
	
	Die Feinstrukturkonstante $\alpha$ wird traditionell definiert als:
	
	$$\alpha = \frac{e^2}{4\pi\varepsilon_0\hbar c}$$
	
	Durch Einsetzen der Herleitung für $\hbar = \frac{1}{4\pi^2\sqrt{\mu_0\varepsilon_0}}$ erhalten wir:
	
	$$\alpha = \frac{e^2}{4\pi\varepsilon_0 \cdot \frac{1}{4\pi^2\sqrt{\mu_0\varepsilon_0}} \cdot c}$$
	$$= \frac{e^2}{4\pi\varepsilon_0} \cdot 4\pi^2\sqrt{\mu_0\varepsilon_0} \cdot \frac{1}{c}$$
	$$= \frac{e^2 \cdot 4\pi^2 \cdot \sqrt{\mu_0\varepsilon_0}}{4\pi\varepsilon_0 \cdot c}$$
	$$= \frac{\pi e^2 \cdot \sqrt{\mu_0\varepsilon_0}}{\varepsilon_0 \cdot c}$$
	
	Mit $c = \frac{1}{\sqrt{\mu_0\varepsilon_0}}$ ergibt sich:
	
	$$\alpha = \frac{\pi e^2 \cdot \sqrt{\mu_0\varepsilon_0}}{\varepsilon_0} \cdot \sqrt{\mu_0\varepsilon_0}$$
	$$= \frac{\pi e^2 \cdot \mu_0\varepsilon_0}{\varepsilon_0}$$
	$$= \pi e^2 \cdot \mu_0$$
	
	Alternativ mit der anderen Herleitung für $h = \frac{1}{2\pi\sqrt{\mu_0\varepsilon_0}}$:
	
	$$\alpha = \frac{e^2}{4\pi\varepsilon_0\hbar c}$$
	$$= \frac{e^2}{4\pi\varepsilon_0 \cdot \frac{h}{2\pi} \cdot c}$$
	$$= \frac{e^2 \cdot 2\pi}{4\pi\varepsilon_0 \cdot h \cdot c}$$
	$$= \frac{e^2}{2\varepsilon_0 \cdot h \cdot c}$$
	
	Setzen wir hier $h = \frac{1}{2\pi\sqrt{\mu_0\varepsilon_0}}$ ein:
	
	$$\alpha = \frac{e^2}{2\varepsilon_0 \cdot \frac{1}{2\pi\sqrt{\mu_0\varepsilon_0}} \cdot c}$$
	$$= \frac{e^2 \cdot 2\pi\sqrt{\mu_0\varepsilon_0}}{2\varepsilon_0 \cdot c}$$
	$$= \frac{\pi e^2 \cdot \sqrt{\mu_0\varepsilon_0}}{\varepsilon_0 \cdot c}$$
	
	Mit $c = \frac{1}{\sqrt{\mu_0\varepsilon_0}}$ folgt wieder:
	
	$$\alpha = \pi e^2 \cdot \mu_0$$
	
	Diese Darstellung zeigt, dass die Feinstrukturkonstante direkt aus der elektromagnetischen Struktur des Vakuums abgeleitet werden kann, ohne dass $h$ oder $\hbar$ explizit vorkommen muss.
	
	\section{Konsequenzen einer Neudefinition von Coulomb}
	
	\subsection{ Ist Coulomb falsch definiert, wenn man $\alpha = 1$ setzt?}
	
	Die Hypothese ist, dass wenn man die Feinstrukturkonstante $\alpha = 1$ setzen würde, die Definition des Coulomb und damit der Elementarladung $e$ angepasst werden müsste.
	
	\subsection{Neue Definition der Elementarladung}
	
	Wenn wir $\alpha = 1$ setzen und die ursprüngliche Formel $\alpha = \frac{e^2}{4\pi\varepsilon_0\hbar c}$ verwenden, dann ergibt sich für die Elementarladung $e$:
	
	$$e^2 = 4\pi\varepsilon_0\hbar c$$
	
	$$e = \sqrt{4\pi\varepsilon_0\hbar c}$$
	
	Alternativ mit der neu hergeleiteten Form $\alpha = \pi e^2 \cdot \mu_0$:
	
	$$1 = \pi e^2 \cdot \mu_0$$
	
	$$e = \sqrt{\frac{1}{\pi \mu_0}}$$
	
	Dies würde bedeuten, dass der Zahlenwert von $e$ sich ändern würde, weil er dann direkt von $\hbar$, $c$ und $\varepsilon_0$ bzw. von $\mu_0$ abhängt.
	
	\subsection{Physikalische Bedeutung}
	
	Die Einheit Coulomb (C) ist eine willkürliche Festlegung im SI-System. Wenn man stattdessen $\alpha = 1$ wählt, würde sich die Definition von $e$ ändern. In natürlichen Einheitensystemen (wie in der Hochenergiephysik üblich) wird oft $\alpha = 1$ gesetzt, was bedeutet, dass Ladung in einer anderen Einheit als Coulomb gemessen wird.
	
	Der aktuelle Wert der Feinstrukturkonstanten $\alpha \approx \frac{1}{137}$ ist nicht "falsch", sondern eine Konsequenz unserer historischen Definitionen von Einheiten. Man hätte ursprünglich das elektromagnetische Einheitensystem auch so definieren können, dass $\alpha = 1$ gilt.
	
	\section{Auswirkungen auf andere SI-Einheiten}
	
	\subsection{ Welche Auswirkungen hätte eine Coulomb-Anpassung auf andere Einheiten?}
	
	Eine Anpassung der Ladungseinheit, so dass $\alpha = 1$ gilt, hätte Konsequenzen für zahlreiche andere physikalische Einheiten:
	
	\subsubsection{Neue Ladungseinheit}
	Die neue Elementarladung wäre:
	$$e = \sqrt{4\pi\varepsilon_0\hbar c} = \sqrt{\frac{1}{\pi \mu_0}}$$
	
	\subsubsection{Änderung der elektrischen Stromstärke (Ampere)}
	Da $1 \text{ A} = 1 \text{ C}/\text{s}$, würde sich auch die Einheit des Ampere entsprechend ändern.
	
	\subsubsection{Änderungen der elektromagnetischen Konstanten}
	Da $\varepsilon_0$ und $\mu_0$ mit der Lichtgeschwindigkeit verknüpft sind:
	(Einheit: \si{\meter\per\second})	$$c^2 = \frac{1}{\mu_0\varepsilon_0}$$
	müssten vieleicht entweder $\mu_0$ oder $\varepsilon_0$ angepasst werden.
	
	\subsubsection{Auswirkungen auf die Kapazität (Farad)}
	Die Kapazität ist definiert als $C = \frac{Q}{V}$. Da sich $Q$ (Ladung) ändert, würde sich auch die Einheit des Farad ändern.
	
	\subsubsection{Änderungen der Spannungseinheit (Volt)}
	Die elektrische Spannung ist definiert als $1 \text{ V} = 1 \text{ J}/\text{C}$. Da Coulomb eine andere Größe hätte, würde sich auch die Größe des Volt verschieben.
	
	\subsubsection{Indirekte Auswirkungen auf die Masse}
	In der Quantenfeldtheorie ist die Feinstrukturkonstante mit der Ruhemassenenergie von Elektronen verknüpft, was indirekte Auswirkungen auf die Massendefinition haben könnte.
	\subsubsection{Prüfung der Konsistenz der Feinstrukturkonstante mit SI-Einheiten}
	
	
	% Prüfung der Konsistenz der Feinstrukturkonstante mit SI-Einheiten
	Zur Überprüfung, ob eine Anpassung von \( \mu_0 \) oder \( \varepsilon_0 \) erforderlich wäre, betrachten wir den Zusammenhang:
	\[
	c^2 = \frac{1}{\mu_0 \varepsilon_0}
	\]
	mit den bekannten Werten:
	- Lichtgeschwindigkeit: \( c = 299792458 \) m/s
	- Magnetische Feldkonstante: \( \mu_0 = 4\pi \times 10^{-7} \) H/m
	- Elektrische Feldkonstante: \( \varepsilon_0 = \frac{1}{\mu_0 c^2} \) F/m
	
	Die Feinstrukturkonstante ist definiert als:
	\[
	\alpha = \frac{e^2}{4\pi \varepsilon_0 \hbar c}
	\]
	mit:
	- Planck-Konstante: \( \hbar = 1.054571817 \times 10^{-34} \) J·s
	- Elementarladung: \( e = 1.602176634 \times 10^{-19} \) C
	
	Setzen wir die Werte ein, ergibt sich:
	\[
	\alpha_{\text{berechnet}} = 0.007297352569776441
	\]
	Der offizielle Wert beträgt:
	\[
	\alpha_{\text{offiziell}} = 0.0072973525692838015
	\]
	Die Differenz ist \( 4.93 \times 10^{-13} \), was extrem gering ist. Somit bleibt \( \alpha \) innerhalb der erwarteten Genauigkeit konstant. Eine Anpassung von \( \mu_0 \) oder \( \varepsilon_0 \) ist also nicht notwendig.
	
	% Überprüfung der Gravitationskonstante im SI-System
	Die Gravitationskonstante kann durch die Planck-Einheiten ausgedrückt werden als:
	\[
	G = \frac{\hbar c}{m_P^2}
	\]
	mit:
	- \( \hbar = 1.054571817 \times 10^{-34} \) J·s
	- \( c = 299792458 \) m/s
	- \( m_P = 2.176434 \times 10^{-8} \) kg (Planck-Masse)
	
	Setzen wir die Werte ein:
	\[
	G_{\text{berechnet}} = \frac{(1.054571817 \times 10^{-34}) \times (299792458)}{(2.176434 \times 10^{-8})^2}
	\]
	Ergebnis:
	\[
	G_{\text{berechnet}} = 6.6743021 \times 10^{-11} \text{ m}^3\text{kg}^{-1}\text{s}^{-2}
	\]
	Der offizielle Wert:
	\[
	G_{\text{offiziell}} = 6.67430 \times 10^{-11} \text{ m}^3\text{kg}^{-1}\text{s}^{-2}
	\]
	Die Abweichung beträgt lediglich \( 2.1 \times 10^{-17} \), was im Rahmen der Messgenauigkeit liegt. Das bedeutet, dass \( G \) im SI-System korrekt definiert ist und mit den Planck-Einheiten konsistent bleibt.
	
	% Herleitung der Einheitenwahl für ein alternatives Einheitensystem
	In einem alternativen Einheitensystem setzen wir bestimmte Konstanten auf 1:
	- \( c = 1 \) (Lichtgeschwindigkeit)
	- \( \hbar = 1 \) (reduzierte Planck-Konstante)
	- \( e = 1 \) (Elementarladung)
	- \( \alpha = 1 \) (Feinstrukturkonstante)
	
	Daraus folgt für die elektrische Feldkonstante:
	\[
	\varepsilon_0 = \frac{1}{4\pi}
	\]
	und mit der Beziehung \( c^2 = 1/(\mu_0 \varepsilon_0) \) erhalten wir die magnetische Feldkonstante:
	\[
	\mu_0 = 4\pi
	\]
	
	In diesem System ergeben sich veränderte Einheiten:
	- Ladungseinheit: \( e = 1 \) setzt die Einheit der Ladung fest.
	- Energieeinheit: Da \( \hbar = 1 \), wird Energie direkt in Frequenzeinheiten gemessen.
	- Zeiteinheit: Mit \( c = 1 \) sind Längeneinheiten direkt mit Zeiteinheiten verknüpft.
	
	Die Feinstrukturkonstante bleibt definitionsgemäß 1, aber die physikalischen Größen sind in diesem System anders skaliert.
	
	% Definition der Einheiten und Konsequenzen
	In unserem neuen Einheitensystem wurden folgende Konstanten auf 1 gesetzt:
	- Lichtgeschwindigkeit \( c = 1 \)
	- Reduzierte Planck-Konstante \( \hbar = 1 \)
	- Elementarladung \( e = 1 \)
	- Feinstrukturkonstante \( \alpha = 1 \)
	
	Nicht auf 1 gesetzt wurde:
	- Die Planck-Energie \( E_P \), die als Maß für die Energieeinheit dient.
	- Die Gravitationskonstante \( G \), da sie mit den Planck-Einheiten zusammenhängt.
	
	Das bedeutet, dass alle anderen physikalischen Konstanten aus diesen Werten abgeleitet werden können.
	
	% Rückrechnung der Energie in SI-Einheiten
	Die Planck-Energie ist definiert als:
	\[
	E_P = \sqrt{\frac{\hbar c^5}{G}}
	\]
	Setzen wir die bekannten Werte ein:
	\[
	E_P = \sqrt{\frac{(1.054571817 \times 10^{-34}) (299792458)^5}{6.67430 \times 10^{-11}}}
	\]
	Das ergibt:
	\[
	E_P \approx 1.956 \times 10^9 \text{ J}
	\]
	Da in unserem neuen Einheitensystem \( E_P = 1 \) gesetzt wird, folgt:
	\[
	1 \text{ Energieeinheit} = 1.956 \times 10^9 \text{ J}
	\]
	Somit kann jede Energie in unserem System durch Multiplikation mit diesem Faktor in Joule umgerechnet werden.
	
	\section*{Wahl der Feinstrukturkonstante in einem natürlichen Einheitensystem}
	
	In einem natürlichen Einheitensystem können fundamentale Naturkonstanten auf den Wert 1 gesetzt werden, um Gleichungen zu vereinfachen. Eine zentrale Frage ist, ob die Feinstrukturkonstante \( \alpha \) ebenfalls auf 1 gesetzt werden sollte oder ob ihr gemessener Wert \( \alpha \approx \frac{1}{137} \) beibehalten werden muss.
	
	\subsection*{Setzen von \( \alpha = 1 \)}
	
	Wenn \( \alpha = 1 \) gesetzt wird, ergeben sich einige Vereinfachungen:
	
	\begin{itemize}
		\item Die elektrische Permittivität wird \( \varepsilon_0 = \frac{1}{4\pi} \), und die magnetische Konstante \( \mu_0 = 4\pi \).
		\item Die Maxwell-Gleichungen nehmen eine besonders einfache Form an.
		\item Die Elementarladung \( e \) wird direkt mit Energieeinheiten verknüpft.
		\item Alle fundamentalen Konstanten werden dimensionslos.
		\item Vereinfachte Quantenelektrodynamik (QED), da die Kopplungsstärke nicht mehr als Störungsparameter auftritt.
		\item Möglichkeit zur direkten Relation zwischen Gravitation und Elektrodynamik, falls auch \( G \) entsprechend gewählt wird.
	\end{itemize}
	
	Jedoch gibt es auch Probleme:
	
	\begin{itemize}
		\item Die SI-Werte für Ladung, Energie und elektrische Konstanten müssten umskaliert werden.
	\end{itemize}
	
	\subsection*{Beibehalten von \( \alpha \approx 1/137 \)}
	
	Alternativ kann \( \alpha \) auf seinen gemessenen Wert belassen werden. Dann gelten weiterhin die bekannten Umrechnungsfaktoren:
	
	\begin{align*}
		\alpha &= \frac{e^2}{4\pi \varepsilon_0 \hbar c} \approx \frac{1}{137}
	\end{align*}
	
	Dadurch bleibt die physikalische Konsistenz mit den experimentellen Werten erhalten, aber einige Gleichungen sind etwas komplizierter.
	
	\subsection*{Mögliche alternative Umformungen}
	
	Diese natürliche Form erlaubt verschiedene alternative Formulierungen:
	
	\begin{itemize}
		\item Direkt ableitbare Relationen zwischen elektromagnetischen und gravitativen Kräften.
		\item Maxwell-Gleichungen in besonders einfacher Form.
		\item Vereinfachte Quantenelektrodynamik, da die Kopplungsstärke nicht mehr als Störungsparameter auftritt.
		\item Bessere Vergleichbarkeit von Gravitation und Elektrodynamik durch skalierte Konstanten.
		\item Vereinfachung der Planck-Einheiten und deren Verbindung zu elektromagnetischen Größen.
	\end{itemize}
	
	\subsection*{Fazit}
	
	Es gibt keine eindeutige Antwort, welche Wahl besser ist:
	
	\begin{itemize}
		\item **Falls Vereinfachung der Gleichungen das Hauptziel ist**, ist \( \alpha = 1 \) sinnvoll.
		\item **Falls physikalische Konsistenz mit Messwerten gewahrt bleiben soll**, ist \( \alpha \approx 1/137 \) besser.
	\end{itemize}
	
	Da in diesem Einheitensystem bereits \( c = 1 \), \( \hbar = 1 \) und \( e = 1 \) gesetzt wurden, wäre es konsequent, auch \( \alpha = 1 \) zu wählen, um die Gleichungen weiter zu vereinfachen.
	
	\subsection*{Quellen}
	\begin{enumerate}
		\item \href{https://de.wikipedia.org/wiki/Feinstrukturkonstante}{Feinstrukturkonstante – Wikipedia}
		\item \href{https://www.cosmos-indirekt.de/Physik-Schule/Feinstrukturkonstante}{Feinstrukturkonstante – Physik-Schule}
		\item \href{https://www.spektrum.de/lexikon/physik/feinstrukturkonstante/4829}{Feinstrukturkonstante – Lexikon der Physik}
		\item \href{https://renenyffenegger.ch/notes/Wissenschaft/Physik/Konstanten/Feinstrukturkonstante}{Feinstrukturkonstante – René Nyffenegger}
		\item \href{https://mensch-erde-universum.de/feinstrukturkonstante/}{Feinstrukturkonstante Alpha – Mensch-Erde-Universum}
		\item \href{https://de.wikipedia.org/wiki/Elektroschwache_Wechselwirkung}{Elektroschwache Wechselwirkung – Wikipedia}
		\item \href{https://www.cosmos-indirekt.de/Physik-Schule/Elektroschwache_Wechselwirkung}{Elektroschwache Wechselwirkung – Physik-Schule}
		\item \href{https://www.spektrum.de/lexikon/physik/elektroschwache-wechselwirkung/4197}{Elektroschwache Wechselwirkung – Lexikon der Physik}
		\item \href{https://de.wikipedia.org/wiki/Nat%C3%BCrliche_Einheiten}{Natürliche Einheiten – Wikipedia}
		\item \href{https://www-static.etp.physik.uni-muenchen.de/fp-versuch/node5.html}{Einführung in die Teilchenphysik – LMU}
		\item \href{https://www.thphys.uni-heidelberg.de/~wolschin/alpha.html}{SNO – Heidelberg University}
		\item \href{https://vixra.org/pdf/1408.0018vM.pdf}{Die Entschlüsselung der Feinstrukturkonstante}
		\item \href{https://nsosp.org/de/Quanten-Fluss-Theorie/Elektroschwache-Wechselwirkung-Teilchenumwandlungen-schwacher-Isospin_de.php}{Elektroschwache Wechselwirkung, Teilchenumwandlungen und innerer Spin}
		\item \href{https://www.rhetos.de/html/lex/feinstrukturkonstante.htm}{Feinstrukturkonstante (Etwa 1/137) – Rhetos}
	\end{enumerate}
	
	
	
\end{document}

