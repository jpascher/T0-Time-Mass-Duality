\documentclass{article}
\usepackage[utf8]{inputenc}
\usepackage[T1]{fontenc}
\usepackage{amsmath}
\usepackage{amssymb}
\usepackage{geometry}
\usepackage{hyperref}
\usepackage{siunitx}

\title{Eine spekulative Erweiterung der Zeit als emergente Eigenschaft: \\Dokumentation der Diskussion mit einer neuen Perspektive}
\author{Unter Mitwirkung von Johann Paschers Konzepten}
\date{März 23, 2025}

\begin{document}
	
	\maketitle
	
	\section{Einführung}
	
	Diese Arbeit basiert auf Johann Paschers Konzept (\textit{Die Feinstrukturkonstante: Verschiedene Darstellungen und Zusammenhänge}, 2025) und untersucht eine neue Perspektive: Zeit als unveränderlich, Energie als variabel, mit vakuum-spezifischem \( c \). Die Diskussion entwickelt sich durch iterative Analysen und Fragen, die kosmologische und beobachtbare Implikationen hinterfragen.
	
	\section{Grundlage: Paschers Konzept der intrinsischen Zeit}
	
	\subsection{Definition und Herleitung}
	
	\[
	E = mc^2 = \frac{\hbar}{T}, \quad T = \frac{\hbar}{mc^2}
	\]
	
	\subsection{Energie-Zeit-Unschärferelation}
	
	\[
	\Delta E \cdot \Delta t \geq \frac{\hbar}{2}
	\]
	
	\subsection{Feinstrukturkonstante}
	
	\[
	\alpha = \frac{e^2}{4\pi \varepsilon_0 \hbar c}
	\]
	
	\section{Ursprünglicher Ansatz: Variable Zeit, feste Energie}
	
	\subsection{Überblick}
	
	\[
	E = mc^2, \quad T = \frac{\hbar}{E}
	\]
	
	\section{Neue Perspektive: Unveränderliche Zeit, variable Energie}
	
	\subsection{Hypothese}
	
	Zeit ist eine feste Konstante \( T_0 \) (z. B. \( t_P = 5.39 \times 10^{-44} \, \text{s} \)):
	
	\[
	E = \frac{\hbar}{T_0}
	\]
	
	\subsection{Mathematische Modifikation}
	
	\subsubsection{Variable Masse bei konstantem \( c \)}
	- \( c = c_0 \) (vakuum-spezifisch):
	\[
	E = m c_0^2 = \frac{\hbar}{T_0}, \quad m = \frac{\hbar}{T_0 c_0^2}
	\]
	- \( m \) variiert mit \( E \).
	
	\subsection{Physikalische Implikationen}
	
	\subsubsection{Quantenmechanik}
	\[
	i\hbar \frac{\partial \Psi}{\partial T_0} = \hat{H}(E) \Psi
	\]
	
	\subsubsection{Relativitätstheorie}
	\[
	\gamma = \frac{1}{\sqrt{1 - \frac{v^2}{c_0^2}}}
	\]
	
	\subsection{Kompatibilität mit \( c \) und ART}
	
	\subsubsection{Mit konstantem \( c \)}
	- \( c = c_0 \) im Vakuum, kompatibel mit ART im beobachtbaren Bereich.
	
	\subsection{Energie: Ausgedehntes Universum vs. Davor}
	
	\subsubsection{Ausgedehntes Universum}
	- \( m_{\text{total}} \approx 10^{53} \, \text{kg} \) (Materie + Dunkle Materie).
	- \( E_{\text{total}} = m_{\text{total}} c_0^2 \approx 10^{70} \, \text{J} \).
	- Dichte: \( \rho \approx 10^{-27} \, \text{kg/m}^3 \).
	
	\subsubsection{Davor existierend}
	- Dichte: \( 10^{113} \, \text{kg/m}^3 \) (Planck-Dichte), Volumen spekulativ (z. B. \( 10^{-25} \, \text{m}^3 \)):
	\[
	m_{\text{total}} = 10^{113} \cdot 10^{-25} = 10^{88} \, \text{kg}, \quad E_{\text{total}} = 10^{105} \, \text{J}
	\]
	- Ohne Energieerhaltung möglich.
	
	\subsection{Folgen im erfahrbaren Bereich}
	
	\subsubsection{Absolute Zeit und Massedilatation}
	- \( T_0 \) fest impliziert keine Zeitdilatation, sondern \( m' = \gamma m \).
	- Dualitätsprinzip: \( m \) und \( T_0 \) austauschbar via Planck-Einheiten.
	
	\subsubsection{Messungen überprüft}
	- **Zeitmessung**: Uhren basieren auf Frequenz \( f = \frac{E}{h} \), \( E = m c_0^2 \).
	- Standard: Zeitdilatation (\( t' = \gamma t \)).
	- Modell: \( m \)-Variation, \( f \) ändert sich.
	- **Masse**: \( m_0 \) konstant (LHC, Gravitation).
	- **Widerlegung**: Zeitdilatation (GPS: \( 38 \, \mu\text{s/Tag} \)), \( m_0 \) invariant.
	
	\subsubsection{Photonenbasierte Zeitmessung}
	- Lichtlaufzeit: \( t = \frac{d}{c_0} \), rein photonenbasiert möglich.
	- Gravitationsrotverschiebung: \( f' = f \sqrt{1 - \frac{2GM}{r c_0^2}} \).
	- Modell: \( T_0 \) fest, \( f \)-Änderung durch \( E \), nicht Zeit.
	- Ergebnis: Zeitdilatation erfordert variable Zeit.
	
	\subsubsection{Kosmologische Messungen}
	- **Rotverschiebung**: Standard: Expansion (\( z \approx H_0 d / c_0 \)), Modell: \( E \)-Verlust oder \( m \)-Variation.
	- **CMB**: Standard: \( T \propto a^{-1} \), Modell: \( m \)-Gradienten.
	- **Gravitationslinsen**: Standard: Zeitverzögerung, Modell: \( m \)-Effekt.
	- **Fehlinterpretation**: Zeitdilatation könnte \( m \)-Änderung sein, aber Standard konsistenter.
	
	\subsection{Spekulative Implikationen}
	
	- Energieerhaltung nicht zwingend; „davor“ mehr Energie möglich.
	- Kosmologische Phänomene könnten falsch als Zeitdilatation interpretiert sein.
	
	\section{Verbindung zu Schwarzen Löchern und Urknall}
	
	\subsection{Hypothese}
	Schwarze Löcher und Urknall als Zustände variabler \( m \) und \( E \), wo \( T_0 \) fest plausibel sein könnte.
	
	\section{Diskussionsverlauf}
	
	\subsection{Startpunkt}
	Paschers \( T = \frac{\hbar}{mc^2} \).
	
	\subsection{Entwicklung}
	- Variable Zeit → Feste \( T_0 \), variable \( m \).
	- Energieerhaltung hinterfragt.
	- Messmethoden (Zeit, Masse, Photonen) geprüft.
	- Kosmologische Messungen analysiert.
	
	\section{Schlussfolgerung}
	
	Die Hypothese einer absoluten Zeit \( T_0 \) und variablen Masse \( m \) ist logisch konsistent und könnte „davor“ mehr Energie (\( 10^{105} \, \text{J} \) vs. \( 10^{70} \, \text{J} \)) erklären. Im erfahrbaren Bereich wird sie durch Zeitdilatation und Masseinvarianz widerlegt. Kosmologische Messungen könnten fehlerbehaftet sein, wenn Zeitdilatation falsch interpretiert wird, doch das Standardmodell bleibt konsistenter.
	
	\begin{thebibliography}{1}
		\bibitem{pascher} Pascher, J. (2025). \textit{Die Feinstrukturkonstante: Verschiedene Darstellungen und Zusammenhänge}. Unveröffentlichtes Manuskript.
	\end{thebibliography}
	
\end{document}