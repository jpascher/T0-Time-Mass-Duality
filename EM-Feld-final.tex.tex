\documentclass[12pt,a4paper]{article}
\usepackage[utf8]{inputenc}
\usepackage[german]{babel}
\usepackage{amsmath}
\usepackage{graphicx}
\usepackage{physics}
\usepackage{siunitx}
\usepackage{url}
\usepackage{tikz}

\title{Das elektromagnetische Feld - Eine neue Perspektive auf Licht und Elektrizität}
\author{Johann Pascher}
\date{2.2.2025}

\begin{document}

\maketitle

\section*{Einleitung}
Unsere alltägliche Vorstellung von elektrischem Strom und Licht ist oft vereinfacht und kann zu Missverständnissen führen. Besonders irreführend sind dabei unsere klassischen Vorstellungen von ``Wellen'' und ``Strahlen''. Diese Konzepte waren zwar für die frühe Naturwissenschaft nützliche Berechnungsgrundlagen, entfernen sich aber weit von der physikalischen Realität. Im Folgenden werden wir anhand verschiedener Beispiele zeigen, wie das elektromagnetische Feld eine fundamentalere Beschreibung der Realität bietet.

\section{Die Täuschung unserer Wahrnehmung}
\subsection{Das Beispiel des geteilten Laserstrahls}
Stellen Sie sich einen Laserstrahl vor, der durch einen halbdurchlässigen Spiegel in zwei Strahlen aufgeteilt wird. Was wir sehen, sind zwei getrennte Lichtwege. Diese Wahrnehmung verführt uns zu der Annahme, dass sich das Licht tatsächlich nur entlang dieser sichtbaren Pfade ausbreitet. Doch dies ist eine Täuschung -- was wir sehen, sind nur die Orte, an denen das Licht mit Materie wechselwirkt und dadurch für uns sichtbar wird.

Das zugrundeliegende EM-Feld ist jedoch nicht auf diese sichtbaren Pfade beschränkt. Es breitet sich im gesamten Raum aus, auch wenn wir seine Präsenz nur an bestimmten Wechselwirkungspunkten wahrnehmen können.

\section{Analogien zur Veranschaulichung}
\subsection{Wasserwellen}
Wenn Sie einen Stein ins Wasser werfen, sehen Sie Wellenringe auf der Oberfläche. Diese Ringe sind aber nicht die eigentliche Ausbreitung der Energie -- sie zeigen nur, wo die Energie an der Wasseroberfläche sichtbar wird. Tatsächlich breitet sich die Energie in alle Richtungen im Wasser aus, auch wenn wir nur die Oberflächeneffekte sehen können.

\subsection{Schallwellen und die Schallmauer}
Schallwellen bieten eine besonders aufschlussreiche Analogie:
\begin{itemize}
    \item Schallwellen breiten sich in der Luft deutlich langsamer aus als Licht (etwa \SI{343}{\meter\per\second} gegenüber \SI{3e8}{\meter\per\second})
    \item Bei Schall sind es Luftteilchen, die vor Ort schwingen -- sie bewegen sich nicht vom Platz weg, sondern geben die Schwingung an benachbarte Teilchen weiter
    \item Dies ist vergleichbar mit einer ``La-Ola-Welle'' im Stadion: Die Menschen bleiben an ihrem Platz, aber die Welle läuft durch das Stadion
    \item Ein besonders interessanter Fall tritt bei Überschallflugzeugen auf: Hier bewegt sich das schallerzeugende Objekt schneller als die maximale Ausbreitungsgeschwindigkeit des Schalls in der Luft, was zum bekannten ``Überschallknall'' führt
\end{itemize}

\section{Elektronen und das EM-Feld}
\subsection{Elektronen im Vakuum}
Ähnlich wie bei der Schallmauer gibt es auch für Elektronen eine absolute Geschwindigkeitsgrenze:
\begin{itemize}
    \item Elektronen können selbst im Vakuum nie die Lichtgeschwindigkeit erreichen
    \item Dies ist keine technische Beschränkung, sondern ein fundamentales Prinzip der Natur
    \item Die Elektronen können sich der Lichtgeschwindigkeit nur annähern, sie aber nie erreichen
    \item Im Gegensatz zur Schallmauer ist die Lichtgeschwindigkeit eine absolute Grenze, die nicht überwunden werden kann
\end{itemize}

\subsection{Die überraschende Wahrheit über elektrischen Strom}
Ein noch deutlicheres Beispiel für die Täuschung unserer Vorstellung finden wir beim elektrischen Strom:
\begin{itemize}
    \item Die Elektronen in einem stromdurchflossenen Leiter bewegen sich erstaunlich langsam -- oft nur wenige Millimeter pro Sekunde
    \item Die elektrische Energie wird dennoch nahezu mit Lichtgeschwindigkeit übertragen
    \item Die Energie wird nicht durch die Bewegung der Elektronen, sondern durch das EM-Feld um den Leiter herum übertragen
    \item Dies erklärt auch die Funktionsweise von Transformatoren, wo Energie ohne direkten Kontakt übertragen wird
\end{itemize}

\subsection{LED und Halbleiter}
Bei der Lichterzeugung in einer LED zeigt sich ein weiterer wichtiger Aspekt:
\begin{itemize}
    \item Die Elektronen bewegen sich nur innerhalb des Halbleitermaterials und bleiben dort
    \item Sie ``fliegen'' nicht weg, um das Licht zu transportieren
    \item Stattdessen regen ihre lokalen Übergänge im Halbleiter das EM-Feld an
    \item Das angeregte EM-Feld breitet sich dann in den Raum aus
    \item Durch geschickte Konstruktion können wir dieses Feld in bestimmte Richtungen bündeln
    \item Dies verdeutlicht nochmals den fundamentalen Unterschied zwischen den lokalen Elektronenbewegungen und der weitreichenden Ausbreitung des EM-Feldes
\end{itemize}
\section{Ein aufschlussreiches Gedankenexperiment: Alice und Bob}
\subsection{Aufbau}
Betrachten wir folgendes Gedankenexperiment:
\begin{itemize}
    \item Eine Sendeantenne sendet ein Signal aus
    \item Alice und Bob befinden sich an verschiedenen Orten, jeweils mit einer Empfangsantenne
    \item Alle drei (Sender, Alice und Bob) haben den gleichen Abstand zueinander -- bilden also ein gleichseitiges Dreieck
\end{itemize}

\subsection{Instantanität und ihre Grenzen}
Das Alice-Bob-Beispiel zeigt einen faszinierenden Aspekt der EM-Feld-Ausbreitung. Wir beobachten zwei Arten von ``Gleichzeitigkeit'':

\begin{enumerate}
    \item \textbf{Die begrenzte Ausbreitung:}
    \begin{itemize}
        \item Das Signal braucht die Zeit $d/c$ vom Sender zu den Empfängern
        \item Diese Verzögerung ist fundamentaler Natur und kann nicht umgangen werden
        \item Sie ergibt sich aus der endlichen Lichtgeschwindigkeit
    \end{itemize}

    \item \textbf{Die absolute Korrelation:}
    \begin{itemize}
        \item Die absolute Entfernung vom Sender spielt keine Rolle
        \item Solange zwei oder mehr Empfänger den gleichen Abstand zum Sender haben, empfangen sie das Signal absolut gleichzeitig
        \item Dies gilt für beliebig große Entfernungen
        \item Ob die Empfänger 1 Meter oder 1 Lichtjahr vom Sender entfernt sind: Wenn ihr Abstand zum Sender gleich ist, ist der Empfang synchron
        \item Die Empfänger können dabei beliebig weit voneinander entfernt sein
        \item Diese Korrelation ist eine fundamentale Eigenschaft des EM-Feldes
    \end{itemize}
\end{enumerate}

\subsection{Energietransport durch statische EM-Felder}
Der Energietransport in einem Gleichstromkreis erfolgt auf überraschende Weise:

\begin{itemize}
    \item \textbf{Poynting-Vektor als Schlüssel:}
    \begin{itemize}
        \item Der Poynting-Vektor \( \mathbf{S} = \mathbf{E} \times \mathbf{H} \) beschreibt den Energiefluss
        \item Er zeigt radial in den Leiter hinein
        \item Die Energie fließt also durch das Feld von außen in den Leiter
    \end{itemize}

    \item \textbf{Rolle der Felder:}
    \begin{itemize}
        \item Das elektrische Feld \( \mathbf{E} \) verläuft entlang des Leiters
        \item Das magnetische Feld \( \mathbf{H} \) umgibt den Leiter zirkulär
        \item Beide Felder erscheinen makroskopisch statisch
        \item Transportieren dennoch kontinuierlich Energie
    \end{itemize}

    \item \textbf{Energiefluss und Dissipation:}
    \begin{itemize}
        \item Die Energie fließt durch das Feld, nicht durch den Leiter
        \item Die Elektronen dienen nur als ``Vermittler''
        \item Im Widerstand wird die Energie als Joulesche Wärme dissipiert
        \item Der Energiefluss bleibt kontinuierlich aufrechterhalten
    \end{itemize}
\end{itemize}

\subsection{Mathematische Beschreibung des Energieflusses}
Die quantitative Analyse des Energietransports zeigt interessante Details:

\begin{itemize}
    \item \textbf{Der Poynting-Vektor:}
    \[ \mathbf{S} = \mathbf{E} \times \mathbf{H} = \frac{1}{\mu_0} \mathbf{E} \times \mathbf{B} \]
    \begin{itemize}
        \item \( \mathbf{S} \) gibt die Energieflussdichte in W/m² an
        \item Die Richtung von \( \mathbf{S} \) zeigt die Energieflussrichtung
        \item Die Magnitude von \( \mathbf{S} \) entspricht der Leistungsdichte
    \end{itemize}

    \item \textbf{Feldkonfiguration im Leiter:}
    \begin{itemize}
        \item Elektrisches Feld: \( \mathbf{E} = E_z \hat{z} = \rho j \hat{z} \)
        \item Magnetisches Feld: \( \mathbf{H} = \frac{I}{2\pi r} \hat{\phi} \)
        \item Daraus resultiert der radiale Energiefluss nach innen
    \end{itemize}

    \item \textbf{Integration über eine Zylinderfläche:}
    \[ P = \oint \mathbf{S} \cdot d\mathbf{A} = I^2R \]
    \begin{itemize}
        \item Dies entspricht exakt der Jouleschen Verlustleistung
        \item Bestätigt den Energietransport durch das Feld
    \end{itemize}
\end{itemize}

\begin{figure}[h]
\centering
\begin{tikzpicture}
% Leiter (Querschnitt)
\draw[fill=gray!20] (0,0) circle (1);

% Elektrisches Feld
\draw[->,thick] (-0.5,-0.5) -- (0.5,0.5);
\draw[->,thick] (-0.3,-0.7) -- (0.7,0.3);
\draw[->,thick] (-0.7,-0.3) -- (0.3,0.7);
\node at (1.5,1) {$\mathbf{E}$};

% Magnetisches Feld (konzentrische Kreise)
\draw[dashed] (0,0) circle (1.5);
\draw[dashed] (0,0) circle (2);
\draw[->] (1.5,0) arc (0:45:1.5);
\node at (2.2,0) {$\mathbf{H}$};

% Poynting-Vektor
\foreach \angle in {0,45,...,315} {
    \draw[->,red,thick] ({\angle+22}:2.3) -- ({\angle+22}:1.2);
}
\node[red] at (2.8,2) {$\mathbf{S}$};

% Legende
\node at (0,-3) {Querschnitt eines stromdurchflossenen Leiters};
\end{tikzpicture}
\caption{Feldkonfiguration und Energiefluss bei einem stromdurchflossenen Leiter}
\end{figure}
\section{Von der Atomhülle zur fundamentalen Struktur der Materie}

\subsection{Eine neue Perspektive auf die Atomhülle}
Die aktuelle Forschung zeigt ein komplexeres Bild der Atomstruktur:

\begin{itemize}
    \item \textbf{Moderne Interpretation der Atomhülle:}
    \begin{itemize}
        \item Keine scharf definierten Bahnen
        \item Stattdessen eine räumlich fast unbegrenzte, schwingende Hülle
        \item Elektronenwolke statt fester Orbitale
        \item Fließende Übergänge statt scharfer Grenzen
    \end{itemize}

    \item \textbf{Wechselspiel mit dem EM-Feld:}
    \begin{itemize}
        \item Elektronenbewegung ist ohne EM-Feld nicht denkbar
        \item Das EM-Feld ist integraler Bestandteil der Hüllenstruktur
        \item Die Elektronen bewegen sich langsamer als die EM-Feld-Ausbreitung
        \item Dies wirft fundamentale Fragen nach dem Mechanismus der Wechselwirkung auf
    \end{itemize}
\end{itemize}

\subsection{Materie als Schwingungsphänomen}
Diese Beobachtungen führen zu einer tiefgreifenden Hypothese:

\begin{itemize}
    \item \textbf{Fundamentale Überlegung:}
    \begin{itemize}
        \item Möglicherweise ist alle Materie, einschließlich der Elektronen und ihrer Bausteine, als Schwingungsknoten darstellbar
        \item Diese Schwingungsknoten könnten fundamentale Strukturen im Raum sein
        \item Die beobachteten Teilcheneigenschaften wären dann emergente Phänomene dieser Schwingungen
    \end{itemize}

    \item \textbf{Analogien zu bekannten Phänomenen:}
    \begin{itemize}
        \item Ähnlich wie stehende Wellen in einem Medium
        \item Vergleichbar mit Resonanzen in schwingenden Systemen
        \item Die Stabilität der Materie könnte aus der Stabilität dieser Schwingungsmuster resultieren
    \end{itemize}

    \item \textbf{Mögliche Eigenschaften:}
    \begin{itemize}
        \item Die Schwingungsknoten könnten verschiedene Symmetrien aufweisen
        \item Unterschiedliche Teilchen könnten verschiedene Schwingungsmoden darstellen
        \item Die Wechselwirkungen zwischen Teilchen wären dann Kopplungen zwischen Schwingungsmoden
    \end{itemize}
\end{itemize}

\subsection{Quantenphänomene als spezielle Randbedingungen}
In diesem Kontext erscheinen Quantenphänomene in einem neuen Licht:

\begin{itemize}
    \item \textbf{Analogien zu klassischen Anomalien:}
    \begin{itemize}
        \item Ähnlich wie Wasser bei 4°C seine höchste Dichte aufweist
        \item Wie Supraleitung nur unter bestimmten Bedingungen auftritt
        \item Quanteneffekte als spezielle Ausprägungen unter extremen Bedingungen
    \end{itemize}

    \item \textbf{Diskrete Energieniveaus:}
    \begin{itemize}
        \item Ähnlich wie Raummoden in Resonatoren
        \item Als natürliche Folge von Randbedingungen
        \item Die Vielfachheit der Energieniveaus entspricht harmonischen Schwingungen
    \end{itemize}
\end{itemize}

\section{Skalenabhängige Phänomene bei universellen Naturgesetzen}

Die fundamentalen Naturgesetze bleiben über alle Größenordnungen erhalten, aber ihre Ausprägung und relative Bedeutung ändert sich:

\begin{itemize}
    \item \textbf{Universelle Naturgesetze:}
    \begin{itemize}
        \item Die Lichtgeschwindigkeit als absolute Grenze
        \item Energie- und Impulserhaltung
        \item Kausalität der Wechselwirkungen
        \item Eigenschaften des Vakuums
    \end{itemize}

    \item \textbf{Im makroskopischen Bereich:}
    \begin{itemize}
        \item Klassische EM-Feld-Eigenschaften dominieren
        \item Welleneigenschaften sind gut beobachtbar
        \item Störungen breiten sich kausal aus
        \item Einzelteilcheneffekte mitteln sich heraus
    \end{itemize}

    \item \textbf{Im mikroskopischen Bereich:}
    \begin{itemize}
        \item Andere Effekte werden dominant
        \item Diskrete Energieniveaus werden wichtig
        \item Tunneleffekte werden bedeutsam
        \item Einzelteilchen-Wechselwirkungen bestimmen das Verhalten
    \end{itemize}
\end{itemize}

\section*{Literatur}

\begin{enumerate}
    \item Griffiths, D. J. (2018). \textit{Introduction to Quantum Mechanics}. Cambridge University Press.
    
    \item Jackson, J. D. (1999). \textit{Classical Electrodynamics}. Wiley.
    
    \item Feynman, R. P., Leighton, R. B., \& Sands, M. (2011). \textit{The Feynman Lectures on Physics, Vol. 2: Mainly Electromagnetism and Matter}. Basic Books.
    
    \item Bohm, D. (1952). \textit{A Suggested Interpretation of the Quantum Theory in Terms of "Hidden" Variables}. Physical Review, 85(2), 166--193.
    
    \item de Broglie, L. (1924). \textit{Recherches sur la théorie des quanta}. Annales de Physique, 10(3), 22--128.
    
    \item Landau, L. D., \& Lifshitz, E. M. (1977). \textit{Quantum Mechanics: Non-Relativistic Theory}. Pergamon Press.
    
    \item Sakurai, J. J., \& Napolitano, J. (2017). \textit{Modern Quantum Mechanics}. Cambridge University Press.
    
    \item Aspect, A., Dalibard, J., \& Roger, G. (1982). \textit{Experimental Test of Bell's Inequalities Using Time-Varying Analyzers}. Physical Review Letters, 49(25), 1804--1807.
    
    \item Haroche, S., \& Raimond, J.-M. (2006). \textit{Exploring the Quantum: Atoms, Cavities, and Photons}. Oxford University Press.
    
    \item Weinberg, S. (1995). \textit{The Quantum Theory of Fields, Vol. 1: Foundations}. Cambridge University Press.
    
    \item Wilczek, F. (2008). \textit{The Lightness of Being: Mass, Ether, and the Unification of Forces}. Basic Books.
    
    \item arXiv.org (\url{https://arxiv.org})
    
    \item Stanford Encyclopedia of Philosophy: Quantum Mechanics (\url{https://plato.stanford.edu/entries/qm})
\end{enumerate}

\end{document}