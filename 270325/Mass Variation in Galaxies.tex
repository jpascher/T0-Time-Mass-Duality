\documentclass[a4paper,12pt]{article}
\usepackage[utf8]{inputenc}
\usepackage[english]{babel}
\usepackage{amsmath, amssymb, amsthm}
\usepackage{physics}
\usepackage{graphicx}
\usepackage{hyperref}
\usepackage{tikz}
\usepackage{setspace}
\usepackage{tcolorbox}
\usepackage{xcolor}
\newtheorem{theorem}{Theorem}
\newtheorem{lemma}[theorem]{Lemma}
\newtheorem{proposition}[theorem]{Proposition}
\newtheorem{corollary}[theorem]{Corollary}
\newtheorem{definition}{Definition}
\begin{document}
	
\title{Mass Variation in Galaxies: \\A Mathematical Analysis in the T0 Model}
\author{Johann Pascher}
\date{March 27, 2025}
\maketitle

\begin{abstract}
	This paper develops a detailed mathematical analysis of galaxy rotation within the framework of the T0 model with absolute time and variable mass. In contrast to the standard dark matter model, it is demonstrated that the observed flat rotation curves can be explained as a consequence of an effective mass variation arising from coupling with dark energy. The document derives the corresponding field equations, quantifies the necessary coupling constants, and compares the predictions with the $\Lambda$CDM model. Finally, specific experimental tests are proposed that could distinguish between the two approaches.
\end{abstract}

\tableofcontents
\newpage

\section{Introduction}

The rotation curves of galaxies exhibit behavior that cannot be explained by visible matter alone. In the outer regions of spiral galaxies, the rotational velocity $v(r)$ remains nearly constant rather than decreasing with $r^{-1/2}$, as predicted by Kepler’s law for isolated masses. The standard cosmological model ($\Lambda$CDM) accounts for this phenomenon by assuming an invisible component, dark matter, which forms an extended halo around galaxies and determines the motion of visible matter through its gravitational field.

This paper pursues an alternative approach based on the T0 model, in which time is absolute and the mass of particles varies instead. Within this framework, dark matter is not considered a separate entity but rather a manifestation of an effective mass variation resulting from interaction with dark energy. This reformulation leads to mathematically equivalent predictions for rotation curves while offering a fundamentally different physical interpretation.

In the following, we will mathematically refine this approach, derive the necessary field equations, and determine the coupling constants from observational data. Subsequently, we will analyze which experimental tests could differentiate between the T0 model and the standard model.

\section{Fundamentals of the T0 Model}

Before examining the specific implications for galaxies, we summarize the basic principles of the T0 model.

\subsection{Fundamental Assumptions}

In contrast to special relativity, where rest mass remains constant and time is variable, the T0 model postulates:

\begin{tcolorbox}[colback=blue!5!white,colframe=blue!75!black,title=Basic Assumptions of the T0 Model]
	\begin{align}
		&\text{1. Time $T_0$ is absolute and universally constant.} \\
		&\text{2. Mass varies according to $m = \gamma m_0$, where $\gamma = \frac{1}{\sqrt{1-v^2/c_0^2}}$.} \\
		&\text{3. Total energy is expressed as $E = \frac{\hbar}{T_0}$.}
	\end{align}
\end{tcolorbox}

For a galaxy, this implies that the time coordinate $T_0$ is identical for all objects, regardless of their velocity or position in the gravitational field. Instead, the mass of particles varies depending on their velocity and local energy gradients.

\subsection{Dynamic Masses and Fields}

In the T0 model, the mass of a particle can be regarded as a dynamic quantity that interacts with a scalar field $\phi_{DE}$ representing dark energy. The effective mass $m_{eff}$ of a particle is then:

\begin{equation}
	m_{eff}(r) = m_0 \cdot f(\phi_{DE}(r))
\end{equation}

where $f$ is a function describing the coupling between the particle and the dark energy field. This coupling can be modeled by a Yukawa-like term in the Lagrangian density:

\begin{equation}
	\mathcal{L}_{int} = -g \phi_{DE} \bar{\psi}\psi
\end{equation}

Here, $g$ is the coupling constant, $\phi_{DE}$ is the dark energy field, and $\bar{\psi}\psi$ is the matter field. This approach resembles the Higgs mechanism, but the coupling here is position-dependent.\\

\section{Mathematical Formulation of Galaxy Dynamics}

We now consider the specific implications of this model for the motion of stars in galaxies.

\subsection{Rotation Curves in the Standard Model}

In Newtonian mechanics, the rotational velocity $v(r)$ of an object in a circular orbit around a mass $M$ is given by:

\begin{equation}
	v^2(r) = \frac{GM(r)}{r}
\end{equation}

where $G$ is the gravitational constant and $M(r)$ is the mass within radius $r$. For a point mass at the center ($M(r) = M_0$), this yields:

\begin{equation}
	v(r) \propto r^{-1/2}
\end{equation}

For an exponentially declining disk density $\rho_{disk}(r) \propto e^{-r/r_d}$, the rotational velocity would also decrease with increasing radius after reaching a maximum.

However, observations show that $v(r)$ remains nearly constant in the outer regions of galaxies. In the $\Lambda$CDM model, this is explained by assuming a dark matter halo, typically modeled with an NFW (Navarro-Frenk-White) profile:

\begin{equation}
	\rho_{DM}(r) = \frac{\rho_0}{\frac{r}{r_s}\left(1 + \frac{r}{r_s}\right)^2}
\end{equation}

This profile, combined with baryonic matter, results in a flat rotation curve.

\subsection{Effective Mass Variation in the T0 Model}

In the T0 model, we instead consider an effective mass variation arising from interaction with dark energy. The rotational velocity is then described by the modified equation:

\begin{equation}
	\frac{G \cdot m_{eff}(r) \cdot M(r)}{r^2} = \frac{v^2(r)}{r} \cdot m_{eff}(r)
\end{equation}

where $m_{eff}(r)$ is the effective mass of a test particle (e.g., a star) at position $r$. Canceling $m_{eff}(r)$ yields:

\begin{equation}
	v^2(r) = \frac{GM(r)}{r}
\end{equation}

This appears formally identical to the Newtonian equation, but $M(r)$ is now the effective total mass within $r$, given by integrating the density weighted by the effective mass:

\begin{equation}
	M_{eff}(r) = \int_0^r 4\pi r'^2 \rho_{baryon}(r') \cdot \frac{m_{eff}(r')}{m_0} \, dr'
\end{equation}

To produce a flat rotation curve, we require $v(r) \approx$ const. for large $r$, implying $M_{eff}(r) \propto r$. This can be achieved if the effective mass function takes a specific form.

\subsection{Field Equations for the Dark Energy Field}

To derive the effective mass function $m_{eff}(r)$, we must model the dynamics of the dark energy field $\phi_{DE}$. We start with the Lagrangian density:

\begin{equation}
	\mathcal{L}_{DE} = -\frac{1}{2}\partial_\mu \phi_{DE} \partial^\mu \phi_{DE} - V(\phi_{DE}) - g\phi_{DE}\bar{\psi}\psi
\end{equation}

The field equation for $\phi_{DE}$ is then:

\begin{equation}
	\nabla^2 \phi_{DE} = \frac{dV}{d\phi_{DE}} + g\rho_{baryon}(r)
\end{equation}

where $\rho_{baryon}(r)$ is the baryonic mass density. For a static, radially symmetric system, this simplifies to:

\begin{equation}
	\frac{1}{r^2}\frac{d}{dr}\left(r^2\frac{d\phi_{DE}}{dr}\right) = \frac{dV}{d\phi_{DE}} + g\rho_{baryon}(r)
\end{equation}

Assuming a massless field ($V(\phi_{DE}) = 0$), we obtain:

\begin{equation}
	\frac{1}{r^2}\frac{d}{dr}\left(r^2\frac{d\phi_{DE}}{dr}\right) = g\rho_{baryon}(r)
\end{equation}

For an exponentially declining baryonic density $\rho_{baryon}(r) = \rho_0 e^{-r/r_0}$, this equation has the solution:

\begin{equation}
	\phi_{DE}(r) = -\frac{g\rho_0 r_0^2}{r}(1 - (1 + \frac{r}{r_0})e^{-r/r_0})
\end{equation}

For $r \gg r_0$ (outside the galactic core), this simplifies to:

\begin{equation}
	\phi_{DE}(r) \approx -\frac{g\rho_0 r_0^2}{r}
\end{equation}

\section{Detailed Analysis of Mass Variation}

We can now examine the effective mass function $m_{eff}(r)$ in detail.

\subsection{Approach to Effective Mass}

Based on the coupling between the dark energy field and matter, we define:

\begin{equation}
	m_{eff}(r) = m_0(1 + \alpha\phi_{DE}(r))
\end{equation}

where $\alpha$ is a coupling constant. Substituting the field profile for large $r$ yields:

\begin{equation}
	m_{eff}(r) = m_0\left(1 - \alpha\frac{g\rho_0 r_0^2}{r}\right)
\end{equation}

This corresponds to an effective mass that decreases with increasing distance from the galactic center.

\subsection{Conditions for Flat Rotation Curves}

To produce a flat rotation curve, $M_{eff}(r) \propto r$ is required. We first calculate the effective mass density $\rho_{eff}(r)$:

\begin{equation}
	\rho_{eff}(r) = \rho_{baryon}(r) \cdot \frac{m_{eff}(r)}{m_0} = \rho_{baryon}(r) \cdot \left(1 - \alpha\frac{g\rho_0 r_0^2}{r}\right)
\end{equation}

For large $r$, where $\rho_{baryon}(r) \approx 0$, this effective density would not suffice to produce a flat rotation curve. We must therefore introduce an additional term generating an effective density $\propto 1/r^2$. This can be achieved through an extended coupling or self-interaction of the dark energy field.

\subsection{Extended Model with Effective Gravitational Constant}

An alternative approach is to introduce an effective gravitational constant dependent on the dark energy field:

\begin{equation}
	G_{eff}(r) = G\left(1 + \beta\phi_{DE}(r)\right) = G\left(1 - \beta\frac{g\rho_0 r_0^2}{r}\right)
\end{equation}

where $\beta$ is a new coupling constant. The rotational velocity becomes:

\begin{equation}
	v^2(r) = \frac{G_{eff}(r)M_{baryon}(r)}{r}
\end{equation}

For large $r$, where $M_{baryon}(r) \approx M_{baryon,total}$ (the total mass of baryonic matter in the galaxy), we obtain:

\begin{equation}
	v^2(r) \approx \frac{GM_{baryon,total}}{r} - \beta g\rho_0 r_0^2 \frac{GM_{baryon,total}}{r^2}
\end{equation}

To produce a flat rotation curve, we need a dominant term independent of $r$. This can be achieved by introducing a modified density profile for the dark energy field with a $1/r^2$ dependence.

\subsection{Dark Energy with $1/r^2$ Profile}

We now consider a dark energy field with a density proportional to $1/r^2$ for large $r$:

\begin{equation}
	\rho_{DE}(r) = \frac{\kappa}{r^2}
\end{equation}

where $\kappa$ is a constant. This density can be generated by an appropriate self-interaction of the field. The dark energy field then modifies the effective gravitational constant:

\begin{equation}
	G_{eff}(r) = G\left(1 + \frac{\kappa}{G\rho_0 r^2}\right)
\end{equation}

The rotational velocity becomes:

\begin{equation}
	v^2(r) = \frac{G_{eff}(r)M_{baryon}(r)}{r} \approx \frac{GM_{baryon}}{r} + \frac{\kappa}{\rho_0 r}
\end{equation}

For large $r$, the second term dominates, yielding:

\begin{equation}
	v^2(r) \approx \frac{\kappa}{\rho_0} = \text{const.}
\end{equation}

This exactly matches the observed behavior of flat rotation curves. The parameter $\kappa$ can be determined from observed rotational velocities.

\section{Quantitative Determination of Coupling Parameters}

We can now concretely calculate the coupling constants from observational data.

\subsection{Determination of $\kappa$ from Rotation Curves}

For a typical spiral galaxy like the Milky Way, the rotational velocity in the outer region is approximately $v \approx 220$ km/s. This yields:

\begin{equation}
	\kappa = v^2 \rho_0 \approx (220 \text{ km/s})^2 \cdot \rho_0
\end{equation}

For a typical baryonic reference density $\rho_0 \approx 0.1$ GeV/cm$^3$, we obtain:

\begin{equation}
	\kappa \approx 4.8 \times 10^{-7} \text{ GeV/cm} \cdot \text{s}^{-2}
\end{equation}

This is the value the dark energy density constant must have to explain the observed flat rotation curves.

\subsection{Relation to Cosmological Redshift}

The parameter $\kappa$ is directly related to the absorption coefficient $\alpha = \frac{H_0}{c} \approx 2.3 \times 10^{-28}$ m$^{-1}$, which explains cosmic redshift. This connection is established through the relation:

\begin{equation}
	\kappa = \frac{\beta^2 H_0^2 M_{Pl}^2}{c^2 \rho_0}
\end{equation}

where $\beta \approx 10^{-3}$ is the dimensionless coupling constant. This fundamental relationship shows how mass variation in galaxies is linked to cosmological expansion.

\subsection{Relation to Yukawa Coupling}

The coupling between the dark energy field and matter can be interpreted as a generalized Yukawa interaction:

\begin{equation}
	g = \sqrt{\frac{\kappa}{M_{baryon} r_0^2}}
\end{equation}

For typical galaxy parameters ($M_{baryon} \approx 10^{11} M_{\odot}$, $r_0 \approx 5$ kpc), we obtain:

\begin{equation}
	g \approx 10^{-26} \text{ eV}^{-1}
\end{equation}

This extremely small value explains why this interaction is undetectable in local laboratory experiments while having significant effects on galactic scales.

\section{Field-Theoretic Formulation of Dark Energy}

To consistently describe the behavior of the dark energy field $\phi_{DE}$, we require an appropriate field theory.

\subsection{Lagrangian Density of Dark Energy}

We start with a general Lagrangian density for the dark energy field:

\begin{equation}
	\mathcal{L}_{DE} = -\frac{1}{2}\partial_\mu \phi_{DE} \partial^\mu \phi_{DE} - V(\phi_{DE}) - \frac{\beta}{M_{Pl}} \phi_{DE} T^{\mu}_{\mu}
\end{equation}

Here, $T^{\mu}_{\mu}$ is the trace of the energy-momentum tensor of baryonic matter, which for non-relativistic matter is approximately $T^{\mu}_{\mu} \approx -\rho_{baryon}$. The term $\frac{\beta}{M_{Pl}}$ represents a dimensionless coupling constant normalized to the Planck mass.

\subsection{Self-Interaction and Potential Term}

To generate the desired $1/r^2$ density profile of dark energy, we need a suitable potential $V(\phi_{DE})$. One approach is:

\begin{equation}
	V(\phi_{DE}) = \frac{1}{2}m_{\phi}^2\phi_{DE}^2 + \lambda \phi_{DE}^4
\end{equation}

where $m_{\phi}$ is the mass of the dark energy field and $\lambda$ is its self-coupling constant. For a nearly massless field ($m_{\phi} \approx 0$) and an appropriate self-coupling $\lambda$, a radially symmetric equilibrium profile can emerge that exhibits the desired $1/r^2$ dependence.

\subsection{Field Equation and Stationary Solutions}

The field equation for $\phi_{DE}$ is:

\begin{equation}
	\nabla^2 \phi_{DE} = \frac{dV}{d\phi_{DE}} + \frac{\beta}{M_{Pl}}\rho_{baryon}
\end{equation}

For a static, spherically symmetric system:

\begin{equation}
	\frac{1}{r^2}\frac{d}{dr}\left(r^2\frac{d\phi_{DE}}{dr}\right) = m_{\phi}^2\phi_{DE} + 4\lambda\phi_{DE}^3 + \frac{\beta}{M_{Pl}}\rho_{baryon}(r)
\end{equation}

This equation has a solution for $m_{\phi} \approx 0$ (massless or very light field) and exponentially declining baryonic density $\rho_{baryon}(r) \approx \rho_0 e^{-r/r_0}$ that behaves like $\phi_{DE}(r) \propto 1/r$ for large $r$.

When this solution is substituted into the term for the effective gravitational constant:

\begin{equation}
	G_{eff}(r) = G\left(1 + \beta\frac{\phi_{DE}(r)}{M_{Pl}}\right)
\end{equation}

we obtain the desired behavior $G_{eff}(r) \approx G(1 + \kappa/r^2)$ for large $r$, leading to flat rotation curves.

\section{Comparison with the Standard Dark Matter Model}

We now analyze how the T0 model with effective mass variation differs from the standard model with dark matter.

\subsection{Mathematical Equivalence and Physical Differences}

At first glance, both models appear mathematically equivalent, as they reproduce the same flat rotation curves. However, the fundamental difference lies in the physical interpretation:

\begin{tcolorbox}[colback=green!5!white,colframe=green!75!black,title=Comparison of Models]
	\textbf{$\Lambda$CDM Model:}
	\begin{itemize}
		\item Dark matter as a separate particle species
		\item NFW density profile: $\rho_{DM}(r) = \frac{\rho_0}{\frac{r}{r_s}(1 + \frac{r}{r_s})^2}$
		\item Time is relative (time dilation), rest mass constant
		\item Dark energy as the driver of cosmic expansion
	\end{itemize}
	
	\textbf{T0 Model:}
	\begin{itemize}
		\item No separate dark matter, but effective mass variation
		\item Effective density profile: $\rho_{eff}(r) \approx \rho_{baryon}(r) + \frac{\kappa}{r^2}$
		\item Time is absolute, mass varies with energy
		\item Dark energy as a medium for energy exchange
	\end{itemize}
\end{tcolorbox}

\subsection{Rotation Curves and Mass Density Profiles}

Both models produce flat rotation curves but with different mass density profiles. In the NFW profile of the $\Lambda$CDM model, the density decreases as $r^{-1}$ for $r \ll r_s$ and as $r^{-3}$ for $r \gg r_s$. In the T0 model, an effective density emerges that decreases as $r^{-2}$ for large $r$.

An important consequence is that the T0 model avoids the "cusp-core problem" present in the $\Lambda$CDM model, where the central density spike (cusp) of NFW profiles often does not match observations of low-surface-brightness galaxies.

\subsection{Galaxy Clusters and Gravitational Lensing}

The gravitational lensing effect provides another way to distinguish between the models. In the T0 model, the effective mass scales with the density distribution of baryonic matter and the dark energy field, whereas in the $\Lambda$CDM model, dark matter is an independent component with its own dynamics.

\section{Quantitative Predictions and Experimental Tests}

To differentiate between the T0 model with mass variation and the standard model with dark matter, precise quantitative predictions and experimental tests are necessary.

\subsection{Tully-Fisher Relation}

The Tully-Fisher relation links the luminosity $L$ of a spiral galaxy to its rotational velocity $v_{max}$ and is empirically described by:

\begin{equation}
	L \propto v_{max}^{4}
\end{equation}

In the standard model, this relation is a consequence of galaxy dynamics with dark matter. In the T0 model, mass variation would modify this relation to:

\begin{equation}
	L \propto v_{max}^{4+\epsilon}
\end{equation}

where $\epsilon$ is a small correction term dependent on the coupling constant $\beta$:

\begin{equation}
	\epsilon \approx \frac{\beta^2 \rho_0 r_0^2}{m_0 G}
\end{equation}

A precise measurement of this deviation could provide a direct test of the T0 model.

\subsection{Mass-Dependent Gravitational Lensing Effects}

A key difference between the models concerns the gravitational lensing effect. In the T0 model, the effective mass of an object is mass-dependent, leading to a modified lensing equation:

\begin{equation}
	\alpha_{lens} \propto \int \nabla(\Phi_{Newton} + \beta\phi_{DE}) dz
\end{equation}

For extended objects like galaxy clusters, this results in a lensing profile that differs from the $\Lambda$CDM model’s prediction, particularly at outer radii. A detailed analysis of gravitational lenses could reveal these differences.

\subsection{Gas-Rich vs. Gas-Poor Galaxies}

A specific prediction of the T0 model concerns galaxies with different gas-to-star ratios. Since effective mass variation is tied to baryonic density, gas-rich galaxies should systematically exhibit different rotation curves than gas-poor galaxies of the same total mass.

\begin{equation}
	\frac{v^2_{gas-rich}(r)}{v^2_{gas-poor}(r)} = 1 + \delta(r)
\end{equation}

where $\delta(r)$ is a function dependent on the radial distribution of gas and stars. This could be empirically tested by analyzing galaxies with similar stellar mass but different HI gas masses.

\section{Summary and Conclusions}

In this paper, we have developed a comprehensive mathematical analysis of galaxy dynamics within the framework of the T0 model, based on the assumptions of absolute time and variable mass. Unlike the standard cosmological model ($\Lambda$CDM), which postulates the existence of dark matter as a separate component, the T0 model explains the observed dynamic effects through an effective mass variation arising from coupling with a dark energy field.

\subsection{Key Results}

The main findings of our analysis are:

\begin{enumerate}
	\item A mathematically consistent field theory for the dark energy field that couples to baryonic matter and induces an effective mass variation.
	
	\item A quantitative derivation of the parameters required to explain flat rotation curves in galaxies, particularly the coupling constant $\kappa \approx 4.8 \times 10^{-7}$ GeV/cm$\cdot$s$^{-2}$.
	
	\item The fundamental relationship between mass variation in galaxies and cosmological redshift through the absorption coefficient $\alpha = \frac{H_0}{c}$.
	
	\item Specific proposals for experimental tests that could distinguish between the T0 model and the standard model.
\end{enumerate}

\subsection{Outlook}

The T0 model offers a conceptually elegant alternative to the standard cosmological model by reinterpreting fundamental assumptions about time and mass. The critical question is whether the model can be confirmed through rigorous experimental tests. The proposed tests, particularly the analysis of galaxies with different gas-to-star ratios and the detailed measurement of gravitational lensing profiles, offer promising avenues to differentiate between the models.

\appendix
\section{Mathematical Appendix}

\subsection{Derivation of the Field Equation for the Dark Energy Field}

Starting from the Lagrangian density:

\begin{equation}
	\mathcal{L}_{DE} = -\frac{1}{2}\partial_\mu \phi_{DE} \partial^\mu \phi_{DE} - V(\phi_{DE}) - \frac{\beta}{M_{Pl}}\phi_{DE}T^{\mu}_{\mu} - \frac{1}{2}\xi \phi_{DE}^2 R
\end{equation}

The Euler-Lagrange equation is:

\begin{equation}
	\frac{\partial \mathcal{L}}{\partial \phi_{DE}} - \partial_\mu \left(\frac{\partial \mathcal{L}}{\partial (\partial_\mu \phi_{DE})}\right) = 0
\end{equation}

Substituting yields:

\begin{equation}
	-\frac{dV}{d\phi_{DE}} - \frac{\beta}{M_{Pl}}T^{\mu}_{\mu} - \xi \phi_{DE} R - \partial_\mu\left(-\partial^\mu \phi_{DE}\right) = 0
\end{equation}

Which simplifies to:

\begin{equation}
	\Box\phi_{DE} - \xi R \phi_{DE} - \frac{dV}{d\phi_{DE}} = \frac{\beta}{M_{Pl}}T^{\mu}_{\mu}
\end{equation}

\subsection{Detailed Derivation of Rotation Curves}

The rotational velocity $v(r)$ of an object in a circular orbit is determined by the balance between gravitational force and centrifugal force:

\begin{equation}
	\frac{v^2}{r} = \frac{GM(r)}{r^2}
\end{equation}

In the T0 model, the effective gravitational constant is modified:

\begin{equation}
	G_{eff}(r) = G\left(1 + \beta\frac{\phi_{DE}(r)}{M_{Pl}}\right)
\end{equation}

With the solution for the dark energy field $\phi_{DE}(r) \approx -\frac{\beta\rho_0 r_0^2}{M_{Pl}r}$ for large $r$ and the relation $\kappa = \frac{\beta^2 H_0^2 M_{Pl}^2}{c^2 \rho_0}$, we obtain:

\begin{equation}
	v^2 \approx \frac{GM_{baryon}(r)}{r} + \frac{\kappa}{\rho_0}
\end{equation}

which leads to a constant rotational velocity for large $r$.

\section{Bibliography}

\begin{thebibliography}{99}
	
	\bibitem{pascher} Pascher, J. (2025). A Model with Absolute Time and Variable Energy: A Detailed Investigation of the Foundations.
	
	\bibitem{pascher2} Pascher, J. (2025). Extensions of Quantum Mechanics through Intrinsic Time.
	
	\bibitem{pascher3} Pascher, J. (2025). Complementary Extensions of Physics: Absolute Time and Intrinsic Time.
	
	\bibitem{rotation} Rubin, V. C., Ford, W. K. (1970). Rotation of the Andromeda Nebula from a Spectroscopic Survey of Emission Regions. The Astrophysical Journal, 159, 379.
	
	\bibitem{nfw} Navarro, J. F., Frenk, C. S., White, S. D. M. (1996). The Structure of Cold Dark Matter Halos. The Astrophysical Journal, 462, 563.
	
	\bibitem{tully} Tully, R. B., Fisher, J. R. (1977). A new method of determining distances to galaxies. Astronomy and Astrophysics, 54, 661.
	
	\bibitem{bullet} Clowe, D., Bradač, M., Gonzalez, A. H., et al. (2006). A Direct Empirical Proof of the Existence of Dark Matter. The Astrophysical Journal, 648, L109.
	
	\bibitem{mond} Milgrom, M. (1983). A modification of the Newtonian dynamics as a possible alternative to the hidden mass hypothesis. The Astrophysical Journal, 270, 365.
	
	\bibitem{scalar} Fujii, Y., Maeda, K. (2003). The Scalar-Tensor Theory of Gravitation. Cambridge University Press.
	
	\bibitem{lsb} McGaugh, S. S., de Blok, W. J. G. (1998). Testing the Dark Matter Hypothesis with Low Surface Brightness Galaxies and Other Evidence. The Astrophysical Journal, 499, 41.
	
	\bibitem{quintessence} Caldwell, R. R., Dave, R., Steinhardt, P. J. (1998). Cosmological Imprint of an Energy Component with General Equation of State. Physical Review Letters, 80, 1582.
	
	\bibitem{euclid} Laureijs, R., et al. (2011). Euclid Definition Study Report. ESA/SRE(2011)12.
	
	\bibitem{jwst} Gardner, J. P., et al. (2006). The James Webb Space Telescope. Space Science Reviews, 123, 485.
	
	\bibitem{ska} Dewdney, P. E., Hall, P. J., Schilizzi, R. T., Lazio, T. J. L. W. (2009). The Square Kilometre Array. Proceedings of the IEEE, 97, 1482.
	
\end{thebibliography}	
	\end{document}
