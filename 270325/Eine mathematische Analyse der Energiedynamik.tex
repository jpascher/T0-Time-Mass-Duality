
\documentclass[a4paper,12pt]{article}
\usepackage[utf8]{inputenc}
\usepackage[ngerman]{babel}
\usepackage{amsmath, amssymb, amsthm}
\usepackage{physics}
\usepackage{graphicx}
\usepackage{hyperref}
\usepackage{tikz}
\usepackage{setspace}
\usepackage{tcolorbox}
\usepackage{xcolor}
\usepackage{textalpha}  % For Unicode Greek
\usepackage{textgreek}  % For \textdelta, \textalpha etc.
%\usepackage{unicode-math}  % Muss VOR fontspec stehen falls verwendet
%\usepackage{fontspec}      % Für XeLaTeX/LuaLaTeX
\newtheorem{theorem}{Theorem}
\newtheorem{lemma}[theorem]{Lemma}
\newtheorem{proposition}[theorem]{Proposition}
\newtheorem{corollary}[theorem]{Korollar}
\newtheorem{definition}{Definition}

\begin{document}


\title{Dunkle Energie im T0-Modell: \\Eine mathematische Analyse der Energiedynamik}
\author{Johann Pascher}
\date{26.03.25}
\maketitle

\begin{abstract}
	Diese Arbeit entwickelt eine detaillierte mathematische Analyse der dunklen Energie im Rahmen des T0-Modells mit absoluter Zeit und variabler Masse. Im Gegensatzf zum $\Lambda$CDM-Standardmodell wird dunkle Energie nicht als treibende Kraft einer kosmischen Expansion betrachtet, sondern als dynamisches Medium für Energieaustausch in einem statischen Universum. Das Dokument leitet die entsprechenden Feldgleichungen her, charakterisiert die Energieübertragungsraten, analysiert das radiale Dichteprofil der dunklen Energie und erklärt die beobachtete Rotverschiebung als Folge von Energieverlust durch Photonen. Abschließend werden konkrete experimentelle Tests vorgeschlagen, die zwischen dieser Interpretation und dem Standardmodell unterscheiden könnten.
\end{abstract}

\tableofcontents
\newpage

\section{Einleitung}

Die Entdeckung der beschleunigten kosmischen Expansion durch Supernovae-Beobachtungen in den späten 1990er Jahren führte zur Einführung der dunklen Energie als dominante Komponente des Universums. Im Standardmodell der Kosmologie ($\Lambda$CDM) wird die dunkle Energie als kosmologische Konstante ($\Lambda$) mit negativem Druck modelliert, die etwa 68\% des Energiegehalts des Universums ausmacht und die beschleunigte Expansion antreibt.

Diese Arbeit verfolgt einen alternativen Ansatz auf der Grundlage des T0-Modells, in dem die Zeit absolut ist und stattdessen die Masse der Teilchen variiert. In diesem Rahmen wird die dunkle Energie nicht als treibende Kraft einer Expansion betrachtet, sondern als Medium für Energieaustausch, das mit Materie und Strahlung wechselwirkt. Die kosmische Rotverschiebung wird nicht durch Raumexpansion, sondern durch Energieverlust von Photonen an die dunkle Energie erklärt.

Wir werden im Folgenden diesen Ansatz mathematisch präzisieren, die notwendigen Feldgleichungen herleiten, die Energiedichte und -verteilung der dunklen Energie bestimmen und die Konsequenzen für astronomische Beobachtungen analysieren. Anschließend werden wir untersuchen, welche experimentellen Tests zwischen dem T0-Modell und dem Standardmodell unterscheiden könnten.

\section{Grundlagen des T0-Modells für Dunkle Energie}

Wir fassen zunächst die grundlegenden Konzepte des T0-Modells in Bezug auf dunkle Energie zusammen.

\subsection{Fundamentale Annahmen}

Im Gegensatz zum Standardmodell, in dem die Raumzeit dynamisch ist und sich ausdehnt, während die Ruhemasse konstant bleibt, postuliert das T0-Modell:

\begin{tcolorbox}[colback=blue!5!white,colframe=blue!75!black,title=Grundannahmen des T0-Modells]
	\begin{align}
		&\text{1. Die Zeit $T_0$ ist absolut und universell konstant.} \\
		&\text{2. Die Masse variiert entsprechend $m = \gamma m_0$, wobei $\gamma = \frac{1}{\sqrt{1-v^2/c_0^2}}$.} \\
		&\text{3. Die Gesamtenergie des Universums ist konstant.} \\
		&\text{4. Die Rotverschiebung entsteht durch Energieverlust: $E_2 = E_1(1+z)^{-1}$.}
	\end{align}
\end{tcolorbox}

Für die dunkle Energie bedeutet dies, dass sie nicht als homogene Hintergrunddichte betrachtet wird, die den Raum ausdehnt, sondern als dynamisches Feld, das Energie mit Materie und Strahlung austauschen kann, während die Gesamtenergie des Universums konstant bleibt.

\subsection{Dunkle Energie als Dynamisches Feld}

Die dunkle Energie wird im T0-Modell als Skalarfeld $\phi_{DE}$ modelliert, das mit der Materie und Strahlung wechselwirkt. Die Energiedichte dieses Feldes ist nicht konstant, sondern weist eine räumliche Struktur auf:

\begin{equation}
	\rho_{DE}(r) = \frac{\kappa}{r^2}
\end{equation}

wobei $\kappa$ eine Konstante ist und $r$ den radialen Abstand bezeichnet. Dieses $1/r^2$-Profil unterscheidet sich deutlich von der konstanten Energiedichte $\rho_\Lambda$ der kosmologischen Konstante im Standardmodell.

Die Kopplung zwischen dunkler Energie und Materie/Strahlung kann durch einen Wechselwirkungsterm in der Lagrange-Dichte beschrieben werden:

\begin{equation}
	\mathcal{L}_{int} = -\frac{\beta}{M_{Pl}} \phi_{DE} T^{\mu}_{\mu}
\end{equation}

Hier ist $\beta$ eine dimensionslose Kopplungskonstante, $M_{Pl}$ die Planck-Masse und $T^{\mu}_{\mu}$ die Spur des Energie-Impuls-Tensors der Materie und Strahlung.

\section{Feldtheoretische Beschreibung der Dunklen Energie}

Wir entwickeln nun eine vollständige feldtheoretische Beschreibung der dunklen Energie im T0-Modell.

\subsection{Lagrange-Dichte der Dunklen Energie}

Die vollständige Lagrange-Dichte für das dunkle Energiefeld lautet:

\begin{equation}
	\mathcal{L}_{DE} = -\frac{1}{2}\partial_\mu \phi_{DE} \partial^\mu \phi_{DE} - V(\phi_{DE}) - \frac{\beta}{M_{Pl}} \phi_{DE} T^{\mu}_{\mu} - \frac{1}{2}\xi \phi_{DE}^2 R
\end{equation}

wobei:
\begin{itemize}
	\item $\partial_\mu \phi_{DE} \partial^\mu \phi_{DE}$ der kinetische Term ist,
	\item $V(\phi_{DE})$ das Selbstwechselwirkungspotential des Feldes,
	\item $\frac{\beta}{M_{Pl}} \phi_{DE} T^{\mu}_{\mu}$ die Kopplung an Materie und Strahlung,
	\item $\frac{1}{2}\xi \phi_{DE}^2 R$ eine nicht-minimale Kopplung an die Raumzeitkrümmung $R$.
\end{itemize}

Für ein statisches Universum im T0-Modell müssen wir ein geeignetes Potential $V(\phi_{DE})$ wählen, das ein stabiles Gleichgewicht ermöglicht:

\begin{equation}
	V(\phi_{DE}) = \frac{1}{2}m_{\phi}^2\phi_{DE}^2 + \lambda \phi_{DE}^4
\end{equation}

wobei $m_{\phi}$ die Masse des dunklen Energiefeldes und $\lambda$ die Selbstkopplungskonstante ist.

\subsection{Feldgleichungen der Dunklen Energie}

Aus der Lagrange-Dichte ergeben sich die Euler-Lagrange-Gleichungen für das dunkle Energiefeld:

\begin{equation}
	\Box\phi_{DE} - \frac{dV}{d\phi_{DE}} - \frac{\beta}{M_{Pl}}T^{\mu}_{\mu} - \xi \phi_{DE} R = 0
\end{equation}

was sich zu

\begin{equation}
	\Box\phi_{DE} - m_{\phi}^2\phi_{DE} - 4\lambda\phi_{DE}^3 - \frac{\beta}{M_{Pl}}T^{\mu}_{\mu} - \xi \phi_{DE} R = 0
\end{equation}

vereinfacht.

Für ein statisches, sphärisch symmetrisches System reduziert sich diese Gleichung auf:

\begin{equation}
	\frac{1}{r^2}\frac{d}{dr}\left(r^2\frac{d\phi_{DE}}{dr}\right) = m_{\phi}^2\phi_{DE} + 4\lambda\phi_{DE}^3 + \frac{\beta}{M_{Pl}}T^{\mu}_{\mu} + \xi \phi_{DE} R
\end{equation}

\subsection{Energiedichteprofil der Dunklen Energie}

Für ein masseloses Feld ($m_{\phi} \approx 0$) und vernachlässigbare Krümmung ($\xi R \approx 0$) vereinfacht sich die Feldgleichung zu:

\begin{equation}
	\frac{1}{r^2}\frac{d}{dr}\left(r^2\frac{d\phi_{DE}}{dr}\right) = 4\lambda\phi_{DE}^3 + \frac{\beta}{M_{Pl}}T^{\mu}_{\mu}
\end{equation}

Wir suchen eine Lösung, die das beobachtete $1/r^2$-Dichteprofil der dunklen Energie reproduziert. Für große Abstände $r$, wo $T^{\mu}_{\mu} \approx 0$ (vernachlässigbare Materiedichte), dominiert der Term mit der Selbstwechselwirkung $\lambda$. Wenn wir einen Ansatz $\phi_{DE}(r) \propto r^{-\alpha}$ wählen, erhalten wir durch Einsetzen und Koeffizientenvergleich $\alpha = 1/2$, also:

\begin{equation}
	\phi_{DE}(r) \approx \left(\frac{1}{8\lambda}\right)^{1/3} r^{-1/2} \quad \text{für } r \gg r_0
\end{equation}

Die Energiedichte der dunklen Energie ist dann:

\begin{equation}
	\rho_{DE}(r) \approx \frac{1}{2}\left(\frac{d\phi_{DE}}{dr}\right)^2 + \frac{1}{2}m_{\phi}^2\phi_{DE}^2 + \lambda\phi_{DE}^4 \approx \frac{\kappa}{r^2}
\end{equation}

mit $\kappa \propto \lambda^{-2/3}$. Dieses $1/r^2$-Profil ist genau das, was wir für die Erklärung flacher Rotationskurven in Galaxien benötigen.

\section{Energieaustausch und Rotverschiebung}

Ein zentraler Aspekt des T0-Modells ist die Interpretation der kosmischen Rotverschiebung als Folge eines Energieverlusts von Photonen an die dunkle Energie, nicht als Folge einer Raumexpansion.

\subsection{Energieverlust von Photonen}

Wir betrachten ein Photon, das sich durch das dunkle Energiefeld bewegt. Die Energieänderung des Photons wird durch die Gleichung:

\begin{equation}
	\frac{dE_{\gamma}}{dx} = -\alpha E_{\gamma}
\end{equation}

beschrieben, wobei $\alpha$ die Absorptionsrate ist. Diese Gleichung hat die Lösung:

\begin{equation}
	E_{\gamma}(x) = E_{\gamma,0} e^{-\alpha x}
\end{equation}

wobei $E_{\gamma,0}$ die ursprüngliche Energie des Photons ist und $x$ die zurückgelegte Strecke.

Die Rotverschiebung $z$ ist definiert als:

\begin{equation}
	1 + z = \frac{E_0}{E} = \frac{\lambda_{obs}}{\lambda_{emit}} = e^{\alpha d}
\end{equation}

wobei $d$ die Entfernung ist. Für kleine $z$ (lokale Entfernungen) gilt:

\begin{equation}
	z \approx \alpha d
\end{equation}

Um Konsistenz mit der beobachteten Hubble-Relation $z \approx H_0 d/c$ zu gewährleisten, muss gelten:

\begin{equation}
	\alpha = \frac{H_0}{c} \approx 2.3 \times 10^{-28} \text{ m}^{-1}
\end{equation}

wobei $H_0 \approx 70 \text{ km/s/Mpc}$ die Hubble-Konstante ist.

Die Energieübertragung an das dunkle Energiefeld kann durch das Konzept der intrinsischen Zeit $T = \frac{\hbar}{mc^2}$ weiter präzisiert werden. Diese intrinsische Zeit ist umgekehrt proportional zur Masse des Teilchens und steht im Zentrum der Zeit-Masse-Dualität des T0-Modells. Für Photonen kann die intrinsische Zeit definiert werden als $T = \frac{\hbar}{E_{\gamma}} e^{\alpha x}$, wobei $\alpha = \frac{H_0}{c}$ die Absorptionsrate ist.

\subsection{Energieübertragung an die Dunkle Energie}

Die vom Photon verlorene Energie wird an das dunkle Energiefeld übertragen. Die Energieerhaltung erfordert:

\begin{equation}
	\frac{d}{dt}(E_{\gamma} + E_{DE}) = 0
\end{equation}

Die Rate, mit der die dunkle Energie Energie aufnimmt, ist:

\begin{equation}
	\frac{dE_{DE}}{dt} = -\frac{dE_{\gamma}}{dt} = \alpha c E_{\gamma}
\end{equation}

Für die Energiedichte der dunklen Energie bedeutet dies:

\begin{equation}
	\frac{d\rho_{DE}}{dt} = \alpha c \rho_{\gamma}
\end{equation}

wobei $\rho_{\gamma}$ die Energiedichte der Photonen ist.

\subsection{Bilanzgleichung für die Gesamtenergie}

In einem statischen Universum mit konstanter Gesamtenergie müssen wir die Energiebilanz betrachten. Die Energiedichte $\rho$ setzt sich zusammen aus:

\begin{equation}
	\rho_{total} = \rho_{matter} + \rho_{\gamma} + \rho_{DE} = const.
\end{equation}

Die Bilanzgleichung für die zeitliche Entwicklung der Energiedichten lautet:

\begin{align}
	\frac{d\rho_{matter}}{dt} &= -\alpha_{m} c \rho_{matter} \\
	\frac{d\rho_{\gamma}}{dt} &= -\alpha_{\gamma} c \rho_{\gamma} \\
	\frac{d\rho_{DE}}{dt} &= \alpha_{m} c \rho_{matter} + \alpha_{\gamma} c \rho_{\gamma}
\end{align}

wobei $\alpha_{m}$ und $\alpha_{\gamma}$ die Energieübertragungsraten für Materie bzw. Photonen sind.

Unter der Annahme, dass $\alpha_{\gamma} = \alpha_{m} = \alpha$ (gleiche Übertragungsrate für alle Energieformen), erhalten wir für die zeitliche Entwicklung der Energiedichten:

\begin{align}
	\rho_{matter}(t) &= \rho_{matter,0} e^{-\alpha c t} \\
	\rho_{\gamma}(t) &= \rho_{\gamma,0} e^{-\alpha c t} \\
	\rho_{DE}(t) &= \rho_{DE,0} + (\rho_{matter,0} + \rho_{\gamma,0})(1 - e^{-\alpha c t})
\end{align}

Für große Zeiten ($t \gg (\alpha c)^{-1}$) strebt das Universum einem Zustand entgegen, in dem die gesamte Energie in Form von dunkler Energie vorliegt:

\begin{equation}
	\lim_{t \rightarrow \infty} \rho_{DE}(t) = \rho_{total} = \rho_{DE,0} + \rho_{matter,0} + \rho_{\gamma,0}
\end{equation}

\section{Quantitative Bestimmung der Parameter}

Basierend auf astronomischen Beobachtungen können wir die Parameter des T0-Modells quantitativ abschätzen.

\subsection{Gesamtenergiedichte des Universums}

Die kritische Dichte des Universums beträgt:

\begin{equation}
	\rho_{crit} = \frac{3H_0^2}{8\pi G} \approx 8.5 \times 10^{-27} \text{ kg/m}^3
\end{equation}

Im Standardmodell macht die dunkle Energie etwa 68\% der kritischen Dichte aus:

\begin{equation}
	\rho_{\Lambda} \approx 0.68 \rho_{crit} \approx 5.8 \times 10^{-27} \text{ kg/m}^3
\end{equation}

Im T0-Modell entspricht diese Dichte nicht einer homogenen Hintergrunddichte, sondern dem Mittelwert eines inhomogenen Feldes mit $1/r^2$-Abhängigkeit.

\subsection{Absorptionskoeffizient und Hubble-Konstante}

Aus der Beziehung $\alpha = H_0/c$ und dem beobachteten Wert $H_0 \approx 70 \text{ km/s/Mpc}$ erhalten wir:

\begin{equation}
	\alpha \approx 2.3 \times 10^{-28} \text{ m}^{-1}
\end{equation}

Diese extrem kleine Absorptionsrate erklärt, warum der Energieverlust von Photonen an die dunkle Energie in Laborexperimenten nicht messbar ist, aber über kosmologische Distanzen signifikant wird.

\subsection{Kopplungskonstante zur Materie}

Die dimensionslose Kopplungskonstante $\beta$, die die Wechselwirkung zwischen dunkler Energie und Materie beschreibt, kann aus der Analyse von Galaxienrotationskurven abgeschätzt werden:

\begin{equation}
	\beta \approx 10^{-3}
\end{equation}

Dieser Wert ist klein genug, um lokale Tests der Gravitation zu bestehen, aber groß genug, um kosmologische Effekte zu erklären.

\subsection{Selbstwechselwirkung des dunklen Energiefeldes}

Die Selbstwechselwirkungskonstante $\lambda$ in $V(\phi_{DE}) = \lambda \phi_{DE}^4$ bestimmt das Dichteprofil der dunklen Energie. Aus der Beziehung $\kappa \propto \lambda^{-2/3}$ und dem beobachteten Wert $\kappa \approx 4.8 \times 10^{-7} \text{ GeV/cm} \cdot \text{s}^{-2}$ (aus Galaxienrotationskurven) können wir $\lambda$ abschätzen:

\begin{equation}
	\lambda \approx 10^{-120}
\end{equation}

Diese extrem kleine Selbstwechselwirkung ist eine Herausforderung für das Modell, ähnlich wie das Hierarchieproblem im Standardmodell.

\section{Dunkle Energie und Kosmologische Beobachtungen}

Wir analysieren nun, wie das T0-Modell verschiedene kosmologische Beobachtungen erklärt, die im Standardmodell der dunklen Energie zugeschrieben werden.

\subsection{Supernovae Typ Ia und kosmische Beschleunigung}

Die Beobachtung, dass Supernovae vom Typ Ia in großen Entfernungen lichtschwächer erscheinen als in einem reinen Materieuniversum erwartet, führte zur Entdeckung der "kosmischen Beschleunigung". Im $\Lambda$CDM-Modell wird dies durch die beschleunigte Expansion des Universums erklärt, angetrieben durch dunkle Energie mit negativem Druck.

Im T0-Modell ergibt sich eine alternative Erklärung: Photonen verlieren auf ihrem Weg durch das Universum Energie an das dunkle Energiefeld, wodurch ihre Wellenlänge zunimmt (Rotverschiebung) und ihre Intensität abnimmt. Die Helligkeit-Rotverschiebungs-Beziehung wird dann durch:

\begin{equation}
	m - M = 5 \log_{10}(d_L) + 25
\end{equation}

mit der Leuchtkraftdistanz:

\begin{equation}
	d_L = \frac{c}{H_0} \ln(1+z) (1+z)
\end{equation}

beschrieben, im Gegensatz zur Standardformel:

\begin{equation}
	d_L^{\Lambda CDM} = \frac{c}{H_0} \int_0^z \frac{dz'}{\sqrt{\Omega_m(1+z')^3 + \Omega_\Lambda}}
\end{equation}

Beide Formeln können die beobachteten Daten gleich gut fitten, jedoch mit unterschiedlichen physikalischen Interpretationen.

\subsection{Kosmischer Mikrowellenhintergrund (CMB)}

Der CMB zeigt eine nahezu perfekte Schwarzkörperstrahlung mit einer Temperatur von T = 2.725 K und winzigen Temperaturschwankungen ($\delta T/T \sim 10^{-5}$). Im $\Lambda$CDM-Modell wird dies als Relikt des frühen, heißen Universums interpretiert, das durch die kosmische Expansion abgekühlt wurde.

Im T0-Modell wird der CMB als statisches thermisches Feld betrachtet, dessen Temperatur durch das Gleichgewicht zwischen Energiezufuhr (z.B. von Sternen und Galaxien) und Energieabgabe an die dunkle Energie bestimmt wird. Die beobachteten Anisotropien entstehen durch lokale Schwankungen in der Energiedichte des dunklen Energiefeldes.

Die Leistungsspektrum des CMB, insbesondere die charakteristischen akustischen Peaks, muss in diesem Rahmen neu interpretiert werden. Während im $\Lambda$CDM-Modell diese Peaks durch Baryon-Akustische Oszillationen vor der Rekombination erklärt werden, müssen sie im T0-Modell als Folge von Dichtefluktuationen im statischen dunklen Energiefeld verstanden werden.

\subsection{Großräumige Struktur und Baryon-Akustische Oszillationen (BAO)}

Die Verteilung von Galaxien zeigt eine charakteristische Längenskala von etwa 150 Mpc, die im $\Lambda$CDM-Modell als Folge von Baryon-Akustischen Oszillationen vor der Rekombination interpretiert wird. Diese Längenskala dient als Standardlineal zur Messung der kosmischen Expansion.

Im T0-Modell muss diese Längenskala anders erklärt werden, ohne auf die Expansion zurückzugreifen. Eine mögliche Erklärung ist, dass die Massenvariation und der Energieaustausch mit dem dunklen Energiefeld charakteristische Längenskalen in der Strukturbildung erzeugen.

Die mathematische Beschreibung dieser Prozesse erfordert eine detaillierte Analyse der Störungsgleichungen im T0-Modell:

\begin{equation}
	\nabla^2 \delta\phi_{DE} - m_{\phi}^2 \delta\phi_{DE} - 12\lambda\phi_{DE}^2 \delta\phi_{DE} = \frac{\beta}{M_{Pl}}\delta T^{\mu}_{\mu}
\end{equation}

wobei $\delta\phi_{DE}$ die Fluktuation des dunklen Energiefeldes und $\delta T^{\mu}_{\mu}$ die Fluktuation in der Materieverteilung ist.

\section{Experimentelle Tests und Vorhersagen}

Das T0-Modell der dunklen Energie macht spezifische Vorhersagen, die es von der kosmologischen Konstante des Standardmodells unterscheiden könnten.

\subsection{Zeitliche Variation der Feinstrukturkonstante}

Da im T0-Modell Photonen Energie an das dunkle Energiefeld abgeben, könnte dies zu einer zeitlichen Variation fundamentaler Konstanten führen, insbesondere der Feinstrukturkonstante $\alpha_{fs}$. Die Änderungsrate wäre:

\begin{equation}
	\frac{d\alpha_{fs}}{dt} \approx \alpha_{fs} \cdot \alpha \cdot c \approx 10^{-18} \text{ Jahr}^{-1}
\end{equation}

Diese Variation ist extrem klein, könnte aber durch hochpräzise Spektroskopie entfernter Quasare gemessen werden. Solche Messungen könnten Hinweise darauf liefern, ob das T0-Modell mit den beobachteten kosmologischen Daten konsistent ist.

\subsection{Umgebungsabhängigkeit der Rotverschiebung}

Da die dunkle Energie im T0-Modell ein dynamisches Feld mit räumlichen Variationen ist, sollte die Absorptionsrate $\alpha$ von der lokalen Energiedichte abhängen:

\begin{equation}
	\alpha(r) = \alpha_0 \cdot \left(1 + \delta\frac{\rho_{baryon}(r)}{\rho_0}\right)
\end{equation}

wobei $\delta$ ein Parameter ist, der die Stärke der Kopplung beschreibt. Dies führt zu einer Vorhersage: Die Rotverschiebung sollte in dichten kosmischen Regionen (z.B. Galaxienhaufen) leicht anders sein als in kosmischen Voids. Konkret:

\begin{equation}
	\frac{z_{cluster}}{z_{void}} \approx 1 + \delta\frac{\rho_{cluster} - \rho_{void}}{\rho_0}
\end{equation}

Diese Abweichung könnte durch präzise Messungen der Rotverschiebung in verschiedenen kosmischen Umgebungen getestet werden.

\subsection{Anomale Lichtausbreitung in starken Gravitationsfeldern}

Da die dunkle Energie im T0-Modell an die Materie koppelt, sollte ihre Dichte in der Nähe massereicher Objekte höher sein. Dies führt zu einer modifizierten Lichtausbreitung, insbesondere in starken Gravitationsfeldern wie in der Nähe von Schwarzen Löchern oder in Galaxienhaufen.

Die effektive Brechungszahl des Raums wäre:

\begin{equation}
	n_{eff}(r) = 1 + \epsilon \frac{\phi_{DE}(r)}{M_{Pl}}
\end{equation}

wobei $\epsilon$ ein Parameter ist, der von der genauen Kopplung zwischen dem dunklen Energiefeld und dem elektromagnetischen Feld abhängt.

Diese anomale Lichtausbreitung könnte sich in subtilen Abweichungen von der durch die allgemeine Relativitätstheorie vorhergesagten Gravitationslinseneffekten manifestieren.

\subsection{Differenzielle Rotverschiebung}

Eine weitere Vorhersage des T0-Modells betrifft die Wellenlängenabhängigkeit der Rotverschiebung. Da die Absorption von Photonen durch das dunkle Energiefeld wellenlängenabhängig sein könnte, insbesondere wenn die Kopplung eine Frequenzabhängigkeit aufweist:

\begin{equation}
	\alpha(\lambda) = \alpha_0 \cdot \left(1 + \eta\frac{\lambda}{\lambda_0}\right)
\end{equation}

Dies würde zu einer differenziellen Rotverschiebung führen, bei der verschiedene Wellenlängen desselben Objekts leicht unterschiedliche Rotverschiebungen zeigen:

\begin{equation}
	\frac{z(\lambda_1)}{z(\lambda_2)} \approx 1 + \eta\frac{\lambda_1 - \lambda_2}{\lambda_0}
\end{equation}

Diese Vorhersage könnte durch hochauflösende Spektroskopie entfernter Quasare getestet werden.

\section{Statistische Analyse und Vergleich mit dem Standardmodell}

Um die Vorhersagen des T0-Modells mit dem Standardmodell zu vergleichen, führen wir eine statistische Analyse durch.

\subsection{Bayesischer Modellvergleich}

Wir verwenden bayesische Statistik, um die Evidenz für das T0-Modell im Vergleich zum $\Lambda$CDM-Modell zu quantifizieren. Die Bayes-Evidenz ist gegeben durch:

\begin{equation}
	E(M) = \int L(\theta|D,M) \pi(\theta|M) d\theta
\end{equation}

wobei $L(\theta|D,M)$ die Likelihood der Daten $D$ gegeben der Parameter $\theta$ im Modell $M$ ist, und $\pi(\theta|M)$ die Prior-Verteilung der Parameter.

Das Bayes-Verhältnis zwischen den Modellen ist:

\begin{equation}
	B_{T_0,\Lambda CDM} = \frac{E(T_0)}{E(\Lambda CDM)}
\end{equation}

Dieses Verhältnis quantifiziert, wie stark die Beobachtungsdaten das eine oder das andere Modell bevorzugen.

\subsection{Anpassung an Supernova-Daten}

Die Supernova-Daten können sowohl mit dem Standardmodell als auch mit dem T0-Modell gefittet werden. Im $\Lambda$CDM-Modell wird die Entfernungsmodul-Rotverschiebungs-Beziehung durch:

\begin{equation}
	\mu(z) = 5 \log_{10}\left[\frac{c}{H_0}(1+z)\int_0^z \frac{dz'}{\sqrt{\Omega_m(1+z')^3 + \Omega_{\Lambda}}}\right] + 25
\end{equation}

beschrieben, während im T0-Modell:

\begin{equation}
	\mu(z) = 5 \log_{10}\left[\frac{c}{H_0}(1+z)\ln(1+z)\right] + 25
\end{equation}

gilt. Beide Modelle haben freie Parameter (($\Omega_m$, $\Omega_{\Lambda}$, $H_0$) für $\Lambda$CDM und ($\alpha$, $H_0$) für T0), die an die Daten angepasst werden können.

\subsection{Analyse des CMB-Leistungsspektrums}

Das Leistungsspektrum des kosmischen Mikrowellenhintergrunds bietet einen kritischen Test für beide Modelle. Im $\Lambda$CDM-Modell wird das Spektrum durch akustische Oszillationen vor der Rekombination bestimmt, während im T0-Modell die Struktur durch Dichtefluktuationen im statischen dunklen Energiefeld erklärt werden muss. Die mathematische Beschreibung des CMB-Leistungsspektrums im T0-Modell erfordert eine detaillierte Behandlung der Dichtefluktuationen im dunklen Energiefeld:

\begin{equation}
	P(k) = \langle|\delta\phi_{DE}(k)|^2\rangle
\end{equation}

wobei $\delta\phi_{DE}(k)$ die Fourier-Transformierte der Fluktuationen des dunklen Energiefeldes ist.

Diese theoretische Vorhersage kann dann mit den beobachteten Daten, insbesondere den Messungen des Planck-Satelliten, verglichen werden.

\section{Implikationen für die Zukunft des Universums}

Die beiden Modelle unterscheiden sich dramatisch in ihren Vorhersagen für die Zukunft des Universums.

\subsection{Zukünftige Entwicklung im $\Lambda$CDM-Modell}

Im Standardmodell führt die konstante Energiedichte der dunklen Energie zu einer beschleunigten Expansion, die immer schneller wird. Die Zukunft des Universums ist ein "Großer Riss" (Big Rip) oder ewige Expansion, je nach der genauen Zustandsgleichung der dunklen Energie.

Die Skalenfaktorentwicklung folgt:

\begin{equation}
	\frac{\ddot{a}}{a} = -\frac{4\pi G}{3}(\rho_m + 3p_\Lambda) = -\frac{4\pi G}{3}\rho_m + \frac{8\pi G}{3}\rho_\Lambda
\end{equation}

Da $\rho_m \propto a^{-3}$ mit der Zeit abnimmt, während $\rho_\Lambda = const.$, wird die Expansion langfristig immer stärker beschleunigt.

\subsection{Zukünftige Entwicklung im T0-Modell}

Im T0-Modell gibt es keine echte Expansion des Universums, sondern eine kontinuierliche Umwandlung von Materie- und Strahlungsenergie in dunkle Energie. Die Energiedichten entwickeln sich gemäß:

\begin{align}
	\rho_{matter}(t) &= \rho_{matter,0} e^{-\alpha c t} \\
	\rho_{\gamma}(t) &= \rho_{\gamma,0} e^{-\alpha c t} \\
	\rho_{DE}(t) &= \rho_{DE,0} + (\rho_{matter,0} + \rho_{\gamma,0})(1 - e^{-\alpha c t})
\end{align}

Langfristig nähert sich das Universum einem Zustand, in dem alle Energie in Form von dunkler Energie vorliegt - ein "thermischer Tod", aber ohne Expansion des Raumes.

\subsection{Vergleich der langfristigen Prognosen}

\begin{tcolorbox}[colback=yellow!5!white,colframe=yellow!75!black,title=Langfristige Entwicklung des Universums]
	\begin{tabular}{|p{0.45\textwidth}|p{0.45\textwidth}|}
		\hline
		\textbf{$\Lambda$CDM-Modell} & \textbf{T0-Modell} \\
		\hline
		Beschleunigte Expansion & Kein expandierender Raum \\
		\hline
		Galaxien entfernen sich immer schneller voneinander & Galaxien bleiben im gleichen Raum, verlieren aber Energie \\
		\hline
		Eventuelle Verdünnung aller Materie & Kontinuierliche Umwandlung von Materie in dunkle Energie \\
		\hline
		Endet im "Großen Riss" oder in ewiger Expansion & Endet in einem vom dunklen Energiefeld dominierten Zustand \\
		\hline
	\end{tabular}
\end{tcolorbox}

\section{Literaturverzeichnis}

\begin{thebibliography}{99}
	
	\bibitem{pascher} Pascher, J. (2025). Ein Modell mit absoluter Zeit und variabler Energie: Eine ausführliche Untersuchung der Grundlagen.
	
	\bibitem{pascher2} Pascher, J. (2025). Erweiterungen der Quantenmechanik durch intrinsische Zeit.
	
	\bibitem{pascher3} Pascher, J. (2025). Komplementäre Erweiterungen der Physik: Absolute Zeit und Intrinsische Zeit.
	
	\bibitem{supernova} Perlmutter, S., et al. (1999). Measurements of $\Omega$ and $\Lambda$ from 42 High-Redshift Supernovae. The Astrophysical Journal, 517, 565.
	
	\bibitem{riess} Riess, A. G., et al. (1998). Observational Evidence from Supernovae for an Accelerating Universe and a Cosmological Constant. The Astronomical Journal, 116, 1009.
	
	\bibitem{planck} Planck Collaboration. (2020). Planck 2018 results. VI. Cosmological parameters. Astronomy \& Astrophysics, 641, A6.
	
	\bibitem{cmb} Bennett, C. L., et al. (2013). Nine-year Wilkinson Microwave Anisotropy Probe (WMAP) Observations: Final Maps and Results. The Astrophysical Journal Supplement Series, 208, 20.
	
	\bibitem{bao} Eisenstein, D. J., et al. (2005). Detection of the Baryon Acoustic Peak in the Large-Scale Correlation Function of SDSS Luminous Red Galaxies. The Astrophysical Journal, 633, 560.
	
	\bibitem{quintessence} Caldwell, R. R., Dave, R., Steinhardt, P. J. (1998). Cosmological Imprint of an Energy Component with General Equation of State. Physical Review Letters, 80, 1582.
	
	\bibitem{euclid} Laureijs, R., et al. (2011). Euclid Definition Study Report. ESA/SRE(2011)12.
	
	\bibitem{tired} Zwicky, F. (1929). On the Red Shift of Spectral Lines through Interstellar Space. Proceedings of the National Academy of Sciences, 15, 773.
	
	\bibitem{alfa} Webb, J. K., et al. (2011). Indications of a Spatial Variation of the Fine Structure Constant. Physical Review Letters, 107, 191101.
	
	\bibitem{vacuum} Weinberg, S. (1989). The Cosmological Constant Problem. Reviews of Modern Physics, 61, 1.
	
	\bibitem{scalar} Fujii, Y., Maeda, K. (2003). The Scalar-Tensor Theory of Gravitation. Cambridge University Press.
	
	\bibitem{lambda} Carroll, S. M. (2001). The Cosmological Constant. Living Reviews in Relativity, 4, 1.
	
\end{thebibliography}

\end{document}

