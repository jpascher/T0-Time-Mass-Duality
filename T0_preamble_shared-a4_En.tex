% ==============================================================================
% T0 Theory: Shared English Preamble – Optimized for Book (approx. 500 Pages)
% Version: 1.2
% Author: Johann Pascher
% Date: 2026
% ==============================================================================
%
% This preamble is optimized for thick books (approx. 500 pages).
% Paper format: 8.25" × 11"
% Binding adjustments:
%   - Enlarged inner margin and bindingoffset for perfect binding
%   - Slightly smaller outer margins for more text per page
%   - Enlarged bottom margin for comfortable readability
%   - Openright for new chapters (with book class)
%
% Recommended document class:
%   \documentclass[11pt,openright,twoside]{book}
%
% ==============================================================================

% =============================================================================
% SECTION 1: Encoding and Language
% =============================================================================
\usepackage[utf8]{inputenc}
\usepackage[T1]{fontenc}
\usepackage[english]{babel}
\usepackage{lmodern}

% =============================================================================
% SECTION 2: Page Geometry – Book Optimization for ~500 Pages (8.25" × 11")
% =============================================================================
\usepackage[paperwidth=8.25 in, paperheight=11in,
top=1.0in,
bottom=1.1in,           % enlarged for better readability
inner=1.0in,            % larger inner margin (binding)
outer=0.75in,           % smaller outer margin → more text per page
bindingoffset=0.75in,   % additional buffer for binding
twoside]{geometry}
\setlength{\headheight}{15pt}

% =============================================================================
% SECTION 3: Mathematics and Physics
% =============================================================================
\usepackage{amsmath,amssymb,amsfonts,amsthm}
\usepackage{mathtools}
\usepackage{physics}
\usepackage{siunitx}
\sisetup{
	locale=US,
	group-separator={,},
	output-decimal-marker={.},
	per-mode=symbol
}

% =============================================================================
% SECTION 4: Graphics and Tables
% =============================================================================
\usepackage{graphicx}
\usepackage[table,xcdraw]{xcolor}
\usepackage{tikz}
\usetikzlibrary{arrows.meta,positioning,shapes.geometric,decorations.pathmorphing,patterns,shapes.arrows,intersections}
\usepackage{pgfplots}
\pgfplotsset{compat=1.18}
\usepackage[most]{tcolorbox}
\tcbuselibrary{breakable}
\usepackage{booktabs}
\usepackage{array}
\usepackage{longtable}
\usepackage{float}
\usepackage{adjustbox}
\usepackage{rotating}
\usepackage{tabularx}
\usepackage{makecell}
\usepackage{multirow}
%\usepackage{alphalph}
% =============================================================================
% SECTION 5: Document Formatting
% =============================================================================
\usepackage{fancyhdr}
\renewcommand{\headrulewidth}{0.4pt}
\renewcommand{\footrulewidth}{0.4pt}
\usepackage{tocloft}
\usepackage{hyperref}
\hypersetup{
	colorlinks=true,
	linkcolor=black,
	citecolor=black,
	urlcolor=black,
	breaklinks=true,
	bookmarksnumbered=true,
	unicode=true
}
\usepackage{bookmark}
\usepackage{cleveref}

\setcounter{tocdepth}{3}
\usepackage{microtype}
\usepackage{enumitem}
\usepackage{setspace}
\usepackage{ragged2e}
\usepackage{multicol}

% =============================================================================
% SECTION 6: Code and Algorithms
% =============================================================================
\usepackage{algorithm}
\usepackage{algorithmic}
\usepackage{listings}
\lstset{
	basicstyle=\ttfamily\footnotesize,
	breaklines=true,
	breakatwhitespace=true,
	columns=flexible,
	keepspaces=true,
	showstringspaces=false,
	frame=single,
	xleftmargin=0pt,
	xrightmargin=0pt
}
\usepackage{mdframed}

% =============================================================================
% SECTION 7: Additional Packages
% =============================================================================
\usepackage{pdflscape}
\usepackage{braket}
\usepackage{cancel}
\usepackage{caption}
\usepackage{csquotes}
\usepackage{gensymb}
\usepackage{hyphenat}
\usepackage{textcomp}
\usepackage{textgreek}
\usepackage{upgreek}
\usepackage{url}
\usepackage{slashed}
\usepackage{bm}
\usepackage{newunicodechar}

% Für bessere Zeilenumbrüche
\emergencystretch=3em
\tolerance=1000
\hbadness=1000
% --- Mathematical Operators and Notation ---
\newcommand{\primrel}[1]{\mathbf{#1}} % Prime relation notation
\newcommand{\vect}[1]{\mathbf{#1}}    % Vector notation (bold)
% Alternative vector notation (if you prefer arrows):
% \newcommand{\vect}[1]{\vec{#1}}
% =============================================================================
% SECTION 8: Citation Commands (Compatibility)
% =============================================================================
\providecommand{\citep}[1]{\cite{#1}}
\providecommand{\citet}[1]{\cite{#1}}

% =============================================================================
% SECTION 9: Colors
% =============================================================================
\definecolor{gold}{RGB}{255,215,0}
\definecolor{blue}{rgb}{0,0,1}
\definecolor{boxgray}{RGB}{240,240,240}
\definecolor{deepblue}{RGB}{0,0,127}
\definecolor{deepgreen}{RGB}{0,127,0}
\definecolor{deepred}{RGB}{191,0,0}
\definecolor{t0blue}{RGB}{33,150,243}
\definecolor{t0green}{RGB}{76,175,80}
\definecolor{t0orange}{RGB}{255,152,0}
\definecolor{t0purple}{RGB}{156,39,176}
\definecolor{t0red}{RGB}{244,67,54}
\definecolor{t0yellow}{RGB}{255,204,0}

% =============================================================================
% SECTION 10: Column Types
% =============================================================================
\newcolumntype{L}[1]{>{\raggedright\arraybackslash}p{#1}}
\newcolumntype{C}[1]{>{\centering\arraybackslash}p{#1}}

% =============================================================================
% SECTION 11: Unicode Character Mappings
% =============================================================================
\newunicodechar{ħ}{$\hbar$}
\newunicodechar{↔}{$\leftrightarrow$}
\newunicodechar{⇐}{$\Leftarrow$}
\newunicodechar{⇒}{$\Rightarrow$}
\newunicodechar{⇔}{$\Leftrightarrow$}
\newunicodechar{∂}{$\partial$}
\newunicodechar{∅}{$\emptyset$}
\newunicodechar{∇}{$\nabla$}
\newunicodechar{∈}{$\in$}
\newunicodechar{∉}{$\notin$}
\newunicodechar{∏}{$\prod$}
\newunicodechar{∑}{$\sum$}
\newunicodechar{√}{\ensuremath{\sqrt{}}}
\newunicodechar{∝}{$\propto$}
\newunicodechar{∞}{$\infty$}
\newunicodechar{∩}{$\cap$}
\newunicodechar{∪}{$\cup$}
\newunicodechar{∫}{$\int$}
\newunicodechar{≈}{$\approx$}
\newunicodechar{≠}{$\neq$}
\newunicodechar{≤}{$\leq$}
\newunicodechar{≥}{$\geq$}
\newunicodechar{ξ}{\ensuremath{\xi}}
\newunicodechar{μ}{\ensuremath{\mu}}
\newunicodechar{ψ}{\ensuremath{\psi}}
\newunicodechar{φ}{\ensuremath{\phi}}
\newunicodechar{π}{\ensuremath{\pi}}
\newunicodechar{λ}{\ensuremath{\lambda}}
\newunicodechar{Δ}{\ensuremath{\Delta}}

% Custom commands for specific documents
\newcommand{\Weyl}{\mathrm{Weyl}}
\newcommand{\ZPinch}{\nabla p = \mathbf{J} \times \mathbf{B}}
\newcommand{\SynchPower}{P_{\text{synch}}}
% Symbole für Tabellen
\newcommand{\checkmarkx}{$\checkmark$}
\newcommand{\warningx}{{\color{orange}$\triangle$}}

% Mathematische Operatoren
\newcommand{\mytimes}{\times}
\newcommand{\xiT}{\xi_{\mathrm{T}}}
\newcommand{\phiT}{\phi_{\mathrm{T}}}
\newcommand{\Lorentz}{\mathcal{L}}
\newcommand{\Riem}{\mathcal{R}}
% In T0_preamble_shared-a4_En.tex oder Hauptdatei:
\newcommand{\lambdabar}{\mathchar'26\mkern-11mu\lambda}

% =============================================================================
% SECTION 12: Hyperref Settings
% =============================================================================
\hypersetup{
	colorlinks=true,
	linkcolor=blue,
	citecolor=blue,
	urlcolor=blue,
	breaklinks=true,
	bookmarksnumbered=true,
	pdfstartview=FitH
}

% =============================================================================
% SECTION 13: Theorem Environments (English)
% =============================================================================
\theoremstyle{plain}
\newtheorem{theorem}{Theorem}[section]
\newtheorem{lemma}[theorem]{Lemma}
\newtheorem{proposition}[theorem]{Proposition}
\newtheorem{corollary}[theorem]{Corollary}

\theoremstyle{definition}
\newtheorem{definition}[theorem]{Definition}
\newtheorem{example}[theorem]{Example}
\newtheorem{insight}[theorem]{Insight}
\newtheorem{discovery}[theorem]{Discovery}
\newtheorem{erkenntnis}[theorem]{Insight}

\theoremstyle{remark}
\newtheorem{remark}[theorem]{Remark}
\newtheorem{axiom}{Axiom}
\newtheorem{principle}{Principle}
\newtheorem{bemerkung}[theorem]{Remark}
\newtheorem{warnung}[theorem]{Warning}

% =============================================================================
% SECTION 14: T0-Specific Commands
% =============================================================================

% --- Core T0 Fields ---
\newcommand{\Tfield}{T(x,t)}
\providecommand{\Tfieldt}{T(\vec{x},t)}
\newcommand{\Efield}{E(x,t)}
\newcommand{\mfield}{m(x,t)}
\providecommand{\vecx}{\vec{x}}

% --- Lagrangian ---
\newcommand{\Lag}{\mathcal{L}}
\newcommand{\calL}{\mathcal{L}}

% --- Greek Letters and Constants ---
\newcommand{\alphaem}{\alpha}
\newcommand{\betaT}{\beta_T}

\newcommand{\xipar}{\xi}

% --- Energy and Planck Units ---
\newcommand{\Ezero}{E_0}
\newcommand{\EPlanck}{E_{\text{Pl}}}
\newcommand{\Mpl}{M_{\text{Pl}}}
\newcommand{\mP}{m_{\text{P}}}
\newcommand{\lP}{\ell_{\text{P}}}
\newcommand{\tP}{t_{\text{P}}}
\newcommand{\LPlanck}{\ell_{\text{Pl}}}
\newcommand{\TPlanck}{t_{\text{Pl}}}

% --- Coupling Constants ---
\newcommand{\Gnat}{G_{\text{nat}}}
\newcommand{\alphaEM}{\alpha_{\text{EM}}}
\newcommand{\alphaSI}{\alpha_{\text{SI}}}
\newcommand{\Hubble}{H_0}
\newcommand{\LCDM}{\Lambda\text{CDM}}
\newcommand{\natunits}{(nat. units)}

% --- T0 Model Parameters ---
\newcommand{\xigeom}{\xi_{\mathrm{geom}}}
\newcommand{\rzero}{r_{0}}
\newcommand{\xirat}{\xi_{\mathrm{rat}}}
\newcommand{\tzero}{t_{0}}
\newcommand{\Lambdat}{\Lambda_{\mathrm{t}}}
\newcommand{\EP}{E_{\mathrm{P}}}
\newcommand{\Emu}{E_{\mu}}
\newcommand{\Ee}{E_{e}}
\newcommand{\Etau}{E_{\tau}}
\newcommand{\alphafine}{\alpha_{\mathrm{fine}}}
\newcommand{\alphal}{\alpha_{\ell}}
\newcommand{\Lzero}{\ell_{0}}
\newcommand{\Lp}{\ell_{\mathrm{P}}}

% --- Additional T0 Commands ---
\newcommand{\Cconv}{C_{\mathrm{conv}}}  % Conversion factor for gravitational constant
\newcommand{\Kfrak}{K_{\mathrm{frak}}}  % Fractal correction factor
\newcommand{\Dfrak}{D_{\text{frak}}}
\newcommand{\betapar}{\ensuremath{\beta_T}}
\newcommand{\alphapar}{\alpha}
\newcommand{\deltafield}{\delta \phi}
\newcommand{\deltam}{\delta m}
\newcommand{\deltaE}{\delta E}
\newcommand{\Exi}{E_{\xi}}
\newcommand{\Lxi}{\ell_{\xi}}
\newcommand{\rhoCMB}{\rho_{\text{CMB}}}
\newcommand{\rhoCasimir}{\rho_{\text{Casimir}}}
\newcommand{\Leff}{L_{\text{eff}}}
\newcommand{\CQCD}{C_{\mathrm{QCD}}}
\newcommand{\Kspec}{K_{\mathrm{spec}}}
\newcommand{\Tzero}{\ensuremath{T_0}}
\newcommand{\Eabs}{E_{\text{abs}}}
\newcommand{\taupar}{\tau}
% ============================================
% MISSING CUSTOM COMMAND DEFINITIONS
% ============================================

% Mathematical symbols and operators
\newcommand{\Df}{D_f}                     % Fractal dimension
\newcommand{\Evis}{E_{\text{vis}}}        % Visual energy scale

\newcommand{\Phiphoton}{\Phi_{\gamma}}    % Photon flux
\newcommand{\etavis}{\eta_{\text{vis}}}   % Visual efficiency
\newcommand{\Cto}{C_{\text{to}}}          % Conversion constant
\newcommand{\lambdazero}{\lambda_0}       % Reference wavelength
\newcommand{\nuzero}{\nu_0}               % Reference frequency

\newcommand{\pichar}{\pi}                 % Pi (character)
\newcommand{\lambdah}{\lambda_h}          % Higgs lambda

\newcommand{\Rzero}{R_0}                  % Reference radius
\newcommand{\alphaWnat}{\alpha_{\text{W}}^{\text{(nat)}}} % Weak coupling (natural)
\newcommand{\alphaWSI}{\alpha_{\text{W}}^{\text{(SI)}}}   % Weak coupling (SI)

% Special characters
\newcommand{\checked}{$\checkmark$}       % Checkmark symbol

% ============================================
% --- Provided Commands (may be redefined elsewhere) ---
\providecommand{\xiconst}{\xi_{\text{const}}}
\providecommand{\DhiggsT}{D_{\text{Higgs-T}}}
\providecommand{\rhoE}{\rho_{E}}
\providecommand{\Echar}{E_{\text{char}}}
\providecommand{\kfrac}{k_{\text{frac}}}
\providecommand{\alphaEMSI}{\alpha_{\text{EM,SI}}}
\providecommand{\alphaEMnat}{\alpha_{\text{EM,nat}}}
\providecommand{\betaTSI}{\beta_{T,\text{SI}}}
\providecommand{\betaTnat}{\beta_{T,\text{nat}}}
\providecommand{\Gsi}{G_{\text{SI}}}
\providecommand{\xiparSI}{\xi_{\text{SI}}}
\providecommand{\xiparnat}{\xi_{\text{nat}}}
\providecommand{\meff}{m_{\text{eff}}}
\providecommand{\Tzerot}{T_{0}(t)}
\providecommand{\mzerot}{m_{0}(t)}
\providecommand{\Ezeroabs}{E_{0,\text{abs}}}
\providecommand{\Epar}{E_{\text{par}}}
\providecommand{\EzeroT}{E_{0,T}}
\providecommand{\EzeroH}{E_{0,H}}
\providecommand{\EzeroF}{E_{0,F}}
\providecommand{\EzeroM}{E_{0,M}}
\providecommand{\EzeroG}{E_{0,G}}
\providecommand{\EzeroQ}{E_{0,Q}}
\providecommand{\EzeroC}{E_{0,C}}
\providecommand{\EzeroP}{E_{0,P}}
\providecommand{\EzeroL}{E_{0,L}}
\providecommand{\EzeroS}{E_{0,S}}
\providecommand{\EzeroB}{E_{0,B}}
\providecommand{\EzeroD}{E_{0,D}}
\providecommand{\EzeroA}{E_{0,A}}
\providecommand{\EzeroR}{E_{0,R}}
\providecommand{\EzeroI}{E_{0,I}}
\providecommand{\EzeroU}{E_{0,U}}
\providecommand{\EzeroN}{E_{0,N}}
\providecommand{\EzeroV}{E_{0,V}}
\providecommand{\EzeroW}{E_{0,W}}
\providecommand{\EzeroY}{E_{0,Y}}
\providecommand{\EzeroK}{E_{0,K}}
\providecommand{\EzeroJ}{E_{0,J}}
\providecommand{\EzeroO}{E_{0,O}}

% =============================================================================
% SECTION 15: Color Boxes
% =============================================================================

% --- CRITICAL: Missing Definitions Added Here ---
\newtcolorbox{keyresult}[1][]{
	colback=green!5!white,
	colframe=green!75!black,
	fonttitle=\bfseries,
	title=#1,
	breakable,
	enhanced,
	before skip=10pt,
	after skip=10pt
}

\newtcolorbox{analysis}[1][]{
	colback=blue!5!white,
	colframe=blue!75!black,
	fonttitle=\bfseries,
	title=#1,
	breakable,
	enhanced,
	before skip=10pt,
	after skip=10pt
}

\newtcolorbox{correct}[1][]{
	colback=yellow!5!white,
	colframe=orange!75!black,
	fonttitle=\bfseries,
	title=#1,
	breakable,
	enhanced,
	before skip=10pt,
	after skip=10pt
}

% --- Existing Definitions (Keep All) ---
\newtcolorbox{fundamental}[1][]{
	colback=boxgray,
	colframe=t0blue,
	fonttitle=\bfseries,
	title=#1,
	sharp corners,
	boxrule=2pt
}

\newtcolorbox{newperspective}[1][]{
	colback=red!5!white,
	colframe=t0red,
	fonttitle=\bfseries,
	title=#1,
	sharp corners,
	boxrule=2pt
}

\newtcolorbox{formula}[1][]{
	colback=blue!5!white,
	colframe=blue!75!black,
	fonttitle=\bfseries,
	title=#1
}

\newtcolorbox{result}[1][]{
	colback=green!5!white,
	colframe=green!75!black,
	fonttitle=\bfseries,
	title=#1
}

\newtcolorbox{derivation}[1][]{
	colback=green!5!white,
	colframe=green!75!black,
	title=#1,
	fonttitle=\bfseries,
	breakable
}

\newtcolorbox{summary}[1][]{
	colback=gray!10!white,
	colframe=gray!75!black,
	title=#1,
	fonttitle=\bfseries,
	breakable
}

\newtcolorbox{comparison}[1][]{
	colback=purple!5!white,
	colframe=purple!75!black,
	title=#1,
	fonttitle=\bfseries,
	breakable
}

\newtcolorbox{relation}[1][]{
	colback=cyan!5!white,
	colframe=cyan!75!black,
	title=#1,
	fonttitle=\bfseries,
	breakable
}

\newtcolorbox{principleBox}[1][]{
	colback=yellow!5!white,
	colframe=yellow!75!black,
	title=#1,
	fonttitle=\bfseries,
	breakable
}

\newtcolorbox{insightBox}[1][]{colback=blue!5,colframe=t0blue,title={#1},fonttitle=\bfseries,breakable}
\newtcolorbox{discoveryBox}[1][]{colback=green!5,colframe=t0green,title={#1},fonttitle=\bfseries,breakable}
\newtcolorbox{revelation}[1][]{colback=red!5,colframe=t0red,title={#1},fonttitle=\bfseries,breakable}
\newtcolorbox{keypoint}[1][]{colback=blue!5,colframe=t0blue,title={#1},fonttitle=\bfseries,breakable}
\newtcolorbox{evidence}[1][]{colback=green!5,colframe=t0green,title={#1},fonttitle=\bfseries,breakable}
\newtcolorbox{conclusionBox}[1][]{colback=gray!5,colframe=gray,title={#1},fonttitle=\bfseries,breakable}
\newtcolorbox{significance}[1][]{colback=yellow!5,colframe=orange,title={#1},fonttitle=\bfseries,breakable}
\newtcolorbox{philosophical}[1][]{colback=purple!5,colframe=purple,title={#1},fonttitle=\bfseries,breakable}
\newtcolorbox{implicationBox}[1][]{colback=cyan!5,colframe=cyan,title={#1},fonttitle=\bfseries,breakable}
\newtcolorbox{perspectiveBox}[1][]{colback=blue!5,colframe=t0blue,title={#1},fonttitle=\bfseries,breakable}
\newtcolorbox{revolutionary}[1][]{colback=red!5,colframe=t0red,title={#1},fonttitle=\bfseries,breakable}

\newtcolorbox{technical}[1][]{colback=gray!5,colframe=gray!75!black,title={#1},fonttitle=\bfseries,breakable}
\newtcolorbox{technicalBox}[1][]{colback=gray!5,colframe=gray!75!black,title={#1},fonttitle=\bfseries,breakable}
\newtcolorbox{notationBox}[1][]{colback=yellow!5,colframe=yellow!75!black,title={#1},fonttitle=\bfseries,breakable}
\newtcolorbox{verification}[1][]{colback=orange!5!white,colframe=orange!75!black,fonttitle=\bfseries,title={#1}}
\newtcolorbox{explanationBox}[1][]{colback=purple!5!white,colframe=purple!75!black,fonttitle=\bfseries,title={#1}}
\newtcolorbox{interpretationBox}[1][]{colback=cyan!5!white,colframe=cyan!75!black,fonttitle=\bfseries,title={#1}}
\newtcolorbox{explanation}[1][]{colback=purple!5!white,colframe=purple!75!black,fonttitle=\bfseries,title={#1},breakable}
\newtcolorbox{interpretation}[1][]{colback=cyan!5!white,colframe=cyan!75!black,fonttitle=\bfseries,title={#1},breakable}
\newtcolorbox{proof_step}[1][]{colback=gray!5!white,colframe=gray!75!black,fonttitle=\bfseries,title={#1},breakable}
\newtcolorbox{experimental}[1][]{colback=teal!5!white,colframe=teal!75!black,fonttitle=\bfseries,title={#1},breakable}

\newtcolorbox{important}[1][]{colback=red!5!white,colframe=red!75!black,title={#1},fonttitle=\bfseries,breakable}
\newtcolorbox{warning}[1][]{colback=orange!5!white,colframe=orange!75!black,title={#1},fonttitle=\bfseries,breakable}
\newtcolorbox{caution}[1][]{colback=yellow!5!white,colframe=yellow!75!black,title={#1},fonttitle=\bfseries,breakable}
\newtcolorbox{highlight}[1][]{colback=yellow!10!white,colframe=yellow!75!black,title={#1},fonttitle=\bfseries,breakable}
\newtcolorbox{critical}[1][]{colback=red!10!white,colframe=red!75!black,title={#1},fonttitle=\bfseries,breakable}

\newtcolorbox{application}[1][]{colback=green!5!white,colframe=green!75!black,title={#1},fonttitle=\bfseries,breakable}
\newtcolorbox{experiment}[1][]{colback=cyan!5!white,colframe=cyan!75!black,title={#1},fonttitle=\bfseries,breakable}
\newtcolorbox{historical}[1][]{colback=brown!5!white,colframe=brown!75!black,title={#1},fonttitle=\bfseries,breakable}
\newtcolorbox{numerical}[1][]{colback=gray!5!white,colframe=gray!75!black,title={#1},fonttitle=\bfseries,breakable}
\newtcolorbox{overview}[1][]{colback=blue!5!white,colframe=blue!75!black,title={#1},fonttitle=\bfseries,breakable}
\newtcolorbox{speculation}[1][]{colback=purple!5!white,colframe=purple!75!black,title={#1},fonttitle=\bfseries,breakable}
\newtcolorbox{question}[1][]{colback=orange!5!white,colframe=orange!75!black,title={#1},fonttitle=\bfseries,breakable}
\newtcolorbox{method}[1][]{colback=teal!5!white,colframe=teal!75!black,title={#1},fonttitle=\bfseries,breakable}
\newtcolorbox{units}[1][]{colback=gray!5!white,colframe=gray!75!black,title={#1},fonttitle=\bfseries,breakable}
\newtcolorbox{achievement}[1][]{colback=gold!5!white,colframe=orange!75!black,title={#1},fonttitle=\bfseries,breakable}
\newtcolorbox{equivalence}[1][]{colback=cyan!5!white,colframe=cyan!75!black,title={#1},fonttitle=\bfseries,breakable}
\newtcolorbox{dimensional}[1][]{colback=purple!5!white,colframe=purple!75!black,title={#1},fonttitle=\bfseries,breakable}

\newtcolorbox{photon}[1][]{colback=yellow!5!white,colframe=yellow!75!black,title={#1},fonttitle=\bfseries,breakable}
\newtcolorbox{neutrino}[1][]{colback=blue!5!white,colframe=blue!75!black,title={#1},fonttitle=\bfseries,breakable}
\newtcolorbox{revolution}[1][]{colback=red!5!white,colframe=red!75!black,title={#1},fonttitle=\bfseries,breakable}
\newtcolorbox{t0box}[1][]{colback=blue!5!white,colframe=t0blue,title={#1},fonttitle=\bfseries,breakable}
\newtcolorbox{documentbox}[1][]{colback=gray!5!white,colframe=gray!75!black,title={#1},fonttitle=\bfseries,breakable}
\newtcolorbox{sibox}[1][]{colback=green!5!white,colframe=green!75!black,title={#1},fonttitle=\bfseries,breakable}
\newtcolorbox{smbox}[1][]{colback=blue!5!white,colframe=blue!75!black,title={#1},fonttitle=\bfseries,breakable}
\newtcolorbox{pvbox}[1][]{colback=purple!5!white,colframe=purple!75!black,title={#1},fonttitle=\bfseries,breakable}
\newtcolorbox{koidebox}[1][]{colback=orange!5!white,colframe=orange!75!black,title={#1},fonttitle=\bfseries,breakable}

\newtcolorbox{formel}[1][]{colback=blue!5!white,colframe=blue!75!black,title={#1},fonttitle=\bfseries,breakable}
\newtcolorbox{schluessel}[1][]{colback=blue!5!white,colframe=blue!75!black,title={#1},fonttitle=\bfseries,breakable}
\newtcolorbox{wichtig}[1][]{colback=red!5!white,colframe=red!75!black,title={#1},fonttitle=\bfseries,breakable}
\newtcolorbox{vorsicht}[1][]{colback=orange!5!white,colframe=orange!75!black,title={#1},fonttitle=\bfseries,breakable}
\newtcolorbox{revolutionaer}[1][]{colback=red!5!white,colframe=red!75!black,title={#1},fonttitle=\bfseries,breakable}
\newtcolorbox{numerisch}[1][]{colback=gray!5!white,colframe=gray!75!black,title={#1},fonttitle=\bfseries,breakable}
\newtcolorbox{experimentell}[1][]{colback=cyan!5!white,colframe=cyan!75!black,title={#1},fonttitle=\bfseries,breakable}
\newtcolorbox{anwendung}[1][]{colback=green!5!white,colframe=green!75!black,title={#1},fonttitle=\bfseries,breakable}
\newtcolorbox{alternative}[1][]{colback=orange!5!white,colframe=orange!75!black,title={#1},fonttitle=\bfseries,breakable}
\newtcolorbox{beziehung}[1][]{colback=cyan!5!white,colframe=cyan!75!black,title={#1},fonttitle=\bfseries,breakable}
\newtcolorbox{folgerung}[1][]{colback=green!5!white,colframe=green!75!black,title={#1},fonttitle=\bfseries,breakable}
\newtcolorbox{abhandlung}[1][]{colback=gray!5!white,colframe=gray!75!black,title={#1},fonttitle=\bfseries,breakable}
\newtcolorbox{prinzipBox}[1][]{colback=blue!5!white,colframe=blue!75!black,title={#1},fonttitle=\bfseries,breakable}
\newtcolorbox{prinzip}[1][]{colback=blue!5!white,colframe=blue!75!black,title={#1},fonttitle=\bfseries,breakable}
\newtcolorbox{beweis}[1][]{colback=gray!5!white,colframe=gray!75!black,title={#1},fonttitle=\bfseries,breakable}
\newtcolorbox{key}[2][]{colback=blue!5!white,colframe=blue!75!black,title={#2},fonttitle=\bfseries,breakable}
\newtcolorbox{category}[1][]{colback=purple!5!white,colframe=purple!75!black,title={#1},fonttitle=\bfseries,breakable}

% =============================================================================
% SECTION 16: Additional Simple Environments
% =============================================================================
\newenvironment{treatise}{\begin{quote}}{\end{quote}}
\newenvironment{gemeinsam}{\begin{quote}}{\end{quote}}
\newenvironment{vergleich}{\begin{quote}}{\end{quote}}
\newenvironment{vorteil}{\begin{quote}}{\end{quote}}
\newenvironment{quantum}{\begin{quote}}{\end{quote}}

% =============================================================================
% SECTION 17: Layout Settings (Kindle-compatible)
% =============================================================================
\raggedbottom

\usepackage{environ}
\let\oldtabular\tabular
\let\endoldtabular\endtabular

\newenvironment{scaledtable}[1][0.85]{%
	\begingroup\footnotesize\setlength{\LTleft}{0pt}\setlength{\LTright}{0pt}%
}{%
	\endgroup%
}

\newcommand{\widetable}[1]{\resizebox{\textwidth}{!}{#1}}

% =============================================================================
% SECTION 18: Table of Contents Formatting
% =============================================================================
\renewcommand{\cftsecfont}{\color{blue}}
\renewcommand{\cftsubsecfont}{\color{blue}}
\renewcommand{\cftsecpagefont}{\color{blue}}
\renewcommand{\cftsubsecpagefont}{\color{blue}}
\renewcommand{\cfttoctitlefont}{\huge\bfseries\color{blue}}

% =============================================================================
% SECTION 19: Default Header and Footer – Book-Style
% =============================================================================
\pagestyle{fancy}
\fancyhf{}
\fancyfoot[C]{\thepage}
% Wichtig: Auch der plain-Stil (Kapitelanfangsseiten) soll genau dasselbe haben
\fancypagestyle{plain}{%
\fancyfoot[C]{\thepage}
}
% =============================================================================
% SECTION 20: Custom T0 Commands (for specific documents)
% =============================================================================
% Plasma cosmology commands (used in 036_T0_peratt)
\providecommand{\Weyl}{\mathrm{Weyl}}
\providecommand{\ZPinch}{\nabla p = \mathbf{J} \times \mathbf{B}}
\providecommand{\SynchPower}{P_{\text{synch}}}

% =============================================================================
% SECTION 21: Additional Book-specific Settings
% =============================================================================
\usepackage{titlesec}
\titleformat{\chapter}[display]
{\normalfont\huge\bfseries}{\chaptertitlename\ \thechapter}{20pt}{\Huge}
\titlespacing*{\chapter}{0pt}{50pt}{40pt}

\setlength{\parindent}{1.5em}
\setlength{\parskip}{0pt}

% ==============================================================================
% End of Shared Preamble – Book Version
% ==============================================================================
% ============================================
% Foundation Environment Definition
% ============================================
\newenvironment{foundation}
{\begin{quote}\itshape}
	{\end{quote}}
% Einfache Definitionen ohne zusätzliche Pakete
\newenvironment{common}
{\par\noindent\textbf{Common Ground:}\par\nopagebreak\medskip}
{\par\medskip}



\newenvironment{advantage}
{\par\noindent\textbf{T0 Advantage:}\par\nopagebreak\medskip}
{\par\medskip}